% A complete graph
% Author: Quintin Jean-Noël
% <http://moais.imag.fr/membres/jean-noel.quintin/>
\documentclass{article}
\usepackage{tikz}
%%%<
\usepackage{verbatim}
\usepackage{verbatim}

\usepackage{color}

\usepackage[active,tightpage]{preview}

\definecolor{colorNodeInner}{RGB}{253,141,60}
\definecolor{colorNodeEdge}{RGB}{217,72,1}
\definecolor{colorEdge}{RGB}{241,105,19}


\PreviewEnvironment{tikzpicture}
\setlength\PreviewBorder{5pt}%
%%%>
\usetikzlibrary[topaths,positioning]
% A counter, since TikZ is not clever enough (yet) to handle
% arbitrary angle systems.
\newcount\mycount
\newcount\mylabel
\begin{document}
\begin{tikzpicture}[transform shape]
  %the multiplication with floats is not possible. Thus I split the loop in two.
  \foreach \number in {1,...,12}{
      % Computer angle:
        \mycount=\number
        \advance\mycount by -1
  \multiply\mycount by 30
        \advance\mycount by 0
      \node[draw=colorNodeEdge,line width=2pt,circle,fill=colorNodeInner, inner sep=0.25cm] (N-\number) at (\the\mycount:5.4cm) {};
    }
  %\foreach \number in {9,...,16}{
      % Computer angle:
  %      \mycount=\number
  %      \advance\mycount by -1
  %\multiply\mycount by 45
  %      \advance\mycount by 22.5
  %    \node[draw,circle,inner sep=0.25cm] (N-\number) at (\the\mycount:5.4cm) {};
  %  }

  %% Another complete graph
  \foreach \number in {13,...,24}{
      % Computer angle:
        \mycount=\number
        \advance\mycount by -1
  \multiply\mycount by 30
        \advance\mycount by 0

	\mylabel=\number
	\advance\mylabel by -12
      \node[right=13cm of N-1, draw=colorNodeEdge,line width=2pt,circle,fill=colorNodeInner, inner sep=0.15cm] (N-\number) at (\the\mycount:5.4cm) {\Huge \the\mylabel};
    }

  \foreach \numbera/\numberb in {13/15,18/13,19/13,21/13,
				14/18,14/23,
				15/17,22/15,15/24,
				16/20,16/23,
				17/18,22/17,17/24,
				18/21,
				19/20,19/23,
				20/24,
				21/24}{
    \path (N-\numbera) edge[-,bend right=4, colorEdge, line width=0.7pt] (N-\numberb);
    \path (N-\numbera) edge[-,bend right=0, colorEdge, line width=0.7pt] (N-\numberb);
    \path (N-\numbera) edge[-,bend right=8, colorEdge, line width=0.7pt] (N-\numberb);
    \path (N-\numbera) edge[-,bend left=4, colorEdge, line width=0.7pt] (N-\numberb);
    %\path (N-\numbera) edge[-,bend left=8, colorEdge, line width=0.7pt] (N-\numberb);

    }


  %\path (N-1) edge[-, bend left = 3, line width = 2pt, colorEdge] (N-19);

\end{tikzpicture}
\end{document}
