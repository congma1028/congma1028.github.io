\pdfminorversion=4
\documentclass[compress,
mathserif,wide,%red,
%handout
]{beamer}

\newcommand{\indicator}{\ensuremath{\mathbbm{1}}}

\usepackage{amsmath,amsfonts,bm,bbm,comment}
\usepackage{latexsym,tabularx,amscd}
\usepackage{etex} 
\usepackage{algorithm}
\usepackage[noend]{algorithmic}
\usepackage{subfigure}
\usepackage{url}
%\usepackage{multimedia}
\usepackage{hyperref}
\usepackage{skull}
%\usepackage{enumitem}
%\usepackage{movie15}

\usepackage{multirow}
\usepackage{cancel}

%\usepackage{ulem}
\usepackage{lmodern}
\usepackage{array} 
\usepackage{etoolbox}
\usepackage{caption}
\usepackage[normalem]{ulem}

\usepackage{mathtools}

\usepackage[space]{grffile}

\usepackage{tikz}
\usetikzlibrary{shapes.arrows}
\usetikzlibrary{arrows,positioning}
\usepackage{pgfplots}
%\usetikzlibrary{pgfplots.groupplots}

\tikzset{
    myarrow/.style={
        draw,
        fill=black,
        single arrow,
        minimum height=6ex,
        single arrow head extend=1ex
    }
}


\newcommand{\vertiii}[1]{{\left\vert\kern-0.25ex\left\vert\kern-0.25ex\left\vert #1 
    \right\vert\kern-0.25ex\right\vert\kern-0.25ex\right\vert}}

\newcommand{\vertiiibig}[1]{{\big\vert\kern-0.25ex\big\vert\kern-0.25ex\big\vert #1 
    \big\vert\kern-0.25ex\big\vert\kern-0.25ex\big\vert}}

\newcommand{\vertiiiplain}[1]{{\vert\kern-0.25ex\vert\kern-0.25ex\vert #1 
    \vert\kern-0.25ex\vert\kern-0.25ex\vert}}



\usepackage{dsfont}
\DeclareMathOperator{\ind}{\mathds{1}}  % Indicator


\newcommand{\abs}[1]{\left|#1\right|}

\mode<presentation>

\usetheme{Madrid}
% other themes: AnnArbor, Antibes, Bergen, Berkeley, Berlin, Boadilla,
% boxes, CambridgeUS, Copenhagen, Darmstadt, default, Dresden,
% Frankfurt, Goettingen, Hannover, Ilmenau, JuanLesPins, Luebeck,
% Madrid, Maloe, Marburg, Montpellier, PaloAlto, Pittsburg, Rochester,
% Singapore, Szeged, classic

% \usecolortheme{lily} 
% color themes: albatross, beaver, beetle, crane, default, dolphin,
% dov, fly, lily, orchid, rose, seagull, seahorse, sidebartab,
% structure, whale, wolverine

% \usefonttheme{serif} 
% font themes: default, professionalfonts, serif, structurebold,
% structureitalicserif, structuresmallcapsserif

%\usefonttheme[onlymath]{serif}
%\usefonttheme[onlylarge]{structuresmallcapsserif} 
  
%\useoutertheme[subsection=false]{smoothbars}
% outer themes:
% default,infolines,miniframes,smoothbars,sidebar,split,shadow,tree,smoothtree
%\useinnertheme{rectangles}
\beamertemplatenavigationsymbolsempty

% \definecolor{fore}{RGB}{249,242,215}
% \definecolor{back}{RGB}{51,51,51}
% \definecolor{title}{RGB}{255,0,90}
% \setbeamercolor{titlelike}{fg=title}
% \setbeamercolor{normal text}{fg=fore,bg=back}
% \setbeamercovered{transparent}
  % or whatever (possibly just delete it)
%\beamertemplatenavigationsymbolsempty

\usepackage[english]{babel}
\usepackage[latin1]{inputenc}
% or whatever
%\usepackage{comicsans} 
%\renewcommand{\sfdefault}{comic} 
%\usepackage{mathptmx}
%\mathversion{bold}
%\usepackage{helvet}
%\usepackage{courier}

%\usepackage[T1]{fontenc}
% Or whatever. Note that the encoding and the font should match. If T1
% does not look nice, try deleting the line with the fontenc.


%%%% Figures %%%%%


\newcommand{\LectureFigs}{../Figures}


\definecolor{bluegray}{rgb}{0.15,0.20,0.40}
\definecolor{bluegraylight}{rgb}{0.35,0.40,0.60}
\definecolor{gray}{rgb}{0.3,0.3,0.3}
\definecolor{lightgray}{rgb}{0.7,0.7,0.7}
\definecolor{brightblue}{rgb}{0.1,0.4,0.7}
\definecolor{darkblue}{rgb}{0.1,0.1,0.6}
\definecolor{darkgreen}{rgb}{0.0,0.5,0.2}
\definecolor{orange}{rgb}{0.7,0.3,0}

\def \captioncolor { }

\setbeamertemplate{itemize item}{{\color{black}$\bullet$}}
\setbeamertemplate{itemize subitem}{{\color{black}$\circ$}}
\setbeamertemplate{frametitle}[default][center]
\setbeamerfont{frametitle}{series=\bfseries}
\addtobeamertemplate{frametitle}{\vskip1ex}{ \vspace{0.5em} \hrule}
\setbeamercolor{frametitle}{fg=,bg=}

\setbeamertemplate{theorems}[numbered]
\setbeamertemplate{caption}[numbered]
\setbeamertemplate{algorithm}[numbered]

\numberwithin{theorem}{subsection}
\numberwithin{equation}{subsection}
\numberwithin{figure}{subsection}
\numberwithin{algorithm}{subsection}


%\setbeamerfont{page number in head/foot}{size=\scriptsize}
\setbeamertemplate{footline}[frame number]
\setbeamercolor{page number in head/foot}{fg=black}
\setbeamersize{text margin left=2em,text margin right=2em}

\setbeamertemplate{footline}
{
    %\hspace{3em} \insertshorttitle \hfill \insertframenumber / \inserttotalframenumber\hspace{3em}
    \hspace{3em} \insertshorttitle \hfill \insertsubsectionnumber-\insertframenumber \hspace{3em}
    \vspace{1.5em}
}

\definecolor{babyblueeyes}{rgb}{0.63, 0.79, 0.95}
\definecolor{bleudefrance}{rgb}{0.19, 0.55, 0.91}
\definecolor{matlabPlotBlue}{rgb}{0.25, 0.58, 0.8}
\definecolor{matlabPlotOrange}{rgb}{0.855, 0.324, 0.098}



\captionsetup[figure]{labelfont={color=black}}

\AtBeginEnvironment{lemma}{%
  \setbeamercolor{block title}{fg=black,bg=babyblueeyes}
  \setbeamercolor{block body}{fg=black,bg=}
  \setbeamerfont*{block title}{series=\bfseries}
}

\AtBeginEnvironment{theorem}{%
  \setbeamercolor{block title}{fg=black,bg=babyblueeyes}
  \setbeamercolor{block body}{fg=black,bg=}
  \setbeamerfont*{block title}{series=\bfseries}
}

\AtBeginEnvironment{corollary}{%
  \setbeamercolor{block title}{fg=black,bg=babyblueeyes}
  \setbeamercolor{block body}{fg=black,bg=}
  \setbeamerfont*{block title}{series=\bfseries}
}

\AtBeginEnvironment{fact}{%
  \setbeamercolor{block title}{fg=black,bg=babyblueeyes}
  \setbeamercolor{block body}{fg=black,bg=}
  \setbeamerfont*{block title}{series=\bfseries}
}

\AtBeginEnvironment{definition}{%
  \setbeamercolor{block title}{fg=black,bg=babyblueeyes}
  \setbeamercolor{block body}{fg=black,bg=}
  \setbeamerfont*{block title}{series=\bfseries}
}

\newenvironment<>{varblock}[2][.9\textwidth]{%
  \vspace{-1em}
  \begin{center}
  \begin{minipage}{#1}
  %\setlength{\textwidth}{#1}
  \begin{actionenv}#3%
    \def\insertblocktitle{#2}%
    \par%
    \usebeamertemplate{block begin}}
  {\par%
    \usebeamertemplate{block end}%
  \end{actionenv}
  \end{minipage}
  \end{center}
}


\newcommand\blankfootnote[1]{%
  \let\thefootnote\relax\footnotetext{#1}%
  \let\thefootnote\svthefootnote%
}


%%%%%
\newcommand{\bb}{\bm b}
\newcommand{\bZ}{\bm Z}
\newcommand{\bC}{\bm C}
\newcommand{\cU}{{\cal U}}
\newcommand{\bR}{{\bm R}}
\newcommand{\bi}{{\bm i}}
\newcommand{\bS}{{\bm S}}
\newcommand{\cE}{{\cal E}}
\newcommand{\cF}{{\cal F}}
\newcommand{\bk}{{\bm k}}
\newcommand{\bu}{{\bm u}}
\newcommand{\bx}{{\bm x}}
\newcommand{\bA}{{\bm A}}
\newcommand{\bh}{{\bm h}}
\newcommand{\by}{{\bm y}}
\newcommand{\bI}{{\bm I}}
\newcommand{\ba}{{\bm a}}
\newcommand{\bw}{{\bm w}}
\newcommand{\bPhi}{{\bm \Phi}}
\newcommand{\bX}{{\bm X}}
\newcommand{\bY}{{\bm Y}}
\newcommand{\bM}{{\bm M}}
\newcommand{\be}{{\bm e}}
\newcommand{\bs}{{\bm s}}
\newcommand{\bB}{{\bm B}}
\newcommand{\bW}{{\bm W}}
\newcommand{\bN}{{\bm N}}
\newcommand{\bU}{{\bm U}}
\newcommand{\bV}{{\bm V}}
\newcommand{\bP}{{\bm P}}
\newcommand{\bH}{{\bm H}}
\newcommand{\bSigma}{{\bm \Sigma}}
\newcommand{\bL}{{\bm L}}



\newcommand{\bv}{\bm v}
\newcommand{\eps}{\epsilon}
\newcommand{\vf}{\varphi}
\newcommand{\de}{\delta}
\newcommand{\<}{\langle}
\renewcommand{\>}{\rangle}
\newcommand{\goto}{\rightarrow}
\newcommand{\argmin}{\mbox{argmin}}
\newcommand{\argmax}{\mbox{argmax}}
\newcommand{\Ave}{\mathop{\rm Ave}\nolimits}
\newcommand{\sgn}{\mbox{sgn}}
\newcommand{\cN}{\mathcal{N}}
\newcommand{\cR}{\mathcal{R}}

\def\C{{\mathbb{C}}}
\def\R{{{\mathbb{R}}}} %seemed too large in subscripts, so I changed --JW 
\def\P{{\hbox{\bf P}}}
\def\Z{{{\mathbb{Z}}}}
\def\T{{\hbox{\bf T}}}


\newcommand{\cA}{\mathcal{A}}
\newcommand{\cP}{\mathcal{P}}
\newcommand{\cD}{\mathcal{D}}
\newcommand{\cL}{\mathcal{L}}
\newcommand{\shrink}{\text{shrink}}


\newcommand{\Obs}{\Omega_{\text{obs}}}


\renewcommand{\P}{\operatorname{\mathbb{P}}}
\newcommand{\E}{\operatorname{\mathbb{E}}}

% Linear algebra macros
%\newcommand{\vct}[1]{#1}
%\newcommand{\mtx}[1]{#1}
\newcommand{\vct}[1]{\bm{#1}}
\newcommand{\mtx}[1]{\bm{#1}}
%\newcommand{\mtx}[1]{\mathsfsl{#1}}

\newcommand{\transp}{T}
\newcommand{\adj}{*}
\newcommand{\psinv}{\dagger}
\newcommand{\lspan}[1]{\operatorname{span}{#1}}

\newcommand{\range}{\operatorname{range}}
\newcommand{\colspan}{\operatorname{colspan}}

\newcommand{\rank}{\text{rank}}

\newcommand{\diag}{\operatorname{diag}}
\newcommand{\trace}{\text{Tr}}

\newcommand{\supp}[1]{\operatorname{supp}(#1)}

\newcommand{\smax}{\sigma_{\max}}
\newcommand{\smin}{\sigma_{\min}}

\newcommand{\restrict}[1]{\big\vert_{#1}}

\newcommand{\Id}{\text{\em I}}
\newcommand{\OpId}{\mathcal{I}}



% define your own colors:
\definecolor{Red}{rgb}{1,0,0}
\definecolor{Blue}{rgb}{0,0,1}
\definecolor{Green}{rgb}{0,1,0}
\definecolor{magenta}{rgb}{1,0,.6}
\definecolor{lightblue}{rgb}{0,.5,1}
\definecolor{lightpurple}{rgb}{.6,.4,1}
\definecolor{gold}{rgb}{.6,.5,0}
\definecolor{orange}{rgb}{1,0.4,0}
\definecolor{hotpink}{rgb}{1,0,0.5}
\definecolor{newcolor2}{rgb}{.5,.3,.5}
\definecolor{newcolor}{rgb}{0,.3,1}
\definecolor{newcolor3}{rgb}{1,0,.35}
\definecolor{darkgreen1}{rgb}{0, .35, 0}
\definecolor{darkgreen}{rgb}{0, .6, 0}
\definecolor{darkred}{rgb}{.75,0,0}

\xdefinecolor{olive}{cmyk}{0.64,0,0.95,0.4}
\xdefinecolor{purpleish}{cmyk}{0.75,0.75,0,0}

\definecolor{lightgrey}{RGB}{186,186,186}
\definecolor{back}{RGB}{51,51,51}
\definecolor{fore}{RGB}{249,242,215}
\definecolor{back}{RGB}{51,51,51}
\definecolor{title}{RGB}{255,0,20}
\definecolor{keywords}{RGB}{255,0,90}
\definecolor{comments}{RGB}{60,179,113}
\definecolor{cobalt}{RGB}{102,153,204}

\def \itemstyle {\bf}
\def \itemhighlight {\bf}  

% \definecolor{blockbodycolor}{RGB}{30,30,40}
% \definecolor{blocktitlecolor}{RGB}{55,55,75}
\definecolor{blockbodycolor}{RGB}{186,206,232}
\definecolor{blocktitlecolor}{RGB}{55,55,75}

\setbeamercolor{titlelike}{fg=darkred,bg=white}
\setbeamercolor{normal text}{fg=black,bg=white}
\setbeamercolor{block title}{fg=white,bg=blocktitlecolor}
\setbeamercolor{block body}{bg=blockbodycolor}
\setbeamercolor{alerted text}{fg=darkred}
\setbeamercolor{title}{fg=darkred,bg=lightgrey}


\newcommand{\alertb}[1]{\textcolor{blue}{ #1}}

\newcommand{\yp}[1]{\textcolor{red}{ #1}}
\newcommand{\zeronorm}[1]{\left\|#1 \right\|_0}
\newcommand{\unorm}[1]{\left\|#1 \right\|_u}
\newcommand{\ynorm}[1]{\left\|#1 \right\|_{\bar{y}}}
\newcommand{\onetwonorm}[1]{\left\|#1\right\|_{1,2}}
\newcommand{\opnorm}[1]{\left\|#1\right\|}
\newcommand{\fronorm}[1]{\left\|#1\right\|_{F}}
\newcommand{\onenorm}[1]{\left\|#1\right\|_{\ell_1}}
\newcommand{\twonorm}[1]{\left\|#1\right\|_{2}}
\newcommand{\oneinfnorm}[1]{\left\|#1\right\|_{1,\infty}}
\newcommand{\infnorm}[1]{\left\|#1\right\|_{\ell_\infty}}
\newcommand{\nucnorm}[1]{\left\|#1\right\|_*}

%\makeatletter
%\def\th@mystyle{%
%    \normalfont % body font
%    \setbeamercolor{block title example}{bg=,fg=black}
%    \setbeamerfont{block title example}{series=\bfseries}
%    \setbeamercolor{block body example}{bg=,fg=black}
%    %\setbeamertemplate{exampleblock}[numbered]
%    \def\inserttheoremblockenv{exampleblock}
%  }
%\makeatother
%\theoremstyle{mystyle}
%\newtheorem*{lem}{Lemma}

\setbeamertemplate{frametitle continuation}{}


\newcommand{\courseTitle}{STAT 37797: Mathematics of Data Science} 



% \subtitle
% {Include Only If Paper Has a Subtitle}

\vspace{.5cm}

\author{Yuxin Chen}

\institute[Princeton] % (optional, but mostly needed)
{Princeton University}


% - Use the \inst command only if there are several affiliations.
% - Keep it simple, no one is interested in your street address.


\subject{Beamer}
% This is only inserted into the PDF information catalog. Can be left
% out.

% Delete this, if you do not want the table of contents to pop up at
% the beginning of each subsection:
% \AtBeginSubsection[]
% {
%   \begin{frame}<beamer>
%     \frametitle{Outline}
%     \tableofcontents[currentsection,currentsubsection]
%   \end{frame}
% }






\graphicspath{{../../../Figures/}}


\title % (optional, use only with long paper titles)
{Generic analysis of local convergence}

\defbeamertemplate*{title page}{customized}[1][]
{

  \hfill {\em \courseTitle}

  \begin{center}
    \vspace{2.5em}
    \usebeamerfont{title} {\Large\bf\inserttitle} \par
  
    \vspace{1.5em}
    \includegraphics[width=2cm]{\LectureFigs/UC_logo.png} 
  
    \vspace{1em}
    {\large Cong Ma \par }

    \vspace{0.2em}
    { \large \quad University of Chicago, Autumn 2021 }
  \end{center}

  \vfill
}

\setcounter{subsection}{8}

\begin{document}

\begin{frame}[plain]
  \titlepage

\end{frame}

\begin{frame}
	\frametitle{Outline}
	\begin{itemize}
		\item Low-rank matrix sensing
		\item Phase retrieval
		\item Low-rank matrix completion
	\end{itemize}
\end{frame}


\begin{frame}[plain]

\vfill
\begin{center}
  {\Large \bf Low-rank matrix sensing}
\end{center}
\vfill

\end{frame}


\begin{frame}
	\frametitle{Low-rank matrix sensing}
	\begin{itemize}
		\item Groundtruth: rank-$r$ matrix $\bm{M}^\star \in \mathbb{R}^{n_1 \times n_2}$
		\item Observations:
			\begin{align*}
				 y_{i} = \langle \bm{A}_i, \bm{M}^\star \rangle, \qquad \text{for } 1 \leq i \leq m
			\end{align*}
		\item Goal: recover $\bm{M}^\star$ based on linear measurements $\{ \bm{A}_i, y_i\}_{ 1 \leq i \leq m}$
	\end{itemize}
\end{frame}

\begin{frame}
	\frametitle{How many measurements are needed}
	\begin{itemize}
		\item $m \geq n_1 n_2$ ``generic'' measurements suffice given theory of solving linear equations
		\item But $\bm{M}^{\star}$ only has $O((n_1 + n_2)r)$ degrees of freedom. Ideally, one hope for using only $O((n_1 + n_2)r)$ measurements
	\end{itemize}

\vfill
	{
\setbeamercolor{block body}{bg=babyblueeyes,fg=black}

\begin{varblock}[\textwidth]{}
\centering
Recovery is possible if $\{A_i\}$'s satisfy restricted isometry property
\end{varblock}
}

\end{frame}

\begin{frame}
	\frametitle{Restricted isometry property (RIP)}

Define linear operator $\mathcal{A}: \mathbb{R}^{n_1 \times n_2 } \mapsto \mathbb{R}^{m}$ t obe
\[
\mathcal{A} (\bm{M}) = [ m^{-1/2} \langle \bm{A}_i, \bm{M} \rangle ]_{1 \leq i \leq m}
\]

\vspace{-0.5em}
	\begin{definition}
		The operator $\mathcal{A}$ is said to satisfy $r$-RIP with RIP constant $\delta_{r} < 1$ if 
		\[
		(1 - \delta_{r}) \| \bm{M} \|_{\mathsf{F}}^{2} \leq \|\mathcal{A} (\bm{M}) \|_{2}^{2} \leq (1 + \delta_{r}) \| \bm{M} \|_{\mathsf{F}}^{2}
		\]
		holds simultaneously for all $\bm{M}$ of rank at most $r$.
	\end{definition}

	\begin{itemize}
			\item Many random designs satisfy RIP with high probability
			\item For instance, when $\bm{A}_{i}$ is composed of i.i.d.~$\mathcal{N}(0,1)$ entries, $\mathcal{A}$ obeys $r$-RIP with constant $\delta_{r}$ as soon as $m \gtrsim (n_1 + n_2) r / \delta_{r}^{2}$
	\end{itemize}
\end{frame}

\begin{frame}
	\frametitle{An optimization-based method}
	Consider the simple case when $\bm{M}^\star$ is psd and rank 1, i.e.,
	\[
		\bm{M}^{\star} = \bm{x}^{\star} \bm{x}^{\star \top}
	\]

	\vfill
	Then least-squares estimation yields
	\[
		\underset{\bm{x} \in \mathbb{R}^{n}}{\text{minimize}}\qquad f(\bm{x}) = \frac{1}{4m} \sum_{i=1}^{m} \left( \langle \bm{A}_{i}, \bm{x} \bm{x}^{\top} \rangle - y_i \right)^{2}
	\]

\end{frame}

\begin{frame}
	\frametitle{Gradient descent}
	Starting from $\bm{x}^0$, one proceeds by
	\begin{align*}
		\bm{x}^{t+1} &= \bm{x}^t - \eta \nabla f ( \bm{x}^t ) \\
		&= \bm{x}^t - \frac{\eta}{m} \sum_{i=1}^{m} \left( \langle \bm{A}_{i}, \bm{x}^t \bm{x}^{t\top} \rangle - y_i \right) \bm{A}_{i} \bm{x}^t
	\end{align*}

	Here we made simplifying assumption that $A_{i}$ is symmetric

	\vfill
	\begin{itemize}
			\item Under random design, when $m \to \infty$, this mirrors PCA problem with loss $\frac{1}{4} \| \bm{x} \bm{x}^\top  - \bm{x}^{\star} \bm{x}^{\star \top} \|_{\mathrm{F}}^{2}$; GD works locally
			\item How about finite-sample case? \\
			{\hfill \em \footnotesize --- RIP helps}
	\end{itemize}
\end{frame}

\begin{frame}
	\frametitle{Local convergence of gradient descent}
	
	\begin{theorem}\label{thm:sensing-rank-1}
		Suppose that $\mathcal{A}$ obeys 4-RIP with constant $\delta_{4} \leq 1 /44$. If $\| \bm{x}^0 - \bm{x}^\star \|_{2} \leq \|\bm{x}^\star\|_{2} / 12$, then GD with $\eta = 1 / (3 \|\bm{x}^\star\|_{2}^{2})$ obeys
		\begin{align*}
			\| \bm{x}^t - \bm{x}^\star \|_2 \leq (\tfrac{11}{12})^t \| \bm{x}^0 - \bm{x}^\star \|_2, \qquad \text{for }t=0,1,2,\ldots
		\end{align*}
	\end{theorem}

	\vfill
	\begin{itemize}
		\item local linear convergence within basin of attraction $\{\bm{x} \mid \| \bm{x} - \bm{x}^\star \|_{2} \leq \|\bm{x}^\star\|_{2} / 12\}$
		\item how do we initialize GD? spectral method
	\end{itemize}
\end{frame}

\begin{frame}
	\frametitle{Proof of Theorem~\ref{thm:sensing-rank-1}}
	In view of theory of gradient descent for locally strongly convex and smooth functions, it suffices to prove that 
	\[
		0.25 \|\bm{x}^\star\|_{2}^{2} \bm{I}_{n} \preceq \nabla^2 f (\bm{x}) \preceq 3 \|\bm{x}^\star\|_{2}^{2} \bm{I}_{n}
	\]
	holds for all 
	\[
	\{\bm{x} \mid \| \bm{x} - \bm{x}^\star \|_{2} \leq \|\bm{x}^\star\|_{2} / 12\}
	\]

	To analyze spectral properties of $\nabla^2 f (\bm{x})$, we focus on quadratic forms
	\[
	\bm{z}^\top  \nabla^2 f (\bm{x}) \bm{z} 
	\]
\end{frame}

\begin{frame}
	\frametitle{Proof of Theorem~\ref{thm:sensing-rank-1} (cont.)}
	Simple calculations show
	\begin{align*}
		\bm{z}^\top  \nabla^2 f (\bm{x}) \bm{z} = \frac{1}{m} \sum_{i=1}^{m} \langle \bm{A}_{i}, \bm{x} \bm{x}^\top  - \bm{x}^{\star} \bm{x}^{\star \top} \rangle (\bm{z}^\top  \bm{A}_{i} \bm{z} ) + 2 (\bm{z}^\top  \bm{A}_{i} \bm{x} )^2,
	\end{align*}
	which admits a more ``compact'' expression
	\begin{align*}
		\bm{z}^\top  \nabla^2 f (\bm{x}) \bm{z} &= \langle \mathcal{A} (\bm{x} \bm{x}^\top  - \bm{x}^{\star} \bm{x}^{\star \top}), \mathcal{A} (\bm{z} \bm{z}^\top ) \rangle \\
		&\quad + \frac{1}{2} \langle \mathcal{A} (\bm{z} \bm{x}^\top  + \bm{x} \bm{z}^{\top}), \mathcal{A} (\bm{z} \bm{x}^\top  + \bm{x} \bm{z}^{\top}) \rangle 
	\end{align*}
	
	\vfill
	{\em \hfill --- requires analyzing $\langle \mathcal{A}(\bm{X}), \mathcal{A}(\bm{Y}) \rangle$}
\end{frame}

\begin{frame}
	\frametitle{RIP preserves inner product}
	
	A consequence of RIP 
	
	\vfill 
	\begin{lemma}\label{lemma:RIP-inner-product}
	Suppose that $\mathcal{A}$ satisfies 2$r$-RIP with constant $\delta_{2r} < 1$, then 
	\[
		\left| \langle \mathcal{A}(\bm{X}), \mathcal{A}(\bm{Y}) \rangle - \langle \bm{X}, \bm{Y} \rangle \right| \leq \delta_{2r} \| \bm{X} \|_{\mathrm{F}} \| \bm{Y} \|_{\mathrm{F}}
	\]
	holds for any $\bm{X}, \bm{Y}$ of rank no more than $r$
	\end{lemma}
\end{frame}




\begin{frame}
	\frametitle{Proof of Theorem~\ref{thm:sensing-rank-1} (cont.)}
	
	Apply Lemma~\ref{lemma:RIP-inner-product} to obtain
	\begin{align*}
		& \left|  \langle \mathcal{A} (\bm{x} \bm{x}^\top  - \bm{x}^{\star} \bm{x}^{\star \top}), \mathcal{A} (\bm{z} \bm{z}^\top ) \rangle - \langle \bm{x} \bm{x}^\top  - \bm{x}^{\star} \bm{x}^{\star \top}, \bm{z} \bm{z}^\top  \rangle \right| \\
		&\quad \leq \delta_{4} \| \bm{x} \bm{x}^\top  - \bm{x}^{\star} \bm{x}^{\star \top} \|_{\mathrm{F}} \| \bm{z} \bm{z}^\top \|_{\mathrm{F}}  \leq 3 \delta_{4} \| \bm{x}^{\star} \|_{2}^{2} \| \bm{z} \|_{2}^{2}, 
	\end{align*}
	while last relation uses $\|\bm{x} - \bm{x}^\star \|_{2} \leq \| \bm{x}^\star \|_{2}$. \\
	
	\vfill
	Similarly, one has
	\begin{align*}
		& \left|  \langle \mathcal{A} (\bm{z} \bm{x}^\top  + \bm{x} \bm{z}^{\top}), \mathcal{A} (\bm{z} \bm{x}^\top  + \bm{x} \bm{z}^{\top}) \rangle -  \| \bm{z} \bm{x}^\top  + \bm{x} \bm{z}^{\top} \|_{\mathrm{F}}^{2} \right| \\
		&\quad \leq \delta_{4} \| \bm{z} \bm{x}^\top  + \bm{x} \bm{z}^{\top} \|_{\mathrm{F}}^2 \leq 4 \delta_{4} \| \bm{x} \|_{2}^{2} \| \bm{z} \|_{2}^{2} \leq 16 \delta_{4} \| \bm{x}^\star \|_{2}^{2} \| \bm{z} \|_{2}^{2}
	\end{align*}		
	
\end{frame}

\begin{frame}
	\frametitle{Proof of Theorem~\ref{thm:sensing-rank-1} (cont.)}
		Define
	\begin{align*}
		g(\bm{x}, \bm{z}) \coloneqq \langle \bm{x} \bm{x}^\top  - \bm{x}^{\star} \bm{x}^{\star \top}, \bm{z} \bm{z}^\top  \rangle + \frac{1}{2} \| \bm{z} \bm{x}^\top  + \bm{x} \bm{z}^{\top} \|_{\mathrm{F}}^{2}
		\end{align*}
		
		\vfill
		Key conclusion so far: when $\|\bm{x} - \bm{x}^\star \|_{2} \leq \| \bm{x}^\star \|_{2}$, $\bm{z}^\top  \nabla^2 f (\bm{x}) \bm{z}$ is close to $g(\bm{x}, \bm{z})$ 
		
		\vfill
		It boils down to upper and lower bounding $g(\bm{x}, \bm{z})$---a much easier task
\end{frame}


\begin{frame}
	\frametitle{Proof of Lemma~\ref{lemma:RIP-inner-product}}
	Without loss of generality, assume that $\| \bm{X} \|_{\mathrm{F}} = \| \bm{Y} \|_{\mathrm{F}} = 1$ \\
	
	Since $\bm{X} + \bm{Y}$ and $\bm{X} - \bm{Y}$ have rank at most $2r$, we can leverage $2r$-RIP to obtain
	\begin{align*}
	(1 - \delta_{2r}) \| \bm{X} + \bm{Y} \|_{\mathsf{F}}^{2} \stackrel{(1)}{\leq} \|\mathcal{A} (\bm{X} + \bm{Y}) \|_{2}^{2} \stackrel{(2)}{\leq} (1 + \delta_{2r}) \| \bm{X} + \bm{Y} \|_{\mathsf{F}}^{2} \\
	(1 - \delta_{2r}) \| \bm{X} - \bm{Y} \|_{\mathsf{F}}^{2} \stackrel{(3)}{\leq} \|\mathcal{A} (\bm{X} - \bm{Y}) \|_{2}^{2} \stackrel{(4)}{\leq} (1 + \delta_{2r}) \| \bm{X} - \bm{Y} \|_{\mathsf{F}}^{2} 
	\end{align*}
	Combine (2) and (3) to see
	\begin{align*}
		4  \langle \mathcal{A}(\bm{X}), \mathcal{A}(\bm{Y}) \rangle & = \|\mathcal{A} (\bm{X} + \bm{Y}) \|_{2}^{2} - \|\mathcal{A} (\bm{X} - \bm{Y}) \|_{2}^{2} \\
		& \leq (1 + \delta_{2r}) \| \bm{X} + \bm{Y} \|_{\mathsf{F}}^{2} - (1 - \delta_{2r}) \| \bm{X} - \bm{Y} \|_{\mathsf{F}}^{2} \\
		&= 4 \delta_{2r} + 4 \langle \bm{X}, \bm{Y} \rangle
	\end{align*}
	
	Combine (1) and (4) to finish the proof
\end{frame}

\begin{frame}
	\frametitle{Spectral initialization}
	Construct a surrogate matrix
	\[
	\bm{M} = \frac{1}{m} \sum_{i=1}^{m} y_i \bm{A}_{i}
	\]
	
	Define adjoint operator of $\mathcal{A}$: $\mathcal{A}^*(\cdot) : \mathbb{R}^{m} \to \mathbb{R}^{n_1 \times n_2}$
	\[
	\mathcal{A}^*(\bm{v}) = \frac{1}{ \sqrt{m} }\sum_{i=1}^{m} v_i \bm{A}_{i}
	\]
	
	As a result, one has $\bm{M} = \mathcal{A}^*( \mathcal{A} (\bm{M}^\star) )$ \\
	
	\vfill
	\begin{itemize}
		\item Let $\lambda_{1} \bm{u}_1 \bm{u}_{1}^\top$ be the top eigendecomposition of $\bm{M}$; return $\bm{x}^0 = \sqrt{\lambda_{1}} \bm{u}_1$
	\end{itemize}
\end{frame}

\begin{frame}
	\frametitle{Performance guarantee of spectral initialization}
	\begin{lemma}\label{lemma:spectral-matrix-sensing}
		Suppose that $\mathcal{A}$ obeys 2-RIP with RIP constant $\delta_{2} \leq 1/4$. Then one has 
		\[
		\| \bm{x}^0 - \bm{x}^\star \|_2 \lesssim \delta_{2} \|\bm{x}^\star\|_{2}.
		\]
	\end{lemma}
	
	\vfill
	\begin{itemize}
		\item as long as $\delta_{4}$ is small enough, spectral initialization + GD works for low-rank matrix sensing since $\delta_{2} \leq \delta_{4}$
		\item under Gaussian design, we only need $O((n_1 + n_2)r)$ linear measurements
	\end{itemize}
\end{frame}

\begin{frame}
	\frametitle{Proof of Lemma~\ref{lemma:spectral-matrix-sensing}}
	By definition, one has
	\begin{align*}
		\| \bm{M} - \bm{M}^\star \| &= \| \mathcal{A}^*( \mathcal{A} (\bm{M}^\star) ) - \bm{M}^\star \| \\
		 &= \sup_{\bm{v}: \|\bm{v}\|_2 = 1} \bm{v}^\top \left( \mathcal{A}^*( \mathcal{A} (\bm{M}^\star) ) - \bm{M}^\star \right) \bm{v} \\
		 &= \sup_{\bm{v}: \|\bm{v}\|_2 = 1} \langle  \mathcal{A}^*( \mathcal{A} (\bm{M}^\star) ) - \bm{M}^\star, \bm{v} \bm{v}^\top \rangle \\
		 &\leq \sup_{\bm{v}: \|\bm{v}\|_2 = 1} \delta_{2} \| \bm{M}^\star \|_{\mathsf{F}} \|\bm{v} \bm{v}^\top\|_{\mathsf{F}} \\
		 &\leq \delta_{2} \| \bm{x}^\star \|_2^2
	\end{align*}
	
	Consequently, by Wely's inequality and Davis-Kahan's theorem, we have
	\begin{align*}
		\lambda_1 - \lambda_1^\star & \leq \|\bm{M} - \bm{M}^\star \| \leq \delta_{2} \| \bm{x}^\star \|_2^2 \\
		\| \bm{u}_1 -  \bm{u}^\star_1 \|_{2} & \lesssim \frac{\|\bm{M} - \bm{M}^\star \|}{ \| \bm{M}^\star \| } \lesssim \delta_{2} 
	\end{align*}
\end{frame}

\begin{frame}
	\frametitle{Proof of Lemma~\ref{lemma:spectral-matrix-sensing}}
	Note that 
	\begin{align*}
		\left \| \sqrt{\lambda_1} \bm{u}_1 - \sqrt{\lambda_1^\star} \bm{u}^\star_1 \right \|_2  &\leq \left \| \left ( \sqrt{\lambda_1} - \sqrt{\lambda_1^\star} \right) \bm{u}_1  \right \|_2 + \left \| \sqrt{\lambda_1^\star} \left(  \bm{u}_1 -  \bm{u}^\star_1 \right) \right \|_2 \\
		&=  \left( \sqrt{\lambda_1} - \sqrt{\lambda_1^\star}  \right) + \| \bm{x}^\star \|_2 \cdot \| \bm{u}_1 -  \bm{u}^\star_1 \|_{2} \\
		&= \frac{ \lambda_1 - \lambda_1^\star }{ \sqrt{\lambda_1} + \sqrt{\lambda_1^\star} } + \| \bm{x}^\star \|_2 \cdot \| \bm{u}_1 -  \bm{u}^\star_1 \|_{2} \\
		&\lesssim \delta_{2} \|\bm{x}^\star\|_2
	\end{align*}
\end{frame}



\begin{frame}
\frametitle{Sampling operators that do NOT satisfy RIP} 


{
\setbeamercolor{block body}{bg=babyblueeyes,fg=black}

\begin{varblock}[\textwidth]{}
   \begin{center}
     Unfortunately, many sampling operators fail to satisfy RIP 
   \end{center}
\end{varblock}
}

\vfill

{\bf Two important examples:}
\bigskip

\begin{itemize}
  \itemsep1em
  \item Phase retrieval / solving random quadratic systems of equations
  \item Matrix completion
\end{itemize}


\end{frame}


\begin{frame}

\vfill


\begin{center}
  {\Large\bf Phase retrieval / solving random quadratic systems of equations}
\end{center}
%\vfill
\vfill

\end{frame}



\begin{frame}
\frametitle{\only<1>{Solving linear systems (linear regression)}\only<2>{Solving quadratic systems of equations}}

\only<1>{
\vspace{-0.5em}
\begin{figure}
	\centering
	\includegraphics[width=0.95\textwidth]{linear-stat.pdf}
\end{figure}

Estimate $\bm{\beta}^{\star} \in \mathbb{R}^d$ from $n$ linear samples
%
\begin{eqnarray*}
	y_i = \bm{x}_i^\top\bm{\beta}^{\star} + \varepsilon_i, \qquad i = 1, \ldots, n  
\end{eqnarray*}
%
%\medskip
\hfill {\small \em --- {assume w.l.o.g.~}$\|\bm{\beta}^{\star}\|_2=1$}
}

\only<2>{
\vspace{-0.5em}
\begin{figure}
	\centering
	\includegraphics[width=0.95\textwidth]{quadratic-stat.pdf}
\end{figure}

Estimate $\bm{\beta}^{\star} \in \mathbb{R}^d$ from $n$ \alert{quadratic} samples
%
\begin{eqnarray*}
	y_i = \big(\bm{x}_i^\top\bm{\beta}^{\star}\big)^{\text{\alert{2}}}, \qquad i = 1, \ldots, n  
\end{eqnarray*}
%

}

%\end{itemize}
\end{frame}




\begin{frame}
  \frametitle{Motivation: phase retrieval}



  \begin{itemize}
	\item electric field $\bm{\beta}^\star(t_1, t_2)$  $\longrightarrow$ Fourier transform $\mathcal{F}\bm{\beta}^\star(f_1, f_2)$
  \end{itemize}
  \vfill
\begin{center}
\includegraphics[width=4cm]{microscopy.png} 
\end{center}

\begin{itemize}
	\item detectors record \alert{intensities} $\big| \mathcal{F}\bm{\beta}^\star(f_1, f_2) \big|^2 $ of Fourier transform
  \end{itemize}
  

%{\small
%\[
%\text{intensity: }\big| \mathcal{F}\bm{\beta}^\star(f_1, f_2) \big|^2 = \Big|{ \int}\bm{\beta}^{\star}(t_{1},t_{2}) e^{-i2\pi(f_{1}t_{1}+f_{2}t_{2})}\mathrm{d}t_{1}\mathrm{d}t_{2}\Big|^{2}
%\]
%}
\pause


\vfill
{
\setbeamercolor{block body}{bg=babyblueeyes,fg=black}

\begin{varblock}[\textwidth]{}
\begin{center}
   {\bf Phase retrieval:}  recover signal $\bm{\beta}^{\star}(t_{1},t_{2})$ from $| \mathcal{F}\bm{\beta}^\star(f_1, f_2) \big|^2$
\end{center}
\end{varblock}
}
%\alert{a huge field by itself}
\end{frame}




\begin{frame}
	\frametitle{Motivation: learning neural nets with quadratic activations}

	\hfill --- {\footnotesize \em Soltanolkotabi, Javanmard, Lee\,'17,  Li, Ma, Zhang\,'17}

\begin{center}
\includegraphics[width=0.5\textwidth]{quadratic_nn-stat.pdf}
\end{center}

\vspace{-0.5em}
input features: $\bm{x}$ \quad weights: $\bm{\beta}^{\star}=[\bm{\beta}_1^{\star},\cdots,\bm{\beta}_r^{\star}]$

\vspace{0.3em}
output: $\quad	y  = \sum_{k=1}^r \sigma(\bm{x}^{\top}\bm{\beta}_k^{\star})  + \varepsilon {\quad \stackrel{\sigma(z)=z^2}{\Longrightarrow} \quad \sum_{k=1}^r (\bx^{\top}\bm{\beta}_k^{\star})^2 + \varepsilon}$
% = \big\|\ba^{\top}\bx^{\star}\big\|_2^2 }


	%\hfill {\em --- reduces to phase retrieval if there is a single neuron}


\end{frame}



\begin{frame}
\frametitle{Rank-one measurements in matrix space} 

Equivalent representation for measurements: 

\[
y_{i}=\bm{a}_{i}^{\top}\underset{\alertb{:=\bm{M}^\star}}{\underbrace{\bm{x}^\star \bm{x}^{\star \top}}}\bm{a}_{i}
= \big\langle \underset{\alertb{:=\bm{A}_{i}}}{\underbrace{\bm{a}_{i}\bm{a}_{i}^{\top}}},\bm{M}^\star \big\rangle , 
\qquad 1 \leq i \leq m
\]

\vfill
Using operator notation
\[
\mathcal{A}\left(\bm{X}\right)=\left[\begin{array}{c}
\langle\bm{A}_{1},\bm{X}\rangle\\
\langle\bm{A}_{2},\bm{X}\rangle\\
\vdots\\
\langle\bm{A}_{m},\bm{X}\rangle
\end{array}\right]=\left[\begin{array}{c}
\langle\bm{a}_{1}\bm{a}_{1}^{\top},\bm{X}\rangle\\
\langle\bm{a}_{2}\bm{a}_{2}^{\top},\bm{X}\rangle\\
\vdots\\
\langle\bm{a}_{m}\bm{a}_{m}^{\top},\bm{X}\rangle
\end{array}\right]
\]


\end{frame}



\begin{frame}[label={slide:PR-example-RIP}]
\frametitle{Does $\mathcal{A}$ obey RIP?} 


	Suppose $\bm{a}_i \overset{\mathrm{ind.}}{\sim} \mathcal{N}( \bm{0}, \bI_n )$
\medskip
% and $m \asymp n$, then
\begin{itemize}
\item<1> If $\bm{x}$ is independent of $\{\bm{a}_i\}$, then
\[
   \big\langle \bm{a}_i \bm{a}_i^{\top} , \bm{x} \bm{x}^{\top}  \big\rangle  =  \big| \bm{a}_i^{\top} \bm{x} \big|^2  \asymp \| \bx \|_2^2 
   %= \| \bX \|_{\mathrm{F}}
   ~~ \Rightarrow ~~   
  \frac{1}{ \sqrt{m}} \left\|  \mathcal{A} \big(\bx \bx^{\top}\big) \right\|_{\mathrm{F}} \asymp \| \bx \bx^{\top} \|_{\mathrm{F}}
   %\| \bX \|_{\mathrm{F}}
\]
\item<1> Consider $\bm{A}_i = \bm{a}_i \bm{a}_i ^{\top}$: with high prob.,
\begin{align*}
   \big\langle \bm{a}_i \bm{a}_i^{\top} , \bm{A}_i  \big\rangle  =  \| \bm{a}_i \|_2^4  \approx n  \| \bm{a}_i \ba_i^{\top} \|_{\mathrm{F}}
   %\begin{cases}  \| \ba_j \|^2 \approx n, \qquad & \text{if } i \neq j  \\ n^2, &\text{else}  \end{cases}
\end{align*}
\[   
   \Longrightarrow \quad   
  \frac{1}{ \sqrt{m}}  \left\| \mathcal{A} ( \bA_i ) \right\|_{\mathrm{F}} \geq \frac{1}{ \sqrt{m}} |\big\langle \bm{a}_i \bm{a}_i^{\top} , \bm{A}_i  \big\rangle|  
   \approx \frac{n}{ \sqrt{m}} \| \bm{A}_i \|_{\mathrm{F}}
\]

{\em \hfill --- fails to obey RIP when $m \approx n$}
\end{itemize}

\end{frame}





\begin{frame}
\frametitle{Why do we lose RIP?} 


\begin{itemize}
  \itemsep1em 
  \item Some low-rank matrices $\bm{X}$ (e.g.~$\bm{a}_i\bm{a}_i^{\top}$) might be too aligned with some (rank-1) measurement matrices 
  \vspace{0.3em}
  \begin{itemize}
    \item loss of ``incoherence'' in some measurements
  \end{itemize}
%  \bigskip
%  \item Some measurements $\langle \bm{A}_i,\bm{X} \rangle$ might have too high of a leverage on $\mathcal{A}(\bm{X})$ when measured in \alert{$\|\cdot\|_{\mathrm{F}}$}
%  \vspace{0.3em}
%  \begin{itemize}
%    \item Solution: replace $\|\cdot\|_{\mathrm{F}}$ by other norms!
%  \end{itemize}
\end{itemize}

\end{frame}





\begin{frame}
\frametitle{A natural least-squares formulation}

\begin{eqnarray*}
	\text{given:}\qquad y_k &= &(\vct{a}_k^\top\vct{x}^{\star})^2, \quad  1\leq k \leq m  
\end{eqnarray*}

\vspace{-2em}
\begin{align*}
	\Downarrow
\end{align*}
\vspace{-2em}
%
%One may attempt recovery by solving
%
\begin{align*} 
	%\label{eq:least-squares-PR}
	\text{minimize}_{\bm{x}\in \mathbb{R}^n}\quad f(\bm{x})=\frac{1}{4m}\sum_{k=1}^{m}\Big[ \big(\bm{a}_{k}^{\top}\bm{x}\big)^{2}-y_{k} \Big]^{2}
\end{align*}
%
%\bigskip
%
%\begin{itemize}
%	\itemsep1em
%	\pause
%	\item {\bf pros:} often exact as long as sample size is sufficiently large
%	\pause
%	\item {\bf cons:} $f(\cdot)$ is highly nonconvex \\
%		\hspace{6em} $\longrightarrow$~ {\em computationally challenging!} 
%\end{itemize}

\end{frame}




\begin{frame}
\frametitle{Wirtinger flow (Cand\`es, Li, Soltanolkotabi\,'14)}

\[
\text{minimize}_{\bm{x}}\quad f(\bm{x})=\frac{1}{4m}\sum_{k=1}^{m}\Big[ \big(\bm{a}_{k}^{\top}\bm{x}\big)^{2}-y_{k} \Big]^{2}
\]


\pause

\vspace{-2em}


\begin{columns}
\begin{column}{0.3\textwidth}

\begin{center}
  \includegraphics[width=1.05\textwidth,angle=-40]{GD2.png}
\end{center}

\end{column}



\begin{column}{0.7\textwidth}

\begin{itemize}
\itemsep1em
\item {\bf spectral initialization:}  $\bm{x}^{0}~\leftarrow~$ leading eigenvector of certain data matrix
	%\[
	%\frac{1}{m}\sum_{i=1}^{m}(\bm{a}_{i}^{\top}\bm{x}^{\star})^{2}\bm{a}_{i}\bm{a}_{i}^{\top}
	%\]
\pause
\item {\bf gradient descent:}  
\[
  \vct{x}^{t+1}= \vct{x}^t- \eta \, \nabla f(\vct{x}^t), \qquad t=0,1,\ldots
\]
\end{itemize}

\end{column}
\end{columns}

\end{frame}



\begin{frame}
\frametitle{Spectral initialization}
{\em \hfill ---cf. homework 1}

	$\lambda^0, \bm{u}^0 $~$\longleftarrow$~ leading eigenvalue, eigenvector of 
	\[
		\bm{M}:=\frac{1}{m}\sum_{k=1}^m y_k \,
		\bm{a}_k\bm{a}_k^{\top}
	\]
	Then set $\bm{x}^0 = \sqrt{\lambda_0} \; \bm{u}^0$

\vfill
	{\bf Rationale:} under random Gaussian design $\bm{a}_i \overset{\mathrm{ind.}}{\sim} \mathcal{N}(\bm{0},\bm{I})$,
	\begin{align*}
		\quad \E[\bm{M}] := \E \left[\frac{1}{m} \sum_{k = 1}^m \bm{y}_k \bm{a}_k \bm{a}_k^{\top} \right] = 
		\underset{\alert{\text{leading eigenvector: }\pm\bm{x}^{\star}}}{\underbrace{ \|\bm{x}^{\star}\|_2^2\, \bm{I} + 2 \bm{x}^{\star}\bm{x}^{\star\top}  }}
	\end{align*}

\end{frame}




\begin{frame}
	\frametitle{First theory of WF}

	

 
\hfill $\mathsf{dist}(\vct{x}^t, \vct{x}^{\star}) := \min \{ \|\vct{x}^t \pm \vct{x}^{\star} \|_2 \}$


\begin{theorem}[Cand\`es, Li, Soltanolkotabi\,'14]
Under i.i.d.~Gaussian design, WF with spectral initialization achieves
\[
	\mathsf{dist}(\vct{x}^t, \vct{x}^{\star}) \lesssim \left( 1 - \frac{\eta}{4} \right)^{t/2} \|\vct{x}^{\star} \|_2, 
\]
%
with high prob., 
	provided that step size $\only<2>{\alert{\eta \lesssim 1/n}}\only<1,3>{\eta \lesssim 1/n}$ and
	sample size: $ \only<3>{\alert{m\,\gtrsim\, n\log n}}\only<1,2>{\alert{m\,\gtrsim\, n\log n}} $.
\end{theorem}
%

%
\begin{itemize}
  \itemsep0.5em
  \item Iteration complexity:  $O\big(\alert{n} \log\frac{1}{\epsilon}\big)$  
  \item Sample complexity: $O(n\log n)$
  \item Derived based on (worst-case) local geometry 
\end{itemize}

\end{frame}

\begin{frame}
	\frametitle{Spectral initialization for phase retrieval}
	 Key: control
		\begin{align*}
			\left \| \frac{1}{m}\sum_{k=1}^m y_k \,
		\bm{a}_k\bm{a}_k^{\top} - ( \| \bm{x}^\star \|_{2}^2 \bm{I}_{n} + 2 \bm{x}^\star \bm{x}^{\star \top} ) \right\|
		\end{align*}
		
	\begin{lemma}\label{lemma:pr-concentration}
	Fix any small constant $\delta > 0$. As long as $m \geq c_{\delta} n \log n$ for some sufficiently large constant $c_{\delta}$ (which potentially depends on $\delta$), the following holds with high probability
	\begin{align*}
	\left \| \frac{1}{m}\sum_{k=1}^m y_k \,
		\bm{a}_k\bm{a}_k^{\top} - ( \| \bm{x}^\star \|_{2}^2 \bm{I}_{n} + 2 \bm{x}^\star \bm{x}^{\star \top} ) \right\| \leq \delta \| \bm{x}^\star \|_{2}^2
	\end{align*}
	\end{lemma}
	
	\vfill
	\begin{itemize}
		\item Proof: truncation-based matrix Bernstein or $\varepsilon$-net  argument
	\end{itemize}
\end{frame}

\begin{frame}
	\frametitle{Spectral initialization}
	Since 
	\[
	\left \| \frac{1}{m}\sum_{k=1}^m y_k \,
		\bm{a}_k\bm{a}_k^{\top} - ( \| \bm{x}^\star \|_{2}^2 \bm{I}_{n} + 2 \bm{x}^\star \bm{x}^{\star \top} ) \right\|
	\]
	is small, by Weyl's inequality and Davis-Kahan's theorem, we know  
	\begin{itemize}
		\item $\lambda^0 - \lambda^\star$ is small
		\item $\| \bm{u}^0 - \bm{u}^\star \|_2$ is small
	\end{itemize}
	
	\vfill
	Consequently, $\bm{x}^0 = \sqrt{\lambda^0} \bm{u}^0 $ is close to $\bm{x}^\star = \sqrt{\lambda^\star} \bm{u}^\star$ in the sense that 
	\begin{align*}
		\| \bm{x}^0 - \bm{x}^\star \|_{2} \ll \| \bm{x}^\star \|_2
	\end{align*}
\end{frame}



\begin{frame}
	\frametitle{Local geometry for phase retrieval}
	Now we move on to local convergence of GD, which boils down to characterizing local geometry of $f(\cdot)$
	\vfill
	\begin{lemma}\label{lemma:local-PR}
		Assume that $m \geq c_{0} n \log n$. Then with high probability, 
		\[
		0.5 \bm{I}_{n} \preceq \nabla^2 f (\bm{x}) \preceq c_2 n  \bm{I}_{n} 
		\]
		holds simultaneously for all $\bm{x}$ obeying $\| \bm{x} - \bm{x}^\star \|_2 \leq c_1 \|\bm{x}^\star\|_{2}$. Here $c_0, c_1, c_2 > 0$ are some universal constants.
	\end{lemma}
\end{frame}

\begin{frame}
	\frametitle{Proof of Lemma~\ref{lemma:local-PR}}
	First, write Hessian as
	\[
	\nabla^2 f (\bm{x}) = \frac{1}{m} \sum_{i=1}^{m} ( 3 (\bm{a}_i^\top \bm{x} )^2 - y_i ) \bm{a}_i \bm{a}_i^\top
	\]
	
	When $\bm{x} = \bm{x}^\star$, one has 
	\begin{align*}
	\nabla^2 f (\bm{x}^\star) &= \frac{1}{m} \sum_{i=1}^{m} 2 (\bm{a}_i^\top \bm{x}^\star )^2 \bm{a}_i \bm{a}_i^\top \\
	& \approx 2 ( \| \bm{x}^\star \|_{2}^2 \bm{I}_{n} + 2 \bm{x}^\star \bm{x}^{\star \top}  )
	\end{align*}
	
	Therefore at minimizer $\bm{x}^\star$, $f(\cdot)$ is strongly convex and smooth; how about nearby points $\bm{x}$
\end{frame}

\begin{frame}
	\frametitle{Local strong convexity}
	Recall Hessian
	\begin{align*}
		\nabla^2 f (\bm{x}) &= \frac{1}{m} \sum_{i=1}^{m} ( 3 (\bm{a}_i^\top \bm{x} )^2 - (\bm{a}_i^\top \bm{x}^\star )^2 ) \bm{a}_i \bm{a}_i^\top \\
		&= \frac{1}{m} \sum_{i=1}^{m} 3 (\bm{a}_i^\top \bm{x} )^2 \bm{a}_i \bm{a}_i^\top - ( \| \bm{x}^\star \|_{2}^2 \bm{I}_{n} + 2 \bm{x}^\star \bm{x}^{\star \top} )  \\
		&\quad + \| \bm{x}^\star \|_{2}^2 \bm{I}_{n} + 2 \bm{x}^\star \bm{x}^{\star \top} - \frac{1}{m} \sum_{i=1}^{m} (\bm{a}_i^\top \bm{x}^\star )^2 \bm{a}_i \bm{a}_i^\top
	\end{align*}
	
	\begin{itemize}
		\item Lemma~\ref{lemma:pr-concentration} guarantees that if $m \geq c_0 n \log n$, then whp.,
			\[
				\left \| \| \bm{x}^\star \|_{2}^2 \bm{I}_{n} + 2 \bm{x}^\star \bm{x}^{\star \top} - \frac{1}{m} \sum_{i=1}^{m} (\bm{a}_i^\top \bm{x}^\star )^2 \bm{a}_i \bm{a}_i^\top \right \| \leq 0.001 \| \bm{x}^\star \|_{2}^2
			\]
	\end{itemize}
\end{frame}


\begin{frame}
	\frametitle{Local strong convexity (cont.)}
	Now we turn to a uniform lower bound over $\bm{x}$
	\begin{align*}
		\frac{1}{m} \sum_{i=1}^{m} (\bm{a}_i^\top \bm{x} )^2 \bm{a}_i \bm{a}_i^\top
	\end{align*}
	
	Observe that for any constant $C > 0$
	\begin{align*}
		\frac{1}{m} \sum_{i=1}^{m} (\bm{a}_i^\top \bm{x} )^2 \bm{a}_i \bm{a}_i^\top \succeq \frac{1}{m} \sum_{i=1}^{m} (\bm{a}_i^\top \bm{x} )^2 \indicator\{ | \bm{a}_i^\top \bm{x} | \leq C \} \bm{a}_i \bm{a}_i^\top
	\end{align*}
	
	\begin{itemize}
		\item Intuition: truncation helps concentration due to better tail behavior
	\end{itemize}
\end{frame}


\begin{frame}
	\frametitle{Local strong convexity (cont.)}
	Using covering argument, it is seen that with high probability 
	\begin{align*}
		\left \| \frac{1}{m} \sum_{i=1}^{m} (\bm{a}_i^\top \bm{x} )^2 \indicator\{ | \bm{a}_i^\top \bm{x} | \leq C \} \bm{a}_i \bm{a}_i^\top - 3 (\beta_1 \bm{x} \bm{x}^\top + \beta_2 \|\bm{x}\|_2^2 \bm{I}_{n} ) \right \| \ll \|\bm{x}\|_{2}^2,
	\end{align*}
	for all $\bm{x}$, where 
	\begin{align*}
		\beta_1 &\coloneqq \mathbb{E} [ \xi^4 \indicator \{ | \xi | \leq C \}] - \mathbb{E} [ \xi^2 \indicator \{ | \xi | \leq C \}], \\
		\beta_2 &\coloneqq \mathbb{E} [ \xi^2 \indicator \{ | \xi | \leq C \}]
	\end{align*}
	
	Observe that $\beta_1 \stackrel{C \to \infty}{ \to } 2$, and $\beta_2 \stackrel{C \to \infty}{ \to } 1$
\end{frame}


\begin{frame}
	\frametitle{Local smoothness}
	Decompose Hessian as
		\begin{align*}
		\nabla^2 f (\bm{x}) &= \frac{1}{m} \sum_{i=1}^{m} ( 3 (\bm{a}_i^\top \bm{x} )^2 - (\bm{a}_i^\top \bm{x}^\star )^2 ) \bm{a}_i \bm{a}_i^\top \\
		&= \frac{3}{m} \sum_{i=1}^{m}   [ (\bm{a}_i^\top \bm{x} )^2 - (\bm{a}_i^\top \bm{x}^\star )^2 ] \bm{a}_i \bm{a}_i^\top \coloneqq \bm{\Lambda}_1 \\
		&\quad +\frac{2}{m} \sum_{i=1}^{m} (\bm{a}_i^\top \bm{x}^\star )^2 \bm{a}_i \bm{a}_i^\top -  2 ( \| \bm{x}^\star \|_{2}^2 \bm{I}_{n} + 2 \bm{x}^\star \bm{x}^{\star \top}  ) \coloneqq \bm{\Lambda}_2 \\
		&\quad + 2 ( \| \bm{x}^\star \|_{2}^2 \bm{I}_{n} + 2 \bm{x}^\star \bm{x}^{\star \top} )   \coloneqq \bm{\Lambda}_3 \\
		&\quad 	\end{align*}
		
Our goal is to upper bound $\| \bm{\Lambda}_1 + \bm{\Lambda}_2 + \bm{\Lambda}_3 \|$
\end{frame}


\begin{frame}
	\frametitle{Local smoothness (cont.)}
	
	\begin{itemize}
		\item Term $\| \bm{\Lambda}_3 \|$ is easy to control
		\item By Lemma~\ref{lemma:pr-concentration}, term $\| \bm{\Lambda}_2 \|$ is also small
		\item We are left with first term, which can be controlled as
		\begin{align*}
			\| \Lambda_{1} \| &\leq 3 \left \| \frac{1}{m} \sum_{i=1}^{m}   [ (\bm{a}_i^\top \bm{x} )^2 - (\bm{a}_i^\top \bm{x}^\star )^2 ] \bm{a}_i \bm{a}_i^\top \right \| \\
			&\leq 3 \left \| \frac{1}{m} \sum_{i=1}^{m}   \left | (\bm{a}_i^\top \bm{x} )^2 - (\bm{a}_i^\top \bm{x}^\star )^2 \right |  \bm{a}_i \bm{a}_i^\top \right \| \\
			&= 3 \left \| \frac{1}{m} \sum_{i=1}^{m}   \left | \bm{a}_i^\top ( \bm{x} - \bm{x}^\star ) \right |  \left | \bm{a}_i^\top ( \bm{x} + \bm{x}^\star ) \right |  \bm{a}_i \bm{a}_i^\top \right \|
		\end{align*}
	\end{itemize}
\end{frame}

\begin{frame}
	\frametitle{Control $\Lambda_{1}$}
	By Cauchy--Schwarz, we have 
	\[
		\left | \bm{a}_i^\top ( \bm{x} - \bm{x}^\star ) \right | \leq \| \bm{a}_i \|_{2} \| \bm{x} - \bm{x}^\star \|_{2} \lesssim \sqrt{n} \| \bm{x}^\star \|_{2},
	\]
	where we have used the fact that $\| \bm{a}_i \|_2 \lesssim \sqrt{n}$ with high probability, and the assumption that $\|\bm{x} - \bm{x}^\star\|_2 \lesssim \|\bm{x}^\star\|_2$
	
	\vfill
	As a result, we obtain
	\begin{align*}
		\| \Lambda_{1} \| \lesssim n \|\bm{x}^\star\|_2^2  \left \| \frac{1}{m} \sum_{i=1}^{m} \bm{a}_{i} \bm{a}_{i}^\top \right \| \asymp n
	\end{align*}
	
	
\end{frame}



\begin{frame}
	\frametitle{A closer look at smoothness}
	
	\begin{itemize}
		\item We obtain $O(n)$ smoothness parameter for coherent points $\bm{x}$ such that $|\bm{a}_i^\top \bm{x}| \asymp \sqrt{n}$
		\item Our prediction of local smoothness is tight; take 
		\[
		\bm{x} = \bm{x}^\star + \delta \frac{ \bm{a}_i }{ \| \bm{a}_i \|_2 }
		\]
		consider $\bm{x}^\top \nabla^2 f( \bm{x} ) \bm{x}$
	\end{itemize}
\end{frame}








\begin{frame}[plain]

\vfill
\begin{center}
  {\Large \bf Low-rank matrix completion}
\end{center}
\vfill

\end{frame}



\begin{frame}
\frametitle{Low-rank matrix completion}



\begin{columns}
\begin{column}{0.5\textwidth}
\[
 \begin{bmatrix}
   {\color{blue} \checkmark} & {\color{red} ?} &{\color{red} ?}  & {\color{red} ?} & {\color{blue} \checkmark} & {\color{red} ?} \\
   {\color{red} ?} & {\color{red} ?} & {\color{blue} \checkmark} & {\color{blue} \checkmark} & {\color{red} ?} & {\color{red} ?} \\
   {\color{blue} \checkmark} & {\color{red} ?} & {\color{red} ?} & {\color{blue} \checkmark} & {\color{red} ?} & {\color{red} ?} \\
   {\color{red} ?} & {\color{red} ?} & {\color{blue} \checkmark}  & {\color{red} ?} &{\color{red} ?}  & {\color{blue} \checkmark} \\
   {\color{blue} \checkmark}  &  {\color{red} ?} & {\color{red} ?} & {\color{red} ?}  & {\color{red} ?} & {\color{red} ?} \\
   {\color{red} ?} & {\color{blue} \checkmark} &{\color{red} ?}  & {\color{red} ?} & {\color{blue} \checkmark} & {\color{red} ?} \\
   {\color{red} ?}  &{\color{red} ?} & {\color{blue} \checkmark} &
   {\color{blue} \checkmark} & {\color{red} ?} & {\color{red} ?}
\end{bmatrix}
\]
\end{column}

\begin{column}{0.5\textwidth}  
\begin{center}
\includegraphics[width=0.9\textwidth]{NetflixMahdi} \\
\hfill {\footnotesize\em figure credit: Cand\`es ~~}
\end{center}
\end{column}

\end{columns}



\begin{itemize}
	\itemsep0.5em
	\item consider a low-rank matrix $\bm{M}^{\star} = \bm{U}^{\star} \bm{\Sigma}^{\star} \bm{U}^{\star\top}$
	\item each entry $M_{i,j}^{\star}$   is observed independently with prob.~$p$
	\item {\bf Goal:} estimate $\bm{M}^{\star}$
\end{itemize}

\end{frame}





\begin{frame}
\frametitle{A natural least-squares loss }

\bigskip

  Represent low-rank matrix by $\bm{X}\bm{X}^{\top}$ with $\underset{\alert{\text{low-rank factor}}}{\underbrace{\bm{X} \in \mathbb{R}^{n \times r}}}$ 


\begin{figure}
  \includegraphics[width=0.35\textwidth]{\LectureFigs/XY_factor-plain.pdf}
\end{figure}

\vfill
{
\setbeamercolor{block body}{bg=babyblueeyes,fg=black}


\begin{varblock}[\textwidth]{}
\[
  \underset{\bm{X} \in \mathbb{R}^{n \times r}}{\text{minimize}}~~ f(\bm{X})=  \sum_{(i,j)\in\Omega}\Big[ \big( \bm{X}\bm{X}^{\top} \big)_{i,j} - M^\star_{i,j} \Big]^{2} 
\]  
\end{varblock}
}

{\hfill \em ---how does local geometry look like?}
\end{frame}


\begin{frame}
	\frametitle{Local geometry of $f(\cdot)$}
\begin{lemma}Suppose that $n^{2}p\geq C\kappa^{2}\mu rn\log n$ for some sufficiently large
constant $C>0$. Then with high probability,
the Hessian $\nabla^{2}f(\bm{X})$ obeys 
\begin{align*}
\mathrm{vec}\left(\bm{V}\right)^{\top}\nabla^{2}f\left(\bm{X}\right)\mathrm{vec}\left(\bm{V}\right) & \geq\frac{\sigma_{\min}}{2}\left\Vert \bm{V}\right\Vert _{\mathrm{F}}^{2}\\ 
\left\Vert \nabla^{2}f\left(\bm{X}\right)\right\Vert &\leq\frac{5}{2}\sigma_{\max}
\end{align*}
for all $\bm{X}$, \alert{$\bm{V} = \bm{Y}\bm{H}_{Y}-\bm{X}^\star$} s.t. $\bm{H}_{Y} \coloneqq \arg\min_{\bm{R}\in\mathcal{O}^{r\times r}}\left\Vert \bm{Y}\bm{R}-\bm{X}^\star\right\Vert _{\mathrm{F}}$,
\alert{\begin{align*}
 \left\Vert \bm{X}-\bm{X}^{\star}\right\Vert _{2,\infty} & \leq\epsilon\left\Vert \bm{X}^{\star}\right\Vert _{2,\infty},
\end{align*}}
where $\epsilon\ll1/\sqrt{\kappa^{3}\mu r\log^{2}n}$.

\end{lemma}
\end{frame}

\begin{frame}
	\frametitle{Restricted local strong convexity}
	\begin{itemize}
	\itemsep 0.5em
		\item Due to rotation ambiguity, $f(\cdot)$ cannot be strongly convex along every direction; it is strongly convex along specific directions $\bm{V} = \bm{Y}\bm{H}_{Y}-\bm{X}^\star$
		\item Instead of $\ell_{\mathrm{F}}$ ball, $f(X)$ is strongly convex in a local $\ell_{2,\infty}$ ball; $\bm{X}$ needs to be incoherent in the sense that 
		\[
			\|\bm{X}\|_{2, \infty} \lesssim \sqrt{ \frac{\mu r }{n} }\| \bm{X}^\star \|
		\]
	\end{itemize}
\end{frame}



\begin{frame}
	\frametitle{Revisit Incoherence}
	\begin{definition}
	Fix an orthonormal matrix $\bm{U}^\star \in \mathbb{R}^{n \times r}$. Define its incoherence to be
	\begin{align*}
	\mu(\bm{U}^\star) \coloneqq \frac{n\|\bm{U}^{\star}\|_{2,\infty}^{2}}{r} 
\end{align*}
\end{definition}

{\hfill \em ---recover incoherence of eigenvector when $r=1$}

\vfill 
\begin{itemize}
	\item For $\bm{M}^{\star} = \bm{U}^{\star}\bm{\Sigma}^{\star}\bm{U}^{\star\top}$, define $\mu(\bm{M}^\star) \coloneqq \mu(\bm{U}^\star)$
\end{itemize}
\end{frame}




\begin{frame}
 \frametitle{Projected gradient descent for matrix completion}

\begin{itemize}
\itemsep1em
\item[{\color{black}(1)}]  \textbf{{Projected spectral initialization}}: let $\bm{U}^{0}\bm{\Sigma}^{0}\bm{U}^{0\top}$
be rank-$r$ eigendecomposition of 
$$ \frac{1}{p} \mathcal{P}_{\Omega}(\bY).$$ 
and set $\bm{Z}^{0}=\bm{U}^{0}\left(\bm{\Sigma}^{0}\right)^{1/2}$, and incoherence set 
\[
	\mathcal{C} \coloneqq \{ \bm{X} \mid \|\bm{X}\|_{2, \infty} \leq \sqrt{ \frac{2 \mu r }{n} }\| \bm{Z}^0 \| \}
\]

let $\bm{X}^0 = \mathcal{P}_{\mathcal{C}} (\bm{Z}^{0})$

\medskip
	\item[{\color{black}(2)}] {\bf Projected gradient descent updates:} 
%\textbf{for} $t=0,1,2,\ldots,T-1$
%\textbf{do} 
$$\bm{X}^{t+1}= \mathcal{P}_{\mathcal{C}} ( \bm{X}^{t}-\eta_{t}\nabla f\big(\bm{X}^{t}\big)),\qquad t=0,1,\cdots $$
\end{itemize}


 

\end{frame}

\begin{frame}
	\frametitle{Projection operator}
	Projection onto can be implemented via a row-wise ``clipping operation''
	
	\begin{align*}
		[\mathcal{P}_{\mathcal{C}} ( \bm{X} ) ]_{i,\cdot} = \min \left \{ 1,  \sqrt{ \frac{2 \mu r}{n} } \frac{\|\bm{Z}^0\|}{ \|\bm{X}_{i,\cdot}\|_2} \right \} \cdot \bm{X}_{i, \cdot}
	\end{align*}
	
\end{frame}


\begin{frame}
	\frametitle{Performance guarantees}
	\begin{theorem}
		Suppose that $n^2 p \geq c_0 \mu^2 r^2 \kappa^2 n \log n$ for some large constant $c_0 > 0$. With high probability, one has for all $t \geq 0$
		\begin{align*}
			\|\bm{X}^t \bm{Q}^t \|_{\mathrm{F}}^2 \leq \left ( 1 - \frac{c_1}{\mu^2 r^2 \kappa^2}\right)^{t} \sigma_{r} (\bm{M}^\star),
		\end{align*}
		provided that step size is chosen as $\eta \asymp \frac{1}{\mu^2 r^2 \kappa \sigma_1(\bm{M}^\star)}$
	\end{theorem}
	Here $\bm{Q}^t$ is the optimal alignment matrix between $\bm{X}^t$ and $\bm{X}^\star$
	\[
\bm{Q}^{t}:=\argmin_{\bm{R}\in\mathcal{O}^{r\times r}}\big\Vert \bm{X}^{t}\bm{R}-\bm{X}^{\star}\big\Vert _{\mathrm{F}}
\]
\end{frame}
 




\begin{frame}
	\frametitle{Regularity condition}
	Key to prove convergence is the following regularity condition
	
	\vfill
	\begin{lemma}
		Suppose that $n^2 p \geq \mu^2 r^2 \kappa^2 n \log n$. Then with high probability, for all $\bm{X} \in \mathcal{C}$, and $\|\bm{X} - \bm{X}^\star \bm{H}\|_{\mathrm{F}}^2 \leq \frac{1}{16}\sigma_r(\bm{M}^\star)$ $f$ obeys
		\begin{align*}
		\langle \nabla f (\bm{X}), \bm{X} - \bm{X}^\star \bm{H} \rangle &\geq \frac{99}{512}\sigma_{r}(\bm{M}^\star) \| \bm{X} - \bm{X}^\star \bm{H} \|_{\mathrm{F}}^2 \\
		&\quad + \frac{1}{13196 \mu^2 r^2 \kappa \sigma_{1}( \bm{M}^\star)} \| \nabla f (\bm{X}) \|_{\mathrm{F}}^2
		\end{align*}
	\end{lemma}
	
	Here $\bm{H}$ is optimal alignment matrix 
\end{frame}

\begin{frame}
	\frametitle{Complete the proof}
\end{frame}
 
%\begin{frame}
%	\frametitle{XXX}
%	only present the result using regularity condition, and mention that projection is not needed
%\end{frame}

\end{document}

