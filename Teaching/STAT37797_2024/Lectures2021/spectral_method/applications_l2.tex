\pdfminorversion=4
\documentclass[compress,
mathserif,wide,%red,
%handout
]{beamer}

\usepackage{amsmath,amsfonts,bm,bbm,comment}
\usepackage{latexsym,tabularx,amscd}
\usepackage{etex} 
\usepackage{algorithm}
\usepackage[noend]{algorithmic}
\usepackage{subfigure}
\usepackage{url}
%\usepackage{multimedia}
\usepackage{hyperref}
\usepackage{skull}
%\usepackage{enumitem}
%\usepackage{movie15}

\usepackage{multirow}
\usepackage{cancel}

%\usepackage{ulem}
\usepackage{lmodern}
\usepackage{array} 
\usepackage{etoolbox}
\usepackage{caption}
\usepackage[normalem]{ulem}

\usepackage{mathtools}

\usepackage[space]{grffile}

\usepackage{tikz}
\usetikzlibrary{shapes.arrows}
\usetikzlibrary{arrows,positioning}
\usepackage{pgfplots}
%\usetikzlibrary{pgfplots.groupplots}

\tikzset{
    myarrow/.style={
        draw,
        fill=black,
        single arrow,
        minimum height=6ex,
        single arrow head extend=1ex
    }
}


\newcommand{\vertiii}[1]{{\left\vert\kern-0.25ex\left\vert\kern-0.25ex\left\vert #1 
    \right\vert\kern-0.25ex\right\vert\kern-0.25ex\right\vert}}

\newcommand{\vertiiibig}[1]{{\big\vert\kern-0.25ex\big\vert\kern-0.25ex\big\vert #1 
    \big\vert\kern-0.25ex\big\vert\kern-0.25ex\big\vert}}

\newcommand{\vertiiiplain}[1]{{\vert\kern-0.25ex\vert\kern-0.25ex\vert #1 
    \vert\kern-0.25ex\vert\kern-0.25ex\vert}}



\usepackage{dsfont}
\DeclareMathOperator{\ind}{\mathds{1}}  % Indicator


\newcommand{\abs}[1]{\left|#1\right|}

\mode<presentation>

\usetheme{Madrid}
% other themes: AnnArbor, Antibes, Bergen, Berkeley, Berlin, Boadilla,
% boxes, CambridgeUS, Copenhagen, Darmstadt, default, Dresden,
% Frankfurt, Goettingen, Hannover, Ilmenau, JuanLesPins, Luebeck,
% Madrid, Maloe, Marburg, Montpellier, PaloAlto, Pittsburg, Rochester,
% Singapore, Szeged, classic

% \usecolortheme{lily} 
% color themes: albatross, beaver, beetle, crane, default, dolphin,
% dov, fly, lily, orchid, rose, seagull, seahorse, sidebartab,
% structure, whale, wolverine

% \usefonttheme{serif} 
% font themes: default, professionalfonts, serif, structurebold,
% structureitalicserif, structuresmallcapsserif

%\usefonttheme[onlymath]{serif}
%\usefonttheme[onlylarge]{structuresmallcapsserif} 
  
%\useoutertheme[subsection=false]{smoothbars}
% outer themes:
% default,infolines,miniframes,smoothbars,sidebar,split,shadow,tree,smoothtree
%\useinnertheme{rectangles}
\beamertemplatenavigationsymbolsempty

% \definecolor{fore}{RGB}{249,242,215}
% \definecolor{back}{RGB}{51,51,51}
% \definecolor{title}{RGB}{255,0,90}
% \setbeamercolor{titlelike}{fg=title}
% \setbeamercolor{normal text}{fg=fore,bg=back}
% \setbeamercovered{transparent}
  % or whatever (possibly just delete it)
%\beamertemplatenavigationsymbolsempty

\usepackage[english]{babel}
\usepackage[latin1]{inputenc}
% or whatever
%\usepackage{comicsans} 
%\renewcommand{\sfdefault}{comic} 
%\usepackage{mathptmx}
%\mathversion{bold}
%\usepackage{helvet}
%\usepackage{courier}

%\usepackage[T1]{fontenc}
% Or whatever. Note that the encoding and the font should match. If T1
% does not look nice, try deleting the line with the fontenc.


%%%% Figures %%%%%


\newcommand{\LectureFigs}{../Figures}


\definecolor{bluegray}{rgb}{0.15,0.20,0.40}
\definecolor{bluegraylight}{rgb}{0.35,0.40,0.60}
\definecolor{gray}{rgb}{0.3,0.3,0.3}
\definecolor{lightgray}{rgb}{0.7,0.7,0.7}
\definecolor{brightblue}{rgb}{0.1,0.4,0.7}
\definecolor{darkblue}{rgb}{0.1,0.1,0.6}
\definecolor{darkgreen}{rgb}{0.0,0.5,0.2}
\definecolor{orange}{rgb}{0.7,0.3,0}

\def \captioncolor { }

\setbeamertemplate{itemize item}{{\color{black}$\bullet$}}
\setbeamertemplate{itemize subitem}{{\color{black}$\circ$}}
\setbeamertemplate{frametitle}[default][center]
\setbeamerfont{frametitle}{series=\bfseries}
\addtobeamertemplate{frametitle}{\vskip1ex}{ \vspace{0.5em} \hrule}
\setbeamercolor{frametitle}{fg=,bg=}

\setbeamertemplate{theorems}[numbered]
\setbeamertemplate{caption}[numbered]
\setbeamertemplate{algorithm}[numbered]

\numberwithin{theorem}{subsection}
\numberwithin{equation}{subsection}
\numberwithin{figure}{subsection}
\numberwithin{algorithm}{subsection}


%\setbeamerfont{page number in head/foot}{size=\scriptsize}
\setbeamertemplate{footline}[frame number]
\setbeamercolor{page number in head/foot}{fg=black}
\setbeamersize{text margin left=2em,text margin right=2em}

\setbeamertemplate{footline}
{
    %\hspace{3em} \insertshorttitle \hfill \insertframenumber / \inserttotalframenumber\hspace{3em}
    \hspace{3em} \insertshorttitle \hfill \insertsubsectionnumber-\insertframenumber \hspace{3em}
    \vspace{1.5em}
}

\definecolor{babyblueeyes}{rgb}{0.63, 0.79, 0.95}
\definecolor{bleudefrance}{rgb}{0.19, 0.55, 0.91}
\definecolor{matlabPlotBlue}{rgb}{0.25, 0.58, 0.8}
\definecolor{matlabPlotOrange}{rgb}{0.855, 0.324, 0.098}



\captionsetup[figure]{labelfont={color=black}}

\AtBeginEnvironment{lemma}{%
  \setbeamercolor{block title}{fg=black,bg=babyblueeyes}
  \setbeamercolor{block body}{fg=black,bg=}
  \setbeamerfont*{block title}{series=\bfseries}
}

\AtBeginEnvironment{theorem}{%
  \setbeamercolor{block title}{fg=black,bg=babyblueeyes}
  \setbeamercolor{block body}{fg=black,bg=}
  \setbeamerfont*{block title}{series=\bfseries}
}

\AtBeginEnvironment{corollary}{%
  \setbeamercolor{block title}{fg=black,bg=babyblueeyes}
  \setbeamercolor{block body}{fg=black,bg=}
  \setbeamerfont*{block title}{series=\bfseries}
}

\AtBeginEnvironment{fact}{%
  \setbeamercolor{block title}{fg=black,bg=babyblueeyes}
  \setbeamercolor{block body}{fg=black,bg=}
  \setbeamerfont*{block title}{series=\bfseries}
}

\AtBeginEnvironment{definition}{%
  \setbeamercolor{block title}{fg=black,bg=babyblueeyes}
  \setbeamercolor{block body}{fg=black,bg=}
  \setbeamerfont*{block title}{series=\bfseries}
}

\newenvironment<>{varblock}[2][.9\textwidth]{%
  \vspace{-1em}
  \begin{center}
  \begin{minipage}{#1}
  %\setlength{\textwidth}{#1}
  \begin{actionenv}#3%
    \def\insertblocktitle{#2}%
    \par%
    \usebeamertemplate{block begin}}
  {\par%
    \usebeamertemplate{block end}%
  \end{actionenv}
  \end{minipage}
  \end{center}
}


\newcommand\blankfootnote[1]{%
  \let\thefootnote\relax\footnotetext{#1}%
  \let\thefootnote\svthefootnote%
}


%%%%%
\newcommand{\bb}{\bm b}
\newcommand{\bZ}{\bm Z}
\newcommand{\bC}{\bm C}
\newcommand{\cU}{{\cal U}}
\newcommand{\bR}{{\bm R}}
\newcommand{\bi}{{\bm i}}
\newcommand{\bS}{{\bm S}}
\newcommand{\cE}{{\cal E}}
\newcommand{\cF}{{\cal F}}
\newcommand{\bk}{{\bm k}}
\newcommand{\bu}{{\bm u}}
\newcommand{\bx}{{\bm x}}
\newcommand{\bA}{{\bm A}}
\newcommand{\bh}{{\bm h}}
\newcommand{\by}{{\bm y}}
\newcommand{\bI}{{\bm I}}
\newcommand{\ba}{{\bm a}}
\newcommand{\bw}{{\bm w}}
\newcommand{\bPhi}{{\bm \Phi}}
\newcommand{\bX}{{\bm X}}
\newcommand{\bY}{{\bm Y}}
\newcommand{\bM}{{\bm M}}
\newcommand{\be}{{\bm e}}
\newcommand{\bs}{{\bm s}}
\newcommand{\bB}{{\bm B}}
\newcommand{\bW}{{\bm W}}
\newcommand{\bN}{{\bm N}}
\newcommand{\bU}{{\bm U}}
\newcommand{\bV}{{\bm V}}
\newcommand{\bP}{{\bm P}}
\newcommand{\bH}{{\bm H}}
\newcommand{\bSigma}{{\bm \Sigma}}
\newcommand{\bL}{{\bm L}}



\newcommand{\bv}{\bm v}
\newcommand{\eps}{\epsilon}
\newcommand{\vf}{\varphi}
\newcommand{\de}{\delta}
\newcommand{\<}{\langle}
\renewcommand{\>}{\rangle}
\newcommand{\goto}{\rightarrow}
\newcommand{\argmin}{\mbox{argmin}}
\newcommand{\argmax}{\mbox{argmax}}
\newcommand{\Ave}{\mathop{\rm Ave}\nolimits}
\newcommand{\sgn}{\mbox{sgn}}
\newcommand{\cN}{\mathcal{N}}
\newcommand{\cR}{\mathcal{R}}

\def\C{{\mathbb{C}}}
\def\R{{{\mathbb{R}}}} %seemed too large in subscripts, so I changed --JW 
\def\P{{\hbox{\bf P}}}
\def\Z{{{\mathbb{Z}}}}
\def\T{{\hbox{\bf T}}}


\newcommand{\cA}{\mathcal{A}}
\newcommand{\cP}{\mathcal{P}}
\newcommand{\cD}{\mathcal{D}}
\newcommand{\cL}{\mathcal{L}}
\newcommand{\shrink}{\text{shrink}}


\newcommand{\Obs}{\Omega_{\text{obs}}}


\renewcommand{\P}{\operatorname{\mathbb{P}}}
\newcommand{\E}{\operatorname{\mathbb{E}}}

% Linear algebra macros
%\newcommand{\vct}[1]{#1}
%\newcommand{\mtx}[1]{#1}
\newcommand{\vct}[1]{\bm{#1}}
\newcommand{\mtx}[1]{\bm{#1}}
%\newcommand{\mtx}[1]{\mathsfsl{#1}}

\newcommand{\transp}{T}
\newcommand{\adj}{*}
\newcommand{\psinv}{\dagger}
\newcommand{\lspan}[1]{\operatorname{span}{#1}}

\newcommand{\range}{\operatorname{range}}
\newcommand{\colspan}{\operatorname{colspan}}

\newcommand{\rank}{\text{rank}}

\newcommand{\diag}{\operatorname{diag}}
\newcommand{\trace}{\text{Tr}}

\newcommand{\supp}[1]{\operatorname{supp}(#1)}

\newcommand{\smax}{\sigma_{\max}}
\newcommand{\smin}{\sigma_{\min}}

\newcommand{\restrict}[1]{\big\vert_{#1}}

\newcommand{\Id}{\text{\em I}}
\newcommand{\OpId}{\mathcal{I}}



% define your own colors:
\definecolor{Red}{rgb}{1,0,0}
\definecolor{Blue}{rgb}{0,0,1}
\definecolor{Green}{rgb}{0,1,0}
\definecolor{magenta}{rgb}{1,0,.6}
\definecolor{lightblue}{rgb}{0,.5,1}
\definecolor{lightpurple}{rgb}{.6,.4,1}
\definecolor{gold}{rgb}{.6,.5,0}
\definecolor{orange}{rgb}{1,0.4,0}
\definecolor{hotpink}{rgb}{1,0,0.5}
\definecolor{newcolor2}{rgb}{.5,.3,.5}
\definecolor{newcolor}{rgb}{0,.3,1}
\definecolor{newcolor3}{rgb}{1,0,.35}
\definecolor{darkgreen1}{rgb}{0, .35, 0}
\definecolor{darkgreen}{rgb}{0, .6, 0}
\definecolor{darkred}{rgb}{.75,0,0}

\xdefinecolor{olive}{cmyk}{0.64,0,0.95,0.4}
\xdefinecolor{purpleish}{cmyk}{0.75,0.75,0,0}

\definecolor{lightgrey}{RGB}{186,186,186}
\definecolor{back}{RGB}{51,51,51}
\definecolor{fore}{RGB}{249,242,215}
\definecolor{back}{RGB}{51,51,51}
\definecolor{title}{RGB}{255,0,20}
\definecolor{keywords}{RGB}{255,0,90}
\definecolor{comments}{RGB}{60,179,113}
\definecolor{cobalt}{RGB}{102,153,204}

\def \itemstyle {\bf}
\def \itemhighlight {\bf}  

% \definecolor{blockbodycolor}{RGB}{30,30,40}
% \definecolor{blocktitlecolor}{RGB}{55,55,75}
\definecolor{blockbodycolor}{RGB}{186,206,232}
\definecolor{blocktitlecolor}{RGB}{55,55,75}

\setbeamercolor{titlelike}{fg=darkred,bg=white}
\setbeamercolor{normal text}{fg=black,bg=white}
\setbeamercolor{block title}{fg=white,bg=blocktitlecolor}
\setbeamercolor{block body}{bg=blockbodycolor}
\setbeamercolor{alerted text}{fg=darkred}
\setbeamercolor{title}{fg=darkred,bg=lightgrey}


\newcommand{\alertb}[1]{\textcolor{blue}{ #1}}

\newcommand{\yp}[1]{\textcolor{red}{ #1}}
\newcommand{\zeronorm}[1]{\left\|#1 \right\|_0}
\newcommand{\unorm}[1]{\left\|#1 \right\|_u}
\newcommand{\ynorm}[1]{\left\|#1 \right\|_{\bar{y}}}
\newcommand{\onetwonorm}[1]{\left\|#1\right\|_{1,2}}
\newcommand{\opnorm}[1]{\left\|#1\right\|}
\newcommand{\fronorm}[1]{\left\|#1\right\|_{F}}
\newcommand{\onenorm}[1]{\left\|#1\right\|_{\ell_1}}
\newcommand{\twonorm}[1]{\left\|#1\right\|_{2}}
\newcommand{\oneinfnorm}[1]{\left\|#1\right\|_{1,\infty}}
\newcommand{\infnorm}[1]{\left\|#1\right\|_{\ell_\infty}}
\newcommand{\nucnorm}[1]{\left\|#1\right\|_*}

%\makeatletter
%\def\th@mystyle{%
%    \normalfont % body font
%    \setbeamercolor{block title example}{bg=,fg=black}
%    \setbeamerfont{block title example}{series=\bfseries}
%    \setbeamercolor{block body example}{bg=,fg=black}
%    %\setbeamertemplate{exampleblock}[numbered]
%    \def\inserttheoremblockenv{exampleblock}
%  }
%\makeatother
%\theoremstyle{mystyle}
%\newtheorem*{lem}{Lemma}

\setbeamertemplate{frametitle continuation}{}


\newcommand{\courseTitle}{STAT 37797: Mathematics of Data Science} 



% \subtitle
% {Include Only If Paper Has a Subtitle}

\vspace{.5cm}

\author{Yuxin Chen}

\institute[Princeton] % (optional, but mostly needed)
{Princeton University}


% - Use the \inst command only if there are several affiliations.
% - Keep it simple, no one is interested in your street address.


\subject{Beamer}
% This is only inserted into the PDF information catalog. Can be left
% out.

% Delete this, if you do not want the table of contents to pop up at
% the beginning of each subsection:
% \AtBeginSubsection[]
% {
%   \begin{frame}<beamer>
%     \frametitle{Outline}
%     \tableofcontents[currentsection,currentsubsection]
%   \end{frame}
% }






\graphicspath{{../../../Figures/}}


\title % (optional, use only with long paper titles)
{Applications of spectral methods ($\ell_{2}$ theory)}

\defbeamertemplate*{title page}{customized}[1][]
{

  \hfill {\em \courseTitle}

  \begin{center}
    \vspace{2.5em}
    \usebeamerfont{title} {\Large\bf\inserttitle} \par
  
    \vspace{1.5em}
    \includegraphics[width=2cm]{\LectureFigs/UC_logo.png} 
  
    \vspace{1em}
    {\large Cong Ma \par }

    \vspace{0.2em}
    { \large \quad University of Chicago, Autumn 2021 }
  \end{center}

  \vfill
}

\setcounter{subsection}{5}

\begin{document}


\begin{frame}[plain]
  \titlepage

\end{frame}

\begin{frame}
	\frametitle{What we have learned so far}
	
	- Classical $\ell_{2}$ matrix perturbation theory: 
	\begin{itemize}
		\item Davis-Kahan's $\sin \bm{\Theta}$ theorem 				\item Wedin's $\sin \bm{\Theta}$ theorem 		
		\item Eigenvector perturbation of probability transition matrices
	\end{itemize}
	
	\vfill 
	- Matrix concentration inequalities:
	\begin{itemize}
		\item Matrix Bernstein inequality
	\end{itemize}
	
	\vfill
	\pause
	{\hfill \em --- we will see their applications today}
\end{frame}

\begin{frame}
\frametitle{Outline}

\begin{itemize}
  \itemsep1em
  %\item Warm-up: matrix denoising
  \item Community recovery in stochastic block model \\
  	{\footnotesize \hfill \em --- application of Davis-Kahan's theorem}
  \item Low-rank matrix completion \\
  	{\footnotesize \hfill \em --- application of Wedin's theorem}
  \item Ranking from pairwise comparisons \\
  	{\footnotesize \hfill \em --- application of eigenvector perturbation of prob. transition matrix}
\end{itemize}

\end{frame}


\begin{frame}[plain]
	\vfill
	\centering
	\Large \bf Community recovery in stochastic block model
	\vfill
\end{frame}


\begin{frame}
	\frametitle{Stochastic block model (SBM)}


\vspace{-1em}

\begin{center}
	\includegraphics[height=0.3\textwidth]{SBM_2.png}  \\
	  {\small $x_i^{\star}=1$: $1^{\text{st}}$ community}   \qquad   {\small $x_i^{\star}=-1$: $2^{\text{nd}}$ community}
\end{center}

\begin{itemize}
	\itemsep0.5em
	\item  $n$ nodes $\{1,\ldots,n\}$
	\item  2 communities
	\item  $n$ unknown variables:  $x_1^{\star}, \ldots, x_n^{\star} \in \{1,-1\}$ 
	\begin{itemize}
		\item encode community memberships
	\end{itemize}

	
\end{itemize}



\end{frame}



\begin{frame}
	\frametitle{Stochastic block model (SBM)}

\vspace{-1em}


\begin{columns}

\begin{column}{0.05\textwidth}
\end{column}


\begin{column}{0.25\textwidth}
\begin{center}
  \includegraphics[height=0.9\textwidth]{SBM_1.png} \\
	$\mathcal{G}$
\end{center}
\end{column}

\begin{column}{0.05\textwidth}
\end{column}


\begin{column}{0.12\textwidth}
\begin{center}
\includegraphics[width=\textwidth,angle=-90]{arrow_up.png} 
\end{center}
\end{column}


\begin{column}{0.5\textwidth}
\begin{center}
  \includegraphics[height=0.45\textwidth]{SBM_2.png}
\end{center}
\end{column}

\begin{column}{0.03\textwidth}
\end{column}

\end{columns}


\bigskip


\begin{itemize}
	\itemsep0.5em
	
	\item observe a graph $\mathcal{G}$
		\vspace{-0.8em}
		\begin{align*}
			(i,j)\in \mathcal{G} \text{ with prob.~} \begin{cases} p,  & \text{if }i\text{ and }j \text{ are from same community} \\ q, & \text{else} \end{cases}
		\end{align*}
		%
		Here, $p>q$
	\item {\bf Goal:}  recover community memberships  of all nodes, i.e., $\{x_i^{\star}\}$

\end{itemize}

\end{frame}




\begin{frame}
\frametitle{Adjacency matrix}



\begin{center}
\includegraphics[width=0.4\textwidth]{adjacency_random.png} 
\end{center}


Consider the adjacency matrix $\bm{A}\in \mathbb{R}^{n\times n}$ of $\mathcal{G}$: (assume $A_{ii} = p$)
%
\[
	A_{i,j} = \begin{cases}  1, \qquad & \text{if } (i,j) \in \mathcal{G} \\ 0, & \text{else} \end{cases}
\]

\vspace{-0.5em}
\begin{itemize}
	\item WLOG, suppose $x_1^{\star}=\cdots=x_{n/2}^{\star}=1$; $x_{n/2+1}^{\star}=\cdots=x_n^{\star}=-1$ 
\end{itemize}


\end{frame}




\begin{frame}
\frametitle{Adjacency matrix}


		
\begin{center}
\begin{tabular}{ccccc}
\includegraphics[width=0.23\textwidth,height=0.23\textwidth]{adjacency_random.png} &   & \includegraphics[width=0.23\textwidth,height=0.23\textwidth]{adjacency_mean.png} &  & \includegraphics[width=0.23\textwidth,height=0.23\textwidth]{adjacency_noise.png}\tabularnewline
	$\bm{A}$ & = & $\underset{\alertb{\text{rank 2}}}{\underbrace{\mathbb{E}[\bm{A}]}}$ & + & $\bm{A}-\mathbb{E}\left[\bm{A}\right]$\tabularnewline
\end{tabular}
\end{center}


{\small $$\mathbb{E}[\bm{A}]=\left[\begin{array}{cc}
p\bm{1}\bm{1}^{\top} & q\bm{1}\bm{1}^{\top}\\
q\bm{1}\bm{1}^{\top} & p\bm{1}\bm{1}^{\top}
\end{array}\right]=\underset{\alertb{\text{uninformative bias}}}{\underbrace{\frac{p+q}{2}\bm{1}\bm{1}^{\top}}} \hspace{-0.3em} +\frac{p-q}{2} \hspace{-0.5em}\underset{\alertb{=\bm{x}^{\star}=[x_i]_{1\leq i\leq n}}}{\underbrace{\left[\begin{array}{c}
\bm{1}\\
-\bm{1}
\end{array}\right]}} \hspace{-1em} \left[\bm{1}^{\top},-\bm{1}^{\top}\right]$$ }

		
\end{frame}



\begin{frame}
\frametitle{Spectral clustering}

		
\begin{center}
\begin{tabular}{ccccc}
\includegraphics[width=0.23\textwidth,height=0.23\textwidth]{adjacency_random.png} &   & \includegraphics[width=0.23\textwidth,height=0.23\textwidth]{adjacency_mean.png} &  & \includegraphics[width=0.23\textwidth,height=0.23\textwidth]{adjacency_noise.png}\tabularnewline
	$\bm{A}$ & = & $\underset{\alertb{\text{rank 2}}}{\underbrace{\mathbb{E}[\bm{A}]}}$ & + & $\bm{A}-\mathbb{E}\left[\bm{A}\right]$\tabularnewline
\end{tabular}
\end{center}

\vspace{-0.5em}
\begin{itemize}
	\item[{\color{black}1.}] computing the leading eigenvector $\bm{u}=[u_i]_{1\leq i\leq n}$ of $\bm{A} - \frac{p+q}{2}\bm{1}\bm{1}^{\top}$
	\item[{\color{black}2.}] rounding:  output
		${x}_{i}=\begin{cases}
			1, & \text{if }u_{i} \geq 0\\
			-1, & \text{if }u_{i}<0
\end{cases}$
\end{itemize}




\end{frame}







\begin{frame}
\frametitle{Analysis of spectral clustering}
Consider ``ground-truth'' matrix
\[
\bm{M}^{\star}\coloneqq\mathbb{E}[\bm{A}]-\frac{p+q}{2}\bm{1}\bm{1}^{\top}=\frac{p-q}{2}\left[\begin{array}{c}
\bm{1}\\
-\bm{1}
\end{array}\right]\left[\begin{array}{cc}
\bm{1}^{\top} & -\bm{1}^{\top}\end{array}\right], 
\] \\
which obeys 
\begin{align*}
	\lambda_{1}(\bm{M}^{\star})\coloneqq\frac{(p-q)n}{2},  
	\quad \text{and} \quad
	\bm{u}^{\star}  \coloneqq \frac{1}{\sqrt{n}}
	\left[\begin{array}{c}
		\bm{1}_{n/2}\\
		-\bm{1}_{n/2}
	\end{array}\right].
\end{align*}
Also, we have perturbed matrix
\[
\bm{M}\coloneqq \bm{A}-\frac{p+q}{2}\bm{1}\bm{1}^{\top}
\]
\vfill
Davis-Kahan implies if $\|\bm{A}-\mathbb{E}[\bm{A}]\| < \lambda_{1}(\bm{M}^{\star})=\frac{(p-q)n}{2}$, then 
%
\begin{equation}
	\label{eq:DK-SBM}
	\mathsf{dist}( {\bm{u}}, \bm{u}^{\star}) \leq\frac{\|{\bm{M}} - \bm{M}^{\star}\|}{\lambda_{1}(\bm{M}^{\star})-\|{\bm{M}} - \bm{M}\|}=\frac{\|\bm{A}-\mathbb{E}[\bm{A}]\|}{\frac{(p-q)n}{2}-\|\bm{A}-\mathbb{E}[\bm{A}]\|}
\end{equation}

%
%\vfill
%
%{
%\setbeamercolor{block body}{bg=babyblueeyes,fg=black}
%
%\begin{varblock}[\textwidth]{}
%\begin{center}
%	{\bf Question:} how to bound $\|\bm{A}-\mathbb{E}[\bm{A}]\|$? 
%\end{center}
%\end{varblock}
%}

	
\end{frame}





%
%\begin{frame}
%\frametitle{A hammer: matrix Bernstein inequality}
%
%Consider a sequence of independent random matrices $\big\{ \bm{X}_{l}\in\mathbb{R}^{d_{1}\times d_{2}}\big\} $
%%
%\begin{itemize}
%	\itemsep0.5em
%	\item $\mathbb{E}[\bm{X}_l]=\bm{0}$ \qquad\qquad\qquad\qquad\qquad $\bullet$ $\|\bm{X}_{l}\|\leq B$ for each $l$
%	\item variance statistic: 
%	$$v:=\max\left\{  \left\Vert  \mathbb{E}\left[ \sum\nolimits_{l} \bm{X}_{l}\bm{X}_{l}^{\top}  \right]  \right\Vert  ,  
%			  \left\Vert  \mathbb{E}\left[ \sum\nolimits_{l} \bm{X}_{l}^{\top}\bm{X}_{l}  \right]  \right\Vert  \right\} $$
%\end{itemize}
%
%\begin{theorem}[Matrix Bernstein inequality]\label{thm:mtx-Bernstein}
%%
%For all $\tau \geq 0$,
%\vspace{-1em}
%\[
%\mathbb{P}\left\{ \left\Vert \sum\nolimits_{l}\bm{X}_{l}\right\Vert \geq \tau\right\} \leq\left(d_{1}+d_{2}\right)\exp\left(\frac{-\tau^{2}/2}{v+B\tau/3}\right)
%\]
%%
%\end{theorem}
%
%\end{frame}
%
%
%
%
%
%
%\begin{frame}
%\frametitle{A hammer: matrix Bernstein inequality}
%
%\[
%\mathbb{P}\left\{ \left\Vert \sum\nolimits_{l}\bm{X}_{l}\right\Vert \geq \tau\right\} \leq\left(d_{1}+d_{2}\right)\exp\left(\frac{-\tau^{2}/2}{v+B\tau/3}\right)
%\]
%
%\bigskip
%
%\begin{itemize}
%	\itemsep1em
%	\item<1> {\bf moderate-deviation regime} ($\tau$ is small): \\
%		  \quad --- sub-Gaussian tail behavior $\exp(-\tau^2/2v)$
%		  
%	\item<1> {\bf large-deviation regime} ($\tau$ is large): \\
%		  \quad --- sub-exponential tail behavior $\exp(-3\tau/2B)$ (slower decay)
%	
%	\item<1> {\bf user-friendly form} (exercise): with prob.~$1-O((d_1+d_2)^{-10})$
%	%
%	\begin{equation} \label{eq:Bernstein-user-friendly}
%		\left\Vert \sum\nolimits_{l}\bm{X}_{l}\right\Vert \lesssim  \sqrt{ v \log(d_1+d_2) } + B \log(d_1+d_2)
%	\end{equation}
%
%\end{itemize}
%
%\end{frame}




\begin{frame}
\frametitle{Bounding $\| \bm{A}-\mathbb{E}[\bm{A}] \|$}

Matrix Bernstein inequality tells us that 
%
\begin{lemma}
\label{lem:perturbation-SBM}
%
Consider SBM with $p>q$ and $p\gtrsim \frac{\log n}{n}$. Then with high prob. 
%
\begin{equation}
	\label{eq:A-perturbation}
	\|\bm{A}-\mathbb{E}[\bm{A}]\|\lesssim \sqrt{np \log n} 
\end{equation}
%
\end{lemma}

{\hfill \em --- better concentration yields $\sqrt{np}$ bound}

\vfill

\begin{itemize}
	\item with high probability in this course often means ``with probability at least $1 - O(n^{-8})$''
\end{itemize}
\end{frame}



\begin{frame}
\frametitle{Statistical accuracy of spectral clustering}

Substitute ineq.~\eqref{eq:A-perturbation} into  ineq.~\eqref{eq:DK-SBM} to reach
%
\begin{align*}
	\mathsf{dist}( {\bm{u}}, \bm{u}^{\star}) &\leq \frac{\|\bm{A}-\mathbb{E}[\bm{A}]\|}{\frac{(p-q)n}{2}-\|\bm{A}-\mathbb{E}[\bm{A}]\|}  \lesssim \frac{\sqrt{np \log n}}{(p-q)n} = o(1)
\end{align*}
%
provided that $\sqrt{np \log n} = o((p-q)n)$

\vfill 

Now question is 


{\em \hfill --- how to transfer from estimation error to mis-clustering error}
%Thus, under condition \alert{$\frac{p-q}{\sqrt{p}}\gg \sqrt{\tfrac{\log n}{n}} $}, with high prob.~one has
%%
%\begin{align*}
%	\mathsf{dist}( {\bm{u}}, \bm{u}^{\star}) \ll 1  \qquad \Longrightarrow \qquad  \text{nearly perfect clustering}
%\end{align*}
%


\end{frame}

\begin{frame}
	\frametitle{From estimation error to mis-clustering error}
	WLOG assume that $\|\bm{u}-\bm{u}^{\star}\|_{2}=\mathsf{dist}\big(\bm{u},\bm{u}^{\star}\big)$. Consider the set
%
$$
	\mathcal{N}\coloneqq \big\{ i\mid |u_{i}-u_{i}^{\star}| \geq {1}/{\sqrt{n}}\big\}
$$
We claim that 

\[
	\frac{1}{n}\sum_{i=1}^{n} \mathbbm{1} \big\{ x_{i}\neq x_{i}^{\star}\big\}
	\leq \frac{1}{n}\sum_{i=1}^{n} \mathbbm{1} \Bigg \{ |u_i - u_{i}^{\star}| \geq \frac{1}{\sqrt{n}} \Bigg\}
	= \frac{|\mathcal{N}|}{n}
\]
	
	To see this, observe that for any $i$ obeying $x_{i}\neq x_{i}^{\star}$, one has $\mathsf{sgn}(u_{i})\neq \mathsf{sgn}(u_{i}^{\star})$,
thus indicating that $|u_i - u_{i}^{\star}| \geq |u_{i}^{\star}| = 1/\sqrt{n}$

In the end, we have
\[
|\mathcal{N}|\leq\frac{\|\bm{u}-\bm{u}^{\star}\|_{2}^{2}}{(1/\sqrt{n})^{2}}=o\big(n\big)
\]
\end{frame}


\begin{frame}
\frametitle{Statistical accuracy of spectral clustering}

\[
	\alert{\frac{p-q}{\sqrt{p}}\gg \sqrt{\frac{\log n}{n}}} \quad \Longrightarrow \quad \text{almost exact recovery} 
\]

\begin{itemize}
	\item {\bf dense regime:} if $p\asymp q \asymp 1$, then this condition reads
	      \[
			p-q \gg \sqrt{\frac{\log n}{n}} \qquad \text{(extremely small gap)}
	      \]
        \item {\bf ``sparse'' regime:} if $p = \frac{a\log n}{n}$ and $q = \frac{b \log n}{n}$ for  $a,b\asymp 1$, then 
	      \[
		      a - b \gg \sqrt{a} 
	      \]

\end{itemize}


{
\setbeamercolor{block body}{bg=babyblueeyes,fg=black}

\begin{varblock}[\textwidth]{}
\begin{center}
	This  condition is information-theoretically optimal (up to log factor) \\
	\hfill --- {\em Mossel, Neeman, Sly\,'15, Abbe\,'18}
\end{center}
\end{varblock}
}


\end{frame}



%\begin{frame}
%	\frametitle{Empirical performance of spectral clustering}
%	\begin{figure}
%	\includegraphics[width=0.6\textwidth]{SBM_experiment.pdf}
%	\end{figure}
%	
%	
%	{
%\setbeamercolor{block body}{bg=babyblueeyes,fg=black}
%
%\begin{varblock}[\textwidth]{}
%\begin{center}
%	$\ell_{2}$ perturbation theory alone cannot explain exact recovery guarantees
%\end{center}
%\end{varblock}
%}
%\hfill --- call for fine-grained analysis
%	\end{frame}
	


\begin{frame}
\frametitle{Proof of Lemma \ref{lem:perturbation-SBM}}
We write $\bm{A}-\mathbb{E}[\bm{A}]$ as sum of independent random matrices
\[
\bm{A}-\mathbb{E}[\bm{A}] = \sum_{i < j} \big(A_{i,j}-\mathbb{E}[A_{i,j}]\big) \big ( \bm{e}_{i}\bm{e}_{j}^{\top} +\bm{e}_{j}\bm{e}_{i}^{\top} )
\]

We only need to consider $\bm{A}_{\mathsf{upper}} \coloneqq \sum_{  i < j} \underbrace{\big(A_{i,j}-\mathbb{E}[A_{i,j}]\big) \bm{e}_{i}\bm{e}_{j}^{\top}}_{\eqqcolon \bm{X}_{i,j}} $



\begin{itemize}
\itemsep0.5em
\item First, $\|\bm{X}_{i,j}\|\leq1 \eqqcolon B$
\item Since $\mathsf{Var}(A_{i,j})\leq p$, one has
$\mathbb{E}\left[\bm{X}_{i,j}\bm{X}_{i,j}^{\top}\right]\preceq p\bm{e}_{i}\bm{e}_{i}^{\top}$, which gives
%
\[
\sum\nolimits_{i < j}\mathbb{E}\left[\bm{X}_{i,j}\bm{X}_{i,j}^{\top}\right]\preceq\sum\nolimits_{i<j}p \bm{e}_{i}\bm{e}_{i}^{\top} \preceq np\,\bm{I}_{n}
\]
%
Similarly, $\sum_{i<j}\mathbb{E}\left[ \bm{X}_{i,j}^{\top} \bm{X}_{i,j} \right]\preceq np\,\bm{I}_{n}$.
As a result,
%
\[
v \coloneqq \max\left\{ \Big\|\sum\nolimits_{i,j}\mathbb{E}\left[\bm{X}_{i,j}\bm{X}_{i,j}^{\top}\right]\Big\|,\Big\|\sum\nolimits_{i,j}\mathbb{E}\left[\bm{X}_{i,j}^{\top}\bm{X}_{i,j}\right]\Big\|\right\} \leq np
\]
%


%
\end{itemize}
%
%




\end{frame}


\begin{frame}
	\frametitle{Proof of Lemma \ref{lem:perturbation-SBM} (cont.)}
		
Take the matrix Bernstein inequality to conclude that with high prob.,
%
\begin{equation*}
	\|\bm{A}-\mathbb{E}[\bm{A}]\|\lesssim\sqrt{v\log n}+B\log n\lesssim \sqrt{np\log n} 
\end{equation*}

{\hfill \em --- as long as $p \gtrsim \frac{\log n}{n}$}
\end{frame}



\begin{frame}[plain]
	\vfill
	\centering
	\Large \bf Low-rank matrix completion
	\vfill
\end{frame}



\begin{frame}
\frametitle{Low-rank matrix completion}



\begin{columns}
\begin{column}{0.5\textwidth}
\[
 \begin{bmatrix}
   {\color{blue} \checkmark} & {\color{red} ?} &{\color{red} ?}  & {\color{red} ?} & {\color{blue} \checkmark} & {\color{red} ?} \\
   {\color{red} ?} & {\color{red} ?} & {\color{blue} \checkmark} & {\color{blue} \checkmark} & {\color{red} ?} & {\color{red} ?} \\
   {\color{blue} \checkmark} & {\color{red} ?} & {\color{red} ?} & {\color{blue} \checkmark} & {\color{red} ?} & {\color{red} ?} \\
   {\color{red} ?} & {\color{red} ?} & {\color{blue} \checkmark}  & {\color{red} ?} &{\color{red} ?}  & {\color{blue} \checkmark} \\
   {\color{blue} \checkmark}  &  {\color{red} ?} & {\color{red} ?} & {\color{red} ?}  & {\color{red} ?} & {\color{red} ?} \\
   {\color{red} ?} & {\color{blue} \checkmark} &{\color{red} ?}  & {\color{red} ?} & {\color{blue} \checkmark} & {\color{red} ?} \\
   {\color{red} ?}  &{\color{red} ?} & {\color{blue} \checkmark} &
   {\color{blue} \checkmark} & {\color{red} ?} & {\color{red} ?}
\end{bmatrix}
\]
\end{column}

\begin{column}{0.5\textwidth}  
\begin{center}
\includegraphics[width=0.9\textwidth]{NetflixMahdi} \\
\hfill {\footnotesize\em figure credit: Cand\`es ~~}
\end{center}
\end{column}

\end{columns}



\begin{itemize}
	\itemsep0.5em
	\item consider a low-rank matrix $\bm{M}^{\star} = \bm{U}^{\star} \bm{\Sigma}^{\star} \bm{V}^{\star\top}$
	\item each entry $M_{i,j}^{\star}$   is observed independently with prob.~$p$
	\item {\bf intermediate goal:} estimate $\bm{U}^{\star}, \bm{V}^{\star}$
\end{itemize}


\end{frame}




\begin{frame}
\frametitle{Spectral method for matrix completion}

\begin{itemize}
	
	\item[{\color{black}1.}] identify the key matrix $\bm{M}^{\star}$
	\item[{\color{black}2.}] construct surrogate matrix ${\bm{M}}\in \mathbb{R}^{n\times n}$ as
	%
	\[
		{M}_{i,j} = \begin{cases} \frac{1}{p} M_{i,j}^{\star}, \quad & \text{if }M_{i,j}^{\star}\text{ is observed} \\ 
					0,  & \text{else}	\end{cases} 
	\]
	%
	\begin{itemize}
		\item {\bf rationale for rescaling:} ensures $\mathbb{E}[{\bm{M}}] = \bm{M}^{\star}$
	\end{itemize}

	\bigskip

\item[{\color{black}3.}] compute the rank-$r$ SVD ${\bm{U}}{\bm{\Sigma}}{\bm{V}}^{\top}$ of ${\bm{M}}$, and return $({\bm{U}}, {\bm{\Sigma}}, {\bm{V}})$

	
\end{itemize}


\end{frame}


\begin{frame}
\frametitle{Statistical accuracy of spectral estimate}

Let's analyze a simple case where $\bm{M}^{\star}=\bm{u}^\star \bm{v}^{\star\top}$ with
%
\begin{align*}
	\bm{u}^\star = \frac{1}{\|\tilde{\bm{u}}\|_2} \tilde{\bm{u}}, \quad \bm{v}^\star = \frac{1}{\|\tilde{\bm{v}}\|_2} \tilde{\bm{v}}, \quad \tilde{\bm{u}}, \tilde{\bm{v}} \overset{\mathsf{indep.}}{\sim} \mathcal{N}(\bm{0},\bm{I}_n)
\end{align*}
%

\vfill

From Wedin's Theorem: if $\|\bm{M} - \bm{M}^\star\| \leq \frac{1}{2} \sigma_{1}(\bm{M}^{\star}) = \frac{1}{2}$, then
%
\begin{align}
	\max\left\{ \mathsf{dist}({\bm{u}},\bm{u}^{\star} ),\mathsf{dist}( {\bm{v}},\bm{v}^{\star} )\right\} 
	& \lesssim \frac{ \| {\bm{M}} - \bm{M}^\star \|  }{\sigma_1(\bm{M}^\star)} \asymp \|\bm{M} - \bm{M}^\star\|
\end{align}
%


\end{frame}

\begin{frame}
	\frametitle{Bounding $\|\bm{M} - \bm{M}^\star\|$}
Matrix Bernstein inequality tells us that 
%
\begin{lemma}
\label{lem:perturbation-SBM}
%
Consider matrix completion with $p \gg \frac{\log^{3} n}{n}$. Then with high prob. 
%
\begin{equation}
	\label{eq:MC-perturbation}
	\|\bm{M}-\bm{M}^{\star}\|\lesssim {\sqrt{\frac{\log^{3} n }{np}}} = o(1)
\end{equation}
%
\end{lemma}

	
\end{frame}



\begin{frame}
\frametitle{Sample complexity}

For rank-1 matrix completion, 
%
\begin{align*}
	p\gg \frac{\log^{3}n}{n}  \qquad \Longrightarrow \qquad \text{nearly accurate estimates}
\end{align*}


\vfill

Sample complexity needed to yield reliable spectral estimates is 
%
\[
	\underset{\alertb{\text{optimal up to log factor}}}{\underbrace{ n^2 p \asymp n \log ^3 n  }} 
\]
%


\vfill 

{\hfill \em \footnotesize --- sub-optimal accuracy though}

\end{frame}




\begin{frame}
	\frametitle{Proof of inequality \eqref{eq:MC-perturbation}}

{


Write ${\bm{M}}-\bm{M}^{\star} =\sum_{i,j}\bm{X}_{i,j}$, where $\bm{X}_{i,j}=({M}_{i,j}-M^{\star}_{i,j})\bm{e}_{i}\bm{e}_{j}^{\top}$
\begin{itemize}

\item First, based on Gaussianity, we have
\[
	\|\bm{X}_{i,j}\|\leq\frac{1}{p}\max_{i,j} |M^{\star}_{i,j}| \lesssim\frac{\log n}{pn}:=B \quad (\text{check})
\]


\item Next, $\mathbb{E}\big[\bm{X}_{i,j}\bm{X}_{i,j}^{\top}\big]=\mathsf{Var}({M}_{i,j})\bm{e}_{i}\bm{e}_{i}^{\top}$
and hence
\[
\mathbb{E}\big[\sum\nolimits _{i,j}\bm{X}_{i,j}\bm{X}_{i,j}^{\top}\big]\preceq\Big\{\max_{i,j}\mathsf{Var}\big({M}_{i,j}\big)\Big\} n\bm{I}\preceq\Big\{\frac{n}{p}\max_{i,j}(M^{\star}_{i,j})^{2}\Big\}\bm{I}
\]
\[
\Longrightarrow\qquad\big\|\mathbb{E}\big[\sum\nolimits _{i,j}\bm{X}_{i,j}\bm{X}_{i,j}^{\top}\big]\big\|\leq\frac{n}{p}\max_{i,j}(M^{\star}_{i,j})^{2}\lesssim\frac{\log^{2}n}{np}\quad(\text{check})
\]
Similar bounds hold for $\big\|\mathbb{E}\big[\sum\nolimits _{i,j}\bm{X}_{i,j}^{\top}\bm{X}_{i,j}\big]\big\|$.
Therefore, 
\[
v:=\max\left\{ \big\|\mathbb{E}\big[\sum\nolimits _{i,j}\bm{X}_{i,j}\bm{X}_{i,j}^{\top}\big]\big\|,\big\|\mathbb{E}\big[\sum\nolimits _{i,j}\bm{X}_{i,j}^{\top}\bm{X}_{i,j}\big]\big\|\right\} \lesssim\frac{\log^{2}n}{np}
\]


\end{itemize}



}

\end{frame}

\begin{frame}
	\frametitle{Proof of inequality \eqref{eq:MC-perturbation} (cont.)}
	Take the matrix Bernstein inequality to yield: if $p\gg (\log^{3}n )/n$,
then
\[
 \|\bm{M} - \bm{M}^\star\| \lesssim\sqrt{v\log n}+B\log n \asymp \sqrt{\frac{\log^{3} n }{np}}\ll1
\]
\end{frame}


\begin{frame}[plain]
	\vfill
	\centering
	\Large \bf Ranking from pairwise comparisons
	\vfill
\end{frame}



\begin{frame}
	\frametitle{Ranking from pairwise comparisons}


	\begin{center}
		\includegraphics[width=0.6\textwidth]{pairwise-comparison-tennis.png} \\
		 pairwise comparisons for ranking tennis players  \\
		\hfill {\footnotesize figure credit: Boz\'{o}ki, Csat\'{o}, Temesi}
	\end{center}

\end{frame}




\begin{frame}
	\frametitle{Bradley-Terry-Luce (logistic) model}


	\begin{center}
		\includegraphics[width=0.45\textwidth]{preference-score-w.pdf} 
	\end{center}

	\vspace{-1.2em}

	\begin{itemize}
		\itemsep0.5em
		\item $n$ items to be ranked

		\item assign  a latent positive score $\{w_i^{\star}\}_{1\leq i\leq n}$ to each item, so that 
			$$\text{item }i\succ\text{item }j \quad \text{if} \quad w_i^{\star} >w_j^{\star}$$

		\item each pair of items $(i,j)$ is compared independently
		\uncover<1>{
		%
		\begin{equation*}
			\mathbb{P}\left\{ \text{item }j\text{ beats item }i\right\} =\frac{w_{j}^{\star}}{w_{i}^{\star}+w_{j}^{\star}}
		\end{equation*}
		%
		}

		\uncover<2>{
		\vspace{-4.5em}
		%
		\begin{equation*}
			y_{i,j} ~ \overset{\text{ind.}}{=} ~ \begin{cases}
			1,\quad & \text{with prob. }\frac{w_{j}^{\star}}{w_{i}^{\star}+w_{j}^{\star}} \\
			0, & \text{else}
			\end{cases}
		\end{equation*}
		%
		\item {\bf intermediate goal:} estimate score vector $\bm{w}^{\star}$ (up to scaling)
		}
			\end{itemize}

\end{frame}







\begin{frame}
	\frametitle{Spectral ranking}



	\begin{itemize}
		\item[{\color{black}1.}] identify key matrix  ${\bm{P}}^{\star}$---\alert{probability transition matrix}
		\[
			{P}^{\star}_{i,j}=\begin{cases}
			\frac{1}{n}\cdot\frac{w_j^{\star}}{w_i^{\star}+w_j^{\star}}, & \text{if }i\neq j\\
			1-\sum\nolimits _{l:l\neq i} {P}^{\star}_{i,l},\qquad & \text{if }i=j
			\end{cases}
		\]
		Rationale:
		\begin{itemize}
		\item  $\bm{P}^{\star}$ obeys
		\[
			w_i^{\star} P_{i,j}^{\star} = w_j^{\star} P_{j,i}^{\star} \qquad \alertb{\text{(detailed balance)}}
		\]
		\item Thus, the stationary distribution $\bm{\pi}^{\star}$ of $\bm{P}^{\star}$ obeys
		\[
			\bm{\pi}^{\star}  ~=~ \frac{1}{\sum_{l}w_l^{\star}} \bm{w}^{\star} \qquad \alertb{(\text{reveals true scores})}
		\]


	\end{itemize}

	\end{itemize}



\end{frame}



\begin{frame}
	\frametitle{Spectral ranking}
	\begin{itemize}
		
		\item[{\color{black}2.}] construct a surrogate matrix ${\bm{P}}$ obeying
		%
		\[
			{P}_{i,j}=\begin{cases}
			\frac{1}{n}y_{i,j}, & \text{if }i\neq j\\
			1-\sum\nolimits _{l:l\neq i} {P}_{i,l},\qquad & \text{if }i=j
			\end{cases}
		\]
		%
		\bigskip

		\item[{\color{black}3.}] return leading left eigenvector ${\bm{\pi}}$ of ${\bm{P}}$ as score estimate
	
	\end{itemize}


	\vfill

	\hfill --- closely related to PageRank

\end{frame}

\begin{frame}
	\frametitle{Analysis of spectral ranking}
	Apply our perturbation bound to see
	\begin{align}
	\|\bm{\pi}-\bm{\pi}^{\star}\|_{\bm{\pi}^{\star}}
	& \leq \frac{\big\Vert \bm{\pi}^{\star\top}\bm{E}\big\Vert _{\bm{\pi}^{\star}}}
	{1-\max\left\{ \lambda_{2}(\bm{P}^{\star}),-\lambda_{n}(\bm{P}^{\star})\right\} -\left\Vert \bm{E}\right\Vert _{\bm{\pi}^{\star}}} \nonumber
\end{align}
%
provided that
%
\begin{align}
	1-\max\big\{ \lambda_{2}(\bm{P}^{\star}),-\lambda_{n}(\bm{P}^{\star})\big\}
	-\left\Vert \bm{E}\right\Vert _{\bm{\pi}^{\star}} > 0
\end{align}

\pause 
\vfill
{\hfill \em --- need to understand spectral gap and noise size}
\end{frame}

\begin{frame}
	\frametitle{Spectral gap of Markov chain}
	Define condition number
	\begin{equation*}
	\kappa \coloneqq\frac{\max_{1\leq i\leq n}w_{i}^{\star}}{\min_{1\leq i\leq n}w_{i}^{\star}}
\end{equation*}
	\begin{lemma}
	\label{lemma:ranking-gap}
	It follows that
%
\[
1-\max\big\{ \lambda_{2}(\bm{P}^{\star}),-\lambda_{n}(\bm{P}^{\star}) \big\} \geq\frac{1}{2\kappa^{2}}.
\]
\end{lemma}

\vfill

\begin{itemize}
	\item We omit the proof; it's based on comparison between two reversible Markov chains
\end{itemize}
\end{frame}

\begin{frame}
	\frametitle{Bound $\|\bm{E}\|_{\bm{\pi}^\star}$}
	Recall that $\bm{E}\coloneqq\bm{P}-\bm{P}^{\star}$
	
	\vfill
\begin{lemma}
	\label{lemma:ranking-noise}

With probability at least $1-O(n^{-8})$,
%
\[
	\|\bm{E}\|_{\bm{\pi}^{\star}}\leq \sqrt{\kappa}\,\|\bm{E}\| \lesssim\sqrt{\frac{\kappa\log n}{n}}.
\]
%
\end{lemma}
\end{frame}

\begin{frame}
	\frametitle{Analysis of spectral ranking (cont.)}
	Recall perturbation bound
	\begin{align}
	\|\bm{\pi}-\bm{\pi}^{\star}\|_{\bm{\pi}^{\star}}
	& \leq \frac{\big\Vert \bm{\pi}^{\star\top}\bm{E}\big\Vert _{\bm{\pi}^{\star}}}
	{1-\max\left\{ \lambda_{2}(\bm{P}^{\star}),-\lambda_{n}(\bm{P}^{\star})\right\} -\left\Vert \bm{E}\right\Vert _{\bm{\pi}^{\star}}} \nonumber \\
	& \leq 4 \kappa^2 \big\Vert \bm{\pi}^{\star\top}\bm{E}\big\Vert _{\bm{\pi}^{\star}} \qquad \alertb{(\text{provided that }n \gg \kappa^5 \log n)} \nonumber
\end{align}

Note that for any $\bm{v}$, one has 
\[
	\|\bm{v}\|_{\bm{\pi}^{\star}} \leq \sqrt{\pi^{\star}_{\max }} \, \|\bm{v}\|_2 , \qquad \text{and} \qquad
	\|\bm{v}\|_2  \leq  \frac{1}{\sqrt{  \pi^{\star}_{\min} } } \, \|\bm{v}\|_{\bm{\pi}^{\star}}
\]
As a result, one has
\begin{align*}
\|\bm{\pi}-\bm{\pi}^{\star}\|_{2} & \leq\frac{1}{\sqrt{\pi_{\min}^{\star}}}\|\bm{\pi}-\bm{\pi}^{\star}\|_{\bm{\pi}^{\star}}\leq\frac{4\kappa^{2}}{\sqrt{\pi_{\min}^{\star}}}\|\bm{\pi}^{\star\top}\bm{E}\|_{\bm{\pi}^{\star}}\\
 & \leq4\kappa^{2.5}\|\bm{\pi}^{\star\top}\bm{E}\|_{2} \leq4\kappa^{2.5}\|\bm{E}\|\,\|\bm{\pi}^{\star}\|_{2}
\end{align*}
\end{frame}

\begin{frame}
	\frametitle{Proof of Lemma~\ref{lemma:ranking-noise}}
	By construction of $\bm{P}$ and $\bm{P}^{\star}$, we see that
%
\begin{equation}
	E_{i,j}=P_{i,j}-P_{i,j}^{\star}=\frac{1}{n} \big( y_{i,j}- \mathbb{E} [ y_{i,j}] \big) \label{eq:ranking-noise-off-diag}
\end{equation}
%
for any $i\neq j$.
In addition, for all $1\leq i\leq n$, it follows that
%
\begin{align}
E_{i,i} & =P_{i,i}-P_{i,i}^{\star}
%=\Big(1-\sum_{j:j\neq i}P_{i,j}\Big)-\Big(1-\sum_{j:j\neq i}P_{i,j}^{\star}\Big)
= -\sum_{j:j\neq i}E_{i,j}
  =-\frac{1}{n}\sum_{j:j\neq i}\big( y_{i,j}- \mathbb{E} [ y_{i,j}] \big) .\label{eq:ranking-noise-diag}
\end{align}
%
We shall decompose the matrix $\bm{E}$
into three parts: upper triangular part, diagonal part, and lower triangular
part:
%
\begin{equation}
\|\bm{E}\|\leq\|\bm{E}_{\mathsf{upper}}\|+\|\bm{E}_{\mathsf{diag}}\|+\|\bm{E}_{\mathsf{lower}}\|\label{eq:ranking-triangle}
\end{equation}
	
\vfill
{\hfill \em --- we will upper bound $\|\bm{E}_{\mathsf{upper}}\|$}
\end{frame}

\begin{frame}
	\frametitle{Control $\|\bm{E}_{\mathsf{diag}}\|$}
	Note that
	\[
\|\bm{E}_{\mathsf{diag}}\|=\max_{1\leq i\leq n}|E_{i,i}|=\max_{1\leq i\leq n} \frac{1}{n} \Big| \underbrace{\sum_{j:j\neq i}\big( y_{i,j}- \mathbb{E} [ y_{i,j}]}_{\eqqcolon X_{j}} \big) \Big|
\]
\begin{itemize}
	\item First, we have $|X_j| \leq 1 \eqqcolon B$
	\item Second, one has
	\[
	\sum_{j : j \neq i} \mathbb{E} [X_j^2] = \sum_{j : j \neq i} \mathsf{Var} (y_{i,j}) \leq n \eqqcolon v
	\]
\end{itemize}
By Bernstein's inequality and union bound, we have w.h.p.
\[
\max_{i} |E_{i,i}| \lesssim \frac{1}{n} \cdot ( \sqrt{v \log n} + B \log n) \asymp \sqrt{\frac{\log n }{n}}
\]
\end{frame}

\begin{frame}
	\frametitle{Control $\|\bm{E}_{\mathsf{upper}}\|$}
	First of all, we have
	\[
	\bm{E}_{\mathsf{upper}} = \sum_{i < j} E_{i,j} \bm{e}_{i} \bm{e}_{j}^{\top} =\sum_{i < j} \underbrace{ \tfrac{1}{n} \big( y_{i,j}- \mathbb{E} [ y_{i,j}] \big) \bm{e}_{i} \bm{e}_{j}^{\top} }_{\eqqcolon \bm{X}_{i,j}}
	\]
	Then 
	\begin{itemize}
		\item $\|\bm{X}_{i,j}\| \leq \frac{1}{n} \eqqcolon B$
		\item Since $\mathsf{Var}(y_{i,j})\leq 1$, one has
$\mathbb{E}\left[\bm{X}_{i,j}\bm{X}_{i,j}^{\top}\right]\preceq \frac{1}{n^2} \bm{e}_{i}\bm{e}_{i}^{\top}$, which gives
%
\[
\sum\nolimits_{i < j}\mathbb{E}\left[\bm{X}_{i,j}\bm{X}_{i,j}^{\top}\right]\preceq\sum\nolimits_{i<j} \frac{1}{n^2} \bm{e}_{i}\bm{e}_{i}^{\top} \preceq \frac{1}{n}\,\bm{I}_{n}
\]
%
Similarly, $\sum_{i<j}\mathbb{E}\left[ \bm{X}_{i,j}^{\top} \bm{X}_{i,j} \right]\preceq \frac{1}{n}\,\bm{I}_{n}$.
As a result,
%
\[
v \coloneqq \max\left\{ \Big\|\sum\nolimits_{i,j}\mathbb{E}\left[\bm{X}_{i,j}\bm{X}_{i,j}^{\top}\right]\Big\|,\Big\|\sum\nolimits_{i,j}\mathbb{E}\left[\bm{X}_{i,j}^{\top}\bm{X}_{i,j}\right]\Big\|\right\} \leq \frac{1}{n}
\]

	\end{itemize}
\end{frame}

\begin{frame}
	\frametitle{Control $\|\bm{E}_{\mathsf{upper}}\|$ (cont.)}
	Invoke matrix Bernstein to obtain
	\[
	\|\bm{E}_{\mathsf{upper}}\| \lesssim \sqrt{v \log n} + B \log n \asymp \sqrt{\frac{\log n}{n}}
	\] 
	
	\vfill
	{\hfill \em --- same bound holds for  $\|\bm{E}_{\mathsf{lower}}\|$}
\end{frame}

\begin{frame}
	\frametitle{Putting pieces together}
	Assuming $\kappa = O(1)$, we have
	\[
	\|\bm{\pi}-\bm{\pi}^{\star}\|_{2} \lesssim \sqrt{\frac{\log n}{n}} \|\bm{\pi}^{\star}\|_{2}
	\]
	\begin{itemize}
		\item vanishing relative error when $n$ goes to infinity
		\item optimal error up to a log factor 
	\end{itemize}
	
	\vfill 	
	\hfill --- {\small Negahban, Oh, Shah\,'16, Chen, Fan, Ma, Wang\,'19}
\end{frame}

%\begin{frame}
%	\frametitle{Statistical guarantees for spectral ranking}
%
%
%
%	\vfill
%
%	Suppose $\max_{i,j} \frac{w_i}{w_j} \lesssim 1$. Then with high prob.
%	%
%	\begin{align*}
%		\frac{\| \hat{\bm{\pi}} - \bm{\pi} \|_2 }{\|\bm{\pi} \|_2} \asymp \frac{\| \hat{\bm{\pi}} - \bm{\pi} \|_{\bm{\pi}} }{\|\bm{\pi} \|_{2}} \lesssim  \hspace{-1.2em} \underset{\alertb{\text{nearly perfect estimate}}}{\underbrace{ \frac{1}{\sqrt{n}} ~\rightarrow~ 0 }}
%	\end{align*}
%	%
%
%	\begin{itemize}
%		\item a consequence of Theorem \ref{thm:mc-perturbation} and  matrix Bernstein (exercise)
%	\end{itemize}
%
%\end{frame}


\end{document}

