\pdfminorversion=4
\documentclass[compress,
mathserif,wide,%red,
%handout
]{beamer}

\usepackage{amsmath,amsfonts,bm,bbm,comment}
\usepackage{latexsym,tabularx,amscd}
\usepackage{etex} 
\usepackage{algorithm}
\usepackage[noend]{algorithmic}
\usepackage{subfigure}
\usepackage{url}
%\usepackage{multimedia}
\usepackage{hyperref}
\usepackage{skull}
%\usepackage{enumitem}
%\usepackage{movie15}

\usepackage{multirow}
\usepackage{cancel}

%\usepackage{ulem}
\usepackage{lmodern}
\usepackage{array} 
\usepackage{etoolbox}
\usepackage{caption}
\usepackage[normalem]{ulem}

\usepackage{mathtools}

\usepackage[space]{grffile}

\usepackage{tikz}
\usetikzlibrary{shapes.arrows}
\usetikzlibrary{arrows,positioning}
\usepackage{pgfplots}
%\usetikzlibrary{pgfplots.groupplots}

\tikzset{
    myarrow/.style={
        draw,
        fill=black,
        single arrow,
        minimum height=6ex,
        single arrow head extend=1ex
    }
}


\newcommand{\vertiii}[1]{{\left\vert\kern-0.25ex\left\vert\kern-0.25ex\left\vert #1 
    \right\vert\kern-0.25ex\right\vert\kern-0.25ex\right\vert}}

\newcommand{\vertiiibig}[1]{{\big\vert\kern-0.25ex\big\vert\kern-0.25ex\big\vert #1 
    \big\vert\kern-0.25ex\big\vert\kern-0.25ex\big\vert}}

\newcommand{\vertiiiplain}[1]{{\vert\kern-0.25ex\vert\kern-0.25ex\vert #1 
    \vert\kern-0.25ex\vert\kern-0.25ex\vert}}



\usepackage{dsfont}
\DeclareMathOperator{\ind}{\mathds{1}}  % Indicator


\newcommand{\abs}[1]{\left|#1\right|}

\mode<presentation>

\usetheme{Madrid}
% other themes: AnnArbor, Antibes, Bergen, Berkeley, Berlin, Boadilla,
% boxes, CambridgeUS, Copenhagen, Darmstadt, default, Dresden,
% Frankfurt, Goettingen, Hannover, Ilmenau, JuanLesPins, Luebeck,
% Madrid, Maloe, Marburg, Montpellier, PaloAlto, Pittsburg, Rochester,
% Singapore, Szeged, classic

% \usecolortheme{lily} 
% color themes: albatross, beaver, beetle, crane, default, dolphin,
% dov, fly, lily, orchid, rose, seagull, seahorse, sidebartab,
% structure, whale, wolverine

% \usefonttheme{serif} 
% font themes: default, professionalfonts, serif, structurebold,
% structureitalicserif, structuresmallcapsserif

%\usefonttheme[onlymath]{serif}
%\usefonttheme[onlylarge]{structuresmallcapsserif} 
  
%\useoutertheme[subsection=false]{smoothbars}
% outer themes:
% default,infolines,miniframes,smoothbars,sidebar,split,shadow,tree,smoothtree
%\useinnertheme{rectangles}
\beamertemplatenavigationsymbolsempty

% \definecolor{fore}{RGB}{249,242,215}
% \definecolor{back}{RGB}{51,51,51}
% \definecolor{title}{RGB}{255,0,90}
% \setbeamercolor{titlelike}{fg=title}
% \setbeamercolor{normal text}{fg=fore,bg=back}
% \setbeamercovered{transparent}
  % or whatever (possibly just delete it)
%\beamertemplatenavigationsymbolsempty

\usepackage[english]{babel}
\usepackage[latin1]{inputenc}
% or whatever
%\usepackage{comicsans} 
%\renewcommand{\sfdefault}{comic} 
%\usepackage{mathptmx}
%\mathversion{bold}
%\usepackage{helvet}
%\usepackage{courier}

%\usepackage[T1]{fontenc}
% Or whatever. Note that the encoding and the font should match. If T1
% does not look nice, try deleting the line with the fontenc.


%%%% Figures %%%%%


\newcommand{\LectureFigs}{../Figures}


\definecolor{bluegray}{rgb}{0.15,0.20,0.40}
\definecolor{bluegraylight}{rgb}{0.35,0.40,0.60}
\definecolor{gray}{rgb}{0.3,0.3,0.3}
\definecolor{lightgray}{rgb}{0.7,0.7,0.7}
\definecolor{brightblue}{rgb}{0.1,0.4,0.7}
\definecolor{darkblue}{rgb}{0.1,0.1,0.6}
\definecolor{darkgreen}{rgb}{0.0,0.5,0.2}
\definecolor{orange}{rgb}{0.7,0.3,0}

\def \captioncolor { }

\setbeamertemplate{itemize item}{{\color{black}$\bullet$}}
\setbeamertemplate{itemize subitem}{{\color{black}$\circ$}}
\setbeamertemplate{frametitle}[default][center]
\setbeamerfont{frametitle}{series=\bfseries}
\addtobeamertemplate{frametitle}{\vskip1ex}{ \vspace{0.5em} \hrule}
\setbeamercolor{frametitle}{fg=,bg=}

\setbeamertemplate{theorems}[numbered]
\setbeamertemplate{caption}[numbered]
\setbeamertemplate{algorithm}[numbered]

\numberwithin{theorem}{subsection}
\numberwithin{equation}{subsection}
\numberwithin{figure}{subsection}
\numberwithin{algorithm}{subsection}


%\setbeamerfont{page number in head/foot}{size=\scriptsize}
\setbeamertemplate{footline}[frame number]
\setbeamercolor{page number in head/foot}{fg=black}
\setbeamersize{text margin left=2em,text margin right=2em}

\setbeamertemplate{footline}
{
    %\hspace{3em} \insertshorttitle \hfill \insertframenumber / \inserttotalframenumber\hspace{3em}
    \hspace{3em} \insertshorttitle \hfill \insertsubsectionnumber-\insertframenumber \hspace{3em}
    \vspace{1.5em}
}

\definecolor{babyblueeyes}{rgb}{0.63, 0.79, 0.95}
\definecolor{bleudefrance}{rgb}{0.19, 0.55, 0.91}
\definecolor{matlabPlotBlue}{rgb}{0.25, 0.58, 0.8}
\definecolor{matlabPlotOrange}{rgb}{0.855, 0.324, 0.098}



\captionsetup[figure]{labelfont={color=black}}

\AtBeginEnvironment{lemma}{%
  \setbeamercolor{block title}{fg=black,bg=babyblueeyes}
  \setbeamercolor{block body}{fg=black,bg=}
  \setbeamerfont*{block title}{series=\bfseries}
}

\AtBeginEnvironment{theorem}{%
  \setbeamercolor{block title}{fg=black,bg=babyblueeyes}
  \setbeamercolor{block body}{fg=black,bg=}
  \setbeamerfont*{block title}{series=\bfseries}
}

\AtBeginEnvironment{corollary}{%
  \setbeamercolor{block title}{fg=black,bg=babyblueeyes}
  \setbeamercolor{block body}{fg=black,bg=}
  \setbeamerfont*{block title}{series=\bfseries}
}

\AtBeginEnvironment{fact}{%
  \setbeamercolor{block title}{fg=black,bg=babyblueeyes}
  \setbeamercolor{block body}{fg=black,bg=}
  \setbeamerfont*{block title}{series=\bfseries}
}

\AtBeginEnvironment{definition}{%
  \setbeamercolor{block title}{fg=black,bg=babyblueeyes}
  \setbeamercolor{block body}{fg=black,bg=}
  \setbeamerfont*{block title}{series=\bfseries}
}

\newenvironment<>{varblock}[2][.9\textwidth]{%
  \vspace{-1em}
  \begin{center}
  \begin{minipage}{#1}
  %\setlength{\textwidth}{#1}
  \begin{actionenv}#3%
    \def\insertblocktitle{#2}%
    \par%
    \usebeamertemplate{block begin}}
  {\par%
    \usebeamertemplate{block end}%
  \end{actionenv}
  \end{minipage}
  \end{center}
}


\newcommand\blankfootnote[1]{%
  \let\thefootnote\relax\footnotetext{#1}%
  \let\thefootnote\svthefootnote%
}


%%%%%
\newcommand{\bb}{\bm b}
\newcommand{\bZ}{\bm Z}
\newcommand{\bC}{\bm C}
\newcommand{\cU}{{\cal U}}
\newcommand{\bR}{{\bm R}}
\newcommand{\bi}{{\bm i}}
\newcommand{\bS}{{\bm S}}
\newcommand{\cE}{{\cal E}}
\newcommand{\cF}{{\cal F}}
\newcommand{\bk}{{\bm k}}
\newcommand{\bu}{{\bm u}}
\newcommand{\bx}{{\bm x}}
\newcommand{\bA}{{\bm A}}
\newcommand{\bh}{{\bm h}}
\newcommand{\by}{{\bm y}}
\newcommand{\bI}{{\bm I}}
\newcommand{\ba}{{\bm a}}
\newcommand{\bw}{{\bm w}}
\newcommand{\bPhi}{{\bm \Phi}}
\newcommand{\bX}{{\bm X}}
\newcommand{\bY}{{\bm Y}}
\newcommand{\bM}{{\bm M}}
\newcommand{\be}{{\bm e}}
\newcommand{\bs}{{\bm s}}
\newcommand{\bB}{{\bm B}}
\newcommand{\bW}{{\bm W}}
\newcommand{\bN}{{\bm N}}
\newcommand{\bU}{{\bm U}}
\newcommand{\bV}{{\bm V}}
\newcommand{\bP}{{\bm P}}
\newcommand{\bH}{{\bm H}}
\newcommand{\bSigma}{{\bm \Sigma}}
\newcommand{\bL}{{\bm L}}



\newcommand{\bv}{\bm v}
\newcommand{\eps}{\epsilon}
\newcommand{\vf}{\varphi}
\newcommand{\de}{\delta}
\newcommand{\<}{\langle}
\renewcommand{\>}{\rangle}
\newcommand{\goto}{\rightarrow}
\newcommand{\argmin}{\mbox{argmin}}
\newcommand{\argmax}{\mbox{argmax}}
\newcommand{\Ave}{\mathop{\rm Ave}\nolimits}
\newcommand{\sgn}{\mbox{sgn}}
\newcommand{\cN}{\mathcal{N}}
\newcommand{\cR}{\mathcal{R}}

\def\C{{\mathbb{C}}}
\def\R{{{\mathbb{R}}}} %seemed too large in subscripts, so I changed --JW 
\def\P{{\hbox{\bf P}}}
\def\Z{{{\mathbb{Z}}}}
\def\T{{\hbox{\bf T}}}


\newcommand{\cA}{\mathcal{A}}
\newcommand{\cP}{\mathcal{P}}
\newcommand{\cD}{\mathcal{D}}
\newcommand{\cL}{\mathcal{L}}
\newcommand{\shrink}{\text{shrink}}


\newcommand{\Obs}{\Omega_{\text{obs}}}


\renewcommand{\P}{\operatorname{\mathbb{P}}}
\newcommand{\E}{\operatorname{\mathbb{E}}}

% Linear algebra macros
%\newcommand{\vct}[1]{#1}
%\newcommand{\mtx}[1]{#1}
\newcommand{\vct}[1]{\bm{#1}}
\newcommand{\mtx}[1]{\bm{#1}}
%\newcommand{\mtx}[1]{\mathsfsl{#1}}

\newcommand{\transp}{T}
\newcommand{\adj}{*}
\newcommand{\psinv}{\dagger}
\newcommand{\lspan}[1]{\operatorname{span}{#1}}

\newcommand{\range}{\operatorname{range}}
\newcommand{\colspan}{\operatorname{colspan}}

\newcommand{\rank}{\text{rank}}

\newcommand{\diag}{\operatorname{diag}}
\newcommand{\trace}{\text{Tr}}

\newcommand{\supp}[1]{\operatorname{supp}(#1)}

\newcommand{\smax}{\sigma_{\max}}
\newcommand{\smin}{\sigma_{\min}}

\newcommand{\restrict}[1]{\big\vert_{#1}}

\newcommand{\Id}{\text{\em I}}
\newcommand{\OpId}{\mathcal{I}}



% define your own colors:
\definecolor{Red}{rgb}{1,0,0}
\definecolor{Blue}{rgb}{0,0,1}
\definecolor{Green}{rgb}{0,1,0}
\definecolor{magenta}{rgb}{1,0,.6}
\definecolor{lightblue}{rgb}{0,.5,1}
\definecolor{lightpurple}{rgb}{.6,.4,1}
\definecolor{gold}{rgb}{.6,.5,0}
\definecolor{orange}{rgb}{1,0.4,0}
\definecolor{hotpink}{rgb}{1,0,0.5}
\definecolor{newcolor2}{rgb}{.5,.3,.5}
\definecolor{newcolor}{rgb}{0,.3,1}
\definecolor{newcolor3}{rgb}{1,0,.35}
\definecolor{darkgreen1}{rgb}{0, .35, 0}
\definecolor{darkgreen}{rgb}{0, .6, 0}
\definecolor{darkred}{rgb}{.75,0,0}

\xdefinecolor{olive}{cmyk}{0.64,0,0.95,0.4}
\xdefinecolor{purpleish}{cmyk}{0.75,0.75,0,0}

\definecolor{lightgrey}{RGB}{186,186,186}
\definecolor{back}{RGB}{51,51,51}
\definecolor{fore}{RGB}{249,242,215}
\definecolor{back}{RGB}{51,51,51}
\definecolor{title}{RGB}{255,0,20}
\definecolor{keywords}{RGB}{255,0,90}
\definecolor{comments}{RGB}{60,179,113}
\definecolor{cobalt}{RGB}{102,153,204}

\def \itemstyle {\bf}
\def \itemhighlight {\bf}  

% \definecolor{blockbodycolor}{RGB}{30,30,40}
% \definecolor{blocktitlecolor}{RGB}{55,55,75}
\definecolor{blockbodycolor}{RGB}{186,206,232}
\definecolor{blocktitlecolor}{RGB}{55,55,75}

\setbeamercolor{titlelike}{fg=darkred,bg=white}
\setbeamercolor{normal text}{fg=black,bg=white}
\setbeamercolor{block title}{fg=white,bg=blocktitlecolor}
\setbeamercolor{block body}{bg=blockbodycolor}
\setbeamercolor{alerted text}{fg=darkred}
\setbeamercolor{title}{fg=darkred,bg=lightgrey}


\newcommand{\alertb}[1]{\textcolor{blue}{ #1}}

\newcommand{\yp}[1]{\textcolor{red}{ #1}}
\newcommand{\zeronorm}[1]{\left\|#1 \right\|_0}
\newcommand{\unorm}[1]{\left\|#1 \right\|_u}
\newcommand{\ynorm}[1]{\left\|#1 \right\|_{\bar{y}}}
\newcommand{\onetwonorm}[1]{\left\|#1\right\|_{1,2}}
\newcommand{\opnorm}[1]{\left\|#1\right\|}
\newcommand{\fronorm}[1]{\left\|#1\right\|_{F}}
\newcommand{\onenorm}[1]{\left\|#1\right\|_{\ell_1}}
\newcommand{\twonorm}[1]{\left\|#1\right\|_{2}}
\newcommand{\oneinfnorm}[1]{\left\|#1\right\|_{1,\infty}}
\newcommand{\infnorm}[1]{\left\|#1\right\|_{\ell_\infty}}
\newcommand{\nucnorm}[1]{\left\|#1\right\|_*}

%\makeatletter
%\def\th@mystyle{%
%    \normalfont % body font
%    \setbeamercolor{block title example}{bg=,fg=black}
%    \setbeamerfont{block title example}{series=\bfseries}
%    \setbeamercolor{block body example}{bg=,fg=black}
%    %\setbeamertemplate{exampleblock}[numbered]
%    \def\inserttheoremblockenv{exampleblock}
%  }
%\makeatother
%\theoremstyle{mystyle}
%\newtheorem*{lem}{Lemma}

\setbeamertemplate{frametitle continuation}{}


\newcommand{\courseTitle}{STAT 37797: Mathematics of Data Science} 



% \subtitle
% {Include Only If Paper Has a Subtitle}

\vspace{.5cm}

\author{Yuxin Chen}

\institute[Princeton] % (optional, but mostly needed)
{Princeton University}


% - Use the \inst command only if there are several affiliations.
% - Keep it simple, no one is interested in your street address.


\subject{Beamer}
% This is only inserted into the PDF information catalog. Can be left
% out.

% Delete this, if you do not want the table of contents to pop up at
% the beginning of each subsection:
% \AtBeginSubsection[]
% {
%   \begin{frame}<beamer>
%     \frametitle{Outline}
%     \tableofcontents[currentsection,currentsubsection]
%   \end{frame}
% }






\graphicspath{{../../../Figures/}}


\title % (optional, use only with long paper titles)
[Matrix concentration]{Matrix concentration inequalities}

\defbeamertemplate*{title page}{customized}[1][]
{

  \hfill {\em \courseTitle}

  \begin{center}
    \vspace{2.5em}
    \usebeamerfont{title} {\Large\bf\inserttitle} \par
  
    \vspace{1.5em}
    \includegraphics[width=2cm]{\LectureFigs/UC_logo.png} 
  
    \vspace{1em}
    {\large Cong Ma \par }

    \vspace{0.2em}
    { \large \quad University of Chicago, Autumn 2021 }
  \end{center}

  \vfill
}

\setcounter{subsection}{4}

\begin{document}


\begin{frame}[plain]
  \titlepage

\end{frame}


\begin{frame}
	\frametitle{Concentration inequalities}
	Let $X_{1}, X_{2}, \ldots, X_{n}$ be i.i.d.~random variables, law of large numbers tells us that
	\[
	\frac{1}{n}\sum_{l=1}^{n} X_{l} - \mathbb{E} \left[ \frac{1}{n}\sum_{l=1}^{n} X_{l} \right] \rightarrow 0, \quad \text{as }n \rightarrow \infty
	\]
	
	\vfill
		{
\setbeamercolor{block body}{bg=babyblueeyes,fg=black}

\begin{varblock}[\textwidth]{}
{\bf Key message: }\\
\begin{center}
sum of independent random variables \emph{concentrate} around its mean
\end{center}
\end{varblock}

}

{\hfill \em --- how fast does it concentrate?}
\end{frame}

\begin{frame}
	\frametitle{Bernstein's inequality}
	Consider a sequence of independent random variables $\{X_{l}\} \in \mathbb{R}$
%
\begin{itemize}
	\itemsep0.5em
	\item $\mathbb{E}[{X}_l]=0$ \qquad\qquad\qquad\qquad\qquad $\bullet$ $|{X}_{l}|\leq B$ for each $l$
	\item variance statistic: 
	$$v \coloneqq \mathbb{E} \big [ \big (\sum_{l} X_l \big)^{2} \big] = \sum_{l=1}^{n} \mathbb{E} \big [X_{l}^{2} \big]$$
\end{itemize}

\begin{theorem}[Bernstein's inequality]\label{thm:mtx-Bernstein}
%
For all $\tau \geq 0$,
\vspace{-1em}
\[
\mathbb{P}\left\{ \left| \sum\nolimits_{l}{X}_{l}\right| \geq \tau\right\} \leq  2 \exp\left(\frac{-\tau^{2}/2}{v+B\tau/3}\right)
\]
%
\end{theorem}
\end{frame}

\begin{frame}
\frametitle{Tail behavior}

\[
\mathbb{P}\left\{ \left| \sum\nolimits_{l}{X}_{l}\right| \geq \tau\right\} \leq  2 \exp\left(\frac{-\tau^{2}/2}{v+B\tau/3}\right)
\]

\bigskip

\begin{itemize}
	\itemsep1em
	\item<1> {\bf moderate-deviation regime} ($\tau$ is small): \\
		  \quad --- sub-Gaussian tail behavior $\exp(-\tau^2/2v)$
		  
	\item<1> {\bf large-deviation regime} ($\tau$ is large): \\
		  \quad --- sub-exponential tail behavior $\exp(-3\tau/2B)$ (slower decay)
	
	\item<1> {\bf user-friendly form} (exercise): with prob.~$1-O(n^{-10})$
	%
	\begin{equation*} 
		\left| \sum\nolimits_{l}{X}_{l}\right| \lesssim  \sqrt{ v \log n } + B \log n
	\end{equation*}

\end{itemize}

\end{frame}


\begin{frame}
\frametitle{Tail behavior (cont.)}

	\begin{center}
		\includegraphics[width=0.8\textwidth]{Gaussian_exponential_tail.pdf}
	\end{center}

\end{frame}

\begin{frame}[plain]
	\vfill
	\centering
	\large \bf There are exponential concentration inequalities for \\ spectral norm of sum of independent random matrices
	\vfill
\end{frame}

\begin{frame}
\frametitle{Matrix Bernstein inequality}

Consider a sequence of independent random matrices $\big\{ \bm{X}_{l}\in\mathbb{R}^{d_{1}\times d_{2}}\big\} $
%
\begin{itemize}
	\itemsep0.5em
	\item $\mathbb{E}[\bm{X}_l]=\bm{0}$ \qquad\qquad\qquad\qquad\qquad $\bullet$ $\|\bm{X}_{l}\|\leq B$ for each $l$
	\item variance statistic: 
	$$v:=\max\left\{  \left\Vert  \mathbb{E}\left[ \sum\nolimits_{l} \bm{X}_{l}\bm{X}_{l}^{\top}  \right]  \right\Vert  ,  
			  \left\Vert  \mathbb{E}\left[ \sum\nolimits_{l} \bm{X}_{l}^{\top}\bm{X}_{l}  \right]  \right\Vert  \right\} $$
\end{itemize}

\begin{theorem}[Matrix Bernstein inequality]\label{thm:mtx-Bernstein}
%
For all $\tau \geq 0$,
\vspace{-1em}
\[
\mathbb{P}\left\{ \left\Vert \sum\nolimits_{l}\bm{X}_{l}\right\Vert \geq \tau\right\} \leq\left(d_{1}+d_{2}\right)\exp\left(\frac{-\tau^{2}/2}{v+B\tau/3}\right)
\]
%
\end{theorem}

\pause
User-friendly form: with probability at least $1- O((d_1 +d_2)^{-10})$
\begin{equation} \label{eq:Bernstein-user-friendly}
		\left\Vert \sum\nolimits_{l}\bm{X}_{l}\right\Vert \lesssim  \sqrt{ v \log(d_1+d_2) } + B \log(d_1+d_2)
	\end{equation}
\end{frame}









\begin{frame}[plain]
\frametitle{}  

\vfill
\centering
	{\large\bf This lecture: detailed introduction to matrix Bernstein}

\bigskip
\bigskip

\hfill {\small\em An introduction to matrix concentration inequalities \\ \hfill --- Joel Tropp\,'15}

\vfill

\end{frame}


\begin{frame}
\frametitle{Outline}

\begin{itemize}
  \itemsep1em
  \item Background on matrix functions
  \item Matrix Laplace transform method
  \item Matrix Bernstein inequality
  % \item Application: random features
  %\item Proof of Lieb's Theorem
\end{itemize}

\end{frame}



\begin{frame}[plain]

\vfill
\begin{center}
  {\Large\bf Background on matrix functions}
\end{center}

\vfill

\end{frame}




\begin{frame}
\frametitle{Matrix function}  

Suppose the eigendecomposition of a symmetric matrix $\bm{A}\in \mathbb{R}^{d\times d}$ is 
%
{\small
\[
\bm{A}=\bm{U}\left[\begin{array}{ccc}
\lambda_{1}\\
 & \ddots\\
 &  & \lambda_{d}
\end{array}\right]\bm{U}^{\top}
\]
}
%
Then we can define 
%
{\small
\[
f(\bm{A}) := \bm{U}\left[\begin{array}{ccc}
f(\lambda_{1})\\
 & \ddots\\
 &  & f(\lambda_{d})
\end{array}\right]\bm{U}^{\top}
\]
}


{\hfill \em --- align with our intuition about $\bm{A}^{k}$}
\end{frame}


\begin{frame}
\frametitle{Examples of matrix functions}  

\begin{itemize}
	\itemsep1em
	\item Let $f(a)=c_0 + \sum_{k=1}^{\infty} c_k a^k$, then
	%
	\[
		f(\bm{A}) := c_0 \bm{I} + \sum_{k=1}^{\infty} c_k {\bm A}^k
	\]
	%
	\item {\bf matrix exponential:} $\mathrm{e}^{\bm{A}} := \bm{I} + \sum_{k=1}^{\infty} \frac{1}{k!} \bm{A}^k$

	%
	\begin{itemize}
		\item monotonicity: if $\bm{A}\preceq \bm{H}$, then $\mathrm{tr}\,\mathrm{e}^{\bm{A}} \leq \mathrm{tr}\,\mathrm{e}^{\bm{H}}$
	\end{itemize}


	\item {\bf matrix logarithm:} $\log (\mathrm{e}^{\bm{A}} ) := \bm{A}$
		
	%
	\begin{itemize}
		\item monotonicity: if $\bm{0}\preceq \bm{A}\preceq \bm{H}$, then $\log {\bm{A}} \preceq \log (\bm{H})$ (does not hold for matrix exponential) 
	\end{itemize}


\end{itemize}

\end{frame}


\begin{frame}
\frametitle{Matrix moments and cumulants}  

	Let $\bm{X}$ be a random symmetric matrix. Then

	\begin{itemize}
		\item {\bf matrix moment generating function (MGF):} 
		%
		\[
			\bm{M}_{\bm{X}}(\theta) := \mathbb{E}[\mathrm{e}^{\theta\bm{X}}]
		\]
		%
	\item {\bf matrix cumulant generating function (CGF):} 
		%
		\[
			\bm{\Xi}_{\bm{X}}(\theta) := \log \mathbb{E}[\mathrm{e}^{\theta\bm{X}}]
		\]
		%
		

	
	\end{itemize}

{\hfill \em --- expectations may not exist for all $\theta$}
\end{frame}




\begin{frame}[plain]

\vfill
\begin{center}
  {\Large\bf Matrix Laplace transform method}
\end{center}

\vfill

\end{frame}





\begin{frame}
\frametitle{Matrix Laplace transform}  

A key step for a scalar random variable $Y$: by Markov's inequality,
	\[
		\mathbb{P}\left\{ Y \geq t\right\} \leq \inf_{\theta>0} \mathrm{e}^{-\theta t} \,\mathbb{E}\big[\mathrm{e}^{\theta {Y}}\big]
	\]


\bigskip
\bigskip

This can be generalized to the matrix case

\end{frame}


\begin{frame}
\frametitle{Matrix Laplace transform} 


\begin{lemma}\label{lem:matrix-Laplace}
Let $\bm{Y}$ be a random symmetric matrix. For all $t\in\mathbb{R}$,
%
\[
	\mathbb{P}\left\{ \lambda_{\max}(\bm{Y}) \geq t\right\} \leq\inf_{\theta>0} \mathrm{e}^{-\theta t} \,\mathbb{E}\big[\mathrm{tr}\,\mathrm{e}^{\theta\bm{Y}}\big]
\]
%
\end{lemma}

\begin{itemize}
	\item can control the extreme eigenvalues of $\bm{Y}$ via the trace of the matrix MGF
	\item similar result holds for minimum eigenvalue
\end{itemize}

\end{frame}




\begin{frame}
\frametitle{Proof of Lemma \ref{lem:matrix-Laplace}}  

For any $\theta>0$,
\begin{align*}
\mathbb{P}\left\{ \lambda_{\max}(\bm{Y})\geq t\right\}  & =\mathbb{P}\left\{ \mathrm{e}^{\theta\lambda_{\max}(\bm{Y})}\geq\mathrm{e}^{\theta t}\right\} \\
	& \leq\frac{\mathbb{E}[\mathrm{e}^{\theta\lambda_{\max}(\bm{Y})}]}{\mathrm{e}^{\theta t}}\qquad ~~\alertb{(\text{Markov's inequality})}\\
 	& =\frac{\mathbb{E}[\mathrm{e}^{\lambda_{\max}(\theta\bm{Y})}]}{\mathrm{e}^{\theta t}}\\
	& =\frac{\mathbb{E}[\lambda_{\max}(\mathrm{e}^{\theta\bm{Y}})]}{\mathrm{e}^{\theta t}}\qquad  \alertb{(\mathrm{e}^{\lambda_{\max}(\bm{Z})}=\lambda_{\max}(\mathrm{e}^{\bm{Z}}))}\\
 	& \leq\frac{\mathbb{E}[\mathrm{tr}\,\mathrm{e}^{\theta\bm{Y}}]}{\mathrm{e}^{\theta t}}
\end{align*}

This completes the proof since it holds for any $\theta>0$ 

\end{frame}




\begin{frame}
\frametitle{Issues of the matrix MGF}  

The Laplace transform method is effective for controlling an independent sum when MGF decomposes

\begin{itemize}
	\item in the scalar case where $X=X_{1}+\cdots+X_{n}$ with independent $\{X_{l}\}$:
	%
	\[
		M_{X}(\theta)=\mathbb{E}[\mathrm{e}^{\theta X_1 + \cdots + \theta X_n}]= \mathbb{E}[\mathrm{e}^{\theta X_{1}}] \cdots \mathbb{E}[\mathrm{e}^{\theta X_{n}}] = \hspace{-1.7em}\underset{\alertb{\text{look at each }X_l\text{ separately}}}{\underbrace{ \prod_{l=1}^{n} M_{X_{l}}(\theta) }}
	\]
	%
\end{itemize}

\vfill

{\bf Issues in the matrix settings:} 
%
\[
	\mathrm{e}^{\bm{X}_{1}+\bm{X}_{2}} ~\neq~ \mathrm{e}^{\bm{X}_{1}}\mathrm{e}^{\bm{X}_{2}}\qquad\text{unless }\bm{X}_{1}\text{ and }\bm{X}_{2}\text{ commute}
\]
%
\[
	\mathrm{tr}\,\mathrm{e}^{\bm{X}_{1}+\cdots+\bm{X}_{n}} ~\nleq~ \mathrm{tr}\,\mathrm{e}^{\bm{X}_{1}}\mathrm{e}^{\bm{X}_{1}}\cdots\mathrm{e}^{\bm{X}_{n}} \quad \text{for } n  \geq 3
\]

\end{frame}

\begin{frame}
	\frametitle{How about matrix CGF?}
	\begin{itemize}
	\item in the scalar case where $X=X_{1}+\cdots+X_{n}$ with independent $\{X_{l}\}$:
	%
	\[
		{\Xi}_{X }(\theta)  = \log M_{X}(\theta)=\underset{\alertb{\text{look at each }X_l\text{ separately}}}{\underbrace{ \sum_{l=1}^{n} \log M_{X_{l}}(\theta) }} = \sum_{l} {\Xi}_{{X}_{l}}(\theta)
	\]
	%
\end{itemize}

\vfill

In matrix case, can we hope for
%
\[
\bm{\Xi}_{\sum_{l }\bm{X}_{l} }(\theta) = \sum_{l} \bm{\Xi}_{\bm{X}_{l}}(\theta) \quad ?
\]

{\hfill \em --- Nope; But...}
\end{frame}




\begin{frame}
\frametitle{Subadditivity of matrix CGF}  

Fortunately, the matrix CGF satisfies certain subadditivity rules, allowing us to decompose independent matrix components 

\bigskip


\begin{lemma}
	\label{lem:subadditive-CGF}
	Consider a finite sequence $\{\bm{X}_l\}_{1\leq l\leq n}$ of independent random symmetric matrices. Then for any $\theta \in \mathbb{R}$, 
	%
	\[
		\underset{\alertb{ \mathrm{tr} \exp\big(\bm{\Xi}_{\Sigma_{l}\bm{X}_{l}}(\theta)\big) }}{\underbrace{ \mathbb{E}\Big[\mathrm{tr}\,\mathrm{e}^{\theta\sum_{l}\bm{X}_{l}}\Big] }} 
		\leq
		\underset{\alertb{ \mathrm{tr} \exp\big(\sum_{l}\bm{\Xi}_{\bm{X}_{l}}(\theta)\big) }}{\underbrace{ \mathrm{tr} \exp\Big(\sum\nolimits _{l}\log\mathbb{E}\big[\mathrm{e}^{\theta\bm{X}_{l}}\big]\Big) }}
	\]
	%

	
\end{lemma}

	
\begin{itemize}
	\item this is a deep result --- based on Lieb's Theorem! 
\end{itemize}



\end{frame}






\begin{frame}
\frametitle{Lieb's Theorem}



\begin{columns}

\begin{column}{0.25\textwidth}

\begin{center}
\includegraphics[width=0.95\textwidth]{Elliott-Lieb.jpg} \\
	{\small Elliott Lieb}
\end{center}

\end{column}

\begin{column}{0.6\textwidth}

\begin{theorem}[Lieb\,'73]
	\label{thm:Lieb}
	Fix a symmetric matrix $\bm{H}$. Then 
	%
	\[
		\bm{A}~\mapsto~\mathrm{tr}\exp(\bm{H}+\log\bm{A})
	\]
	%
	is concave on positive-definite cone
\end{theorem}

\end{column}
\end{columns}


\vfill

Lieb's Theorem immediately implies  \alertb{(exercise: Jensen's inequality)}
%
\begin{align}
	\label{eq:Lieb-corollary}
	\mathbb{E}\big[\mathrm{tr}\exp(\bm{H}+\bm{X})\big]\leq\mathrm{tr}\exp\big(\bm{H}+\log\mathbb{E}\big[\mathrm{e}^{\bm{X}}\big]\big)
\end{align}
%


\end{frame}





\begin{frame}
	\frametitle{Proof sketch of Lieb's Theorem}


Main observation: $\mathrm{tr}(\cdot)$ admits a variational formula

\bigskip

\begin{lemma}
	\label{lem:trace-variational}
	For any $\bm{M}\succeq \bm{0}$, one has
	\vspace{-0.5em}
	%
	\begin{align*}
		\mathrm{tr} \bm{M} = \sup_{\bm{T}\succ \bm{0}}  \mathrm{tr}~\big[ \hspace{-2em} \underset{\alertb{\text{relative entropy is }- \bm{T} \log \bm{M} + \bm{T} \log \bm{T} - \bm{T} + \bm{M}}}{\underbrace{ \bm{T} \log \bm{M} - \bm{T} \log \bm{T} + \bm{T} }} \hspace{-2em}\big]
	\end{align*}
	%
\end{lemma}


\end{frame}




\begin{frame}
\frametitle{Proof of Lemma \ref{lem:subadditive-CGF}}  


\begin{align*}
\mathbb{E}\big[\mathrm{tr}\,\mathrm{e}^{\theta\sum_{l}\bm{X}_{l}}\big] & =\mathbb{E}\big[\mathrm{tr}\exp\big(\theta\sum\nolimits _{l=1}^{n-1}\bm{X}_{l}+\theta\bm{X}_{n}\big)\big]\\
	& \leq\mathbb{E}\Big[\mathrm{tr}\exp\Big(\theta\sum\nolimits _{l=1}^{n-1}\bm{X}_{l}+\log\mathbb{E}\big[ \mathrm{e}^{\theta\bm{X}_{n}} \big]\Big)\Big] \qquad \alertb{(\text{by }\eqref{eq:Lieb-corollary})}\\
 	& \leq\mathbb{E}\Big[\mathrm{tr}\exp\Big(\theta\sum\nolimits _{l=1}^{n-2}\bm{X}_{l}+\log\mathbb{E}\big[ \mathrm{e}^{\theta\bm{X}_{n-1}} \big]+\log\mathbb{E}\big[\mathrm{e}^{\theta\bm{X}_{n}}\big]\Big)\Big]\\
 	& \leq\cdots\\
 	& \leq\mathrm{tr}\exp\Big(\sum\nolimits _{l=1}^{n}\log\mathbb{E}\big[\mathrm{e}^{\theta\bm{X}_{l}}\big]\Big)
\end{align*}


\end{frame}



\begin{frame}
\frametitle{Master bounds}  


Combining the Laplace transform method with the subadditivity of CGF yields:

\bigskip

\begin{theorem}[Master bounds for sum of independent matrices]
	\label{thm:master-bounds}
	Consider a finite sequence $\{\bm{X}_l\}$ of independent random symmetric matrices. Then 
	\vspace{-0.5em}
	%
	\[
		\mathbb{P}\left\{ \lambda_{\max}\big(\sum\nolimits _{l}\bm{X}_{l}\big)\geq t\right\} \leq\inf_{\theta>0}\frac{\mathrm{tr}\exp\big(\sum\nolimits _{l}\log\mathbb{E}[\mathrm{e}^{\theta\bm{X}_{l}}]\big)}{\mathrm{e}^{\theta t}}
	\]
	%
\end{theorem}

\begin{itemize}
	\item this is a general result underlying the proofs of the matrix Bernstein inequality and beyond (e.g.,~matrix Chernoff)
\end{itemize}


\end{frame}





\begin{frame}[plain]

\vfill
\begin{center}
  {\Large\bf Matrix Bernstein inequality}
\end{center}

\vfill

\end{frame}


\begin{frame}
\frametitle{Matrix CGF}  

\[
	\mathbb{P}\left\{ \lambda_{\max}\big(\sum\nolimits _{l}\bm{X}_{l}\big)\geq t\right\} \leq\inf_{\theta>0}\frac{\mathrm{tr}\exp\big(\sum\nolimits _{l}\log\mathbb{E}[\mathrm{e}^{\theta\bm{X}_{l}}]\big)}{\mathrm{e}^{\theta t}}
\]


\bigskip
\bigskip

To invoke the master bound, one needs to \hspace{-1.5em} $\underset{\alertb{\text{main step for proving matrix Bernstein}}}{\underbrace{\text{control the matrix CGF}}}$ 
	



\end{frame}






\begin{frame}
\frametitle{Symmetric case}  


Consider a sequence of independent random symmetric matrices $\big\{ \bm{X}_{l}\in\mathbb{R}^{d \times d}\big\} $
%
\begin{itemize}
	\itemsep0.5em
	\item $\mathbb{E}[\bm{X}_l]=\bm{0}$ \qquad\qquad\qquad\qquad\qquad $\bullet$ $\lambda_{\max}(\bm{X}_{l}) \leq B$ for each $l$
	\item variance statistic: 
	$v:= \left\Vert  \mathbb{E}\left[ \sum\nolimits_{l} \bm{X}_{l}^2  \right]  \right\Vert  $
\end{itemize}

\begin{theorem}[Matrix Bernstein inequality: symmetric case]
\label{thm:mtx-Bernstein-symmetric}
%
For all $\tau \geq 0$,
\vspace{-1em}
\[
	\mathbb{P}\left\{ \lambda_{\max}\left( \sum\nolimits_{l}\bm{X}_{l}\right ) \geq \tau\right\} \leq d \exp\left(\frac{-\tau^{2}/2}{v+B\tau/3}\right)
\]
%
\end{theorem}

{\hfill \em \footnotesize--- left as exercise to prove extension to rectangular case}
\end{frame}




\begin{frame}
\frametitle{Bounding matrix CGF}  


For bounded random matrices, one can control the matrix CGF as follows:

\bigskip
\bigskip

\begin{lemma}
	\label{lem:bound-matrix-CGF}
	Suppose $\mathbb{E}[\bm{X}]=\bm{0}$ and $\lambda_{\max}(\bm{X})\leq B$. Then for $0<\theta<3/B$,
	\vspace{-0.5em}
	%
	\[
		\log\mathbb{E}\big[\mathrm{e}^{\theta\bm{X}}\big]\preceq\frac{\theta^{2}/2}{1-\theta B/3}\mathbb{E}[\bm{X}^{2}]
	\]
	%
\end{lemma}



\end{frame}




\begin{frame}
\frametitle{Proof of Theorem \ref{thm:mtx-Bernstein-symmetric}}  


Let $g(\theta):=\frac{\theta^{2}/2}{1-\theta B/3}$, then it follows
from the master bound that
\begin{align*}
\mathbb{P}\left\{ \lambda_{\max}\big(\sum\nolimits _{i}\bm{X}_{i}\big)\geq t\right\}  & \leq\inf_{\theta>0}\frac{\mathrm{tr}\exp\big(\sum\nolimits _{i=1}^{n}\log\mathbb{E}[\mathrm{e}^{\theta\bm{X}_{i}}]\big)}{\mathrm{e}^{\theta t}}\\
	& \hspace{-1.5em}\overset{\alertb{\text{Lemma \ref{lem:bound-matrix-CGF}}}}{\leq}  \hspace{-1em} \inf_{0<\theta<3/B}\frac{\mathrm{tr}\exp\big(g(\theta)\sum\nolimits _{i=1}^{n}\mathbb{E}[\bm{X}_{i}^{2}]\big)}{\mathrm{e}^{\theta t}}\\
 & \leq \inf_{0<\theta<3/B}\frac{d\,\exp\big(g(\theta)v\big)}{\mathrm{e}^{\theta t}}
\end{align*}
Taking $\theta=\frac{t}{v+Bt/3}$ and simplifying the above expression, we establish matrix Bernstein

\end{frame}





\begin{frame}
\frametitle{Proof of Lemma \ref{lem:bound-matrix-CGF}} 

{\small

Define $f(x)=\frac{\mathrm{e}^{\theta x}-1-\theta x}{x^{2}}$, then
for any $\bm{X}$ with $\lambda_{\max}(\bm{X})\leq B$:
\begin{align*}
\mathrm{e}^{\theta\bm{X}} & =\bm{I}+\theta\bm{X}+\big(\mathrm{e}^{\theta\bm{X}}-\bm{I}-\theta\bm{X}\big)=\bm{I}+\theta\bm{X}+\bm{X}\cdot f(\bm{X})\cdot\bm{X}\\
 & \preceq\bm{I}+\theta\bm{X}+f(B)\cdot\bm{X}^{2}
\end{align*}
In addition, we note an elementary inequality: for any $0<\theta<3/B$,
%
\[
f(B)=\frac{\mathrm{e}^{\theta B}-1-\theta B}{B^{2}}=\frac{1}{B^{2}}\sum_{k=2}^{\infty}\frac{(\theta B)^{k}}{k!}\leq\frac{\theta^{2}}{2}\sum_{k=2}^{\infty}\frac{(\theta B)^{k-2}}{3^{k-2}}=\frac{\theta^{2}/2}{1-\theta B/3}
\]
%
\begin{align*}
	\Longrightarrow \qquad \mathrm{e}^{\theta\bm{X}} & \preceq\bm{I}+\theta\bm{X}+\frac{\theta^{2}/2}{1-\theta B/3}\cdot\bm{X}^{2}
\end{align*}
%
Since $\bm{X}$ is zero-mean, one further has
\[
\mathbb{E}\big[\mathrm{e}^{\theta\bm{X}}\big]\preceq\bm{I}+\frac{\theta^{2}/2}{1-\theta B/3}\mathbb{E}[\bm{X}^{2}]\preceq\exp\left(\frac{\theta^{2}/2}{1-\theta B/3}\mathbb{E}[\bm{X}^{2}]\right)
\]

}
Finish by observing $\log$ is monotone
\end{frame}





\begin{frame}
\frametitle{Appendix: asymptotic notation}

\begin{itemize}
   \itemsep1em
   \item  $f(n) \lesssim g(n)$ or $f(n) = O(g(n))$ means 
   \[
     \limsup_{n\rightarrow\infty} \frac{|f(n)|}{|g(n)|} ~\leq~ \text{const} 
   \]
   \item  $f(n) \gtrsim g(n)$ or $f(n) = \Omega(g(n))$ means 
   \[
     \liminf_{n\rightarrow\infty} \frac{|f(n)|}{|g(n)|} ~\geq~ \text{const} 
   \]
   \item  $f(n) \asymp g(n)$ or $f(n) = \Theta(g(n))$ means 
   \[
      f(n) \lesssim g(n) \quad \text{and} f(n) \gtrsim g(n)
   \]


   \item  $f(n) = o(g(n))$ means 
   \[
       \lim_{n\rightarrow\infty} \frac{|f(n)|}{|g(n)|} ~=~ 0
   \]

\end{itemize}

\end{frame}

%
%\begin{frame}
%\frametitle{Kernel trick} 
%
%A modern idea in machine learning:  replace the inner product by kernel evaluation (i.e.~certain similarity measure)
%
%\bigskip
%\bigskip
%\bigskip
%
%Advantage: work beyond the Euclidean domain via task-specific similarity measures
%
%\end{frame}
%
%
%
%\begin{frame}
%\frametitle{Similarity measure}  
%
%Define the similarity measure $\Phi$
%
%	\begin{itemize}
%		\itemsep0.5em
%		\item $\Phi(\bm{x}, \bm{x})=1$
%		\item $|\Phi(\bm{x}, \bm{y})| \leq 1$
%		\item $\Phi(\bm{x}, \bm{y}) = \Phi(\bm{y}, \bm{x})$
%	\end{itemize}
%
%\vfill
%
%Example: angular similarity
%%
%\[
%	\Phi(\bm{x},\bm{y})=\frac{2}{\pi}\arcsin\frac{\langle\bm{x},\bm{y}\rangle}{\|\bm{x}\|_{2}\|\bm{y}\|_{2}}=1-\frac{2\angle(\bm{x},\bm{y})}{\pi}
%\]
%%
%
%
%\end{frame}
%
%
%
%\begin{frame}
%\frametitle{Kernel matrix} 
%
%	Consider $N$ data points $\bm{x}_1,\cdots,\bm{x}_N\in \mathbb{R}^d$. Then the kernel matrix $\bm{G}\in \mathbb{R}^{N\times N}$ is 
%	%
%	\begin{equation*}
%		G_{i,j} = \Phi (\bm{x}_i, \bm{x}_j) \qquad 1\leq  i,j \leq N
%	\end{equation*}
%	%
%
%
%	\begin{itemize}
%		\item Kernel $\Phi$ is said to be positive semidefinite if $\bm{G}\succeq \bm{0}$ for any  $\{\bm{x}_i\}$
%	\end{itemize}
%
%
%	\vfill
%
%	{\bf Challenge:} kernel matrices are usually large 
%	\begin{itemize}
%		\itemsep0.3em
%		\item cost of constructing $\bm{G}$ is $O(dN^2)$ 
%	\end{itemize}
%
%	\medskip
%	{\bf Question:} can we approximate $\bm{G}$ more efficiently?
%
%\end{frame}
%
%
%\begin{frame}
%\frametitle{Random features}  
%
%	Introduce a random variable $\bm{w}$ and a feature map $\psi$ such that
%	%
%	\[
%		\Phi(\bm{x},\bm{y})=\mathbb{E}_{\bm{w}}[ \underset{\alertb{\text{decouple }\bm{x}\text{ and }\bm{y}}}{\underbrace{ \psi(\bm{x};\bm{w})\cdot\psi(\bm{y};\bm{w}) }} ]
%	\]
%	%
%	\begin{itemize}
%		\item<1> {\bf example (angular similarity)}
%		%
%		\begin{align}
%			\label{eq:angular-similarity-random}
%			\underset{\alertb{\text{Grothendieck's identity}}}{\underbrace{ \Phi(\bm{x},\bm{y})=1-\frac{2\angle(\bm{x},\bm{y})}{\pi}=\mathbb{E}_{\bm{w}}[\mathrm{sgn}\langle\bm{x},\bm{w}\rangle \cdot \mathrm{sgn}\langle\bm{y},\bm{w}\rangle] }}
%		\end{align}
%		%
%		with $\bm{w}$ uniformly drawn from the unit sphere
%
%
%		\vspace{-7em}
%		\item<2> this results in a {\bf random feature vector}
%		%
%		\[
%			\bm{z}=\left[\begin{array}{c}
%				z_{1}\\
%				\vdots\\
%				z_{N}
%			\end{array}\right]
%			=\left[\begin{array}{c}
%				\psi(\bm{x}_{1};\bm{w})\\
%				\vdots\\
%				\psi(\bm{x}_{N};\bm{w})
%			\end{array}\right]
%		\]
%		%
%		\begin{itemize}
%			\item $\underset{\alertb{\text{rank 1}}}{\underbrace{\bm{z}\bm{z}^{\top}}}$ is an \alert{unbiased estimate} of $\bm{G}$, i.e.~$\bm{G}=\mathbb{E}[\bm{z}\bm{z}^{\top}]$
%		\end{itemize}
%		
%	\end{itemize}
%
%
%\end{frame}
%
%
%\begin{frame}
%	\frametitle{Example}
%
%
%	Angular similarity:
%	\begin{align*}
%\Phi(\bm{x},\bm{y}) & =1-\frac{2\angle(\bm{x},\bm{y})}{\pi}\\
% & =\mathbb{E}_{\bm{w}}\left[\mathsf{sign}\langle\bm{x},\bm{w}\rangle\,\mathsf{sign}\langle\bm{y},\bm{w}\rangle\right]
%\end{align*}
%where $\bm{w}$ is uniformly drawn from the unit sphere
%
%\bigskip
%\bigskip
%
%	As a result, the random feature map is $\psi(\bm{x},\bm{w})=\mathsf{sign}\langle\bm{x},\bm{w}\rangle$
%
%	
%\end{frame}
%
%
%
%\begin{frame}
%\frametitle{Random feature approximation}  
%
%	Generate $n$ independent copies of $\bm{R}=\bm{z}\bm{z}^{\top}$, i.e.~$\{\bm{R}_l\}_{1\leq l\leq n}$
%
%	\bigskip
%	\bigskip
%	
%	{\bf Estimator} of the kernel matrix $\bm{G}$:
%	%
%	\[
%		\hat{\bm{G}} = \frac{1}{n} \sum_{l=1}^n \bm{R}_l
%	\]
%
%
%	\bigskip
%{
%\setbeamercolor{block body}{bg=babyblueeyes,fg=black}
%
%\begin{varblock}[\textwidth]{}
%\begin{center}
%	{\bf Question:} ~how many random features are needed to guarantee accurate estimation?
%\end{center}
%\end{varblock}
%}
%
%	
%\end{frame}
%
%
%
%
%\begin{frame}
%\frametitle{Statistical guarantees for random feature approximation}  
%
%Consider the angular similarity example \eqref{eq:angular-similarity-random}: 
%%
%\begin{itemize}
%	\itemsep0.5em
%		
%	\item To begin with,
%%
%\[
%	\mathbb{E}[\bm{R}_l^2]=\mathbb{E}[\bm{z}\bm{z}^{\top}\bm{z}\bm{z}^{\top}]=N\mathbb{E}[\bm{z}\bm{z}^{\top}]=N\bm{G}
%\]
%%
%\[
%	\Longrightarrow\qquad v= \Big\| \frac{1}{n^2} \sum\nolimits _{l=1}^{n}\mathbb{E}[\bm{R}_{l}^{2}] \Big\| = \frac{N}{n} \|\bm{G}\|
%\]
%%
%
%	\item Next, $ \frac{1}{n}\|\bm{R}\| = \frac{1}{n} \|\bm{z}\|_2^2 = \frac{N}{n} \eqqcolon B$
%
%	\item Applying the matrix Bernstein inequality yields: with high prob.
%	%
%	\begin{align*}
%		\|\hat{\bm{G}}-\bm{G}\| &\lesssim\sqrt{v\log N}+B\log N\lesssim  \sqrt{ \frac{N}{n} \|\bm{G}\| \log N } + \frac{N}{n}\log N  \\
%					&\lesssim  \sqrt{ \frac{N}{n} \underset{\alertb{\geq 1}}{\underbrace{ \|\bm{G}\| }} \log N }   \qquad \qquad \alertb{\text{(for sufficiently large }n)}
%	\end{align*}
%	%
%
%\end{itemize}
%
%\end{frame}
%
%
%
%\begin{frame}
%\frametitle{Sample complexity}
%
%	Define the intrinsic dimension of $\bm{G}$ as 
%	\[
%		\mathsf{intdim}(\bm{G})= \frac{\mathrm{tr} \bm{G}}{ \|\bm{G} \|} = \frac{N}{\|\bm{G}\|}
%	\]
%
%	If $n \gtrsim \varepsilon^{-2} \mathsf{intdim}(\bm{G}) \log N$, then we have
%	%
%	\[
%		\frac{\| \hat{\bm{G}} - \bm{G} \|}{\|\bm{G}\|} \leq \varepsilon
%	\]
%
%
%\end{frame}


%
%\begin{frame}[allowframebreaks]
%\frametitle{Reference}
%
%{\small
%\begin{itemize}  \itemsep0.3em
%  \item ''\textit{An introduction to matrix concentration inequalities},'' J.~Tropp, \textit{Foundations and Trends in Machine Learning}, 2015.
%  \item ''\textit{Convex trace functions and the Wigner-Yanase-Dyson conjecture},'' E.~Lieb, \textit{Advances in Mathematics}, 1973.
%  %\item ''\textit{Introduction to the non-asymptotic analysis of random matrices},'' R.~Vershinyn, \textit{arXiv:1011.3027}, 2010.
%  \item ''\textit{User-friendly tail bounds for sums of random matrices},'' J.~Tropp, \textit{Foundations of computational mathematics}, 2012.
%  \item ''\textit{Random features for large-scale kernel machines},'' A.~Rahimi, B.~Recht, \textit{Neural Information Processing Systems}, 2008.  
%
%
%\end{itemize}
%}
%
%
%\end{frame}



\end{document}

