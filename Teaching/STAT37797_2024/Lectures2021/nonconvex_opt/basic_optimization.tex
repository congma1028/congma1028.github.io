\pdfminorversion=4
\documentclass[compress,
mathserif,wide,%red,
%handout
]{beamer}

\usepackage{amsmath,amsfonts,bm,bbm,comment}
\usepackage{latexsym,tabularx,amscd}
\usepackage{etex} 
\usepackage{algorithm}
\usepackage[noend]{algorithmic}
\usepackage{subfigure}
\usepackage{url}
%\usepackage{multimedia}
\usepackage{hyperref}
\usepackage{skull}
%\usepackage{enumitem}
%\usepackage{movie15}

\usepackage{multirow}
\usepackage{cancel}

%\usepackage{ulem}
\usepackage{lmodern}
\usepackage{array} 
\usepackage{etoolbox}
\usepackage{caption}
\usepackage[normalem]{ulem}

\usepackage{mathtools}

\usepackage[space]{grffile}

\usepackage{tikz}
\usetikzlibrary{shapes.arrows}
\usetikzlibrary{arrows,positioning}
\usepackage{pgfplots}
%\usetikzlibrary{pgfplots.groupplots}

\tikzset{
    myarrow/.style={
        draw,
        fill=black,
        single arrow,
        minimum height=6ex,
        single arrow head extend=1ex
    }
}


\newcommand{\vertiii}[1]{{\left\vert\kern-0.25ex\left\vert\kern-0.25ex\left\vert #1 
    \right\vert\kern-0.25ex\right\vert\kern-0.25ex\right\vert}}

\newcommand{\vertiiibig}[1]{{\big\vert\kern-0.25ex\big\vert\kern-0.25ex\big\vert #1 
    \big\vert\kern-0.25ex\big\vert\kern-0.25ex\big\vert}}

\newcommand{\vertiiiplain}[1]{{\vert\kern-0.25ex\vert\kern-0.25ex\vert #1 
    \vert\kern-0.25ex\vert\kern-0.25ex\vert}}



\usepackage{dsfont}
\DeclareMathOperator{\ind}{\mathds{1}}  % Indicator


\newcommand{\abs}[1]{\left|#1\right|}

\mode<presentation>

\usetheme{Madrid}
% other themes: AnnArbor, Antibes, Bergen, Berkeley, Berlin, Boadilla,
% boxes, CambridgeUS, Copenhagen, Darmstadt, default, Dresden,
% Frankfurt, Goettingen, Hannover, Ilmenau, JuanLesPins, Luebeck,
% Madrid, Maloe, Marburg, Montpellier, PaloAlto, Pittsburg, Rochester,
% Singapore, Szeged, classic

% \usecolortheme{lily} 
% color themes: albatross, beaver, beetle, crane, default, dolphin,
% dov, fly, lily, orchid, rose, seagull, seahorse, sidebartab,
% structure, whale, wolverine

% \usefonttheme{serif} 
% font themes: default, professionalfonts, serif, structurebold,
% structureitalicserif, structuresmallcapsserif

%\usefonttheme[onlymath]{serif}
%\usefonttheme[onlylarge]{structuresmallcapsserif} 
  
%\useoutertheme[subsection=false]{smoothbars}
% outer themes:
% default,infolines,miniframes,smoothbars,sidebar,split,shadow,tree,smoothtree
%\useinnertheme{rectangles}
\beamertemplatenavigationsymbolsempty

% \definecolor{fore}{RGB}{249,242,215}
% \definecolor{back}{RGB}{51,51,51}
% \definecolor{title}{RGB}{255,0,90}
% \setbeamercolor{titlelike}{fg=title}
% \setbeamercolor{normal text}{fg=fore,bg=back}
% \setbeamercovered{transparent}
  % or whatever (possibly just delete it)
%\beamertemplatenavigationsymbolsempty

\usepackage[english]{babel}
\usepackage[latin1]{inputenc}
% or whatever
%\usepackage{comicsans} 
%\renewcommand{\sfdefault}{comic} 
%\usepackage{mathptmx}
%\mathversion{bold}
%\usepackage{helvet}
%\usepackage{courier}

%\usepackage[T1]{fontenc}
% Or whatever. Note that the encoding and the font should match. If T1
% does not look nice, try deleting the line with the fontenc.


%%%% Figures %%%%%


\newcommand{\LectureFigs}{../Figures}


\definecolor{bluegray}{rgb}{0.15,0.20,0.40}
\definecolor{bluegraylight}{rgb}{0.35,0.40,0.60}
\definecolor{gray}{rgb}{0.3,0.3,0.3}
\definecolor{lightgray}{rgb}{0.7,0.7,0.7}
\definecolor{brightblue}{rgb}{0.1,0.4,0.7}
\definecolor{darkblue}{rgb}{0.1,0.1,0.6}
\definecolor{darkgreen}{rgb}{0.0,0.5,0.2}
\definecolor{orange}{rgb}{0.7,0.3,0}

\def \captioncolor { }

\setbeamertemplate{itemize item}{{\color{black}$\bullet$}}
\setbeamertemplate{itemize subitem}{{\color{black}$\circ$}}
\setbeamertemplate{frametitle}[default][center]
\setbeamerfont{frametitle}{series=\bfseries}
\addtobeamertemplate{frametitle}{\vskip1ex}{ \vspace{0.5em} \hrule}
\setbeamercolor{frametitle}{fg=,bg=}

\setbeamertemplate{theorems}[numbered]
\setbeamertemplate{caption}[numbered]
\setbeamertemplate{algorithm}[numbered]

\numberwithin{theorem}{subsection}
\numberwithin{equation}{subsection}
\numberwithin{figure}{subsection}
\numberwithin{algorithm}{subsection}


%\setbeamerfont{page number in head/foot}{size=\scriptsize}
\setbeamertemplate{footline}[frame number]
\setbeamercolor{page number in head/foot}{fg=black}
\setbeamersize{text margin left=2em,text margin right=2em}

\setbeamertemplate{footline}
{
    %\hspace{3em} \insertshorttitle \hfill \insertframenumber / \inserttotalframenumber\hspace{3em}
    \hspace{3em} \insertshorttitle \hfill \insertsubsectionnumber-\insertframenumber \hspace{3em}
    \vspace{1.5em}
}

\definecolor{babyblueeyes}{rgb}{0.63, 0.79, 0.95}
\definecolor{bleudefrance}{rgb}{0.19, 0.55, 0.91}
\definecolor{matlabPlotBlue}{rgb}{0.25, 0.58, 0.8}
\definecolor{matlabPlotOrange}{rgb}{0.855, 0.324, 0.098}



\captionsetup[figure]{labelfont={color=black}}

\AtBeginEnvironment{lemma}{%
  \setbeamercolor{block title}{fg=black,bg=babyblueeyes}
  \setbeamercolor{block body}{fg=black,bg=}
  \setbeamerfont*{block title}{series=\bfseries}
}

\AtBeginEnvironment{theorem}{%
  \setbeamercolor{block title}{fg=black,bg=babyblueeyes}
  \setbeamercolor{block body}{fg=black,bg=}
  \setbeamerfont*{block title}{series=\bfseries}
}

\AtBeginEnvironment{corollary}{%
  \setbeamercolor{block title}{fg=black,bg=babyblueeyes}
  \setbeamercolor{block body}{fg=black,bg=}
  \setbeamerfont*{block title}{series=\bfseries}
}

\AtBeginEnvironment{fact}{%
  \setbeamercolor{block title}{fg=black,bg=babyblueeyes}
  \setbeamercolor{block body}{fg=black,bg=}
  \setbeamerfont*{block title}{series=\bfseries}
}

\AtBeginEnvironment{definition}{%
  \setbeamercolor{block title}{fg=black,bg=babyblueeyes}
  \setbeamercolor{block body}{fg=black,bg=}
  \setbeamerfont*{block title}{series=\bfseries}
}

\newenvironment<>{varblock}[2][.9\textwidth]{%
  \vspace{-1em}
  \begin{center}
  \begin{minipage}{#1}
  %\setlength{\textwidth}{#1}
  \begin{actionenv}#3%
    \def\insertblocktitle{#2}%
    \par%
    \usebeamertemplate{block begin}}
  {\par%
    \usebeamertemplate{block end}%
  \end{actionenv}
  \end{minipage}
  \end{center}
}


\newcommand\blankfootnote[1]{%
  \let\thefootnote\relax\footnotetext{#1}%
  \let\thefootnote\svthefootnote%
}


%%%%%
\newcommand{\bb}{\bm b}
\newcommand{\bZ}{\bm Z}
\newcommand{\bC}{\bm C}
\newcommand{\cU}{{\cal U}}
\newcommand{\bR}{{\bm R}}
\newcommand{\bi}{{\bm i}}
\newcommand{\bS}{{\bm S}}
\newcommand{\cE}{{\cal E}}
\newcommand{\cF}{{\cal F}}
\newcommand{\bk}{{\bm k}}
\newcommand{\bu}{{\bm u}}
\newcommand{\bx}{{\bm x}}
\newcommand{\bA}{{\bm A}}
\newcommand{\bh}{{\bm h}}
\newcommand{\by}{{\bm y}}
\newcommand{\bI}{{\bm I}}
\newcommand{\ba}{{\bm a}}
\newcommand{\bw}{{\bm w}}
\newcommand{\bPhi}{{\bm \Phi}}
\newcommand{\bX}{{\bm X}}
\newcommand{\bY}{{\bm Y}}
\newcommand{\bM}{{\bm M}}
\newcommand{\be}{{\bm e}}
\newcommand{\bs}{{\bm s}}
\newcommand{\bB}{{\bm B}}
\newcommand{\bW}{{\bm W}}
\newcommand{\bN}{{\bm N}}
\newcommand{\bU}{{\bm U}}
\newcommand{\bV}{{\bm V}}
\newcommand{\bP}{{\bm P}}
\newcommand{\bH}{{\bm H}}
\newcommand{\bSigma}{{\bm \Sigma}}
\newcommand{\bL}{{\bm L}}



\newcommand{\bv}{\bm v}
\newcommand{\eps}{\epsilon}
\newcommand{\vf}{\varphi}
\newcommand{\de}{\delta}
\newcommand{\<}{\langle}
\renewcommand{\>}{\rangle}
\newcommand{\goto}{\rightarrow}
\newcommand{\argmin}{\mbox{argmin}}
\newcommand{\argmax}{\mbox{argmax}}
\newcommand{\Ave}{\mathop{\rm Ave}\nolimits}
\newcommand{\sgn}{\mbox{sgn}}
\newcommand{\cN}{\mathcal{N}}
\newcommand{\cR}{\mathcal{R}}

\def\C{{\mathbb{C}}}
\def\R{{{\mathbb{R}}}} %seemed too large in subscripts, so I changed --JW 
\def\P{{\hbox{\bf P}}}
\def\Z{{{\mathbb{Z}}}}
\def\T{{\hbox{\bf T}}}


\newcommand{\cA}{\mathcal{A}}
\newcommand{\cP}{\mathcal{P}}
\newcommand{\cD}{\mathcal{D}}
\newcommand{\cL}{\mathcal{L}}
\newcommand{\shrink}{\text{shrink}}


\newcommand{\Obs}{\Omega_{\text{obs}}}


\renewcommand{\P}{\operatorname{\mathbb{P}}}
\newcommand{\E}{\operatorname{\mathbb{E}}}

% Linear algebra macros
%\newcommand{\vct}[1]{#1}
%\newcommand{\mtx}[1]{#1}
\newcommand{\vct}[1]{\bm{#1}}
\newcommand{\mtx}[1]{\bm{#1}}
%\newcommand{\mtx}[1]{\mathsfsl{#1}}

\newcommand{\transp}{T}
\newcommand{\adj}{*}
\newcommand{\psinv}{\dagger}
\newcommand{\lspan}[1]{\operatorname{span}{#1}}

\newcommand{\range}{\operatorname{range}}
\newcommand{\colspan}{\operatorname{colspan}}

\newcommand{\rank}{\text{rank}}

\newcommand{\diag}{\operatorname{diag}}
\newcommand{\trace}{\text{Tr}}

\newcommand{\supp}[1]{\operatorname{supp}(#1)}

\newcommand{\smax}{\sigma_{\max}}
\newcommand{\smin}{\sigma_{\min}}

\newcommand{\restrict}[1]{\big\vert_{#1}}

\newcommand{\Id}{\text{\em I}}
\newcommand{\OpId}{\mathcal{I}}



% define your own colors:
\definecolor{Red}{rgb}{1,0,0}
\definecolor{Blue}{rgb}{0,0,1}
\definecolor{Green}{rgb}{0,1,0}
\definecolor{magenta}{rgb}{1,0,.6}
\definecolor{lightblue}{rgb}{0,.5,1}
\definecolor{lightpurple}{rgb}{.6,.4,1}
\definecolor{gold}{rgb}{.6,.5,0}
\definecolor{orange}{rgb}{1,0.4,0}
\definecolor{hotpink}{rgb}{1,0,0.5}
\definecolor{newcolor2}{rgb}{.5,.3,.5}
\definecolor{newcolor}{rgb}{0,.3,1}
\definecolor{newcolor3}{rgb}{1,0,.35}
\definecolor{darkgreen1}{rgb}{0, .35, 0}
\definecolor{darkgreen}{rgb}{0, .6, 0}
\definecolor{darkred}{rgb}{.75,0,0}

\xdefinecolor{olive}{cmyk}{0.64,0,0.95,0.4}
\xdefinecolor{purpleish}{cmyk}{0.75,0.75,0,0}

\definecolor{lightgrey}{RGB}{186,186,186}
\definecolor{back}{RGB}{51,51,51}
\definecolor{fore}{RGB}{249,242,215}
\definecolor{back}{RGB}{51,51,51}
\definecolor{title}{RGB}{255,0,20}
\definecolor{keywords}{RGB}{255,0,90}
\definecolor{comments}{RGB}{60,179,113}
\definecolor{cobalt}{RGB}{102,153,204}

\def \itemstyle {\bf}
\def \itemhighlight {\bf}  

% \definecolor{blockbodycolor}{RGB}{30,30,40}
% \definecolor{blocktitlecolor}{RGB}{55,55,75}
\definecolor{blockbodycolor}{RGB}{186,206,232}
\definecolor{blocktitlecolor}{RGB}{55,55,75}

\setbeamercolor{titlelike}{fg=darkred,bg=white}
\setbeamercolor{normal text}{fg=black,bg=white}
\setbeamercolor{block title}{fg=white,bg=blocktitlecolor}
\setbeamercolor{block body}{bg=blockbodycolor}
\setbeamercolor{alerted text}{fg=darkred}
\setbeamercolor{title}{fg=darkred,bg=lightgrey}


\newcommand{\alertb}[1]{\textcolor{blue}{ #1}}

\newcommand{\yp}[1]{\textcolor{red}{ #1}}
\newcommand{\zeronorm}[1]{\left\|#1 \right\|_0}
\newcommand{\unorm}[1]{\left\|#1 \right\|_u}
\newcommand{\ynorm}[1]{\left\|#1 \right\|_{\bar{y}}}
\newcommand{\onetwonorm}[1]{\left\|#1\right\|_{1,2}}
\newcommand{\opnorm}[1]{\left\|#1\right\|}
\newcommand{\fronorm}[1]{\left\|#1\right\|_{F}}
\newcommand{\onenorm}[1]{\left\|#1\right\|_{\ell_1}}
\newcommand{\twonorm}[1]{\left\|#1\right\|_{2}}
\newcommand{\oneinfnorm}[1]{\left\|#1\right\|_{1,\infty}}
\newcommand{\infnorm}[1]{\left\|#1\right\|_{\ell_\infty}}
\newcommand{\nucnorm}[1]{\left\|#1\right\|_*}

%\makeatletter
%\def\th@mystyle{%
%    \normalfont % body font
%    \setbeamercolor{block title example}{bg=,fg=black}
%    \setbeamerfont{block title example}{series=\bfseries}
%    \setbeamercolor{block body example}{bg=,fg=black}
%    %\setbeamertemplate{exampleblock}[numbered]
%    \def\inserttheoremblockenv{exampleblock}
%  }
%\makeatother
%\theoremstyle{mystyle}
%\newtheorem*{lem}{Lemma}

\setbeamertemplate{frametitle continuation}{}


\newcommand{\courseTitle}{STAT 37797: Mathematics of Data Science} 



% \subtitle
% {Include Only If Paper Has a Subtitle}

\vspace{.5cm}

\author{Yuxin Chen}

\institute[Princeton] % (optional, but mostly needed)
{Princeton University}


% - Use the \inst command only if there are several affiliations.
% - Keep it simple, no one is interested in your street address.


\subject{Beamer}
% This is only inserted into the PDF information catalog. Can be left
% out.

% Delete this, if you do not want the table of contents to pop up at
% the beginning of each subsection:
% \AtBeginSubsection[]
% {
%   \begin{frame}<beamer>
%     \frametitle{Outline}
%     \tableofcontents[currentsection,currentsubsection]
%   \end{frame}
% }






\graphicspath{{../../../Figures/}}


\title % (optional, use only with long paper titles)
{Introduction to nonconvex optimization}

\defbeamertemplate*{title page}{customized}[1][]
{

  \hfill {\em \courseTitle}

  \begin{center}
    \vspace{2.5em}
    \usebeamerfont{title} {\Large\bf\inserttitle} \par
  
    \vspace{1.5em}
    \includegraphics[width=2cm]{\LectureFigs/UC_logo.png} 
  
    \vspace{1em}
    {\large Cong Ma \par }

    \vspace{0.2em}
    { \large \quad University of Chicago, Autumn 2021 }
  \end{center}

  \vfill
}

\setcounter{subsection}{7}

\begin{document}


\begin{frame}[plain]
  \titlepage

\end{frame}

\begin{frame}
	\frametitle{Unconstrained optimization}
	Consider an unconstrained optimization problem
\[
	\text{minimize}_{\bm{x}}\qquad f(\bm{x})
\]

\vfill

\begin{itemize}
\item For simplicity, we assume $f(\bm{x})$ is twice differentiable
\item We assume the minimizer $\bm{x}_{\mathsf{opt}}$ exists, i.e., 
\[
\bm{x}_{\mathsf{opt}} \coloneqq \underset{\bm{x}}{\arg\min} \; f(\bm{x})
\]
\end{itemize}

\end{frame}

\begin{frame}
	\frametitle{Critical/stationary points}
	\begin{definition}
A first-order critical point  of $f$ satisfies
	$$ \nabla f(\bx) = \bm{0} $$ 
\end{definition} 

\vfill

\begin{itemize}
	\item If $f$ is convex, any $1$st-order critical point is a global minimizer
	\item Finding $1$st-order stationary point is sufficient for convex optimization
	\item Example: gradient descent (GD)
\end{itemize}
\end{frame}


\begin{frame}
	\frametitle{How about nonconvex optimization?}
	First-order critical points could be global min, local min, local max, saddle points...
	
	\vfill
	 \begin{center}
\includegraphics[width=\textwidth]{critical_points}
\end{center}
\hfill {\em\footnotesize figure credit: Li et al.\,'16}

\vfill
{
\setbeamercolor{block body}{bg=babyblueeyes,fg=black}

\begin{varblock}[\textwidth]{}
Simple algorithms like GD could stuck at undesired stationary points
\end{varblock}
}


\end{frame}




\frame{
\frametitle{Types of critical points}

\begin{definition}
	\label{def:2nd-criticals}
	A second-order critical point $\bm{x}$ satisfies  
	\[
		\nabla f(\bx)= \bm{0} \quad \text{and} \quad \nabla^2 f(\bx) \succeq \bm{0}
	\]
\end{definition}

\vfill
For any first-order critical point $\bx$:
\begin{itemize}
	\itemsep0.1em
	\item $ \nabla^2 f(\bx) \prec \bm{0}$ \qquad\qquad $\rightarrow$ \quad   local maximum
	\item $ \nabla^2 f(\bx) \succ \bm{0}$ \qquad\qquad $\rightarrow$ \quad   local minimum
	%\item $\lambda_{\min}(\nabla^2 f(\bx))=0$, \quad $\rightarrow$ \quad    local minimum or degenerate saddle;
	\item $\lambda_{\min}(\nabla^2 f(\bx))<0$ \qquad $\rightarrow$ \quad   {\em strict} saddle point
%\footnote{Note that a local maximum may also belong to the set of  strict saddle points. }
\end{itemize}




}
	
%\begin{frame}
%\frametitle{Outline}
%
%\begin{itemize}
%  \itemsep1em
%  \item Algorithms for optimization
%  \item Geometry of optimization problems
%  \item An example
%\end{itemize}
%
%\end{frame}
%	
%
%\begin{frame}
%	\frametitle{Convex vs. nonconvex optimization}
%	\begin{itemize}
%		\item XX
%	\end{itemize}
%\end{frame}

\begin{frame}
	
\vfill

\begin{center}
	{\Large \bf When are nonconvex problems solvable?}
\end{center}


%\centering
%{\large This talk:  a case study --- phase retrieval}

\vfill


\end{frame}

\begin{frame}
	\frametitle{(Local) strong convexity and smoothness}
	\begin{definition}
		A twice differentiable function $f : \mathbb{R}^{n} \mapsto \mathbb{R}$ is said to be $\alpha$-strongly convex in a set $\mathcal{B}$ if for all $\bm{x} \in \mathcal{B}$
		\[
		\nabla^2 f(\bm{x}) \succeq \alpha \bm{I}_{n}.
		\]
	\end{definition}
	
	\vfill
	\begin{definition}
		A twice differentiable function $f : \mathbb{R}^{n} \mapsto \mathbb{R}$ is said to be $\beta$-smooth in a set $\mathcal{B}$ if for all $\bm{x} \in \mathcal{B}$
		\[
		\| \nabla^2 f(\bm{x}) \| \leq \beta.
		\]
	\end{definition}
\end{frame}


\begin{frame}
	\frametitle{Gradient descent theory revisited}
	Gradient descent method with step size $\eta > 0$
	\[
		\bm{x}^{t+1}=\bm{x}^t - \eta \nabla f(\bm{x}^t) 
	\]
	
	\begin{lemma}\label{lemma:GD}
		Suppose $f$ is $\alpha$-strongly convex and $\beta$-smooth in the local ball $\mathcal{B}_{\delta}(\bm{x}_{\mathsf{opt}}) \coloneqq \{\bm{x} \mid \|\bm{x} - \bm{x}_{\mathsf{opt}}\|_2 \leq \delta \}$. Running gradient descent from $\bm{x}^{0} \in \mathcal{B}_{\delta}(\bm{x}_{\mathsf{opt}})$ with $\eta = 1 / \beta$ achieves linear convergence
		\[
		\|\bm{x}^{t} -\bm{x}_{\mathsf{opt}}\|_2 \leq \left(1- {\frac{\alpha}{\beta}}  
\right)^{t}\|\bm{x}^{0} -\bm{x}_{\mathsf{opt}}\|_2, \quad t = 0, 1, 2, \ldots
		\]
	\end{lemma}	
	

\end{frame}

\begin{frame}
	\frametitle{Implications}
		\begin{itemize}
	\itemsep0.5em
	\item Condition number $\alert{\beta / \alpha}$ determines rate of convergence
	\item Attains $\varepsilon$-accuracy (i.e., $\|\bm{x}^{t} -\bm{x}_{\mathsf{opt}}\|_2 \leq \varepsilon \|\bm{x}_{\mathsf{opt}}\|_2$) within 
	\[
	O\left( \alert{ \frac{\beta}{\alpha} } \log\frac{1}{\varepsilon} \right)
	\]
	 iterations
	 \item Needs initialization $\bm{x}^{0} \in \mathcal{B}_{\delta}(\bm{x}_{\mathsf{opt}})$: basin of attraction
\end{itemize}
\end{frame}

\begin{frame}
	\frametitle{Proof of Lemma~\ref{lemma:GD}}
	Since $\nabla f(\bm{x}_{\mathsf{opt}}) = \bm{0}$, we can rewrite GD as
	\begin{align*}
		\bm{x}^{t+1} - \bm{x}_{\mathsf{opt}} &= \bm{x}^t - \eta \nabla f(\bm{x}^t) - [\bm{x}_{\mathsf{opt}} - \eta \nabla f(\bm{x}_{\mathsf{opt}})] \\
		&= \left[ \bm{I}_{n} - \eta \int_{0}^{1} \nabla^2 f (\bm{x}(\tau)) \mathsf{d}\tau \right ] (\bm{x}^t - \bm{x}_{\mathsf{opt}}),
	\end{align*}
	where $\bm{x}(\tau) \coloneqq \bm{x}_{\mathsf{opt}} + \tau (\bm{x}^t - \bm{x}_{\mathsf{opt}})$. By local strong convexity and smoothness, one has
	\[
	\alpha \bm{I}_{n} \preceq \nabla^2 f (\bm{x}(\tau)) \preceq \beta \bm{I}_{n}, \qquad \text{for all } 0 \leq \tau \leq 1
	\]
	Therefore $\eta = 1 / \beta$ yields
	\[
	\bm{0} \preceq \bm{I}_{n} - \eta \int_{0}^{1} \nabla^2 f (\bm{x}(\tau)) \mathsf{d}\tau \preceq (1 - \frac{\alpha}{\beta}) \bm{I}_{n},
	\]
	which further implies 
	\[
	\|\bm{x}^{t+1} -\bm{x}_{\mathsf{opt}}\|_2 \leq \left(1- {\frac{\alpha}{\beta}}  
\right)\|\bm{x}^{t} -\bm{x}_{\mathsf{opt}}\|_2
	\]
\end{frame}



\begin{frame}


\frametitle{Regularity condition}
More generally, for update rule
\[
\bm{x}^{t+1} = \bm{x}^t - \eta \bm{g}(\bm{x}^t),
\]
where $g(\cdot) : \mathbb{R}^{n} \mapsto \mathbb{R}^{n}$

\vfill
\begin{definition}
\label{def:reg-condition}
 $\bm{g}(\cdot)$ is said to obey $\mathsf{RC}( \mu,\lambda, \delta )$ for some $ \mu,\lambda, \delta >0 $ if 
\[
2 \langle\, \bm{g}(\bm{x}), \bm{x}-   \bm{x}_{\mathsf{opt}} \rangle\geq {\mu}\| \bm{g}(\bm{x})\|^2_2 + \lambda \left\| \bm{x}-   \bm{x}_{\mathsf{opt}} \right\|_2^2 \quad \forall \bm{x} \in \mathcal{B}_{\delta}(\bm{x}_{\mathsf{opt}})
\]
\end{definition}

\begin{itemize}
	\item Negative search direction $\bm{g}$ is positively correlated with error $\bm{x}-   \bm{x}_{\mathsf{opt}}$ $\Longrightarrow$ one-step improvement
	\item $\mu \lambda \leq 1$ by Cauchy-Schwarz
\end{itemize}
\end{frame}


\begin{frame}

\frametitle{RC = one-point strong convexity + smoothness}


\vspace{-4em}
\uncover<1>{

\begin{itemize}
	\item One-point $\alpha$-strong convexity: 
		\begin{align}
			\label{eq:1point-cvx}
			f(\bm{x}_{\mathsf{opt}}) - f(\bm{x}) \geq \langle \nabla f(\bm{x}), \bm{x}_{\mathsf{opt} }  - \bm{x}  \rangle + \frac{\alpha}{2} \| \bm{x} - \bm{x}_{\mathsf{opt}} \|_2^2 
		\end{align}

	\bigskip
		
	\item $\beta$-smoothness: 
\begin{align}
f(\bm{x}_{\mathsf{opt}})- f(\bm{x}) & \leq f \Big(\bm{x} - \frac{1}{\beta}\nabla f(\bm{x}) \Big)- f(\bm{x})  \nonumber \\
& \leq \Big\langle \nabla f(\bm{x}), - \frac{1}{\beta}\nabla f(\bm{x}) \Big\rangle + \frac{\beta}{2} \Big\| \frac{1}{\beta}\nabla f(\bm{x}) \Big\|_2^2 \nonumber \\
	&= -\frac{1}{2\beta} \left\| \nabla f(\bm{x}) \right\|_2^2  \label{eq:smoothness-grad}
\end{align}


\end{itemize}
}


\uncover<2>{

\vspace{-13em}

Combining relations~\eqref{eq:1point-cvx} and \eqref{eq:smoothness-grad} yields

\begin{align*}
	\langle \nabla f(\bm{x}), \bm{x} - \bm{x}_{\mathsf{opt} }    \rangle  \geq \frac{\alpha}{2} \| \bm{x} - \bm{x}_{\mathsf{opt}} \|_2^2 + \frac{1}{2\beta} \left\| \nabla f(\bm{x}) \right\|_2^2  
\end{align*}

\hfill --- \alert{\em RC holds with $\mu= 1/\beta$ and $\lambda = \alpha$}
}


 \end{frame}

\begin{frame}
	\frametitle{Example of nonconvex functions}
	When $\bm{g} (\bm{x}) = \nabla f (\bm{x})$, $f$ is not necessarily convex
	\begin{center}
	\includegraphics[width=0.5\textwidth]{regularity-example.pdf}
\end{center}

\[
f(x)=\begin{cases}
x^{2}, & |x|\leq6,\\
x^{2}+1.5|x|(\cos(|x|-6)-1), & |x|>6
\end{cases}
\]
\end{frame}

\begin{frame}

\frametitle{Convergence  under RC}

 \begin{lemma}\label{lemma:RC}
 Suppose $\bm{g}(\cdot)$ obeys $\mathsf{RC}( \mu,\lambda, \delta )$. The update rule ($\bm{x}^{t+1} = \bm{x}^t - \eta \bm{g}(\bm{x}^t)$) with $\eta = \mu$  and $\bm{x}^{0} \in \mathcal{B}_{\delta}(\bm{x}_{\mathsf{opt}})$ obeys
	\[
		\|\bm{x}^{t} -\bm{x}_{\mathsf{opt}} \|_2^{2} \leq \left(1- \only<1->{\alert{\mu\lambda}} \right)^{t} \|\bm{x}^{0} -\bm{x}_{\mathsf{opt}} \|_2^{2}
	\]
 \end{lemma}


\vfill
\begin{itemize}
	\itemsep0.5em
	\item $\bm{g}(\cdot)$: more general search directions
	\begin{itemize}
		\item example: in vanilla GD, $\bm{g}(\bx)=\nabla f(\bx)$
	\end{itemize}
	\item The product $\alert{\mu\lambda}$ determines the rate of convergence
	\item Attains $\varepsilon$-accuracy within $O\big( \alert{ \alert{\frac{1}{\mu\lambda}}} \log\frac{1}{\varepsilon} \big)$ iterations
\end{itemize}


\end{frame}


\begin{frame}
	\frametitle{Proof of Lemma~\ref{lemma:RC}}
	By definition, one has
	\begin{align*}
&\|\bm{x}^{t+1}-\bm{x}_{\mathsf{opt}}\|_{2}^{2} =\|\bm{x}^{t}-\eta\bm{g}(\bm{x}^{t})-\bm{x}_{\mathsf{opt}}\|_{2}^{2}\\
 &\quad =\|\bm{x}^{t}-\bm{x}_{\mathsf{opt}}\|_{2}^{2}+\eta^{2}\|\bm{g}(\bm{x}^{t})\|_{2}^{2}-2\eta\left\langle \bm{g}(\bm{x}^{t}),\bm{x}^{t}-\bm{x}_{\mathsf{opt}}\right\rangle \\
 &\quad \leq\|\bm{x}^{t}-\bm{x}_{\mathsf{opt}}\|_{2}^{2}+\eta^{2}\|\bm{g}(\bm{x}^{t})\|_{2}^{2}-\eta\left(\lambda\|\bm{x}^{t}-\bm{x}_{\mathsf{opt}}\|_{2}^{2}+\mu\|\bm{g}(\bm{x}^{t})\|_{2}^{2}\right)\\
 &\quad =(1-\eta\lambda)\|\bm{x}^{t}-\bm{x}_{\mathsf{opt}}\|_{2}^{2}+\eta(\eta-\mu)\|\bm{g}(\bm{x}^{t})\|_{2}^{2}\\
 &\quad \leq(1-\mu \lambda)\|\bm{x}^{t}-\bm{x}_{\mathsf{opt}}\|_{2}^{2}
\end{align*}
\end{frame}





\begin{frame}
	
\vfill

\begin{center}
	{\Large \bf A toy example: rank-1 matrix factorization}
\end{center}


%\centering
%{\large This talk:  a case study --- phase retrieval}

\vfill


\end{frame}


\begin{frame}
\frametitle{Principal component analysis}


 \begin{center}
\includegraphics[width=0.5\textwidth]{PCA3D.png}
\end{center}


\uncover<1>{
	Given $\bm{M}\succeq \bm{0} \in\mathbb{R}^{n\times n}$ (not necessarily low-rank), find its best rank-$r$ approximation:
$$ \underset{\alert{\text{nonconvex optimization!}}}{\underbrace{ \widehat{\bm{M}} = \text{argmin}_{\bm{Z}}\; \| \bm{Z}-\bm{M} \|_{\mathrm{F}}^2 \quad \mbox{s.t.} \quad \mbox{rank}(\bm{Z})\leq r }} $$
}
%this is a nonconvex optimization problem.

\uncover<2>{
	\vspace{-8em}
This problem admits a closed-form solution 
	%(\alertb{Eckart-Young theorem})
\begin{itemize}
\item let $\bM =\sum_{i=1}^n \lambda_i \bm{u}_i\bm{u}_i^{\top}$ be eigen-decomposition of $\bm{M}$  ($\lambda_1\geq \cdots \lambda_{r} > \lambda_{r+1} \geq \lambda_n$), then 
$$\widehat{\bm{M}} = \sum_{i=1}^r \lambda_i \bm{u}_i\bm{u}_i^{\top}$$
\end{itemize}

\hfill --- \alert{\em nonconvex, but tractable}

}

\end{frame}


 

\frame{

\frametitle{Optimization viewpoint}

If we factorize $\bZ =\bX\bX^{\top}$ with $\bX\in\mathbb{R}^{n\times r}$, then it leads to a nonconvex problem:
$$ \text{minimize}_{\bX\in \mathbb{R}^{n\times r}}\quad  f(\bX) = \frac{1}{4} \| \bm{X}\bm{X}^{\top}-\bm{M} \|_{\mathrm{F}}^2$$


\medskip
\bigskip

 
To simplify exposition,  set $r=1$:
$$ \text{minimize}_{\bx}\quad f(\bx) =\frac{1}{4} \| \bm{x}\bm{x}^{\top}-\bm{M} \|_{\mathrm{F}}^2$$



}




\frame{

\frametitle{Interesting questions}

$$ \text{minimize}_{\bx \in \mathbb{R}^{n}}\quad f(\bx) =\frac{1}{4} \| \bm{x}\bm{x}^{\top}-\bm{M} \|_{\mathrm{F}}^2$$


\bigskip
\bigskip

	\begin{itemize}
		\itemsep0.5em
		\item What does the curvature behave like, at least locally around the global minimizer? 
		%\item Is carefully-designed initialization necessary for fast convergence? 
		\item Where\,/\,what are the critical points? (Global geometry) 
		
	\end{itemize}


}

\begin{frame}
	\frametitle{Local linear convergence of GD}
	\begin{theorem}\label{thm:PCA}
	Suppose that $\|\bm{x}_{0} - \sqrt{\lambda_1} \bm{u}_{1} \|_{2} \leq \frac{\lambda_{1}-\lambda_{2}}{15\sqrt{\lambda_{1}}}$ and set $\eta = \frac{1}{ 4.5 \lambda_{1} }$, GD obeys
	\[
	\big\|\bm{x}^{t}-\sqrt{\lambda_{1}}\bm{u}_{1}\big\|_{2}\leq\left(1-\frac{\lambda_{1}-\lambda_{2}}{18\lambda_{1}}\right)^{t}\big\|\bm{x}^{0}-\sqrt{\lambda_{1}}\bm{u}_{1}\big\|_{2}, ~~ t\geq 0,
\]
	\end{theorem}
	
	\vfill
	
	\begin{itemize}
		\item condition number/eigengap determines rate of convergence
		\item Requires initialization: use spectral method?
	\end{itemize}
\end{frame}

\begin{frame}
	\frametitle{Proof of Theorem~\ref{thm:PCA}}
	It suffices to show that for all $\bm{x}$ obeying $\underbrace{\|\bm{x} - \sqrt{\lambda_1} \bm{u}_{1} \|_2 \leq \frac{\lambda_{1}-\lambda_{2}}{15\sqrt{\lambda_{1}}}}_{\text{\alert{basin of attraction}}}$, 
\[
	0.25(\lambda_{1}-\lambda_{2})\bm{I}_n \preceq \nabla^2 f(\bm{x})\preceq4.5\lambda_1 \bm{I}_n
\]

	\vfill
	Express gradient and Hessian as
	\begin{align*}
		\nabla f(\bx) &= (\bx\bx^{\top}-\bM) \bx \\
		\nabla^2 f(\bx) &= 2\bx\bx^{\top} + \|\bm{x}\|_2^2 \bm{I}_{n} - \bM
	\end{align*}
	
\end{frame}

\begin{frame}
	\frametitle{Preliminary facts}
	Let $\bm{\Delta} \coloneqq \bm{x} - \sqrt{\lambda_1} \bm{u}_{1}$. It is seen that when $\| \bm{\Delta} \|_{2} \leq \frac{\lambda_{1}-\lambda_{2}}{15\sqrt{\lambda_{1}}}$, one has
	\begin{align*}
	\lambda_1 - 0.25 (\lambda_1 - \lambda_2) &\leq \|\bm{x}\|_2^2 \leq 1.15 \lambda_{1}; \\
	\| \bm{\Delta} \|_{2} &\leq \| \bm{x} \|_2; \\
	\| \bm{\Delta} \|_{2} \|\bm{x} \|_2 &\leq (\lambda_{1} - \lambda_{2}) / 12
	\end{align*}
\end{frame}

\begin{frame}
	\frametitle{Local smoothness}
	Triangle inequality gives
	\begin{align*}
		\| \nabla^{2} f (\bm{x}) \| &\leq \| 2\bx\bx^{\top} \| +  \|\bm{x}\|_2^2 + \| \bm{M} \| \\
		& \leq 3 \|\bm{x}\|_2^2 + \lambda_{1} < 4.5 \lambda_{1},
	\end{align*}
	where the last relation follows from $\|\bm{x} \|_2^2 \leq 1.15 \lambda_{1}$
\end{frame}

\begin{frame}
	\frametitle{Local strong convexity}
	Recall that $\bm{\Delta} = \bm{x} - \sqrt{\lambda_1} \bm{u}_{1}$
	\begin{align*}
		\bm{x} \bm{x}^{\top} &= \lambda_{1} \bm{u}_1 \bm{u}_1^\top + \bm{\Delta} \bm{x}^\top 
		+ \bm{x} \bm{\Delta}^\top - \bm{\Delta} \bm{\Delta}^\top \\
		& \succeq \lambda_{1} \bm{u}_1 \bm{u}_1^\top - 3 \| \bm{\Delta} \|_{2} \| \bm{x} \|_2 \bm{I}_{n} \qquad \qquad \alertb{(\|\bm{\Delta}\|_2 \leq  \|\bm{x}\|_2)} \\
		&\succeq \lambda_{1} \bm{u}_1 \bm{u}_1^\top - 0.25 ( \lambda_{1} - \lambda_{2} ) \bm{I}_{n},
	\end{align*}
	where last line relies on $\| \bm{\Delta} \|_{2} \|\bm{x} \|_2 \leq (\lambda_{1} - \lambda_{2}) / 12$. Consequently,
	\begin{align*}
		& \nabla^2 f(\bx) = 2\bx\bx^{\top} + \|\bm{x}\|_2^2 \bm{I}_{n} - \lambda_{1} \bm{u}_1 \bm{u}_1^\top - \sum\nolimits_{i=2}^{n} \lambda_{i} \bm{u}_i \bm{u}_i^\top \\
		&\quad \succeq 2 \lambda_{1} \bm{u}_1 \bm{u}_1^\top + (\|\bm{x}\|_2^2 - 0.5) ( \lambda_{1} - \lambda_{2} ) \bm{I}_{n} - \lambda_{1} \bm{u}_1 \bm{u}_1^\top - \sum\nolimits_{i=2}^{n} \lambda_{i} \bm{u}_i \bm{u}_i^\top \\
		&\quad \succeq (\|\bm{x}\|_2^2  - 0.5 ( \lambda_{1} - \lambda_{2} ) +\lambda_{1})  \bm{u}_1 \bm{u}_1^\top \\
		&\quad\quad + \sum\nolimits_{i=2}^{n} ( \|\bm{x}\|_2^2 - 0.5 ( \lambda_{1} - \lambda_{2} ) - \lambda_{i} ) \bm{u}_i \bm{u}_i^\top \\
		&\quad \succeq ( \|\bm{x}\|_2^2 - 0.5 ( \lambda_{1} - \lambda_{2} ) - \lambda_{2} ) \bm{I}_{n} \\
		&\quad \succeq 0.25 ( \lambda_{1} - \lambda_{2} ) \bm{I}_{n} \qquad \qquad \alertb{(\lambda_1 - 0.25 (\lambda_1 - \lambda_2) \leq \|\bm{x}\|_2^2)}
	\end{align*}
\end{frame}


\frame{
\frametitle{Critical points of $f(\cdot)$}

%Any $\bu\in\mathbb{R}^{n}$ satisfies
$$ \bm{x}\text{ is a critical point, ~i.e., } \nabla f(\bx) = (\bx\bx^{\top}-\bM) \bx = \bm{0} $$
$$\Updownarrow $$
$$ \bM\bx = \|\bx\|_2^2  \bx $$
$$\Updownarrow $$
$$\bx~\mbox{aligns with an eigenvector of}~\bM \quad \text{or}\quad \bx = \bm{0}$$
%$$\text{or}\quad  $$

\vfill
 
\alert{Since $ \bM\bu_i = \lambda_i  \bu_i $,  the set of critical points is given by
$$\{\bm{0} \} \cup \{ \pm \sqrt{\lambda_i } \bu_i, ~~i=1,\ldots, n\}$$}
}
 


\begin{frame}
\frametitle{Categorization of critical points}

The critical points can be further categorized based on the \textbf{Hessian}:
$$ \nabla^2 f(\bx) = 2\bx\bx^{\top} + \|\bm{x}\|_2^2 \bm{I}_{n} - \bM $$

\begin{itemize}
\item<1> For any non-zero critical point $\bx_k= \pm \sqrt{\lambda_k } \bu_k$:
\begin{align*}
 \nabla^2 f( {\bx}_k)  & = 2\lambda_k \bu_k \bu_k^{\top} + \lambda_k \bI - \bM \\
 & =  2\lambda_k \bu_k \bu_k^{\top} + \lambda_k \left(\sum_{i=1}^n \bu_i\bu_i^{\top}\right) - \sum_{i=1}^n \lambda_i \bu_i\bu_i^{\top} \\
 & =  \sum_{i:i\neq k} (\lambda_k - \lambda_i ) \bu_i\bu_i^{\top}  + 2\lambda_k \bu_k\bu_k^{\top}
\end{align*}


\end{itemize}



\end{frame}

\begin{frame}
	\frametitle{Categorization of critical points (cont.)}
	If $\lambda_1 \alert{>}\lambda_2\geq \ldots \geq \lambda_n \alert{\geq} 0$, then
\vspace{0.2em}
\begin{itemize}
\itemsep0.4em
\item  $\nabla^2 f( {\bx}_1)\succ \bm{0}$ \qquad\qquad $\rightarrow$ \quad \alertb{local minima}
\item $1< k\leq n$: $\lambda_{\min}(\nabla^2 f( {\bx}_k))< 0$, $\lambda_{\max}(\nabla^2 f( {\bx}_k))>0$ \\ \qquad\qquad\qquad\qquad\qquad $\rightarrow$ \quad \alertb{strict saddle}
\item $\bm{x}=\bm{0}$: $\nabla^2 f(\bm{0}) = - \bm{M} \preceq \bm{0}$ ~\quad $\rightarrow$ ~\quad \alertb{local maxima, strict saddle}   
	%(or \alertb{strict saddle})
\end{itemize}

\vfill
{
\setbeamercolor{block body}{bg=babyblueeyes,fg=black}

\begin{varblock}[\textwidth]{}
\centering
all local are global; all saddle are strict
\end{varblock}
}

\end{frame}


\frame{
\frametitle{A pictorial example}

For example, for 2-dimensional case {\small $f(\bm{x})= \left\| \bm{x}\bm{x}^{\top} - \begin{bmatrix}1 & 1\\1 & 1\end{bmatrix} \right\|_{\mathrm{F}}^2$ }
 \begin{center}
\includegraphics[width=0.65\textwidth]{Saddle_2d}
\end{center}
 
{\small
global minima: $\bm{x} =\pm \begin{bmatrix} 1 \\ 1 \end{bmatrix}$; strict saddles: $\bm{x}=\begin{bmatrix} 0 \\ 0 \end{bmatrix}$,  and $\pm \begin{bmatrix} 1 \\ -1 \end{bmatrix}$\\
	
 {\hfill \em --- No ``spurious'' local minima!}}

}


 

\begin{comment}

\begin{frame}
\frametitle{Extension to the low-rank case}

$$ \  f(\bX): = \frac{1}{4} \| \bm{X}\bm{X}^{\top}-\bm{M} \|_{\mathrm{F}}^2, \quad\quad \bX\in\mathbb{R}^{n\times r} $$
\medskip

\alert{If $\sigma_r > \sigma_{r+1}$:}

\begin{itemize}
\itemsep0.5em
\item \textbf{all local minima are global:} $\bX$ contains top-$r$ eigenvectors (up to orthonormal transformation)
\item \textbf{strict saddle points:} all stationary points are saddle points except global optimum
\item \textbf{restricted strong convexity:} locally around the global optimum
\end{itemize}
% \begin{center}
%\includegraphics[width=0.35\textwidth]{Saddle_2d}
%\end{center}
\end{frame}

\end{comment}


\begin{frame}
\frametitle{Two vignettes}

\begin{columns}
\begin{column}{0.5\textwidth}
\uncover<1->{
{\bf Two-stage approach:} 

\begin{center}
 \includegraphics[width=0.82\textwidth]{non_convex_iterate.pdf}
 \end{center}
 
 \medskip
 
 \begin{center}
{\em smart initialization \\
   + \\
  local refinement}  
  \end{center}
  
 }
 
 \end{column}
 \begin{column}{0.5\textwidth}
 
 \uncover<2->{
 
{\bf  Global landscape:}
 \begin{center}
  \includegraphics[width=0.79\textwidth]{saddle_escaping_illustration.pdf}
   \end{center}
   
    \begin{center}
{\em benign landscape \\
   + \\
  saddle-point escaping}
  \end{center}
  
   }
 \end{column}
  \end{columns}



\end{frame}


\begin{frame}
\frametitle{Global landscape}


\begin{columns}
\begin{column}{0.68\textwidth}
 
{\bf Benign landscape: }
\begin{itemize}
\item all local minima = global minima
\item other critical points = strict saddle points
\end{itemize}
 
 \medskip
 
 {\bf Saddle-point escaping algorithms:}
 
 \begin{itemize}
 \item trust-region methods
 \item perturbed gradient descent
 \item perturbed SGD
 \item ...
 \end{itemize}
 
 \end{column}
 \begin{column}{0.32\textwidth}
 
 \vspace{-0.5in}
 \begin{center}
  \includegraphics[width=1.05\textwidth]{saddle_escaping_illustration.pdf}
   \end{center}
 
 
 \end{column}
  \end{columns}
  
 
\end{frame}

\begin{frame}
	\frametitle{Next steps}
	\begin{itemize}
		\item Generic local analysis of (regularized) gradient descent
		\item Refined local analysis for gradient descent
		\item Global landscape analysis
		\item Gradient descent with random initialization
		\item (Maybe) Gradient descent with arbitrary initialization 
	\end{itemize}
\end{frame}



\end{document}

