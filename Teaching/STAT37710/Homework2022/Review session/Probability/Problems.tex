\documentclass{article}
\usepackage{color,cite,array,comment}
\usepackage[dvipsnames]{xcolor}
\usepackage{amsmath, amsfonts, amssymb,amsthm}
\usepackage{subfigure,epsfig}
\usepackage{algorithm,algpseudocode}
\usepackage{caption}
\newtheorem{lemma}{Lemma}
\newtheorem{theorem}{Theorem}
\title{Problems}
\begin{document}
\maketitle
\begin{enumerate}
\item Let $X_1,\cdots,X_n$ be a sequence of i.i.d. random variables, of which cdf is $F_X(\cdot)$.

\begin{enumerate}


\item What is CDF of $X_\text{max}=\max_i X_i$ ?



\item What is CDF of $X_\text{min}=\min_i X_i$ ?

\end{enumerate}

\item Show how the Chebyshev inequality can be derived from Markov inequality.

\item If $X$ is a continuous random variable having CDF $F_X$, show that the random variable $Y=F_X(X)$ is uniformly distributed in $(0,1)$.

\item Suppose you can generate a random variable $U$ uniformly distributed in $(0,1)$. How would you use it to simulate a continuous random variable $X$ having a arbitrary distribution function $F(\cdot)$ ? 


\item  Suppose you have access to two independent random variables $U_1$ and $U_2$, both uniformly distributed in $[0,1]$.
How would you use them to simulate two continuous random variables $X_1$ and $X_2$ with a given joint distribution $F(\cdot,\cdot)$ ?


\item Prove the weak law of large number using Chebyshev's inequality.


\item Let $W\sim\mathcal{N}(0,1)$ and $Q(x)=\mathbb{P}[W>x]$.

\begin{enumerate}
\item Show that
	\begin{align*}
	Q(x)<\frac{1}{\sqrt{2\pi}x}\exp\left(-\frac{x^2}{2}\right).	
	\end{align*}
	

\item Show that
	\begin{align*}
	Q(x)>\frac{1}{\sqrt{2\pi}x}\left(1-\frac{1}{x^2}\right)\exp\left(-\frac{x^2}{2}\right)\quad\forall~x>1.
	\end{align*}
	

\end{enumerate}
\item Prove Cauchy-Schwarz inequality. 
	\begin{align*}
	\mathbb{E}[XY]^2\leq \mathbb{E}[X^2]\mathbb{E}[Y^2].	
	\end{align*}

\item The amount of weight, $W$, that a bridge can withstand without damage, is a Gaussian random variable with mean $\mu_W$ and variance $\sigma_W^2$. 
Suppose the weight of cars $X_1,X_2,\cdots,X_n$ are i.i.d. random variables with mean $\mu_X$ and variance $\sigma_X^2$.
How many cars would have to be on the bridge for the probability of damage to exceed 0.1 ?

\item Show that if $X$ and $Y$ are independent and 
	\begin{align*}
	&X\sim\mathcal{N}(\mu_X,\sigma_X^2),\\
	&Y\sim\mathcal{N}(\mu_Y,\sigma_Y^2),\\
	&Z=X+Y,	
	\end{align*}
then 
	\begin{align*}
	Z\sim\mathcal{N}(\mu_X+\mu_Y,\sigma_X^2+\sigma_Y^2).	
	\end{align*}


\end{enumerate}
\end{document}
