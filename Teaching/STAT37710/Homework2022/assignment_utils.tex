\usepackage{amsmath}
\usepackage{amssymb}
\usepackage{graphicx}
\usepackage{bbm}
\usepackage{fullpage}
\usepackage{color}
\usepackage{bm}


\input testpoints

\newcommand\courseName{STAT 37710: Machine Learning}
\newcommand\semester{Autumn 2022}


%%%%%%%%%%%%%%%%%%%%%%%%%%%%%%
% Assignments include homeworks, quizzes, and exams.

\newcommand\createAssignment[1]{
  \noindent
  \begin{center}
  \framebox{
    \vbox{
      \hbox to 5.78in {\small  {\bf \courseName} \hfill \semester}
      \vspace{4mm}
        \hbox to 5.78in {{\bf \hfill #1   \hfill} } 
      \vspace{2mm}
    \hbox to 5.78in {\em \hfill Due date: \dueDate   \hfill} 
    }
  }
  
    \end{center}
  \vspace*{4mm}

\newcommand\solution[1]{}
\newcommand\solutionWithSpacing[2]{\vspace{##1}}
\newcommand\answerSpace{\newpage}
}

\newcommand\createAssignmentSolutions[1]{
  \noindent
  \begin{center}
  \framebox{
    \vbox{
      \hbox to 5.78in {\small  {\bf \courseName} \hfill \semester}
      \vspace{4mm}
        \hbox to 5.78in {{\bf \hfill #1  Solutions \hfill} } 
      \vspace{2mm}
    \hbox to 5.78in {\em  \hfill Please do not distribute. \hfill} 
    }
  }
    \end{center}
  \vspace*{4mm}

\newcommand\solution[1]{ \medskip { {\bf Solution:} \em ##1} \medskip}
\newcommand\solutionWithSpacing[2]{ \medskip { {\bf Solution:} \em ##2} \medskip}
\newcommand\answerSpace{}
}

%%%%%%%%%%%%%%%%%%%%%%%%%%%%%%
% Specific to homeworks.

\newcommand\createHomework[1]{
\setcounter{problemspacing}{2} % integer value in em
\createAssignment{Homework #1}
% Please structure your writeups hierarchically: convey the overall plan
% before diving into details.  You should justify with words why something's true (by algebra, convexity, etc.).
% There's no need to step through a long sequence of trivial algebraic operations.
% % Be careful not to mix assumptions with things which are derived.
{\centering{\em You are allowed to drop 1 subproblem without penalty. But you cannot drop problems on simulation.}}
}

\newcommand\createHomeworkSolutions[1]{
\setcounter{problemspacing}{2} % integer value in em; 6 em / inch
\createAssignmentSolutions{Homework #1}
}

\newcommand\w{\mathbf{w}}
\renewcommand\P{\mathbb{P}}
\newcommand\E{\mathbb{E}}
\newcommand\R{\mathbb{R}}
\newcommand\sI{\mathcal{I}}
\newcommand\sN{\mathcal{N}}
\newcommand\sX{\mathcal{X}}
\newcommand\sY{\mathcal{Y}}
\newcommand\sA{\mathcal{A}}
\newcommand\sL{\mathcal{L}}
\newcommand\bE{\mathbb{E}}
\newcommand\bP{\mathbb{P}}
\newcommand{\1}{\mathbb{I}} % Indicator (don't use \mathbbm{1} because bbm is not TrueType)
\newcommand\tr{\text{tr}}
\newcommand\sign{\text{sign}}
\DeclareMathOperator{\rank}{rank}
\DeclareMathOperator{\Var}{Var}
\DeclareMathOperator{\Cov}{Cov}
\DeclareMathOperator{\Regret}{Regret}
\newcommand\inv[1]{\ensuremath{\frac{1}{#1}}}
\newcommand\inner[2]{\ensuremath{\left< #1, #2 \right>}} % Inner product
\newcommand\KL[2]{\ensuremath{\text{KL}\left( #1 \| #2 \right)}} % KL-divergence
\newcommand\sH{\ensuremath{\mathcal{H}}}
\newcommand\sD{\ensuremath{\mathcal{D}}}
\newcommand\sF{\ensuremath{\mathcal{F}}}
\newcommand\vc{\text{VC}}
\newcommand\eqdef{\ensuremath{\stackrel{\rm def}{=}}} % Equal by definition

\newcommand\pb[1]{\ensuremath{\left[ #1 \right]}} % []
\newcommand\p[1]{\ensuremath{\left( #1 \right)}} % Parenthesis ()

\usepackage{color}
\newcommand{\hl}[2]{\colorbox{#2}{#1}}
\newcommand{\hly}[1]{\hl{yellow}{#1}}
\def\todo#1{\hl{{\bf TODO:} #1}{yellow}}
\def\needfig{\hl{Figure X}{green}}
