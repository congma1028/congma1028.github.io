\documentclass{article}
\usepackage{url}

\usepackage{cite,enumitem,amsmath, amsfonts, amssymb}
\usepackage{epsfig,subfigure}
\usepackage{comment}
\usepackage{array}


\newcommand\dueDate{XXX}

\input assignment_utils
\usepackage{listings}


\begin{document}
\createHomework{0}
%\createHomeworkSolutions{1}



%%%%%%%%%%%%%%%%%%%%%%%%%%%%%%%%%%%%%%%%%%%%%%%%%%%%%%%%%%%%

\section{Probability and statistics}

\begin{problem}{Probability and Statistics}{20}


	

\newpart{10}
	After your yearly checkup, the doctor has bad news and good news. 
    The bad news is that you tested positive for a serious disease, and that the test is 99\% accurate (i.e., the probability of testing positive given that you have the disease is 0.99, as is the probability of testing negative given that you don't have the disease).
    The good news is that this is a rare disease, striking only one in 10,000 people.
    What are the chances that you actually have the disease?


\solution{
Let $\bm{A}=\bm{U}\bm{\Lambda}\bm{U}^{\top}$ be the eigen-decomposition
of $\bm{A}$, where $\bm{U}=[\bm{u}_{1},\bm{u}_{2},\cdots,\bm{u}_{n}]$
and $\bm{\Lambda}=\text{diag}(\lambda_{1}(\bm{A}),\lambda_{2}(\bm{A}),\cdots,\lambda_{n}(\bm{A}))$.
Pick $V$ to be the subpace spaned by the top-$i$ eigenvectors $\{\bm{u}_{1},\bm{u}_{2},\cdots,\bm{u}_{i}\}$.
Then every $\bm{v}\in V$ has the following decomposition 
\[
\bm{v}=\sum_{k=1}^{i}\alpha_{k}\bm{u}_{k}.
\]
As a consequence, we have 
\begin{align*}
\bm{v}^{\top}\bm{A}\bm{v} & =\bm{v}^{\top}\bm{U}\bm{\Lambda}\bm{U}^{\top}\bm{v}\\
 & =\sum_{k=1}^{i}\alpha_{k}^{2}\lambda_{k}\left(\bm{A}\right),
\end{align*}
which in turn leads to 
\[
\inf_{\bm{v}\in V:\|\bm{v}\|_{2}=1}\bm{v}^{\top}\bm{A}\bm{v}=\inf_{\|\bm{\alpha}\|_{2}=1}\sum_{k=1}^{i}\alpha_{k}^{2}\lambda_{k}\left(\bm{A}\right)=\lambda_{i}\left(\bm{A}\right).
\]
Therefore we obtain 
\[
\lambda_{i}\left(\bm{A}\right)\leq\sup_{V:\text{dim}\left(V\right)=i}\inf_{\bm{v}\in V:\|\bm{v}\|_{2}=1}\bm{v}^{\top}\bm{A}\bm{v}.
\]
Now we move on to show the reverse inequality, i.e. for every $V$
with dimension $i$, we can find some $\bm{v}\in V, \|\bm{v}\|_{2}=1$ such that $\bm{v}^{\top}\bm{A}\bm{v}\leq\lambda_{i}(\bm{A})$.
Let $W$ be a space spanned by $\{\bm{u}_{i},\bm{u}_{i+1},\cdots,\bm{u}_{n}\}$
which has dimension $n-i+1$ and codimension $i-1$. In light of this, it must have a nontrivial
intersection with $V$. Let $v\in V\cap W$ and $v=\sum_{k=i}^{n}\alpha_{k}\bm{u}_{k}$.
Then one has 
\[
\bm{v}^{\top}\bm{A}\bm{v}=\sum_{k=i}^{n}\alpha_{k}^{2}\lambda_{k}\left(\bm{A}\right)\leq\lambda_{i}\left(\bm{A}\right).
\]
This completes the proof. 
}

	\newpart{10}  Let $X_1, X_2, ..., X_n \sim \mathcal{N}(\mu, \sigma^2)$ be i.i.d random variables. Compute the following:
    \begin{enumerate}
        \item  $a,b$ such that $aX_1+b \sim \mathcal{N}(0,1)$.
        \item  $\E{X_1 + 2X_2}, \Var{X_1 + 2X_2}$.
        \item Setting $\widehat{\mu}_n = \frac{1}{n} \sum_{i=1}^n X_i$, the mean and variance of $\sqrt{n}(\widehat{\mu}_n - \mu)$.
    \end{enumerate}


\solution{
By (a), one sees that 
\begin{align*}
\lambda_{i}\left(\bm{A}\right) & =\sup_{V:\text{dim}\left(V\right)=i}\inf_{\bm{v}\in V:\|\bm{v}\|_{2}=1}\bm{v}^{\top}\bm{A}\bm{v}\\
 & =\sup_{V:\text{dim}\left(V\right)=i}\inf_{\bm{v}\in V:\|\bm{v}\|_{2}=1}\bm{v}^{\top}\bm{B}\bm{v}+\bm{v}^{\top}\left(\bm{A}-\bm{B}\right)\bm{v}.
\end{align*}
In light of the fact that 
\[
\left|\bm{v}^{\top}\left(\bm{A}-\bm{B}\right)\bm{v}\right|\leq\left\Vert \bm{A}-\bm{B}\right\Vert ,
\]
we have 
\begin{align*}
\lambda_{i}\left(\bm{A}\right) & \leq\sup_{V:\text{dim}\left(V\right)=i}\inf_{\bm{v}\in V:\|\bm{v}\|_{2}=1}\bm{v}^{\top}\bm{B}\bm{v}+\left\Vert \bm{A}-\bm{B}\right\Vert \\
 & =\lambda_{i}\left(\bm{B}\right)+\left\Vert \bm{A}-\bm{B}\right\Vert 
\end{align*}
and similarly 
\begin{align*}
\lambda_{i}\left(\bm{A}\right) & \geq\sup_{V:\text{dim}\left(V\right)=i}\inf_{\bm{v}\in V:\|\bm{v}\|_{2}=1}\bm{v}^{\top}\bm{B}\bm{v}-\left\Vert \bm{A}-\bm{B}\right\Vert \\
 & =\lambda_{i}\left(\bm{B}\right)-\left\Vert \bm{A}-\bm{B}\right\Vert .
\end{align*}
Combining the above two inequalities yields the desired result. 
}


\end{problem}

\begin{problem}

\newpart{10}
For any function $g \colon \mathbb{R} \to \mathbb{R}$ and random variable $X$ with PDF $f(x)$, recall that the expected value of $g(X)$ is defined as $\mathbb{E}{g(X)} = \int_{-\infty}^\infty g(y) f(y) \diff y$. For a boolean event $A$, define $\1\{ A \}$ as $1$ if $A$ is true, and $0$ otherwise. Fix $F(x) = \E{\1\{X \leq x\}}$. Let $X_1,\ldots,X_n$ be independent and identically distributed random variables with CDF $F(x)$. Define $\widehat{F}_n(x) = \frac{1}{n} \sum_{i=1}^n \1\{X_i \leq x\}$. Note, for every $x$, that $\widehat{F}_n(x)$ is an empirical estimate of  $F(x)$.
    \begin{enumerate}
      \item For any $x$, what is $\E{\widehat{F}_n(x)}$?
      \item For any $x$, the variance of $\widehat{F}_n(x)$ is $\E{\left( \widehat{F}_n(x) -  F(x) \right)^2 }$.  Show that $\Var{\widehat{F}_n(x)} = \frac{F(x)(1-F(x))}{n}$.
      \item Using your answer to b, show that for all $x\in \R$, we have  $\displaystyle \E{ \left( \widehat{F}_n(x) - F(x) \right)^2 } \leq \tfrac{1}{4n}$.  
    \end{enumerate}
    
    \solution{XXX}
\end{problem}


\begin{problem}{Linear algebra}{20}
Consider two orthonormal matrices $\bm{U},\bm{U}^{\star} \in\mathbb{R}^{n\times r}$,
	satisfying $\bm{U}^{\top}\bm{U}= \bm{U}^{\star\top} \bm{U}^{\star}=\bm{I}_r$ with $r<n$. We have discussed extensively the distance using projection matrices
\[
\|\bm{U}\bm{U}^{\top} - \bm{U}^{\star}\bm{U}^{\star\top} \|, \quad \text{and}\quad \|\bm{U}\bm{U}^{\top} - \bm{U}^{\star}\bm{U}^{\star\top} \|_{\mathrm{F}}.
\]
Also, our default choice of distance is the one using optimal rotation matrix: 
\[
\min_{\bm{R}\in \mathcal{O}^{r\times r}}\big\|\bm{U}\bm{R}-\bm{U}^{\star}\big\| , \quad \text{and}\quad \min_{\bm{R}\in\mathcal{O}^{r\times r}}\left\Vert \bm{U}\bm{R}-\bm{U}^{\star}\right\Vert _{\mathrm{F}}.
\]
Here $\mathbb{O}^{r\times r}:=\{\bm{R}\in\mathbb{R}^{r\times r}\mid\bm{R}\bm{R}^{\top}=\bm{R}^{\top}\bm{R}=\bm{I}_{r}\}$ is the set of all $r\times r$ orthonormal matrices. 


	\newpart{10} Show that 
	\[
	\|\bm{U}\bm{U}^{\top} - \bm{U}^{\star}\bm{U}^{\star\top} \|
	\leq
	\min_{\bm{R}\in \mathcal{O}^{r\times r}}\big\|\bm{U}\bm{R}-\bm{U}^{\star}\big\|
	%\mathsf{dist} \big(\bm{U},\bm{U}^{\star}\big)
	\leq \sqrt{2} \|\bm{U}\bm{U}^{\top} - \bm{U}^{\star}\bm{U}^{\star\top} \|.
	\]


\solution{

As before, suppose that the SVD of $\bm{U}^{\top}\bm{U}^{\star}$
is given by $\bm{X}\bm{\Sigma}\bm{Y}^{\top}$, where $\bm{X}$ and $\bm{Y}$
are $r\times r$ orthonormal matrices whose columns contain the left singular vectors and the right singular vectors of $\bm{U}^{\top}\bm{U}^{\star}$, respectively, and $\bm{\Sigma}\in \mathbb{R}^{r\times r}= \cos\bm{\Theta}$ is a diagonal matrix whose diagonal entries correspond to the singular values of $\bm{U}^{\top}\bm{U}^{\star}$.





\paragraph{The spectral norm upper bound.}
We first observe that
%
\begin{align}
\|\bm{U}\bm{X}\bm{Y}^{\top}-\bm{U}^{\star}\|^{2} & =\|(\bm{U}\bm{X}\bm{Y}^{\top}-\bm{U}^{\star})^{\top}(\bm{U}\bm{X}\bm{Y}^{\top}-\bm{U}^{\star})\|\nonumber\\
 & =\|2\bm{I}_{r}-\bm{Y}\bm{X}^{\top}\bm{U}^{\top}\bm{U}^{\star}-\bm{U}^{\star\top}\bm{U}\bm{X}\bm{Y}^{\top}\| \nonumber\\
	& =\|2\bm{I}_{r}-\bm{Y}\bm{X}^{\top}\bm{X}\bm{\Sigma}\bm{Y}^{\top}-\bm{Y}\bm{\Sigma}\bm{X}^{\top}\bm{X}\bm{Y}^{\top}\| \nonumber\\
 & =2\|\bm{Y}(\bm{I}_{r}-\bm{\Sigma})\bm{Y}^{\top}\|=2\|\bm{I}_{r}-\bm{\Sigma}\|.
	\label{eq:UXT-Ustar-UB1}
\end{align}
%
Here, the penultimate line relies on the singular value decomposition $\bm{U}^{\top}\bm{U}^{\star}=\bm{X}\bm{\Sigma}\bm{Y}^{\top}$,
while the two identities in the last line result from  the orthonormality of $\bm{X}$ and $\bm{Y}$, respectively. In addition, note that
%
\begin{align*}
	\|\bm{I}_{r}-\bm{\Sigma}\| &= \|\bm{I}_{r}-\cos\bm{\Theta}\|\leq\|\bm{I}_{r}-\cos^{2}\bm{\Theta}\| \\
	& =\|\sin^{2}\bm{\Theta}\|=\|\sin\bm{\Theta}\|^2.
\end{align*}
%
This taken together with \eqref{eq:UXT-Ustar-UB1} leads to
%
\begin{align*}
	\min_{\bm{R}\in \mathcal{O}^{r\times r}}\big\|\bm{U}\bm{R}-\bm{U}^{\star}\big\|
	\leq \big\|\bm{U}\bm{X}\bm{Y}^{\top}-\bm{U}^{\star}\big\| \leq \sqrt{2} \|\sin \bm{\Theta} \| ,
	% \mathsf{dist}(\bm{U},\bm{U}^{\star}).
\end{align*}
%
where the first inequality holds since $\bm{X}$ and $\bm{Y}$ are both orthonormal matrices and hence $\bm{X}\bm{Y}^{\top}$ is also orthonormal.



\paragraph{The spectral norm lower bound.}
On the other hand, we make the observation that
%
\begin{align}
	& \min_{\bm{R}\in\mathcal{O}^{r\times r}}\big\|\bm{U}\bm{R}-\bm{U}^{\star}\big\|^{2}  =\min_{\bm{R}\in\mathcal{O}^{r\times r}}\big\|(\bm{U}\bm{R}-\bm{U}^{\star})^{\top}(\bm{U}\bm{R}-\bm{U}^{\star})\big\|\nonumber \\
 & \qquad\qquad =\min_{\bm{R}\in\mathcal{O}^{r\times r}}\big\|\bm{R}^{\top}\bm{U}^{\top}\bm{U}\bm{R}+\bm{U}^{\star\top}\bm{U}^{\star}-\bm{R}^{\top}\bm{U}^{\top}\bm{U}^{\star}-\bm{U}^{\star\top}\bm{U}\bm{R}\big\|\nonumber \\
 & \qquad\qquad{=}\min_{\bm{R}\in\mathcal{O}^{r\times r}}\big\|2\bm{I}_{r}-\bm{R}^{\top}\bm{X}\bm{\Sigma}\bm{Y}^{\top}-\bm{Y}\bm{\Sigma}\bm{X}^{\top}\bm{R}\big\|,
	\label{eq:relation-i-123456}
\end{align}
%
where the last relation holds since $\bm{X}\bm{\Sigma}\bm{Y}^{\top}$ is the
SVD of $\bm{U}^{\top}\bm{U}^{\star}$. Continue the derivation to obtain
%
\begin{align}
\eqref{eq:relation-i-123456}  & \overset{(\mathrm{i})}{=}\min_{\bm{Q}\in\mathcal{O}^{r\times r}}\big\|2\bm{I}_{r}-\bm{Q}\bm{\Sigma}\bm{Y}^{\top}-\bm{Y}\bm{\Sigma}\bm{Q}^{\top}\big\|\nonumber \\
 & \overset{(\mathrm{ii})}{=}\min_{\bm{Q}\in\mathcal{O}^{r\times r}}\big\|2\bm{Q}^{\top}\bm{Q}-\bm{Q}^{\top}\bm{Q}\bm{\Sigma}\bm{Y}^{\top}\bm{Q}-\bm{Q}^{\top}\bm{Y}\bm{\Sigma}\bm{Q}^{\top}\bm{Q}\big\|\nonumber \\
 &  =\min_{\bm{Q}\in\mathcal{O}^{r\times r}}\big\|2\bm{I}_{r}-\bm{\Sigma}\bm{Y}^{\top}\bm{Q}-\bm{Q}^{\top}\bm{Y}\bm{\Sigma}\big\|\nonumber \\
 &  \overset{(\mathrm{iii})}{=}\min_{\bm{O}\in\mathcal{O}^{r\times r}}\big\|2\bm{I}_{r}-\bm{\Sigma}\bm{O}-\bm{O}^{\top}\bm{\Sigma}\big\|.\label{eq:UR-Ustar-identity}
\end{align}
%
Here, (i) follows by setting $\bm{Q}=\bm{R}^{\top}\bm{X}$
(since both $\bm{X}$ and $\bm{R}$ are orthonormal matrices), (ii)
results from the unitary invariance of the spectral norm, whereas (iii) holds
by setting $\bm{O}=\bm{Y}^{\top}\bm{Q}$. Moreover, recognizing that
$\|\bm{\Sigma}\bm{O}\|\leq\|\bm{\Sigma}\| \cdot \|\bm{O}\|\leq1$ (and hence
$2\bm{I}_{r}-\bm{\Sigma}\bm{O}-\bm{O}^{\top}\bm{\Sigma}\succeq\bm{0}$),
one can obtain
%
\begin{align}
\min_{\bm{O}\in\mathcal{O}^{r\times r}}\big\|2\bm{I}_{r}-\bm{\Sigma}\bm{O}-\bm{O}^{\top}\bm{\Sigma}\big\|
	& = \min_{\bm{O}\in\mathcal{O}^{r\times r}}\lambda_{\max} \big( 2\bm{I}_{r}-\bm{\Sigma}\bm{O}-\bm{O}^{\top}\bm{\Sigma}\big) \notag\\
	& =\min_{\bm{O}\in\mathcal{O}^{r\times r}}\max_{\bm{u}:\|\bm{u}\|_{2}=1}\bm{u}^{\top}\big(2\bm{I}_{r}-\bm{\Sigma}\bm{O}-\bm{O}^{\top}\bm{\Sigma}\big)\bm{u}\nonumber \\
 & =\min_{\bm{O}\in\mathcal{O}^{r\times r}}\max_{\bm{u}:\|\bm{u}\|_{2}=1}\big(2-2\bm{u}^{\top}\bm{\Sigma}\bm{O}\bm{u}\big)\nonumber \\
 & \geq\min_{\bm{O}\in\mathcal{O}^{r\times r}}\big(2-2\bm{e}_{r}^{\top}\bm{\Sigma}\bm{O}\bm{e}_{r}\big)\nonumber \\
 & = 2-2\cos\theta_{r}\max_{\bm{O}\in\mathcal{O}^{r\times r}}\bm{e}_{r}^{\top}\bm{O}\bm{e}_{r}\nonumber \\
 & \geq 2-2\cos\theta_{r}
	=4\sin^{2}(\theta_{r}/2).\label{eq:UR-star-inequality}
\end{align}
%
Here, the inequality follows by taking $\bm{u}$ to be $\bm{e}_{r}$
(recall that by construction, $\sigma_{r}=\cos\theta_{r}\geq 0$ is the
smallest singular value of $\bm{\Sigma}$), and the penultimate line
holds by combining the facts $|\bm{e}_{r}^{\top}\bm{O}\bm{e}_{r}|\leq\|\bm{O}\|=1$
and $\bm{e}_{r}^{\top}\bm{e}_{r}=1$. Putting (\ref{eq:UR-star-inequality})
and (\ref{eq:UR-Ustar-identity}) together yields
%
\begin{align*}
\min_{\bm{R}\in\mathcal{O}^{r\times r}}\big\|\bm{U}\bm{R}-\bm{U}^{\star}\big\| & \geq\sqrt{4\sin^{2}(\theta_{r}/2)}=2\sin(\theta_{r}/2)=\|2\sin(\bm{\Theta}/2)\|\nonumber \\
 & \geq\|\sin\bm{\Theta}\|,
	%=\mathsf{dist}(\bm{U},\bm{U}^{\star})
\end{align*}
%
where we again use the inequality $2\sin(\theta/2) \geq \sin \theta$ for all $\theta\in [0,\pi/2]$.

Finally, invoking the relation $\|\sin\bm{\Theta}\|=  \|\bm{U}\bm{U}^{\top} - \bm{U}^{\star}\bm{U}^{\star\top}\|$ establishes the claimed spectral norm bounds.






}


\newpart{10} Show that 
\[
\tfrac{1}{\sqrt{2}} \|\bm{U}\bm{U}^{\top} - \bm{U}^{\star}\bm{U}^{\star\top} \|_{\mathrm{F}}
	%\mathsf{dist}_{\mathrm{F}} \big(\bm{U},\bm{U}^{\star}\big)
	\leq
	\min_{\bm{R}\in\mathcal{O}^{r\times r}}\left\Vert \bm{U}\bm{R}-\bm{U}^{\star}\right\Vert _{\mathrm{F}}
	%\mathsf{dist}_{\mathrm{F}} \big(\bm{U},\bm{U}^{\star}\big)
	\leq \|\bm{U}\bm{U}^{\top} - \bm{U}^{\star}\bm{U}^{\star\top} \|_{\mathrm{F}}.
\] 


\solution{ 


\paragraph{The Frobenius norm upper bound.}
  Regarding the Frobenius norm upper bound, one sees that
%
\begin{align}
	& \big\Vert \bm{U}\bm{X}\bm{Y}^{\top}-\bm{U}^{\star}\big\Vert _{\mathrm{F}}^{2}  =\left\Vert \bm{U}\right\Vert _{\mathrm{F}}^{2}+\big\|\bm{U}^{\star}\big\|_{\mathrm{F}}^{2}-2\mathsf{Tr}\big(\bm{Y}\bm{X}^{\top}\bm{U}^{\top}\bm{U}^{\star}\big) \notag\\
 & \qquad\qquad \overset{(\mathrm{i})}{=}r+r-2\mathsf{Tr}\big(\bm{Y}\bm{X}^{\top}\bm{X}\bm{\Sigma}\bm{Y}^{\top}\big)
  \overset{(\mathrm{ii})}{=}2r-2\mathsf{Tr}\left(\bm{\Sigma}\right) ,
	\label{eq:relation-UB-12689}
\end{align}
%
where (i) holds since $\bm{U}$ and $\bm{U}^{\star}$ are both $n\times r$
matrices with orthonormal columns, and (ii) follows since $\bm{X}^{\top}\bm{X}=\bm{Y}^{\top}\bm{Y}=\bm{I}$ (and hence $\mathsf{Tr}(\bm{Y}\bm{X}^{\top}\bm{X}\bm{\Sigma}\bm{Y}^{\top})=\mathsf{Tr}(\bm{Y}^{\top}\bm{Y}\bm{X}^{\top}\bm{X}\bm{\Sigma})=\mathsf{Tr}(\bm{\Sigma})$).
Furthermore,
%
\begin{align*}
2r-2\mathsf{Tr}\left(\bm{\Sigma}\right)
	& \overset{(\mathrm{iii})}{=}2\sum\nolimits_{i} (1-\cos\theta_{i})
	\leq2\sum\nolimits_{i} (1-\cos^{2}\theta_{i})\\
 & =2\left\Vert \sin\bm{\Theta}\right\Vert _{\mathrm{F}}^{2} = \big\Vert \bm{U}\bm{U}^{\top}-{\bm{U}}^{\star}{\bm{U}}^{\star\top}\big\Vert _{\mathrm{F}}^2,
\end{align*}
%
where (iii) holds by construction, and the last identity results from the lemma in class. This
taken collectively with  \eqref{eq:relation-UB-12689}
reveals that
%
\begin{align*}
\min_{\bm{R}\in\mathcal{O}^{r\times r}}\left\Vert \bm{U}\bm{R}-\bm{U}^{\star}\right\Vert _{\mathrm{F}}^{2}
	& \leq\big\Vert \bm{U}\bm{X}\bm{Y}^{\top}-\bm{U}^{\star}\big\Vert _{\mathrm{F}}^{2}
	\leq
	%2\left\Vert \sin\bm{\Theta}\right\Vert _{\mathrm{F}}^{2} \\
	%=\mathsf{dist}_{\mathrm{F}}^{2}\big(\bm{U},\bm{U}^{\star}\big).
	\big\Vert \bm{U}\bm{U}^{\top}-{\bm{U}}^{\star}{\bm{U}}^{\star\top}\big\Vert _{\mathrm{F}}^2  ,
\end{align*}
where the first inequality holds since $\bm{X}$ and $\bm{Y}$ are both orthonormal matrices and hence $\bm{X}\bm{Y}^{\top}$ is also orthonormal.



\paragraph{The Frobenius norm lower bound.}

With regards to the Frobenius norm lower bound, it is seen that
%
\begin{align}
\min_{\bm{R}\in\mathcal{O}^{r\times r}}\big\|\bm{U}\bm{R}-\bm{U}^{\star}\big\|_{\mathrm{F}}^{2} & =\min_{\bm{R}\in\mathcal{O}^{r\times r}}\Big\{\|\bm{U}\bm{R}\|_{\mathrm{F}}^{2}+\|\bm{U}^{\star}\|_{\mathrm{F}}^{2}-2\big\langle\bm{U}\bm{R},\bm{U}^{\star}\big\rangle\Big\}\nonumber \\
 & \overset{(\mathrm{i})}{=} 2\min_{\bm{R}\in\mathcal{O}^{r\times r}}\Big\{ r-\big\langle\bm{R},\bm{U}^{\top}\bm{U}^{\star}\big\rangle\Big\}\nonumber \\
 & \overset{(\mathrm{ii})}{=} 2\min_{\bm{R}\in\mathcal{O}^{r\times r}}\Big\{ r-\big\langle\bm{R},\bm{X}\bm{\Sigma}\bm{Y}^{\top}\big\rangle\Big\} ,
	\label{eq:relation-ii-67890}
\end{align}
%
where (i) holds since $\|\bm{U}\|_{\mathrm{F}}=\|\bm{U}^{\star}\|_{\mathrm{F}}=\sqrt{r}$,
and (ii) relies on the SVD $\bm{X}\bm{\Sigma}\bm{Y}^{\top}$ of $\bm{U}^{\top}\bm{U}^{\star}$.
Continue the derivation to obtain
%
\begin{align}
\eqref{eq:relation-ii-67890}
	& \overset{(\mathrm{iii})}{=}  2\min_{\bm{Q}\in\mathcal{O}^{r\times r}}\Big\{ r-\big\langle\bm{Q},\cos\bm{\Theta}\big\rangle\Big\}
	 \overset{(\mathrm{iv})}{\geq} 2\min_{\bm{Q}\in\mathcal{O}^{r\times r}}\Big\{ r-\|\bm{Q}\|\,\|\cos\bm{\Theta}\|_{*}\Big\}\nonumber \\
 & =2 \big( r-\sum\nolimits_{i}\cos\theta_{i}\big).\label{eq:min-UR-Ustar-fro-1}
\end{align}
%
Here, (iii) sets $\bm{Q}=\bm{X}^{\top}\bm{R}\bm{Y}$ and identifies
$\bm{\Sigma}$ as $\cos\bm{\Theta}$,
(iv) comes from the elementary inequality $\langle \bm{A}, \bm{B} \rangle \leq \|\bm{A}\|\,\|\bm{B}\|_*$,
whereas the last line follows
since $\cos\theta_{i}\geq0$. Additionally, it is easily seen that
%
\begin{align}
%2r-2\sum\nolimits_{i} \cos\theta_{i}
\eqref{eq:min-UR-Ustar-fro-1}
	& =2\sum\nolimits_{i} (1-\cos\theta_{i})=4\sum\nolimits_{i} \sin^{2}(\theta_{i}/2)\nonumber \\
 & \geq \sum\nolimits_{i} \sin^{2}\theta_{i}
	%=\|\sin\bm{\Theta}\|_{\mathrm{F}}^{2}\nonumber \\
  =\frac{1}{2} \big\Vert \bm{U}\bm{U}^{\top}-{\bm{U}}^{\star}{\bm{U}}^{\star\top}\big\Vert _{\mathrm{F}}^2 ,
	\label{eq:min-UR-Ustar-fro-2}
\end{align}
%
where the penultimate relation follows from the elementary inequality
$2\sin(\theta/2)\geq\sin\theta$ (which holds for any $0\leq\theta\leq\pi/2$).
Combining the inequalities (\ref{eq:min-UR-Ustar-fro-1}) and (\ref{eq:min-UR-Ustar-fro-2}),
we establish the claimed lower bound.




}


\end{problem}







\begin{comment}

\begin{problem}{Ranking from pairwise comparisons with missing data}{20}
%
Consider $n$ items to be ranked, where each item is associated with a latent score $w_i>0$ (which determines the rank of the  items).  
For each pair of items $(i,j)$, we observe a pairwise comparison independently with probability $p$.  The pairwise comparison outcome $y_{i,j}$ is generated according to the Bradley-Terry-Luce (or logistic) model such that: if a pairwise comparison between $i$ and $j$ is obtained, then the outcome is given by  
%
\[
y_{i,j}=\begin{cases}
1,\quad & \text{with probability }\frac{w_{j}}{w_{i}+w_{j}},\\
0, & \text{with probability }\frac{w_{i}}{w_{i}+w_{j}},
\end{cases}
\qquad \text{provided that }(i,j) \text{ has been compared.} 
\]
%

Recall that the spectral ranking method computes the leading left eigenvector of $ {\bm{\pi}}$ of a certain matrix $ {\bm{P}}=[ {P}_{i,j}]_{1\leq i,j\leq n}$.  In the missing data case, this matrix is given by
\[
 {P}_{i,j}=\begin{cases}
	\frac{1}{2np}y_{i,j}, & \text{if }i\neq j \text{ and a paired comparison of } (i,j) \text{ is observed},\\
	0,  & \text{if }i\neq j \text{ and } (i,j) \text{ is not compared}, \\
1-\sum_{j:j\neq i} {P}_{i,j}, & \text{if }i=j.
\end{cases}
	%\qquad \text{provided that }(i,j) \text{ has been compared.} 
\]
%
Here, one can think of $np$ as the average number of comparisons  involving any item.   Let $ {\bm{\pi}}$ be the leading left eigenvector of $ {\bm{P}}$, and define $\bm{\pi}:=\frac{1}{\sum_{i=1}^n w_i} [w_1,\cdots, w_n]$. 
	
	
	\newpart{10} Use the eigenvector perturbation theory to derive an eigenvector perturbation upper bound on $\|  {\bm{\pi}} - \bm{\pi} \|_{\bm{\pi}}$.  \\
	
	(Hint) You can assume (without proof) that the spectral gap $1- \max\big\{ \lambda_2 \big( \mathbb{E}_{\bm{y}}[ {\bm{P}}] \big), - \lambda_n \big( \mathbb{E}_{\bm{y}}[ {\bm{P}}] \big) \big\}>c$ for some universal constant $c>0$, where  $\mathbb{E}_{\bm{y}}[ {\bm{P}}]$ denotes the conditional expectation of $ {\bm{P}}$ given the set of indicator variables $ \mathbb{I}_{\{(i,j)\text{ is observed}\}}  $ (${1\leq i,j\leq n}$). 

\solution{

}

	\newpart{10}  Suppose that the observed data $y_{i,j}$ is instead an average of $L$ independent pairwise comparisons, namely, 
	\[
y_{i,j}=\frac{1}{L}\sum_{l=1}^{L}y_{i,j}^{(l)},
\]
	where $y_{i,j}^{(l)}$ are independently generates obeying
	$$y_{i,j}^{(l)} = \begin{cases}
1,\quad & \text{with probability }\frac{w_{j}}{w_{i}+w_{j}}\\
0, & \text{with probability }\frac{w_{i}}{w_{i}+w_{j}}
\end{cases} 
	\qquad \text{provided that }(i,j) \text{ is compared.}
	$$
	%are independently generated.  
	Show that $ {\bm{\pi}}$ converges to ${\bm{\pi}}$ as $L\rightarrow \infty$. 

\solution{

}

\end{problem}

\end{comment}


\begin{problem}{Variant of Wedin's theorem}{10}
Consider the setting and notation used in class. Wedin's $\sin \bm{\Theta}$ theorem tells us that if $\|\bm{E}\| < \sigma_{r}^{\star} - \sigma_{r+1}^{\star}$, then there exist two orthonormal matrices $\bm{R}_{\bm{U}}, \bm{R}_{\bm{V}} \in \mathbb{R}^{r \times r}$ such that 
\[
\max\left\{ \left\Vert \bm{U}\bm{R}_{\bm{U}}-\bm{U}^{\star}\right\Vert _{\mathrm{F}} , \left\Vert \bm{V}\bm{R}_{\bm{V}}-\bm{V}^{\star}\right\Vert _{\mathrm{F}}\big)\right\} \leq \frac{\sqrt{2}\max\big\{ \|\bm{E}^{\top}\bm{U}^{\star}\|_{\mathrm{F}},\|\bm{E}\bm{V}^{\star}\|_{\mathrm{F}}\big\} }{\sigma_{r}^{\star}-\sigma_{r+1}^{\star}-\|\bm{E}\|}.
\]
However, in some cases, we hope for a single rotation matrix that could align both $(\bm{U}, \bm{U}^{\star})$ and  $(\bm{V}, \bm{V}^{\star})$. It turns out that this is achievable. Show that if $\|\bm{E}\| < \sigma_{r}^{\star} - \sigma_{r+1}^{\star}$, there exists a single orthonormal matrix $\bm{R} \in \mathcal{O}^{r \times r}$ such that
\[
\big(\left\Vert \bm{U}\bm{R}-\bm{U}^{\star}\right\Vert _{\mathrm{F}}^{2} + \left\Vert \bm{V}\bm{R}-\bm{V}^{\star}\right\Vert _{\mathrm{F}}^{2}\big)^{1/2} \leq \frac{\sqrt{2}\big( \|\bm{E}^{\top}\bm{U}^{\star}\|_{\mathrm{F}}^{2} + \|\bm{E}\bm{V}^{\star}\|_{\mathrm{F}}^{2}\big)^{1/2} }{\sigma_{r}^{\star}-\sigma_{r+1}^{\star}-\|\bm{E}\|}.
\]

You are allowed to invoke the general Davis-Kahan $\sin\bm{\Theta}$ theorem given in class.

\solution{
Apply Davis-Kahan to the symmetric dilation of $\bm{M}$ and $\bm{M}^\star$.}

\end{problem}

\begin{problem}{Quadratic systems of equations}{10}
Suppose that our goal is to estimate an unknown vector $\bm{x}^{\star} \in \mathbb{R}^n$ (obeying $\|\bm{x}^{\star}\|_2=1$) based on $m$ i.i.d.~samples of the form
	\[
		y_i = ( \bm{a}_i^{\top} \bm{x}^{\star} )^2, \qquad i=1,\ldots,m,
	\]
	where $\bm{a}_i\in \mathbb{R}^n$ are independent vectors (known {\em a priori}) obeying $\bm{a}_i \sim \mathcal{N}(\bm{0},\bm{I}_n)$. 

%\newpart{10}
	Suggest a spectral method for estimating $\bm{x}^{\star}$ that is consistent with either $\bm{x}^{\star}$ or $-\bm{x}^{\star}$ in the limit of infinite data, i.e., as $m$ goes to infinity.

%\newpart{10}
%	Derive a perturbation bound for the estimate output by your spectral method for sufficiently large $m$. 


\solution{
Construct a surrogate matrix
\[
\bm{Y}=\frac{1}{m}\sum_{i=1}^{m}y_{i}\bm{a}_{i}\bm{a}_{i}^{\top}=\frac{1}{m}\sum_{i=1}^{m}(\bm{a}_{i}^{\top}\bm{x})^{2}\bm{a}_{i}\bm{a}_{i}^{\top}.
\]
Then compute the leading eigenvalue $\bm{u}$ of $\bm{Y}.$ When $m\rightarrow\infty$,
\[
\bm{Y}\rightarrow\mathbb{E}[\bm{Y}]=\|\bm{x}\|_{2}^{2}\bm{I}+2\bm{x}\bm{x}^{\top}=\bm{I}+2\bm{x}\bm{x}^{\top},
\]
whose leading eigenvector is exactly $\pm\bm{x}$. 


}

\end{problem}





\begin{problem}{Matrix completion}{20}
% Preview source code for paragraph 0
Suppose that the ground-truth matrix is given by
\[
\bm{M}^\star=\bm{u}^\star\bm{v}^{\star\top}\in\mathbb{R}^{n\times n},
\]
where $\bm{u}^\star=\tilde{\bm{u}}/\|\tilde{\bm{u}}\|_{2}$ and $\bm{v}^\star=\tilde{\bm{v}}/\|\tilde{\bm{v}}\|_{2}$,
with $\tilde{\bm{u}},\tilde{\bm{v}}\sim\mathcal{N}(\bm{0}, \bm{I}_{n})$
generated independently. Each entry of $\bm{M}^\star=[M^\star_{i,j}]_{1\leq i,j\leq n}$
is observed independently with probability $p$. In the lecture,
we have constructed a matrix ${\bm{M}}=[{M}_{i,j}]_{1\leq i,j\leq n}$,
where
\[
{M}_{i,j}=\begin{cases}
\frac{1}{p}M^\star_{i,j}, & \text{if }M^\star_{i,j}\text{ is observed};\\
0, & \text{else}.
\end{cases}
\]
We have shown in class that with high probability, the leading left singular vector ${\bm{u}}$
of ${\bm{M}}$ is a reliable estimate of $\bm{u}^\star$, provided that
$p\gg\frac{\log^{3}n}{n}$.


	Now, consider a new matrix ${\bm{M}}^{(1)}=[{M}_{i,j}^{(1)}]_{1\leq i,j\leq n}$ obtained by zeroing out the 1st column and 1st row of ${\bm{M}}$. More precisely, for any $1\leq i,j\leq n$, 
\[
{M}_{i,j}^{(1)}=\begin{cases}
{M}_{i,j}, & \text{if }i\neq1\text{ and }j\neq1;\\
0, & \text{else}.
\end{cases}
\]
Let ${\bm{u}}^{(1)}$ (resp.~${\bm{v}}^{(1)}$) be the leading left (resp.~right) singular vector of ${\bm{M}}^{(1)}$. 

	\newpart{10} Recall that  Wedin's $\sin\bm{\Theta}$ Theorem states that: for any two matrices $\bm{A}$ and $\bm{B}$, their leading left singular vectors (denoted by $\bm{u}_A$ and $\bm{u}_B$ respectively) satisfy
%
\[
	\mathsf{dist}(\bm{u}_A, \bm{u}_B)\leq\frac{\big\|\bm{A}-\bm{B}\big\|}{\sigma_{1}\big(\bm{A}\big)-\sigma_2(\bm{A}) -\|\bm{A}-\bm{B}\|}.
\]
Use it to derive an upper bound on $\mathsf{dist}({\bm{u}}^{(1)},{\bm{u}})$ in terms of $n$ and $p$. 

\solution{
%


To begin with, using Matlab notation we have
\begin{align*}
\sigma_{1}( {\bm{M}}^{(1)}) & \geq\sigma_{1}(\bm{M}^\star)-\|\bm{M}^\star- {\bm{M}}^{(1)}\|\\
 & \geq1-\|\bm{M}^\star_{2:n,2:n}- {\bm{M}}_{2:n,2:n}\|-\|\bm{M}^\star_{1:n,1}\|_{2}-\|\bm{M}^\star_{1,1:n}\|_{2}\\
 & \geq1-o(1),
\end{align*}
with high probability. Here, the last inequality follows since 
\begin{itemize}
\item it has been shown in the lecture notes that $\|\bm{M}^\star_{2:n,2:n}- {\bm{M}}_{2:n,2:n}\| \leq \|\bm{M}^\star- {\bm{M}} \| \ll1$
if $p\gg\frac{\log^{3}n}{n}$; 
\item $\|\bm{M}^\star_{1:n,1}\|_{2}=|v^\star_{1}|\cdot\|\bm{u}^\star\|_{2}=|v^\star_{1}|=\frac{|\tilde{v}_{1}|}{\|\tilde{\bm{v}}\|_{2}}\lesssim\sqrt{\frac{\log n}{n}}$
with high probability (because $|\tilde{v}_{1}|\lesssim\sqrt{\log n}$
and $\|\tilde{\bm{v}}\|_{2}=(1-o(1))\sqrt{n}$ with high probability);
\item Similarly, $\|\bm{M}^\star_{1,1:n}\|_{2}\lesssim\sqrt{\frac{\log n}{n}}$
with high probability.
\end{itemize}

Similarly, with high probability one has
\begin{align*}
\sigma_{2}( {\bm{M}}^{(1)}) & \leq\sigma_{2}(\bm{M}^\star) + \|\bm{M}^\star- {\bm{M}}^{(1)}\|
 %& \geq1-\|\bm{M}_{2:n,2:n}- {\bm{M}}_{2:n,2:n}\|-\|\bm{M}_{1:n,1}\|_{2}-\|\bm{M}_{1,1:n}\|_{2}\\
  \leq o(1).
\end{align*}
%
The above two bounds taken collectively give
\begin{align*}
  \sigma_{1}( {\bm{M}}^{(1)}) - \sigma_{2}( {\bm{M}}^{(1)}) & \geq 1 - o(1).
\end{align*}
%

% Preview source code for paragraph 4

As a result, applying Wedin's $\sin\bm{\Theta}$ Theorem gives
\begin{align}
\mathsf{dist}\big( {\bm{u}}, {\bm{u}}^{(1)}\big) & \lesssim\frac{\| {\bm{M}}- {\bm{M}}^{(1)}\|}{\sigma_{1}( {\bm{M}}^{(1)})-\sigma_{2}( {\bm{M}}^{(1)})}\lesssim\| {\bm{M}}- {\bm{M}}^{(1)}\|.\label{eq:dist-u-bound}
\end{align}
In addition, 
\begin{align*}
\| {\bm{M}}- {\bm{M}}^{(1)}\| & \leq\| {\bm{M}}_{1:n,1}\|_{2}+\| {\bm{M}}_{1,1:n}\|_{2}\\
 & \leq\| {\bm{M}}_{1:n,1}\|_{\infty}\sqrt{\| {\bm{M}}_{1:n,1}\|_{0}}+\| {\bm{M}}_{1,1:n}\|_{\infty}\sqrt{\| {\bm{M}}_{1,1:n}\|_{0}}\\
 & \lesssim\frac{1}{p}\|\bm{u}^\star\|_{\infty}\|\bm{v}^\star\|_{\infty}\sqrt{np}\\
 & \lesssim\frac{1}{p}\cdot\frac{\log n}{n}\cdot\sqrt{np}\asymp\frac{\log n}{\sqrt{np}},
\end{align*}
where we have used the fact that $\| {\bm{M}}_{1:n,1}\|_{0}\asymp np$
as long as $p\gg\frac{\log n}{n}$ (Chernoff bound). Substitution
into (\ref{eq:dist-u-bound}) gives
\begin{align}
\mathsf{dist}\big( {\bm{u}}, {\bm{u}}^{(1)}\big) & \lesssim\frac{\log n}{\sqrt{np}}.\label{eq:dist-u-bound-1}
\end{align}



}

\newpart{10}
 Recall that a more refined version of Wedin's $\sin\bm{\Theta}$ Theorem states that: for any two matrices $\bm{A}$ and $\bm{B}$, their leading left singular vectors (denoted by $\bm{u}_A$ and $\bm{u}_B$ respectively) satisfy
%
\[
\mathsf{dist}(\bm{u}_{A},\bm{u}_{B})\leq\frac{\max\big\{\big\|(\bm{A}-\bm{B})\bm{v}_{A}\big\|,\big\|(\bm{A}-\bm{B})^{\top}\bm{u}_{A}\big\|\big\}}{\sigma_{1}\big(\bm{A}\big)-\sigma_{2}(\bm{A})-\|\bm{A}-\bm{B}\|}
\]
%
where $\bm{v}_A$ is the leading right singular vector of $\bm{A}$. Can you use this refined version to derive a sharper upper bound on $\mathsf{dist}({\bm{u}}^{(1)},{\bm{u}})$?  Here, you can assume without proof that $\| {\bm{u}} \|_{\infty} , \| {\bm{u}}^{(1)} \|_{\infty}, \| {\bm{v}} \|_{\infty} , \| {\bm{v}}^{(1)} \|_{\infty} \lesssim \sqrt{ \frac{\log n}{n} }$ with high probability. 


\solution{
% Preview source code from paragraph 5 to 7
%
Applying the refined version of Wedin's $\sin\bm{\Theta}$ Theorem
gives
\begin{align}
\mathsf{dist}\big( {\bm{u}}, {\bm{u}}^{(1)}\big) & \lesssim\frac{\max\left\{ \|( {\bm{M}}- {\bm{M}}^{(1)}) {\bm{v}}^{(1)}\|_{2},\|( {\bm{M}}- {\bm{M}}^{(1)})^{\top} {\bm{u}}^{(1)}\|_{2}\right\} }{\sigma_{1}( {\bm{M}}^{(1)})-\sigma_{2}( {\bm{M}}^{(1)})}\label{eq:dist-u-bound-2}\\
 & \lesssim\max\left\{ \|( {\bm{M}}- {\bm{M}}^{(1)}) {\bm{v}}^{(1)}\|_{2},\|( {\bm{M}}- {\bm{M}}^{(1)})^{\top} {\bm{u}}^{(1)}\|_{2}\right\} .
\end{align}

To bound $\|( {\bm{M}}- {\bm{M}}^{(1)}) {\bm{v}}^{(1)}\|_{2}$,
we have
\begin{align*}
\|( {\bm{M}}- {\bm{M}}^{(1)}) {\bm{v}}^{(1)}\|_{2}\leq & \big| {\bm{M}}_{1,1:n} {\bm{v}}^{(1)}\big|+\| {\bm{M}}_{1:n,1}\|_{2}\big| {v}_{1}^{(1)}\big|.
\end{align*}
It has been shown above that $\| {\bm{M}}_{1:n,1}\|_{2}\lesssim\frac{\log n}{\sqrt{np}}$,
which together with the assumption $\| {\bm{v}}^{(1)}\|_{\infty}\lesssim\sqrt{\frac{\log n}{n}}$
gives
\[
\| {\bm{M}}_{1:n,1}\|_{2}\big| {v}_{1}^{(1)}\big|\lesssim\sqrt{\frac{\log^{3}n}{n^{2}p}}.
\]
In addition, given that $ {\bm{M}}_{1,1:n}$ and $ {\bm{v}}^{(1)}$
are statistically independent, we have
%\begin{align*}
%\mathsf{Var}\left(\big| {\bm{M}}_{1,1:n} {\bm{v}}^{(1)}\big|\right) & \leq\mathbb{E}\left[\big| {\bm{M}}_{1,1:n} {\bm{v}}^{(1)}\big|^{2}\right]\\
% & \leq\frac{1}{p}\sum_{i=1}^{n}\left(M_{1,i} {v}_{i}^{(1)}\right)^{2}\\
% & \leq\frac{1}{p}\|\bm{M}_{1,1:n}\|_{\infty}^{2}\| {\bm{v}}^{(1)}\|_{2}^{2}\\
% & =\frac{1}{p}\|\bm{M}_{1,1:n}\|_{\infty}^{2}\\
% & \lesssim\frac{1}{p}\frac{\log^{2}n}{n^{2}},
%\end{align*}

\begin{align*}
\mathsf{Var}\left(\big| {\bm{M}}_{1,1:n} {\bm{v}}^{(1)}\big|\right) & \leq\mathbb{E}\left[\big| {\bm{M}}_{1,1:n} {\bm{v}}^{(1)}\big|^{2}\right]\\
 & =\sum_{i=1}^{n} \mathbb{E}\left[  {M}^2_{1,i} ( {v}_{i}^{(1)})^2 \right] \\
  & =\sum_{i=1}^{n} \mathbb{E}\left[  {M}^2_{1,i} \right] \mathbb{E}\left[ ( {v}_{i}^{(1)})^2 \right] \\
 & =\sum_{i=1}^{n} \frac{1}{p} M^{\star2}_{1,i} \mathbb{E}\left[ ( {v}_{i}^{(1)})^2 \right] \\
 & \leq \frac{1}{p}\|\bm{M}^\star_{1,1:n} \|_{\infty}^{2} \mathbb{E}\left[ \sum_{i=1}^{n} ( {v}_{i}^{(1)})^2 \right] \\
 & = \frac{1}{p}\|\bm{M}^\star_{1,1:n} \|_{\infty}^{2} \\
 & \lesssim\frac{1}{p}\frac{\log^{2}n}{n^{2}},
\end{align*}
and hence by Chebyshev's inequality, 
\[
\big| {\bm{M}}_{1,1:n} {\bm{v}}^{(1)}\big|\lesssim\sqrt{\mathsf{Var}\left(\big| {\bm{M}}_{1,1:n} {\bm{v}}^{(1)}\big|\right)\log n}\lesssim\sqrt{\frac{\log^{3}n}{n^{2}p}}
\]
with high probability. In summary, 
\[
\|( {\bm{M}}- {\bm{M}}^{(1)}) {\bm{v}}^{(1)}\|_{2}\lesssim\sqrt{\frac{\log^{3}n}{n^{2}p}}.
\]

Similarly, 
\[
\|( {\bm{M}}- {\bm{M}}^{(1)})^{\top} {\bm{u}}^{(1)}\|_{2}\lesssim\sqrt{\frac{\log^{3}n}{n^{2}p}}.
\]
Putting the above bounds together, we obtain
\begin{align}
\mathsf{dist}\big( {\bm{u}}, {\bm{u}}^{(1)}\big) & \leq\sqrt{\frac{\log^{3}n}{n^{2}p}}.
\end{align}
This bound is significantly tighter than the one obtain in Part (a). 
 


}

\end{problem}


\begin{problem}{Community detection experiments}{20}
Consider the SBM model discussed in class. Fix the number $n$ of nodes in a graph to be $100$. Set $p=\frac{1+ \varepsilon}{2}$ and $q=\frac{1-\varepsilon}{2}$ for some quantity $\varepsilon \in [0,1/2]$. Generate a random graph and then use the spectral method to cluster the nodes. Please plot the mis-clustering rate vs. the probability gap $\varepsilon$. At the minimum, you should take 50 different values of $\varepsilon$ (with linear spacing) in $[0, 1/2]$. For each value of $\varepsilon$, you need to run the experiment with at least 200 Monte-Carlo trials to calculate the average mis-clustering rate across trials. 
\end{problem}




\end{document}
