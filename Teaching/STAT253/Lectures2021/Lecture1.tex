%\documentclass[letterpaper,draft]{beamer}
\documentclass[letterpaper,mathserif,handout]{beamer}
%\documentclass[letterpaper]{beamer}

%---multiple pages on one sheet, ADD for handout--
%\usepackage{pgfpages}
%\pgfpagesuselayout{4 on 1}[letterpaper, landscape, border shrink=1mm]
%-------------------------------------------------
\usepackage{amsmath,amsfonts}
%\usepackage[misc]{ifsym} % for the dice symbol \Cube{}
%\usepackage{booktabs}
%\usepackage{mdwlist}
%\usepackage{pgf,tikz}
%\usetheme{Copenhagen}
%\usetheme{warsaw}
\setbeamertemplate{navigation symbols}{}
\usepackage[english]{babel}
\def\ul{\underline}
% or whatever

\usepackage[latin1]{inputenc}
\subject{Talks}

\def\P{\mathbb{P}}
\def\p{\mathrm P}
\def\E{\mathbb E}
\def\Sum{\sum\nolimits}
\def\X{\mathfrak{X}}
\def\typo#1{\alert{#1}}
%-------------Answers------------
\def\Hide#1#2{\ul{~~~\onslide<#1>{\alert{#2}}~~~}}
\def\hide#1#2{\ul{~~\onslide<#1>{\alert{#2}}~~}}
\def\hid#1#2{\onslide<#1>{\alert{#2}}}
%------Centered Page Number------
\defbeamertemplate{footline}{centered page number}
{%
  \hspace*{\fill}%
  %\usebeamercolor[fg]{page number in head/foot}%
  %\usebeamerfont{page number in head/foot}%
  \small Lecture \chapnum\ - \insertframenumber%
  \hspace*{\fill}\vskip2pt%
}

%\usepackage{tikz}
%\usebackgroundtemplate{%
%\tikz\node[opacity=0.3] {\includegraphics[height=\paperheight,widht=\paperwidth]{ctanlion}};}

%\usebackgroundtemplate{%
%  %\rule{0pt}{\paperheight}%
%  \parbox[c][\paperheight][c]{\paperwidth}{\centering\includegraphics[width=.65\paperwidth]{UClogo.pdf}}
%  %\hspace*{\paperwidth}
%}

\def\chapnum{1}
%--------------------------------
\setbeamertemplate{footline}[centered page number]
\title{STAT253/317 Lecture \chapnum}
\date{}
\author{Cong Ma}
\begin{document}
% ----------------------------------------------------------------------
\begin{frame}
\maketitle\bigskip
\begin{center}
4.1 Introduction to Markov Chains
\end{center}
\end{frame}
% ----------------------------------------------------------------------
\begin{frame}{Stochastic Processes}
A stochastic process is a family of random variables $\{X_t: t\in\mathcal{T}\}$ such that
\begin{itemize}
\item For each $t\in\mathcal{T}$, $X_t$ is a random variable
\end{itemize}
The index set $\mathcal{T}$ can be discrete or continuous
\begin{itemize}\normalsize
\item $\mathcal{T}=\{0,1,2,3,4\}$
\item $\mathcal{T}=\mathbb{R}, \mathbb{R}^+, \mathbb{R}^2, \mathbb{R}^3$
\end{itemize}

Examples:
\begin{itemize}
\item Discrete Time Markov Chains \dotfill Chapter 4
\item Poisson Processes, Counting Processes \dotfill Chapter 5
\item Continuous Time Markov Chains \dotfill Chapter 6
\item Renewal Theory  \dotfill Chapter 7
\item Queuing Theory  \dotfill Chapter 8
\item Brownian Motion \dotfill Chapter 10
\end{itemize}
\end{frame}
% ----------------------------------------------------------------------
\begin{frame}{4.1 Introduction to Markov Chain}
Consider a stochastic process $\{X_n : n=0,1,2,\ldots\}$ taking values in a finite or countable set $\X$.
\begin{itemize}
\item $\X$ is called the {\bf state space}
\item If $X_n=i$, $i\in \X$, we say the process is in state $i$ at time $n$
\item Since $\X$ is countable, there is a 1-1 map from $\X$ to the set of non-negative integers $\{0,1,2,3,\ldots\}$\\
    From now on, we assume $\X=\{0,1,2,3,\ldots\}$
\end{itemize}
\begin{block}{Definition}
A stochastic process $\{X_n:\; n=0,1,2,\ldots\}$ is called a \structure{\bf Markov chain} if it has the following property:
\begin{align*}
&P(X_{n+1}=j \mid X_n=i,X_{n-1}=i_{n-1},\ldots,X_2=i_2,X_1=i_1,X_0=i_0)\\
&=P(X_{n+1}=j \mid X_n=i)
\end{align*}
for all states $i_0$, $i_1$, $i_2,\ldots, i_{n-1}$, $i$, $j\in \X$ and $n\ge 0$.\par
\end{block}
\end{frame}
% ----------------------------------------------------------------------
\begin{frame}{Transition Probability Matrix}
If $\alert{P(X_{n+1}=j|X_n=i)=P_{ij}}$ does not depend on $\alert{n}$, then the process $\{X_n:\; n=0,1,2,\ldots\}$ is called a \structure{\bf stationary Markov chain}. From now on, we consider stationary Markov chain only.\medskip\pause

$\alert{\{P_{ij}\}}$  is called the \structure{\bf transition probabilities}.\pause

The matrix
$$
\P=
\begin{pmatrix}
P_{00} & P_{01} & P_{02} & \cdots &P_{0j} & \cdots\\
P_{10} & P_{11} & P_{12} & \cdots &P_{1j} & \cdots\\
\vdots & \vdots & \vdots & \ddots &\vdots & \ddots\\
P_{i0} & P_{i1} & P_{i2} & \cdots &P_{ij} & \cdots\\
\vdots & \vdots & \vdots & \ddots &\vdots & \ddots
\end{pmatrix}
$$
is called the \structure{\bf transition probability matrix}. \pause

Naturally, the transition probabilities $\{P_{ij}\}$ satisfy the following
\begin{itemize}
\item $P_{ij}\ge 0$ for all $i,j$
\item Rows sums are 1: $\Sum_j P_{ij}=1$ for all $i$.
\end{itemize}
In other words, $\mathbb{P}\mathbf{1}=\mathbf{1}$, where $\mathbf{1}=(1,1,\ldots,1,\ldots)^\top$
\end{frame}


\begin{frame}{Example 1: construct Markov Chain from i.i.d.~sequence}
Let $\{ Y_n \}_{n \geq 0}$ be an i.i.d.~sequence. The following two stochastic 
processes $\{ X_n \}_{n \geq 0}$ are Markov chains

\begin{itemize}
	\item $X_n = Y_n$
	\item $X_n = \sum_{k = 0}^{n} Y_n$
\end{itemize}
\end{frame}



% ----------------------------------------------------------------------
%\begin{frame}{Example 1: Busy Phone Line}
%Consider the status of a phone line on discrete time intervals: 1, 2, \ldots.
%\begin{itemize}
%\item Suppose calls come in independently at constant rate: for each time interval, there is a probability \alert{$\alpha$} that one call comes in that interval.  Assume there is at most one call per interval.
%
%\item An incoming call can go through only if the line is free when the call comes in, and will occupy the line until the call ends.
%\item All unanswered calls are missed. (Cannot ``stay on the line'')
%\item Suppose the length of calls is fixed at 2 (time intervals).
%\end{itemize}
%Can you find a Markov chain in this model?
%\end{frame}
%% ----------------------------------------------------------------------
%
%
%
%\begin{frame}{Example 1: Busy Phone Line (Cont'd)}
%
%Let $X_n$ be the status (busy or free) of the phone line in the $n$th time interval.
%Is $\{X_n\}$ a Markov chain?\pause\bigskip
%
%{\bf Answer}: No. Observe that
%\begin{align*}
%P(X_3=\text{free}|X_2=\text{busy}, X_1=\text{free})&=0\\
%P(X_3=\text{free}|X_2=\text{busy}, X_1=\text{busy})&>0
%\end{align*}
%The distribution of $X_3$ depends on not just $X_2$ but also
%the $X_1$ and hence $\{X_n\}$ is NOT a Markov Chain.
%
%\end{frame}
%% ----------------------------------------------------------------------
%\begin{frame}{Example 1: Busy Phone Line (Cont'd)}
%
%Let $Y_n$ be the remaining time of the call in the $n$th time interval if the line is busy,
%and $Y_n=0$ if the line is free in the $n$th time interval. Is $\{Y_n\}$ a Markov chain?\pause
%
%\[
%Y_{n+1}=
%\begin{cases}
%Y_n - 1 & \text{if }Y_n>0\\
%2 & \text{with prob $\alpha$ if $Y_n=0$}\\
%0 & \text{with prob $1-\alpha$ if $Y_n=0$}
%\end{cases}
%\]
%The transition matrix is
%$$\P=\bordermatrix{%
% & 0   & 1 & 2\cr
%0& 1-\alpha & 0 & \alpha\cr
%1& 1   & 0 & 0\cr
%2& 0   & 1 & 0\cr
%}$$
%
%\end{frame}
%% ----------------------------------------------------------------------
%\begin{frame}{Example 1: Busy Phone Line (Cont'd)}
%Is $\{Y_n\}$ still a Markov chain if the lengths of calls are random:
%50\% of the calls are of length 1, 30\% of length 2 and 20\% of length 3?
%Assume the lengths of calls are independent of each other.\pause
%$$
%Y_{n+1}=
%\begin{cases}
%Y_n - 1 & \text{if }Y_n>0\\
%1 & \text{with prob $0.5\alpha$ if $Y_n=0$}\\
%2 & \text{with prob $0.3\alpha$ if $Y_n=0$}\\
%3 & \text{with prob $0.2\alpha$ if $Y_n=0$}\\ 0 & \text{with prob $1-\alpha$ if $Y_n=0$} \end{cases}
%$$\pause
%
%The transition matrix is
%$$\P=\bordermatrix{%
% & 0   & 1 & 2 & 3\cr
%0& 1-\alpha & 0.5\alpha & 0.3\alpha & 0.2\alpha\cr
%1& 1   & 0 & 0 & 0\cr
%2& 0   & 1 & 0 & 0\cr
%3& 0   & 0 & 1 & 0\cr
%}$$
%
%\end{frame}
% ----------------------------------------------------------------------
\begin{frame}{Example 2: Random Walk}
Consider the following random walk on integers
$$
X_{n+1}=
\begin{cases}
X_n + 1 &\text{with prob } p\\
X_n - 1 &\text{with prob } 1-p
\end{cases}
$$\pause
This is a Markov chain because given $X_n, X_{n-1}, X_{n-2}, \ldots$,
 the distribution of $X_{n+1}$ depends only on $X_n$ but not $X_{n-1}, X_{n-2}, \ldots.$\pause

The state space is
\begin{align*}
\X &= \{\cdots, -3, -2, -1, 0, 1, 2, 3,\cdots\}=\mathbb{Z}=\mbox{all integers}
\end{align*}\pause

The transition probability is
$$
P_{ij}=
\begin{cases}
p   & \mbox{if }j=i+1\\
1-p & \mbox{if }j=i-1\\
0   & \mbox{otherwise}
\end{cases}
$$
\end{frame}
% ----------------------------------------------------------------------
\begin{frame}{Example 2: Random Walk (Cont'd)}
\small
$$\P=
\bordermatrix{%
       & \cdots & -3  & -2  & -1  &  0  &  1  &  2  &  3   & \cdots\cr
\vdots & \ddots &\ddots&    &     &     &     &     &      &       \cr
-3     & \ddots &  0  &  p  &     &     &     &     &      &       \cr
-2     &        & 1\!-\!p &  0  &  p  &     &     &     &      &       \cr
-1     &        &     & 1\!-\!p &  0  &  p  &     &     &      &       \cr
 0     &        &     &     & 1\!-\!p &  0  &  p  &     &      &       \cr
 1     &        &     &     &     & 1\!-\!p &  0  &  p  &      &       \cr
 2     &        &     &     &     &     & 1\!-\!p &  0  &  p   &       \cr
 3     &        &     &     &     &     &     & 1\!-\!p &  0   & \ddots\cr
\vdots &        &     &     &     &     &     &     &\ddots& \ddots
}
$$
\end{frame}
% ----------------------------------------------------------------------
\begin{frame}{Example 3: Ehrenfest Diffusion Model}
Two containers $A$ and $B$, containing a sum of $\alert{K}$ balls.
At each stage, a ball is selected at random from the totality of $K$ balls, and move to the other container.
Let
\begin{align*}
X_0 &= \mbox{\# of balls in container $A$ in the beginning}\\
X_n &= \mbox{\# of balls in container $A$ after $n$ movements},\;n=1,2,3,\ldots
\end{align*}
$$
\X=\{\hid{2-}{0,1,2,\ldots,K}\}
$$

$$
P_{ij}=
\begin{cases}
\dfrac{i}{K}    & \mbox{if }j=i-1\\[10pt]
\dfrac{K-i}{K} & \mbox{if }j=i+1\\
0   & \mbox{otherwise}
\end{cases}
$$
\end{frame}


\begin{frame}{Joint Distribution of Random Variables in a Markov Chain}
Suppose $\{X_n:\; n=0,1,2,\ldots\}$ is a stationary Markov chain with
\begin{itemize}
\item state space $\X$ and
\item transition probabilities $\{P_{ij}: i,j\in\X\}$.
\end{itemize}
Define $\pi_0(i)=\p(X_0=i)$, $i\in\X$ to be the distribution of $X_0$.
\medskip

What is the joint distribution of $X_0, X_1, X_2$?\pause
\begin{align*}
&\p(X_0=i_0,X_1=i_1,X_2=i_2)\\
&=\p(X_0=i_0)\p(X_1=i_1|X_0=i_0)\p(X_2=i_2|X_1=i_1,X_0=i_0)\\
&=\p(X_0=i_0)\p(X_1=i_1|X_0=i_0)\p(X_2=i_2|X_1=i_1)\quad(\text{Markov})\\
&=\pi_0(i_0)P_{i_0i_1}P_{i_1i_2}
\end{align*}\pause

In general,
\begin{align*}
&\p(X_0=i_0,X_1=i_1,X_2=i_2,\ldots,X_{n-1}=i_{n-1},X_n=i_n)\\
&=\pi_0(i_0)P_{i_0i_1}P_{i_1i_2}\ldots P_{i_{n-1}i_n}
\end{align*}
\end{frame}
% ----------------------------------------------------------------------
%\begin{frame}{Example 4: Discrete Queuing Process}
%A line of customers await in front of 1 server.
%\begin{itemize}
%\item It takes one unit of time to serve 1 customer
%\item During each period of time, only 1 customer is served.
%\item If no customer awaits, the server idles
%\end{itemize}
%Let $\xi_n=\#$ of customers arriving in the $n$-th period.
%Suppose $\{\xi_n,n=0,1,2,\ldots\}$ are i.i.d. with
%\begin{align*}
%\p(\xi_n=k)&=a_k,\quad k=0,1,2,\ldots\\
%a_k \ge 0&\mbox{ for all }k,\quad\mbox{and } \Sum_{k=0}^{\infty}a_k = 1
%\end{align*}
%Let $X_n=\#$ of customers await during the $n$-th period, including the one being served. Then
%$$
%X_{n+1}=
%\begin{cases}
%X_{n}-1+\xi_{n}  & \mbox{if }X_n\ge 1\\
%\xi_{n}          & \mbox{if }X_n = 0
%\end{cases}
%$$
%\end{frame}
%% ----------------------------------------------------------------------
%\begin{frame}{Example 4: Discrete Queuing Process (Cont'd)}
%First, observe that $P_{0k}=P(X_{n+1}=k|X_n=0)=P(\xi_n=k)=a_k$ since $X_{n+1}=\xi_n$ if $X_n = 0$.\par\medskip
%
%Second, as $X_{n+1}=X_{n}-1+\xi_{n}=i-1+\xi_{n}$ if $X_n=i>0$,
%$X_{n+1}=k$ implies $\xi_n=k-i+1$ and hence,
%$$P_{ik}=P(X_{n+1}=k|X_n=i)=P(\xi_{n}=k-i+1)=a_{k-i+1}.$$\pause
%The transition probability matrix is then
%$$\P=
%\bordermatrix{%
%   &  0  &  1  &  2  &  3  &  4  &\cdots\cr
%0  & a_0 & a_1 & a_2 & a_3 & a_4 &\cdots \cr
%1  & a_0 & a_1 & a_2 & a_3 & a_4 &\cdots \cr
%2  &  0  & a_0 & a_1 & a_2 & a_3 &\cdots \cr
%3  &  0  &  0  & a_0 & a_1 & a_2 &\cdots \cr
%4  &  0  &  0  &  0  & a_0 & a_1 &\cdots \cr
%\vdots&\vdots&\vdots&\vdots&\vdots&\vdots& \ddots
%}
%$$
%\end{frame}
\end{document} 