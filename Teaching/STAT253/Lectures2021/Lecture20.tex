%\documentclass[letterpaper,draft]{beamer}
%\documentclass[letterpaper,handout]{beamer}
\documentclass[letterpaper]{beamer}

%---multiple pages on one sheet, ADD for handout--
%\usepackage{pgfpages}
%\pgfpagesuselayout{4 on 1}[letterpaper, landscape, border shrink=1mm]
%-------------------------------------------------
\usepackage{amsmath,amsfonts}
%\usepackage{booktabs}
%\usepackage{mdwlist}
\usepackage{amsfonts}
%\usetheme{Copenhagen}
%\usetheme{warsaw}
\setbeamertemplate{navigation symbols}{}
\usepackage[english]{babel}
\def\ul{\underline}
% or whatever

\usepackage[latin1]{inputenc}
\subject{Talks}

\def\Sum{\sum\nolimits}
\def\Prod{\prod\nolimits}
\def\p{\mathrm P}
\def\E{\mathbb E}
\def\V{\mathrm{Var}}
\def\CV{\mathrm{Cov}}
\def\X{\mathcal{X}}
\def\dt{\Delta}
\def\typo#1{\alert{#1}}
%-------------Answers------------
\def\Hide#1#2{\ul{~~~\onslide<#1>{\alert{#2}}~~~}}
\def\hide#1#2{\ul{~~\onslide<#1>{\alert{#2}}~~}}
\def\hid#1#2{\onslide<#1>{\alert{#2}}}
%------Centered Page Number------
\defbeamertemplate{footline}{centered page number}
{%
  \hspace*{\fill}%
  %\usebeamercolor[fg]{page number in head/foot}%
  %\usebeamerfont{page number in head/foot}%
  \small Lecture \chapnum\ - \insertframenumber%
  \hspace*{\fill}\vskip2pt%
}

%\usepackage{tikz}
%\usebackgroundtemplate{%
%\tikz\node[opacity=0.3] {\includegraphics[height=\paperheight,widht=\paperwidth]{ctanlion}};}

%\usebackgroundtemplate{%
%  %\rule{0pt}{\paperheight}%
%  \parbox[c][\paperheight][c]{\paperwidth}{\centering\includegraphics[width=.65\paperwidth]{UClogo.pdf}}
%  %\hspace*{\paperwidth}
%}

\def\chapnum{20}
%--------------------------------
\setbeamertemplate{footline}[centered page number]

\title{STAT253/317 Winter 2022 Lecture \chapnum} \date{} \author{Cong Ma}
\begin{document}
% ----------------------------------------------------------------------
\begin{frame}\maketitle
\bigskip

\begin{center}\large
\begin{tabular}{ll}
8.2.2 & Steady-State Probabilities\\
8.5 & The System $M/G/1$
\end{tabular}
\end{center}
\end{frame}
% ----------------------------------------------------------------------
\begin{frame}{8.2.2. Steady-State Probabilities}
For a general queueing model, we are interested in three different limiting probabilities:
\begin{align*}
P_n &=\lim_{t\to\infty}\p(X(t)=n),\\[-3pt]
    &\qquad\mbox{where $X(t)=\#$ of customers in the system at time $t$}\\
a_n &=\mbox{proportion of customers arrive finding $n$ in the system}\\
d_n &=\mbox{proportion of customers depart leaving $n$ behind in the system}
\end{align*}
Here we assume they exist.

Though the three are defined differently, the latter two are identical in most of the queueing models.\bigskip

\textbf{Proposition 8.1} In any system in which customers arrive and depart one at a time
$$
\mbox{the rate at which arrivals find }n = \mbox{the rate at which departures leave }n
$$
and
$$a_n = d_n$$
\end{frame}
% ----------------------------------------------------------------------
\begin{frame}{Proof of Proposition 8.1}
Let
\begin{align*}
N_{i,j}(t)&=\mbox{number of times the number of customers in the system}\\
&\quad\mbox{ goes from $i$ to $j$ by time }t\\
A(t)&=\mbox{number of customers arrived by time }t\\
D(t)&=\mbox{number of customers departed by time }t
\end{align*}
Note that an arrival will see $n$ in the system whenever the number in the system
goes from $n$ to $n + 1$; similarly, a departure will leave behind $n$ whenever the
number in the system goes from $n + 1$ to $n$.
Thus we know
\begin{align*}
&\mbox{the rate at which arrivals find }n = \lim_{t\to\infty}\frac{N_{n,n+1}(t)}{t}\\
&\mbox{the rate at which departures leave }n = \lim_{t\to\infty}\frac{N_{n+1,n}(t)}{t}\\
&a_n=\lim_{t\to\infty}\frac{N_{n,n+1}(t)}{A(t)},\quad d_n=\lim_{t\to\infty}\frac{N_{n+1,n}(t)}{D(t)}
\end{align*}
\end{frame}
% ----------------------------------------------------------------------
\begin{frame}{Proof of Proposition 8.1 (Cont'd)}
Since between any two transitions from $n$ to $n+1$, there must be one from
$n + 1$ to $n$, and vice versa, we have
$$N_{n,n+1}(t)=N_{n+1,n}(t)\pm 1\quad \mbox{for all }t.$$
Thus
\begin{align*}
\mbox{rate at which arrivals find }n
&= \lim_{t\to\infty}\frac{N_{n,n+1}(t)}{t}\\
&= \lim_{t\to\infty}\frac{N_{n+1,n}(t)\pm 1}{t}\\
&=\mbox{rate at which departures leave }n
\end{align*}
\end{frame}
% ----------------------------------------------------------------------
\begin{frame}{Proof of Proposition 8.1 (Cont'd)}
For $a_n$ and $d_n$, obviously $A(t)\ge D(t)$ and hence
\[
\lim_{t\to\infty}\frac{A(t)}{t}\ge\lim_{t\to\infty}\frac{D(t)}{t}
\]
Combining with the fact $\lim_{t\to\infty}\frac{N_{n,n+1}(t)}{t}
= \lim_{t\to\infty}\frac{N_{n+1,n}(t)}{t}$ we just shown, we obtain
\[
a_n%=\lim_{t\to\infty}\frac{N_{n,n+1}(t)}{A(t)}
=\lim_{t\to\infty}\frac{N_{n,n+1}(t)\hid{2-}{/t}}{A(t)\hid{2-}{/t}}
\le
\lim_{t\to\infty}\frac{N_{n+1,n}(t)\hid{2-}{/t}}{D(t)\hid{2-}{/t}}=d_n
\]

There are two possibilities:
\begin{itemize}
\item if $\displaystyle\lim_{t\to\infty}A(t)/t=\lim_{t\to\infty}D(t)/t$, then obviously $a_n=d_n$ for all $n$
\item if $\displaystyle\lim_{t\to\infty}A(t)/t>\lim_{t\to\infty}D(t)/t$, then the queue size will go to infinity, implying that
$a_n = d_n = 0.$ The equality is still valid.
\end{itemize}
\end{frame}
% ----------------------------------------------------------------------
\begin{frame}{Example 8.1}
Here is an example where $P_n\neq a_n$. Consider a queueing model in which
\begin{itemize}
\item service times $= 1$, always
\item interarrival times  are always $> 1$ [e.g., Uniform(1.5,2)].
\end{itemize}
Hence, as every arrival finds the system empty and every departure leaves it empty, we have
$$a_0 = d_0 = 1$$
However,
$P_0 \neq 1$
as the system is not always empty of customers.\bigskip


\end{frame}
% ----------------------------------------------------------------------
\begin{frame}{PASTA}

\textbf{Proposition 8.2}\;\ (PASTA Principle)
\begin{center}
\fbox{\ul{P}oisson \ul{A}rrivals \ul{S}ee \ul{T}ime \ul{A}verages}
\end{center}
If the arrival process is Poisson, then
$$P_n=a_n,$$
and hence $P_n=d_n.$


\begin{itemize}
\item By time $T$, the total amount of time there are $n$ customers in the system is about $P_nT$
\item Regardless of how many customers in the system, Poisson arrivals always arrive at rate $\lambda$. Thus
by time $T$, the total number of arrivals that find $n$ in the system  is $\approx \lambda P_nT$.
\item the overall number of customers arrived by time $T$ is $\approx\lambda T$
\item the proportion of arrivals that find the system in state $n$ is
$$a_n=\frac{\lambda P_n T}{\lambda T} = P_n$$
\end{itemize}
\end{frame}


\begin{frame}{Example 5.5 (M/M/1 Queueing w/ Finite Capacity)}
\begin{itemize}
\item single-server service station. Service times are i.i.d. $\sim Exp(\mu)$
\item Poisson arrival of customers with rate $\lambda$
\item Upon arrival, a customer would
\begin{itemize}\normalsize
\item go into service if the server is free (queue length $=0$)
\item join the queue if 1 to $N-1$ customers in the station, or
\item \alert{walk away} if $N$ or more customers in the station
\end{itemize}
\end{itemize}
{\bf Q}: What fraction of potential customers are lost?\par\medskip

Let $X(t)$ be the number of customers in the station at time $t$. \smallskip

$\{X(t), ~t\ge 0\}$ is a birth-death process with the birth and death rates below
$$
\mu_n=
\begin{cases}
0   & \text{if }n = 0\\
\mu & \text{if }n\ge 1
\end{cases}\quad\text{and}\quad
\lambda_n =
\begin{cases}
\lambda&\text{if $0\le n<N$}\\
0& \text{if }n\ge N
\end{cases}
$$\end{frame}
% ----------------------------------------------------------------------
\begin{frame}{Example 5.5 (M/M/1 Queueing w/ Finite Capacity)}
Solving $\lambda_n P_n= \mu_{n+1}P_{n+1}$ for the limiting distribution
\begin{align*}
P_1&=(\lambda/\mu)P_0\\
P_2&=(\lambda/\mu)P_1=(\lambda/\mu)^2P_0\\
&\vdots\\
P_i&=(\lambda/\mu)^iP_0, & i=1,2,\ldots,N
\end{align*}
Plugging $P_i=(\lambda/\mu)^iP_0$ into $\sum_{i=0}^N P_i=1$, one can solve for $P_0$ and get
$$
P_i=\frac{1-\lambda/\mu}{1-(\lambda/\mu)^{N+1}}(\lambda/\mu)^{i}
$$
Answer: The fraction of customers lost is $P_N=\frac{1-\lambda/\mu}{1-(\lambda/\mu)^{N+1}}(\lambda/\mu)^{N}$
\end{frame}

\begin{frame}
	\vfill
	\center
	\bf \large M/G/1
	\vfill
\end{frame}

% ----------------------------------------------------------------------
\begin{frame}{$M/G/1$}
The $M/G/1$ model assumes
\begin{itemize}
\item Poisson arrivals at rate $\lambda$;
\item i.i.d service times with a general distribution G, $S_i\sim G$;
\item a single server; and
\item first come, first serve
\end{itemize}
A necessary condition for an $M/G/1$ to be stable is that the mean of service time $\E[S_n]$ must satisfies
$$\lambda\E[S_n] < 1.$$
This condition is necessary. Otherwise if
\begin{align*}
&\mbox{the average service time }\E[S_n] \\
&>\;\mbox{the average interarrival time of customers }1/\lambda,
\end{align*}
the queue will become longer and longer and the system will ultimately explode.
\end{frame}
% ----------------------------------------------------------------------
\begin{frame}{A Markov Chain embedded in $M/G/1$}
Let $X(t)=\#$ of customers in the system at time $t$.

Unlike $M/M/k$ or $M/M/\infty$ systems, the process $\{X(t),t\ge 0\}$ in a $M/G/1$ system is NOT a continuous time Markov chain.% because the service time is not exponential. %Future events will depend on the current service time.
\par\medskip

Fortunately, there is a discrete-time Markov chain embedded in an $M/G/1$ system.

 
Let
\begin{align*}
Y_0 &=0\\
Y_n &=\#\mbox{ of customers in the system}\\
&\quad\mbox{ leaving behind at the $n$th departure},\; n\ge 1\\
\end{align*}


\begin{center}
\fbox{$\{Y_n, n\ge 0\}$ is a Markov chain.}
\end{center}
\end{frame}

\begin{frame}{A Markov Chain embedded in $M/G/1$ (Cont'd)}


To see this, let us define
\begin{align*}
A_n &=\#\mbox{ of customers that enter the system}\\
&\quad\mbox{ during the service time of the $n$th customer},\; n\ge 1
\end{align*}
Observed that $\{Y_n, n\ge 0\}$ and $\{A_n,n\ge 1\}$ are related as follows
$$
Y_{n+1} =A_{n+1}+(Y_n-1)_+=
\begin{cases}
Y_n-1+A_{n+1} & \mbox{if } Y_n>0\\
A_{n+1} & \mbox{if } Y_n=0
\end{cases}
$$

Example: $Y_1 = A_1$, $Y_2 = A_2 + (Y_1 - 1)_+$ \\

%Need to show $A_n$'s are i.i.d.
\end{frame}
% ----------------------------------------------------------------------
\begin{frame}{A Markov Chain embedded in $M/G/1$ (Cont'd)}

Recall that $S_n$ denotes the length of time to serve the $n$th customer.\par\smallskip

Given $S_n$, $A_n$ is Poisson with mean $\lambda S_n$.
From this we can conclude that $A_1,A_2,\ldots$ are i.i.d. since
\begin{itemize}
\item the service times $S_1, S_2,\ldots$ are i.i.d., and
\item there is only 1 server, the service times of different customers are disjoint, and the number of events occurred in disjoint intervals are independent in a Poisson process.
\end{itemize}
That $\{A_n,n\ge 1\}$ are i.i.d. and $Y_n$ is independent of $A_{n+1}$ implies that $Y_n$ forms a Markov chain.

%Recall we have seen this Markov chain in Lecture 1 and in HW4.
\end{frame}
% ----------------------------------------------------------------------
\begin{frame}{Transition probabilities of the Markov chain}
Moreover, as $A_n$ given $S_n$ is Poisson with mean $\lambda S_n$, we can find the distribution of $A_n$
\begin{align*}
\alpha_k = \p(A_n=k)&=\int_0^{\infty} \p(A_n=k|S_n=y)G(dy)\\
&=\int_0^{\infty} \frac{(\lambda y)^k}{k!}e^{-\lambda y}G(dy)
\end{align*}
from which we can find the transition probability $P_{ij}$ for the Markov chain $\{Y_n, n\ge 0\}$:
\begin{align*}
P_{ij} &= \p(Y_{n+1}=j|Y_n=i)=\p(A_{n+1}=j-(i-1)^+)\\
&=
\begin{cases}
\alpha_j,       & \mbox{if }i=0\\
\alpha_{j-i+1}, & \mbox{if }i\ge 1, j\ge i-1\\
0 &\mbox{if }i\ge 1, j<i-1
\end{cases}
\end{align*}
We can show that the Markov chain is irreducible and aperiodic and has a limiting distribution if and only if $\lambda \E[S_1]<1.$
\end{frame}
% ----------------------------------------------------------------------
\begin{frame}{Idle Periods in $M/G/1$}
Using the equation $Y_{n+1} =A_{n+1}+(Y_n-1)^+$, we can find many properties of the Markov chain. First write the equation as
$$
Y_{n+1}=A_{n+1}+Y_n-1+\mathbf{1}_{\{Y_n=0\}}
$$
Taking expectations we get
\[\E[Y_{n+1}]=\underbrace{\E[A_{n+1}]}_{{\alert{=\lambda\E[S]}}}+\E[Y_n]-1+\p(Y_n=0)\]
where $\E[A_{n+1}]=\lambda\E[S_{n+1}]$ since $A_{n+1}$ given $S_{n+1}$ is Poisson with mean $\lambda S_{n+1}$ and
$\E[S_{n+1}]=\E[S]$ since $S_i$'s are i.i.d.\medskip

Let $n\to\infty$, since the MC has a limiting distribution, we have $\lim_{n\to\infty}\E[Y_{n+1}]=\lim_{n\to\infty}\E[Y_n]$ and from which we can get
$$
\lim_{n\to\infty}\p(Y_n=0)=1-\lambda\E[S]
$$
By the PASTA principle, $\lim_{n\to\infty}\p(Y_n=0)=d_0=P_0$ is also the long-run proportion of time that the system is idle.
\end{frame}
% ----------------------------------------------------------------------
\begin{frame}{Length of Busy Periods in $M/G/1$}
As in a birth \& death queueing model,
there is a alternating renewal process embedded in an $M/G/1$ system.
We say a renewal occurs if the system become empty, then the system idles for a period of time until the next customer enters the system, and then a busy period begins until the system become empty again.

Using the alternating renewal theory, the long-run proportion of time that the system is empty is
$$
\frac{\E[\mbox{Idle}]}{\E[\mbox{Idle}]+\E[\mbox{Busy}]},
$$
and we just derived that it is $\lim_{t\to\infty}\p(X(t)=0)=1-\lambda\E[S]$.
Since the length of an idle period $\sim Exp(\lambda)$, we have $\E[\mbox{Idle}]=1/\lambda.$
In summary, we have that
$$
1-\lambda\E[S]=\frac{1/\lambda}{(1/\lambda)+\E[\mbox{Busy}]}\quad\Rightarrow\quad
\E[\mbox{Busy}]=\frac{\E[S]}{1-\lambda\E[S]}
$$
\end{frame}
% ----------------------------------------------------------------------
\begin{frame}{$L$ of $M/G/1$ (Cont'd)}
By the PASTA principle, we know $\displaystyle\lim_{n\to\infty}\E[Y_n]=\lim_{t\to\infty}\E[X(t)]=L$.
From the equation $Y_{n+1}=A_{n+1}-1+Y_n+\mathbf{1}_{\{Y_n=0\}}$, we have
\begin{align}
&\V(Y_{n+1})\cr
={}&\V(A_{n+1}-1+Y_n+\mathbf{1}_{\{Y_n=0\}})\cr
={}&\V(A_{n+1})+\V(Y_n+\mathbf{1}_{\{Y_n=0\}})\qquad(\mbox{$A_{n+1}$ and $Y_n$ are indep.})\cr
={}&\V(A_{n+1})+\V(Y_n)\cr
&\quad+2\CV(Y_n,\mathbf{1}_{\{Y_n=0\}})+\V(\mathbf{1}_{\{Y_n=0\}}),&\label{eq:var}
\end{align}
in which

\vspace{-18pt}
\begin{align}
\V(\mathbf{1}_{\{Y_n=0\}})&=\p(Y_n=0)(1-\p(Y_n=0))\label{eq:V1}\\
\CV(Y_n,\mathbf{1}_{\{Y_n=0\}})
&=\E[{Y_n\mathbf{1}_{\{Y_n=0\}}}]-\E[Y_n]\p(Y_n=0)\cr
&=-\E[Y_n]\p(Y_n=0)\label{eq:cov}\\
\V(A_n)&=\E[\V(A_n|S_n)]+\V(\E[A_n|S_n])\cr
&=\E[\lambda S_n]+\V(\lambda S_n)\cr
&=\lambda \E[S]+\lambda^2\V(S)\label{eq:ES}
\end{align}

\end{frame}
% ----------------------------------------------------------------------
\begin{frame}{$L$ of $M/G/1$ (Cont'd)}
Plugging in \eqref{eq:V1} \eqref{eq:cov} \eqref{eq:ES} into \eqref{eq:var}, letting $n\to\infty$, we have
\begin{align*}
\lim_{n\to\infty}\V(Y_{n+1})&=\lambda \E[S]+\lambda^2\V(S)+\lim_{n\to\infty}\V(Y_n)\\
&\quad-2\lim_{n\to\infty}\E[Y_n]\p(Y_n=0)\\
&\quad+\lim_{n\to\infty}\p(Y_n=0)(1-\p(Y_n=0))\\
&=\lambda \E[S]+\lambda^2\V(S)+\lim_{n\to\infty}\V(Y_n)\\
&\quad-2\lim_{n\to\infty}\E[Y_n](1-\lambda\E[S])+(1-\lambda\E[S])\lambda\E[S]
\end{align*}
Again since the MC has a limiting distribution, we have $\lim_{n\to\infty}\V[Y_{n+1}]=\lim_{n\to\infty}\V[Y_n]$,
and can get
\begin{align*}
\lim_{n\to\infty}\E[Y_n]&=\frac{\lambda \E[S]+\lambda^2\V(S)}{2(1-\lambda\E[S])}+\frac{\lambda\E[S]}{2}\\
&=\frac{\lambda^2\E[S^2]}{2(1-\lambda\E[S])}+\lambda\E[S]\quad (\text{since }\V(S)=\E[S^2]-(\E[S])^2)
\end{align*}
\end{frame}
% ----------------------------------------------------------------------
\begin{frame}{$L$ of $M/G/1$ (Cont'd)}

From the cost identity $L=\lambda_a W$ and $L_Q=\lambda_a W_Q$, and that $\lambda_a=\lambda$, we have
\begin{align*}
L&=\frac{\lambda^2\E[S^2]}{2(1-\lambda\E[S])}+\lambda\E[S]\\
W&=L/\lambda = \frac{\lambda\E[S^2]}{2(1-\lambda\E[S])}+\E[S]\\
W_Q &= W-\E[S]=\frac{\lambda\E[S^2]}{2(1-\lambda\E[S])}\\
L_Q &=\lambda W_Q=\frac{\lambda^2\E[S^2]}{2(1-\lambda\E[S])}
\end{align*}
Since $\E[S^2]=(\E[S])^2+\V(S)$, from the equations above we see for fixed mean service time $\E[S]$,
\begin{center}
$L$, $L_Q$, $W$, and $W_Q$ all increase as $\V(S)$ increases.
\end{center}
\end{frame}
% ----------------------------------------------------------------------
\begin{frame}{Example}
For an $M/M/1$ system, we have shown that
if the service time is exponential with mean $1/\mu$ that the average waiting time is
$$W=\frac{1}{\mu-\lambda}$$

If the service time is exactly $1/\mu$, the average waiting time can be reduced to
\begin{align*}
W&=\frac{\lambda\E[S^2]}{2(1-\lambda\E[S])}+\E[S]=\frac{\lambda/\mu^2}{2(1-\lambda/\mu)}+1/\mu=\frac{1}{\mu-\lambda}-\frac{\lambda/\mu}{2(\mu-\lambda)}
\end{align*}
For example, for $\lambda=1/12$, $\mu=1/8$
$$
W=\begin{cases}
24 & \mbox{for } M/M/1\\
16 & \mbox{if service time is exactly $1/\mu=8$}
\end{cases}
$$
For $\lambda=1/10$, $\mu=1/8$
$$
W=\begin{cases}
40 & \mbox{for } M/M/1\\
24 & \mbox{if service time is exactly $1/\mu=8$}
\end{cases}
$$
\end{frame}
\end{document}
% ----------------------------------------------------------------------
\begin{frame}{8.5.1 Work and Another Cost Identity}
For an arbitrary queueing system, the \structure{\em work} in the system at any time $t$ is defined as
\begin{align*}
V(t)&=\mbox{the sum of the remaining service times of}\\
    &\quad\quad\mbox{all customers in the system at time }t.
\end{align*}
For example, suppose the $i$th customer enters the system at time $T_i$, spends $W^Q_i$ units of time in queue, and his service time is $S_i$. That is, the $i$th customer will start being served at time $T_i+W^Q_i$, and leave the system at time $T_i+W^Q_i+S_i$. At time $t$ the remaining service time $R_i(t)$ of the $i$th customer is
$$
R_i(t)=
\begin{cases}
0          &\mbox{if } t < T_i\\
S_i        &\mbox{if }T_i\le t < T_i+W^Q_i\\
S_i-(t-T_i-W^Q_i)&\mbox{if }T_i+W^Q_i\le t\le T_i+S_i+W^Q_i\\
0          &\mbox{if }t > T_i+S_i+W^Q_i
\end{cases}
$$
Then $V(t)=\Sum_{i=1}^{\infty}R_i(t)$
\end{frame}
% ----------------------------------------------------------------------
\begin{frame}{Another Cost Identity}
%Define $V$ as the (time) average work in the system
Recall the fundamental cost equation:
\begin{align*}
&\mbox{average rate at which the system earns}\\
&= \lambda_a\times \mbox{average amount an entering customer pays}
\end{align*}
Consider the following cost rule:
\begin{quote}
Each customer pays at a rate of $R(t)$/unit time
when his remaining service time is $R(t)$,
whether he is in queue or in service
\end{quote}

Then the amount the system earns by time $t$ is
$$\int_0^t\sum_{i=1}^{\infty}R_i(s)ds=\int_0^t V(s)ds$$
and hence the average rate the system earns is
$$
V = \lim_{t\to\infty}\frac{\int_0^t V(s)ds}{t}
$$
$V$ is called \structure{\em the (time) average work in the system}.
\end{frame}
% ----------------------------------------------------------------------
\begin{frame}{Another Cost Identity (Cont'd)}
The total amount paid by the $i$th customer is
\begin{align*}
\int_0^{\infty} R_i(s)ds&=\int_0^{T_i}0 ds + \int_{T_i}^{T_i+W^Q_i}S_ids\\
&\quad+\int_{T_i+W^Q_i}^{T_i+W^Q_i+S_i}S_i-(t-T_i-W^Q_i)ds + \int_{T_i+W^Q_i+S_i}^{\infty}0 ds\\
&=S_iW^Q_i+S_i^2/2
\end{align*}
The average amount an entering customer pays is thus
$$
\lim_{n\to\infty}\sum_{i=1}^{n}\frac{S_iW^Q_i+S_i^2/2}{n}=\E[S_iW^Q_i]+\E[S_i^2]/2
$$
From the Cost Equation, we have a new cost identity
$$
V=\lambda_a\E[S_iW^Q_i]+\lambda_a\E[S_i^2]/2
$$
\end{frame}
% ----------------------------------------------------------------------
\begin{frame}{Application of Work to $M/G/1$}
Recall the new cost identity $V=\lambda_a\E[S_iW^Q_i]+\lambda_a\E[S_i^2]/2.$
If a customer's service time $S_i$ is independent of his wait in queue $W^Q_i$,
then we have
$$
V=\lambda_a\E[S_i]\E[W^Q_i]+\lambda_a\E[S_i^2]/2.
$$
This condition is true for $M/G/1$ and most of the queueing models. (There are exceptions.)
\medskip

For an arbitrary customer in an $M/G/1$ system, since there is only a single server
\begin{align*}
&\mbox{the $i$th customer's wait in queue }W^Q_i\\
&=\mbox{ work in the system when he arrives } V(T_i)
\end{align*}
\end{frame}
% ----------------------------------------------------------------------
\end{document}
\begin{frame}{}
\end{frame}
% ----------------------------------------------------------------------
\begin{frame}
\end{frame}
% ----------------------------------------------------------------------
