%\documentclass[letterpaper,draft]{beamer}
\documentclass[letterpaper,handout]{beamer}
%\documentclass[letterpaper]{beamer}

%---multiple pages on one sheet, ADD for handout--
%\usepackage{pgfpages}
%\pgfpagesuselayout{4 on 1}[letterpaper, landscape, border shrink=1mm]
%-------------------------------------------------
\usepackage{amsmath,amsfonts}
%\usepackage{booktabs}
%\usepackage{mdwlist}
\usepackage{amsfonts}
%\usetheme{Copenhagen}
%\usetheme{warsaw}
\setbeamertemplate{navigation symbols}{}
\usepackage[english]{babel}
\def\ul{\underline}
% or whatever

\usepackage[latin1]{inputenc}
\subject{Talks}

\def\Sum{\sum\nolimits}
\def\Prod{\prod\nolimits}
\def\p{\mathrm P}
\def\E{\mathbb E}
\def\V{\mathrm Var}
\def\X{\mathcal{X}}
\def\typo#1{\alert{#1}}
%-------------Answers------------
\def\Hide#1#2{\ul{~~~\onslide<#1>{\alert{#2}}~~~}}
\def\hide#1#2{\ul{~~\onslide<#1>{\alert{#2}}~~}}
%------Centered Page Number------
\defbeamertemplate{footline}{centered page number}
{%
  \hspace*{\fill}%
  %\usebeamercolor[fg]{page number in head/foot}%
  %\usebeamerfont{page number in head/foot}%
  \small Lecture \chapnum\ - \insertframenumber%
  \hspace*{\fill}\vskip2pt%
}

%\usepackage{tikz}
%\usebackgroundtemplate{%
%\tikz\node[opacity=0.3] {\includegraphics[height=\paperheight,widht=\paperwidth]{ctanlion}};}

%\usebackgroundtemplate{%
%  %\rule{0pt}{\paperheight}%
%  \parbox[c][\paperheight][c]{\paperwidth}{\centering\includegraphics[width=.65\paperwidth]{UClogo.pdf}}
%  %\hspace*{\paperwidth}
%}

\def\chapnum{13}
%--------------------------------
\setbeamertemplate{footline}[centered page number]

\title{STAT253/317 Lecture \chapnum} \date{} \author{Yibi Huang}

\begin{document}
% ----------------------------------------------------------------------
%\begin{frame}\maketitle
%\begin{center}
%Chapter 6 \ Continuous-Time Markov Chains\par\medskip
%
%6.5. Limiting Probabilities\par
%6.6. Time Reversibility
%\end{center}
%\end{frame}
% ----------------------------------------------------------------------
\begin{frame}{STAT253/317 Lecture \chapnum}
\begin{block}{\bf 6.5. Limiting Probabilities}
\textbf{Definition.} Just like discrete-time Markov chains, if the probability that
a continuous-time Markov chain will be in state $j$ at time $t$, $P_{ij}(t)$, converges to
a limiting value $P_j$ independent of the initial state $i$, for all $i\in\X$
$$
P_j = \lim_{t\to\infty}P_{ij}(t) \typo{>0}
$$
then we say $P_j$ is the \structure{\em limiting probability} of state $j$.
If $P_j$ exists for all $j\in\X$, we say $\{P_j\}_{j\in\X}$ is the \structure{\em limiting distribution} of the process.

\bigskip

\textbf{Remark.}
If $\lim_{t\to\infty}P_{ij}(t)$ exists, we must have
$$\lim_{t\to\infty}P'_{ij}(t)=0.$$
\end{block}
\end{frame}

\begin{frame}{Balanced equations}
 Recall the forward equations $\mathbf{P}'(t) = \mathbf{P}(t) \mathbf{Q}$ \\
 
 \medskip
 If you set $t \to \infty$, you have 
 \begin{align*}
 	0 = p^\top \mathbf{Q}, 
 \end{align*}
 where $p = (P_1, P_2, \ldots)^\top$
 
 This is the same as saying that 
 $$
\nu_jP_j= \sum_{k\in\X,k\neq j}P_k q_{kj} \quad\mbox{for all }j\in\X
$$
\end{frame}
% ----------------------------------------------------------------------
%\begin{frame}
%Recall the forward equations are
%$$P'_{ij} (t) =\left(\sum_{k\in\X,k\neq j}P_{ik}(t)q_{kj}\right)-\nu_jP_{ij} (t)$$
%
%If we let $t\to\infty$, and assume that we can interchange limit and summation, we obtain
%$$
%\begin{array}{c@{\;}c@{\;}c@{\;}c@{\;}c}
%\displaystyle{\lim_{t\to\infty}}P'_{ij} (t)
%&=&\displaystyle{\lim_{t\to\infty}}\left(\sum_{k\in\X,k\neq j}P_{ik}(t)q_{kj}\right)&-&\nu_jP_{ij} (t)\\
%\downarrow && \downarrow && \downarrow\\
%0&=&\sum_{k\in\X,k\neq j}P_k q_{kj}&-&\nu_jP_j
%\end{array}
%$$
%Hence we get the \structure{\em balanced equations}.
%$$
%\nu_jP_j= \sum_{k\in\X,k\neq j}P_k q_{kj} \quad\mbox{for all }j\in\X
%$$
%\end{frame}
% ----------------------------------------------------------------------
%\begin{frame}{Taking $\lim_{t\to\infty}$ of the Backward Equation Leads to $\ldots$}
%Taking the limit $t\to\infty$ of the Backward Equation $\mathbf{P}'(t) =  \mathbf{Q} \mathbf{P}(t)$
%$$
%\begin{array}{c@{\;}c@{\;}c@{\;}c@{\;}c}
%\displaystyle{\lim_{t\to\infty}}P'_{ij} (t)
%&=&\displaystyle{\lim_{t\to\infty}}\left(\sum_{k\in\X,k\neq i}q_{ik}P_{kj}(t)\right)&-&\nu_jP_{ij} (t)\\
%\downarrow && \downarrow && \downarrow\\
%0&=&\sum_{k\in\X,k\neq j} q_{ik}P_j&-&\nu_jP_j
%\end{array}
%$$
%we get the identity
%$$
%P_j\sum_{k\in\X,k\neq j} q_{ik}=\nu_jP_j.
%$$
%which is trivial since $\sum_{k\in\X,k\neq j} q_{ik}=\nu_j.$
%\end{frame}
% ----------------------------------------------------------------------
\begin{frame}{Interpretation of the Balanced Equations}
$$\nu_jP_j= \sum_{k\in\X,k\neq j}P_k q_{kj}$$
\begin{align*}
\nu_jP_j &= \mbox{rate at which the process {\bf leaves} state } j\\
\sum_{k\in\X,k\neq j}P_{k}q_{kj} &= \mbox{rate at which the process {\bf enters} state }j
\end{align*}
Balanced equations means that the rates at which the
process enters and leaves state $j$ are equal.\medskip

\hrule\medskip

The limiting distribution $\{P_j\}_{j\in\X}$ can be obtained by solving the balanced equations
along with the equation $\sum_{j\in\X}P_j=1$.\medskip

\hrule\medskip
\textbf{Remarks.}
Just like discrete-time Markov chains, a sufficient condition for the existence of a limiting distribution is that
the chain is irreducible and positive recurrent.
\end{frame}
% ----------------------------------------------------------------------
\begin{frame}{Examples}
\begin{itemize}[<+->]
\item {\bf Poisson processes}: $\mu_n=0$, $\lambda_n=\lambda$ for all $n\ge 0$
$$\nu_i=\lambda,\;P_{i,i+1}=1,\quad q_{i,i+1}=\nu_iP_{i,i+1}=\lambda$$
Balanced equations:
$$\nu_jP_j= P_{j-1}q_{j-1,j}\quad\Rightarrow\quad \lambda P_j=\lambda P_{j-1}
\quad\Rightarrow\quad P_j=P_{j-1}$$
No limiting distribution exists. The chain is not irreducible. All states are transient.

\item {\bf Pure birth processes with $\lambda_n>0$ for all $n$}

No limiting distribution exists. All states are transient.

%\item {\bf Pure birth processes} with
%$$\lambda_n>0\; \mbox{for } n\le 10,\mbox{ and }\lambda_n=0\mbox{ for all }n>10.$$
%State space $\X=\{0,1,2,\ldots,10\}$.\par
%State $10$ is the only absorbing state. All others are transient.
\end{itemize}
\end{frame}
% ----------------------------------------------------------------------
\begin{frame}{Birth and Death Processes}
For a birth and death process,
$$
\begin{array}{r@{\;\,}l@{\;\,}ll}
\nu_0 &= \lambda_0,\\
\nu_i &= \lambda_i + \mu_i,& i>0\\
P_{01} &= 1,\\
P_{i,i+1} &= \frac{\lambda_i}{\lambda_i +\mu_i}, &i>0\\
P_{i,i-1} &= \frac{    \mu_i}{\lambda_i +\mu_i}, &i>0\\
P_{i,j}&=0 &\mbox{if }|i-j|>1
\end{array}\Rightarrow
\begin{array}{l}
q_{i,i+1} =\nu_iP_{i,i+1}=\lambda_i,\; i\ge 0\\
q_{i,i-1} =\nu_iP_{i,i-1}=\mu_i, i\ge 1
\end{array}
$$
\end{frame}
% ----------------------------------------------------------------------
\begin{frame}{Balanced Equations for Birth and Death Processes}
The balanced equations $\nu_jP_j= \sum_{k\in\X,k\neq j}P_k q_{kj}$ for
a birth and death process are
\begin{align*}
\lambda_0 P_0 &= \mu_1  P_1\\
(\mu_1+\lambda_1) P_1&=\lambda_0 P_{0}+\mu_2 P_{2},\\
(\mu_2+\lambda_2) P_2&=\lambda_1 P_{1}+\mu_3 P_{3},\\
&\vdots\\
(\mu_{n-1}+\lambda_{n-1}) P_{n-1}&=\lambda_{n-2} P_{n-2}+\mu_n P_n\\
(\mu_n+\lambda_n) P_n&=\lambda_{n-1} P_{n-1}+\mu_{n+1} P_{n+1}
\end{align*}
Adding up all the equations above and eliminating the common terms on both sides, we get
$$\lambda_n P_n= \mu_{n+1}P_{n+1}\quad n\ge 0,$$

We hence just need to solve $\lambda_n P_n= \mu_{n+1}P_{n+1}$ for the limiting distribution.
\end{frame}
% ----------------------------------------------------------------------
\begin{frame}{6.6. Time Reversibility}
\textbf{Definition.} A continuous-time Markov chain with state space $\X$ is \structure{\em time reversible} if
$$
P_iq_{ij} = P_j q_{ji},\quad\mbox{for all }i, j\in\X\quad(\mbox{detailed balanced equation})
$$
If a distribution $\{P_j\}$ on $\X$ satisfies the detailed balanced equation, then it is a stationary distribution for the process.\par\bigskip

\textbf{Example.} We have just shown that for Birth and Death processes,
the balanced equations would lead to the detailed balanced equations, which are
$$\lambda_n P_n= \mu_{n+1}P_{n+1},\quad n \ge 0$$
\end{frame}
% ----------------------------------------------------------------------
\begin{frame}{Limiting Dist'n for Birth and Death Processes}
Solving $\lambda_n P_n= \mu_{n+1}P_{n+1}$, $n\ge 0$ for the limiting distribution, we get
$$
P_n=\frac{\lambda_{n-1}}{\mu_{n}}P_{n-1}=\frac{\lambda_{n-1}}{\mu_{n}}\frac{\lambda_{n-2}}{\mu_{n-1}}P_{n-2}
=\ldots=\frac{\lambda_{n-1}\lambda_{n-2}\cdots\lambda_{0}}{\mu_{n}\mu_{n-1}\cdots\mu_{1}}P_0
$$
To meet the requirement $\sum_{n=0}^{\infty}P_n=1$, we need
$$\sum_{n=0}^{\infty}P_n=P_0+P_0\sum_{n=1}^{\infty}\frac{\lambda_{n-1}\lambda_{n-2}\cdots\lambda_{0}}{\mu_{n}\mu_{n-1}\cdots\mu_{1}}=1$$
In other words, to have a limiting distribution, it is necessary that
$$
\sum_{n=1}^{\infty}\frac{\lambda_{0}\lambda_{1}\cdots\lambda_{n-1}}{\mu_{1}\mu_{2}\cdots\mu_{n}}< \infty
$$
\end{frame}
% ----------------------------------------------------------------------
\begin{frame}{Limiting Dist'n for Birth and Death Processes (Cont'd)}

If $\sum_{n=1}^{\infty}\frac{\lambda_{0}\lambda_{1}\cdots\lambda_{n-1}}{\mu_{1}\mu_{2}\cdots\mu_{n}}$ is finite,
 the limiting distribution is
$$P_0=\frac{1}{1+\sum_{n=1}^{\infty}\frac{\lambda_{0}\lambda_{1}\cdots\lambda_{n-1}}{\mu_{1}\mu_{2}\cdots\mu_{n}}}$$
and
$$P_k=\frac{\lambda_{0}\lambda_{1}\cdots\lambda_{k-1}/(\mu_{1}\mu_{2}\cdots\mu_{k})}{1+\sum_{n=1}^{\infty}\frac{\lambda_{0}\lambda_{1}\cdots\lambda_{n-1}}{\mu_{1}\mu_{2}\cdots\mu_{n}}},\quad k\ge 1$$
\end{frame}


\begin{frame}
\textbf{Lemma: (Ratio Test)}
If $a_n\ge 0$ for all $n$, then
$$
\sum_{n=1}^{\infty}a_n
\begin{cases}
< \infty & \mbox{if } \lim_{n\to\infty}a_n/a_{n-1}<1\\
= \infty & \mbox{if } \lim_{n\to\infty}a_n/a_{n-1}> 1
\end{cases}
$$

For $a_n=\displaystyle{\frac{\lambda_{0}\lambda_{1}\cdots\lambda_{n-1}}{\mu_{1}\mu_{2}\cdots\mu_{n}}}$,
$a_n/a_{n-1}=\lambda_{n-1}/\mu_n$. By the ratio test, if
$$\lim_{n\to\infty}\frac{\lambda_{n-1}}{\mu_n}<1,$$
then $\sum_{n=1}^{\infty}a_n=\sum_{n=1}^{\infty}\frac{\lambda_{0}\lambda_{1}\cdots\lambda_{n-1}}{\mu_{1}\mu_{2}\cdots\mu_{n}}<\infty$, the limiting distribution exists.

\textbf{Example 6.4 Linear Growth Model with Immigration}
$$\mu_n=n\mu,\quad \lambda_n=n\lambda+\theta$$
Using the Ratio Test,
$$
\lim_{n\to\infty}\frac{\lambda_{n-1}}{\mu_n}=\lim_{n\to\infty}\frac{(n-1)\lambda+\theta}{n\mu}=\frac{\lambda}{\mu}
$$
So the linear growth model has a limiting distribution if $\lambda<\mu.$
\end{frame}

% ----------------------------------------------------------------------
\begin{frame}{Example 5.5 (M/M/1 Queueing w/ Finite Capacity)}
\begin{itemize}
\item single-server service station. Service times are i.i.d. $\sim Exp(\mu)$
\item Poisson arrival of customers with rate $\lambda$\pause
\item Upon arrival, a customer would
\begin{itemize}\normalsize
\item go into service if the server is free (queue length $=0$)
\item join the queue if 1 to $N-1$ customers in the station, or
\item \alert{walk away} if $N$ or more customers in the station
\end{itemize}\pause
\end{itemize}
{\bf Q}: What fraction of potential customers are lost?\par\pause\medskip

Let $X(t)$ be the number of customers in the station at time $t$. \smallskip

$\{X(t), ~t\ge 0\}$ is a birth-death process with the birth and death rates below
$$
\mu_n=
\begin{cases}
0   & \text{if }n = 0\\
\mu & \text{if }n\ge 1
\end{cases}\quad\text{and}\quad
\lambda_n =
\begin{cases}
\lambda&\text{if $0\le n<N$}\\
0& \text{if }n\ge N
\end{cases}
$$\end{frame}
% ----------------------------------------------------------------------
\begin{frame}{Example 5.5 (M/M/1 Queueing w/ Finite Capacity)}
Solving $\lambda_n P_n= \mu_{n+1}P_{n+1}$ for the limiting distribution
\begin{align*}
P_1&=(\lambda/\mu)P_0\\
P_2&=(\lambda/\mu)P_1=(\lambda/\mu)^2P_0\\
&\vdots\\
P_i&=(\lambda/\mu)^iP_0, & i=1,2,\ldots,N
\end{align*}
Plugging $P_i=(\lambda/\mu)^iP_0$ into $\sum_{i=0}^N P_i=1$, one can solve for $P_0$ and get
$$
P_i=\frac{1-\lambda/\mu}{1-(\lambda/\mu)^{N+1}}(\lambda/\mu)^{i}
$$
Answer: The fraction of customers lost is $P_N=\frac{1-\lambda/\mu}{1-(\lambda/\mu)^{N+1}}(\lambda/\mu)^{N}$
\end{frame}
% ----------------------------------------------------------------------

% ----------------------------------------------------------------------
\begin{frame}{Duration Times for Birth and Death Processes}
Let
$$T_i=\mbox{time to move from state }i\mbox{ to state }i+1,\quad i=0,1,\ldots.$$
%Then $T_i$, $i=0,1,\ldots$ are independent random variables.
Suppose at some moment $X(t)=i$.
Let
\begin{align*}
B_i&=\text{time until the next birth}\sim Exp(\lambda_i)\\
D_i&=\text{time until the next death}\sim Exp(\mu_i)
\end{align*}
Then
\begin{align*}
T_i &=
\begin{cases}
B_i &\text{if the next step is }i\to i+1,\text{ i.e., }B_i<D_i\\
D_i+T_{i-1}+T_i^* &\text{if the next step is }i\to i-1,\text{ i.e., }D_i<B_i
\end{cases}\\
&=\min(B_i, D_i)+
\begin{cases}
0 &\text{if }B_i<D_i\text{, occur w/ prob. }\frac{\lambda_i}{\lambda_i+\mu_i}\\
T_{i-1}+T_i^* & \text{if }D_i<B_i\text{, occur w/ prob. }\frac{\mu_i}{\lambda_i+\mu_i}
\end{cases}
\end{align*}
Note
\begin{itemize}
\item $T_i^*$ has the same distribution as $T_i$
\item $T_{i-1}$ and $T_i^*$ are indep. of $B_i$ and $D_i$ because it's Markov
\end{itemize}
\end{frame}
% ----------------------------------------------------------------------
\begin{frame}{Duration Times for Birth and Death Processes}

Taking expected value of
$$
T_i
=\min(B_i, D_i)+
\begin{cases}
0 &\text{if }B_i<D_i\text{, occur w/ prob. }\frac{\lambda_i}{\lambda_i+\mu_i}\\
T_{i-1}+T_i^* & \text{if }D_i<B_i\text{, occur w/ prob. }\frac{\mu_i}{\lambda_i+\mu_i}
\end{cases}
$$
we get
\begin{align*}
\E[T_i]&=\E[\min(B_i, D_i)]+\left(\E[T_{i-1}]+\E[T_i]\right)\frac{\mu_i}{\lambda_i+\mu_i}\\
&=\frac{1}{\lambda_i+\mu_i}+\frac{\mu_i}{\lambda_i+\mu_i}(\E[T_{i-1}]+\E[T_i])
\end{align*}
We obtain the recursive formula
$$
\lambda_i\E[T_i]=1+\mu_i\E[T_{i-1}]
$$
\end{frame}
% ----------------------------------------------------------------------
\begin{frame}{Duration Times for Birth and Death Processes (Cont'd)}
Since $T_0\sim Exp(\lambda_0)$, $\E[T_0]=1/\lambda_0$.

Using the recursive formula $\lambda_i\E[T_i]=1+\mu_i\E[T_{i-1}]$,
we have
\begin{align*}
\E[T_0]&=\frac{1}{\lambda_0}\\
\E[T_1]&=\frac{1}{\lambda_1}+\frac{\mu_1}{\lambda_1}\E[T_0]=\frac{1}{\lambda_1}+\frac{\mu_1}{\lambda_1\lambda_0}\\
\E[T_2]&=\frac{1}{\lambda_2}+\frac{\mu_2}{\lambda_2}\left(\frac{1}{\lambda_1}+\frac{\mu_1}{\lambda_1\lambda_0}\right)
=\frac{1}{\lambda_2}+\frac{\mu_2}{\lambda_2\lambda_1}+\frac{\mu_2\mu_1}{\lambda_2\lambda_1\lambda_0}\\
&\vdots\\
\E[T_i]&=\frac{1}{\lambda_i}+\frac{\mu_i}{\lambda_i}\E[T_{i-1}]
=\frac{1}{\lambda_i}+\frac{\mu_i}{\lambda_i\lambda_{i-1}}+\cdots+\frac{\mu_i\mu_{i-1}\cdots\mu_2\mu_1}{\lambda_i\lambda_{i-1}\cdots\lambda_2\lambda_1\lambda_0}\\
&=\frac{1}{\lambda_i}\left(1+\sum_{k=1}^i\prod_{j=1}^k\frac{\mu_{i-j+1}}{\lambda_{i-j}}\right)
\end{align*}
\end{frame}
% ----------------------------------------------------------------------
\end{document}
% ----------------------------------------------------------------------
\begin{frame}
\end{frame}
% ----------------------------------------------------------------------
