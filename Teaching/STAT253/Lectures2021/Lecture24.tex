%\documentclass[letterpaper,draft]{beamer}
\documentclass[letterpaper,handout]{beamer}
%\documentclass[letterpaper]{beamer}

%---multiple pages on one sheet, ADD for handout--
%\usepackage{pgfpages}
%\pgfpagesuselayout{4 on 1}[letterpaper, landscape, border shrink=1mm]
%-------------------------------------------------
\usepackage{amsmath,amsfonts}
%\usepackage{booktabs}
%\usepackage{mdwlist}
\usepackage{amsfonts}
%\usetheme{Copenhagen}
%\usetheme{warsaw}
\setbeamertemplate{navigation symbols}{}
\usepackage[english]{babel}
\def\ul{\underline}
% or whatever

\usepackage[latin1]{inputenc}
\subject{Talks}

\def\Sum{\sum\nolimits}
\def\Prod{\prod\nolimits}
\def\P{\mathbb{P}}
\def\p{\mathrm P}
\def\E{\mathbb E}
\def\V{\mathrm{Var}}
\def\CV{\mathrm{Cov}}
\def\X{\mathcal{X}}
\def\dt{\Delta}
\def\typo#1{\alert{#1}}
%-------------Answers------------
\def\Hide#1#2{\ul{~~~\onslide<#1>{\alert{#2}}~~~}}
\def\hide#1#2{\ul{~~\onslide<#1>{\alert{#2}}~~}}
%------Centered Page Number------
\defbeamertemplate{footline}{centered page number}
{%
  \hspace*{\fill}%
  %\usebeamercolor[fg]{page number in head/foot}%
  %\usebeamerfont{page number in head/foot}%
  \small Lecture \chapnum\ - \insertframenumber%
  \hspace*{\fill}\vskip2pt%
}

%\usepackage{tikz}
%\usebackgroundtemplate{%
%\tikz\node[opacity=0.3] {\includegraphics[height=\paperheight,widht=\paperwidth]{ctanlion}};}

%\usebackgroundtemplate{%
%  %\rule{0pt}{\paperheight}%
%  \parbox[c][\paperheight][c]{\paperwidth}{\centering\includegraphics[width=.65\paperwidth]{UClogo.pdf}}
%  %\hspace*{\paperwidth}
%}

\def\chapnum{24}
%--------------------------------
\setbeamertemplate{footline}[centered page number]

\title{STAT253/317 Winter 2021 Lecture \chapnum} \date{} \author{Yibi Huang}
\begin{document}
% ----------------------------------------------------------------------
\begin{frame}\maketitle
\bigskip
\begin{center}\large
\begin{tabular}{ll}
$\bullet$ & Brownian Motion with Drift\\
$\bullet$ & Stopping Time, Strong Markov Property (Review)\\
$\bullet$ & Wald's Identities for Brownian Motion\\
\end{tabular}
\end{center}
\end{frame}
% ----------------------------------------------------------------------
\begin{frame}{Brownian Motion with Drift}
A stochastic process $\{B(t), t \ge 0\}$ is said to be a \structure{\em Brownian
motion process with drift coefficient $\mu$ and variance parameter $\sigma^2$} if
\begin{itemize}
\item[(i)] $B(0) = 0$;
\item[(ii)] $\{B(t), t \ge 0\}$ has stationary and independent increments;
\item[(iii)] for every $t \ge 0, s\ge 0$, $B(t+s)-B(s)\sim N(\mu t,\sigma^2 t)$
\end{itemize}
\end{frame}
% ----------------------------------------------------------------------
\begin{frame}{Stopping Time (Review)}\normalsize
For a continuous time stochastic process $\{X(t),t\ge 0\}$,
a \structure{\em stopping time $T$ with respect to} $\{X(t),t\ge 0\}$ is a nonnegative random variable, such that
the event $\{T\le t\}$ depends only on $\{X(s),0\le s\le t\}$ but not $\{X(s), s> t\}$.\bigskip

\textbf{Remark:} If $T$ is a stopping time with respect to $\{X(t),t\ge 0\}$, for each non-random $n>0$,
the stopping time truncated at $n$
$$(T\wedge n)\mbox{ defined as } \min(T,n)$$
is also a stopping time with respective to $\{X(t),t\ge 0\}$.

{\bf Reason}:
$\{(T\wedge n)\le t\} = \{T\le t\}\cup\{n\le t\}$
\begin{itemize}
\item The event $\{n\le t\}$ is non-random, does not depend on $\{X(s)\}$
\item The event $\{T\le t\}$ depends only on $\{X(s),0\le s\le t\}$ but not $\{X(s), s>t\}$ since $T$ is a stopping time
\end{itemize}
Hence the event $\{(T\wedge n)\le t\}$ depends on $\{X(s),0\le s\le t\}$ only but not $\{X(s), s> t\}$,
which shows $(T\wedge n)$ is also a stopping time.
\end{frame}
% ----------------------------------------------------------------------
\begin{frame}{Strong Markov Property (Review)}
Let $\{B(t),t\ge 0\}$ be a Brownian Motion (with drift $\mu$), and let $T$ be a stopping time respective to $\{B(t),t\ge 0\}$.
Then
\begin{itemize}
\item[(a)] Define $Z(t)=B(t+T)-B(T)$, $t\ge 0$.\\
 Then $\{Z(t), t\ge 0\}$ is also a Brownian Motion with drift $\mu$
\item[(b)] For each $t>0$, $\{Z(s), 0\le s\le t\}$ is independent of $\{B(s), 0\le s\le T\}$
\end{itemize}\medskip

\textbf{Remark:} If $T$ is not a stopping time, the Strong Markov Property may not be true.
For example, let
$$T=T_{\max}=\min\Big\{t: B(t)=\max_{0\le s\le 1}B(s)\Big\},$$\vspace{-10pt}

where $\{B(t), t \ge 0\}$ is a standard Brownian motion.
\begin{itemize}
\item $T_{\max}$ is not a stopping time since the event $\{T_{\max}\le t\}$ depends not just $\{B(s),0\le s\le t\}$, but on the entire $\{B(s),0\le s\le 1\}.$
\item Since $B(T_{\max})$ will be the maximum of $\{B(s),0\le s\le 1\}$, $B(t+T_{\max})-B(T_{\max})$ will be $\le 0$ for $t\le 1-T_{\max},$ and hence is not Brownian motion
\end{itemize}
\end{frame}
% ----------------------------------------------------------------------
\begin{frame}{Wald's Identities for Brownian Motion}
If $\{B(t), t \ge 0\}$ is a Brownian motion process
with drift $\mu$  and variance parameter $\sigma^2$,
and $T$ is a \structure{\bf bounded stopping time} with respect to $\{B(t)\}$, then
\begin{itemize}
\item[(i)] $\E[B(T)]=\mu\E[T]$,
\item[(ii)] $\E[B^2(T)]=\sigma^2\E[T]+\mu^2\E[T^2]$,
\item[(iii)] $\E[e^{\theta B(T)-(\theta\mu+\frac{\theta^2\sigma^2}{2})T}]=1$ for all $\theta\in\mathbb{R}$
\end{itemize}

\underline{Remark}:
\begin{itemize}
\item For {\em nonrandom} times $T=t$, the identities follows from the elementary properties of the normal distribution%: for example, for $Z\sim N(\mu,\sigma^2t)$, its moment generating function (MGF) is $\E[e^{aZ}]=e^{a^2t/2}.$
\item If $T$ is {\em unbounded}, the identities may not be true\\
\begin{itemize}
\item Example: if $T=T_1$ be the hitting time to value 1 of a standard Brownian motion, then $B(T)=1$. So $\E[B(T)]\neq 0.$
\end{itemize}
\item If $T$ is not a stopping time, the identities may also fail.\\
\begin{itemize}
\item Example: if $T=T_{\max}=\min\{t: B(t)=\max_{0\le s\le 1}B(s)\}$ \\
 then $\E[B(T_{\max})]=\E[\max_{0\le s\le 1}B(s)]>0.$
 \end{itemize}
\end{itemize}
\end{frame}
% ----------------------------------------------------------------------
\begin{frame}{Application of Wald's Identities}
For constants $a,b>0$
Let $T=T_{-a,b}$ be the first time $t$ such that the standard Brownian Motion process hit $-a$ or $b$

$$T_{-a,b}=\min\{t: B(t)=-a,\mbox{ or }B(t)=b\}$$\vspace{-15pt}

\begin{itemize}
\item $T$ is a stopping time since the event
$$\{T\le t\}=\Big\{\max_{0\le s\le t}B(s)\ge b\Big\}\bigcup\Big\{\min_{0\le s\le t}B(s)\le -a\Big\},$$
depends on $\{B(s),0\le s\le t\}$ only.
\item $T$ is finite, but \underline{unbounded} $\Rightarrow$ Wald's identities may \underline{not} apply.
\item However, for each integer $n\ge 1$, the random variable $T\wedge n=\min(T,n)$ is a bounded stopping time.\\
By the first and second Wald's identities, we have
$$
\E[B(T\wedge n)]=0\quad\mbox{and}\quad\E[B^2(T\wedge n)]=\E[T\wedge n]
$$
\end{itemize}
\end{frame}
% ----------------------------------------------------------------------
\begin{frame}{Application of Wald's Identities (Cont'd)}
%$$\E[B(T\wedge n)]=0\quad\mbox{and}\quad\E[B^2(T\wedge n)]=\E[T\wedge n]$$
\begin{itemize}
\item From that $-a\le B(T\wedge n)\le b$, we know $|B(T\wedge n)|$ is uniformly bounded by $a+b$ for all $n$
\item As $\p(T\!<\!\infty)\!=\!1$, we have $\displaystyle\lim_{n\to\infty}B(T\wedge n)=B(T)$ w/ prob. 1.
\item By Bounded Convergence Theorem,
\begin{align}
\E[B(T)]&=\lim_{n\to\infty}\E[B(T\wedge n)]=0\label{eq:ETab}\\
\E[B^2(T)]&=\lim_{n\to\infty}\E[B^2(T\wedge n)]=\lim_{n\to\infty}\E[T\wedge n]=\E[T]\label{eq:VTab}
\end{align}
\item Because $B(T)=-a$ or $b$, from that
$$\E[B(T)]=-a\p(B(T)=-a)+b\p(B(T)=b)=0$$
and that $\p(B(T)=-a)+\p(B(T)=b)=1$, it follows that
$$\p(B(T)=-a)=\frac{b}{a+b},\quad\p(B(T)=b)=\frac{a}{a+b}$$
\item From the above and \eqref{eq:VTab}, one may easily deduce that
$$\E[T]=\E[B^2(T)]=a^2\p(B(T)=-a)+b^2\p(B(T)=b)=ab$$
\end{itemize}
\end{frame}
% ----------------------------------------------------------------------
\begin{frame}{Exercise 10.22: $T_{-a,b}$ for Brownian with Drift }
Let $\{B(t),t\ge 0\}$ be Brownian Motion with drift coefficient $\mu\neq0$ and variance parameter $\sigma^2$.
For constants $a,b>0$ let
$$T=T_{-a,b}=\min\{t: B(t)=-a,\mbox{ or }B(t)=b\}$$
$T$ is again a finite but \underline{unbounded} stopping time, so Wald's identities may \underline{not} be applied directly. However, using the truncated stopping time $T\wedge n=\min(T,n)$ and Bounded Convergence Theorem, we can prove that the first Wald's identity holds for $T$
$$\mu\E[T]=\E[B(T)]=-a\p(B(T)=-a)+b\p(B(T)=b).$$
However, when $\mu\neq 0$, we cannot use this equation and that $\p(B(T)=-a)+\p(B(T)=b)=1$ to solve for $\p(B(T)=-a)$ and $\p(B(T)=b)$ since $\E[T]$ is unknown. Instead we will use the third Wald's identity.
\end{frame}
% ----------------------------------------------------------------------
\begin{frame}{Exercise 10.22: $T_{-a,b}$ for Brownian with Drift (Cont'd)}
\begin{itemize}
\item By the third Wald's identity, we have
\begin{equation}\label{eq:Wald3}
\E[e^{\theta B(T\wedge n)-(\theta\mu+\frac{\theta^2\sigma^2}{2})(T\wedge n)}]=1\quad\text{for all }\theta\in\mathbb{R}.
\end{equation}
\item Let us choose $\theta=\theta_0=-2\mu/\sigma^2$ so that the 2nd term in the exponent of \eqref{eq:Wald3} vanishes. So
$$\E[e^{\theta_0 B(T\wedge n)}]=1$$
\item $-a\le B(T\wedge n)\le b\Rightarrow |B(T\wedge n)|\le a+b $\\[5pt]
$\phantom{-a\le B(T\wedge n)\le b}\Rightarrow e^{\theta_0 B(T\wedge n)}\le e^{\theta_0 (a+b)}$
\item By the Bounded Convergence Theorem,
\begin{align*}
1&=\lim_{n\to\infty}\E[e^{\theta_0 B(T\wedge n)}]=\E[e^{\theta_0 B(T)}]\\
&=e^{-\theta_0 a}\p(B(T)=-a)+e^{\theta_0 b}\p(B(T)=b)
\end{align*}
\end{itemize}
\end{frame}
% ----------------------------------------------------------------------
\begin{frame}{Exercise 10.22: $T_{-a,b}$ for Brownian with Drift (Cont'd)}
Solving the equation
$$1=e^{-\theta_0 a}\p(B(T)=-a)+e^{\theta_0 b}\p(B(T)=b)$$
and the equation $\p(B(T)=-a)+\p(B(T)=b)=1$ for $\p(B(T)=-a)$ and $\p(B(T)=b)$,
one can get that
$$
\p(B(T)=-a)=\frac{1-e^{\theta_0b}}{e^{-\theta_0a}-e^{\theta_0b}},\quad
\p(B(T)= b)=\frac{e^{-\theta_0a}-1}{e^{-\theta_0a}-e^{\theta_0b}}
$$
\textbf{Theorem 1}. Let $\{B(t),t\ge 0\}$ be a Brownian Motion with drift coefficient $\mu \neq 0$ and variance parameter $\sigma^2$, the probability that the process reach $b>0$ before hitting $-a<0$ is given by
$$
\p(B(T_{-a,b})=b)=\frac{\exp(2\mu a/\sigma^2)-1}{\exp(2\mu a/\sigma^2)-\exp(-2\mu b/\sigma^2)}
$$
\end{frame}
% ----------------------------------------------------------------------
\begin{frame}{Proof of Wald's Identities for Brownian Motion}
\begin{itemize}
\itemsep=8pt
\item Since $T$ is bounded, there is a nonrandom $N<\infty$ such that $\p(T< N)=1$
\item By the Strong Markov Property, the post-$T$ process $B(t+T)-B(T)$ is
\begin{itemize}\normalsize
\item also a Brownian Motion process with drift $\mu$ and variance parameter $\sigma^2$, and
\item independent of $\{B(s), 0\le s\le T\}$, and in particular, independent of the random vector $(T, B(T))$.
\end{itemize}
%\item Hence,
%\begin{align*}
%0&=\E[B(N)]=\E[B(N)-B(T)+B(T)]\\
%&=\E[B(N)-B(T)]\E[B(T)]\quad(\mbox{since }B(N)-B(T)\mbox{ is indep. of }B(T))
%\end{align*}
%Thus either $\E[B(T)]=0$ or $\E[B(N)-B(T)]=0$, but the latter also implies $\E[B(T)]=0$ since $\E[B(N)]=0$.
%\item Moreover, since $\E[B^2(N)]=\V(B(N))$ for $\E[B(N)]=0$
%\begin{align*}
%N=\E[B^2(N)]&=\V(B(N))=\V(B(N)-B(T)+B(T))\\
%&\V[B(N)-B(T)]+\V(B(T))\quad(\mbox{since }B(N)-B(T)\mbox{ is indep. of }B(T))
%\end{align*}
\item Hence, given that $T=s$ the conditional distribution of $B(N)-B(T)$
is normal with mean $\mu(N-s)$ and variance $\sigma^2(N-s)$. It follows that
$$\E\Big[e^{\theta [B(N)-B(T)]-\theta\mu(N-T)- \frac{\theta^2\sigma^2(N-T)}{2}}\Big|T, B(T)\Big]=1$$
\end{itemize}
\end{frame}
% ----------------------------------------------------------------------
\begin{frame}{Proof of Wald's Identities (Cont'd)}
Therefore
\begin{align*}
&\E[e^{\theta B(T)-\theta\mu T-\frac{\theta^2\sigma^2T}{2}}]=\E[e^{\theta B(T)-\theta\mu T-\frac{\theta^2\sigma^2T}{2}}]\times 1\\
={}&\E[e^{\theta B(T)-\theta\mu T-\frac{\theta^2\sigma^2T}{2}}]\\
&\qquad\qquad\times\E\Big[e^{\theta [B(N)-B(T)]-\theta\mu(N-T)-\frac{\theta^2\sigma^2(N-T)}{2}}\Big|T, B(T)\Big]\\
={}&\E\Big[\E\Big[e^{\theta B(T)-\theta\mu T-\frac{\theta^2\sigma^2T}{2}
        +\theta [B(N)-B(T)]-\theta\mu(N-T)-\frac{\theta^2\sigma^2(N-T)}{2}}\Big|T, B(T)\Big]\Big]\\
={}&\E\Big[\E\Big[e^{\theta B(N)-\theta\mu N-\frac{\theta^2\sigma^2N}{2}}\Big|T, B(T)\Big]\Big]\\
={}&\E[e^{\theta B(N)-\theta\mu N-\frac{\theta^2\sigma^2N}{2}}]=1
\end{align*}
This proves the third identity.\par
The first and second identity can be derived by differentiating the third identity with respective to $\theta$ once and twice respectively, and letting $\theta=0$.
\end{frame}
% ----------------------------------------------------------------------
\end{document}
\begin{frame}
\end{frame}
% ----------------------------------------------------------------------
