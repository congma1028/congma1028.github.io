%\documentclass[letterpaper,draft]{beamer}
\documentclass[letterpaper,handout]{beamer}
%\documentclass[letterpaper]{beamer}

%---multiple pages on one sheet, ADD for handout--
%\usepackage{pgfpages}
%\pgfpagesuselayout{4 on 1}[letterpaper, landscape, border shrink=1mm]
%-------------------------------------------------
\usepackage{amsmath,amsfonts}
%\usepackage[misc]{ifsym} % for the dice symbol \Cube{}
%\usepackage{booktabs}
%\usepackage{mdwlist}
%\usepackage{pgf,tikz}
%\usetheme{Copenhagen}
%\usetheme{warsaw}
\setbeamertemplate{navigation symbols}{}
\usepackage[english]{babel}
\def\ul{\underline}
% or whatever

\usepackage[latin1]{inputenc}
\subject{Talks}

\def\P{\mathbb{P}}
\def\p{\mathrm P}
\def\E{\mathbb E}
\def\Sum{\sum\nolimits}
\def\X{\mathfrak{X}}
\def\typo#1{\alert{#1}}
%-------------Answers------------
\def\Hide#1#2{\ul{~~~\onslide<#1>{\alert{#2}}~~~}}
\def\hide#1#2{\ul{~~\onslide<#1>{\alert{#2}}~~}}
\def\hid#1#2{\onslide<#1>{\alert{#2}}}
%------Centered Page Number------
\defbeamertemplate{footline}{centered page number}
{%
  \hspace*{\fill}%
  %\usebeamercolor[fg]{page number in head/foot}%
  %\usebeamerfont{page number in head/foot}%
  \small Lecture \chapnum\ - \insertframenumber%
  \hspace*{\fill}\vskip2pt%
}

%\usepackage{tikz}
%\usebackgroundtemplate{%
%\tikz\node[opacity=0.3] {\includegraphics[height=\paperheight,widht=\paperwidth]{ctanlion}};}

%\usebackgroundtemplate{%
%  %\rule{0pt}{\paperheight}%
%  \parbox[c][\paperheight][c]{\paperwidth}{\centering\includegraphics[width=.65\paperwidth]{UClogo.pdf}}
%  %\hspace*{\paperwidth}
%}

\def\chapnum{4}
%--------------------------------
\setbeamertemplate{footline}[centered page number]

\title{STAT253/317 Winter 2013 Lecture \chapnum}
\date{January 14, 2013}
\author{Yibi Huang}
\begin{document}
% ----------------------------------------------------------------------
%\begin{frame}\maketitle\bigskip
%\begin{center}4.4 Limiting Distribution\end{center}
%\end{frame}
% ----------------------------------------------------------------------
\begin{frame}{STAT253/317 Lecture \chapnum: 4.4 Limiting Distribution I}

\begin{block}{Stationary Distribution}
Define $\pi^{(n)}_i=\p(X_n=i)$, $i\in\X$ to be the marginal distribution of $X_n$, $n=1,2,\ldots,$
and let $\pi^{(n)}$ be the row vector
$$
\pi^{(n)}=(\pi^{(n)}_0,\,\pi^{(n)}_1,\,\pi^{(n)}_2,\ldots),
$$
From Chapman-Kolmogrov Equation, we know that
$$\pi^{(n)}=\pi^{(n-1)}\P\quad\text{i.e.}\quad\pi^{(n)}_j=\Sum_{i\in\X}\pi^{(n-1)}_iP_{ij}\text{ for all }j\in\X,$$

If $\pi$ is a distribution on $\X$ satisfying
$$\pi\P=\pi\quad\text{i.e.}\quad\pi_j=\Sum_{i\in\X}\pi_iP_{ij}\text{ for all }j\in\X,$$
then $\pi^{(0)}=\pi$ implies $\pi^{(n)}=\pi$ for all $n$.\medskip

We say $\pi$ is a \structure{\bf stationary distribution} of the Markov chain.
\end{block}
\end{frame}
% ----------------------------------------------------------------------
\begin{frame}{Example 1: 2-state Markov Chain}
$$
\X = \{0, 1\},\qquad
\P=\bordermatrix{%
 & 0  & 1\cr
0& 1-\alpha  & \alpha\cr
1& \beta & 1-\beta\cr
}$$
\begin{align*}
\pi\P=\pi&\Rightarrow
\begin{cases}
\pi_0&=(1-\alpha)\pi_0+\beta\pi_1\\
\pi_1&=   \alpha \pi_0+(1-\beta)\pi_1
\end{cases}\\
&\Rightarrow
\begin{cases}
\alpha\pi_0&=\beta\pi_1\\
\beta \pi_1&=\alpha\pi_0
\end{cases}
\end{align*}
Need one more constraint: $\pi_0+\pi_1=1$
$$
\Rightarrow \pi=(\pi_0,\pi_1)=\left(\frac{\beta}{\alpha+\beta},\frac{\alpha}{\alpha+\beta}\right)
$$
\end{frame}
% ----------------------------------------------------------------------
\begin{frame}{Example 2: Ehrenfest Diffusion Model with $N$ Balls}
$$
P_{ij}=
\begin{cases}
\dfrac{i}{N}    & \text{if }j=i-1\\[5pt]
\dfrac{N-i}{N}  & \text{if }j=i+1\\
0   & \text{otherwise}
\end{cases}
$$
$$
\arraycolsep=1pt
\begin{array}{crrcl}
\pi_0=&            \pi_1P_{10}=&                     &\frac{1}{N}\pi_1&\Rightarrow\pi_1=N\pi_0={N\choose 1}\pi_0\\[5pt]
\pi_1=&\pi_0P_{01}+\pi_2P_{21}=&               \pi_0+&\frac{2}{N}\pi_2&\Rightarrow\pi_2=\frac{N(N-1)}{2}\pi_0={N\choose 2}\pi_0\\[5pt]
\pi_2=&\pi_1P_{12}+\pi_3P_{32}=&\frac{N-1}{N}\pi_1+&\frac{3}{N}\pi_3&\Rightarrow\pi_3=\frac{N(N-1)(N-2)}{6}\pi_0={N\choose 3}\pi_0\\
\vdots&\vdots
\end{array}
$$
In general, you'll get $\pi_i={N\choose i}\pi_0$.\par
As $1=\sum_{i=0}^N\pi_i=\pi_0\sum_{i=0}^N{N\choose i}$ and $\sum_{i=0}^N{N\choose i}=2^N,$ we have
$$
\pi_i={N\choose i}\left(\frac{1}{2}\right)^{N}\quad\text{for }i=0,1,2,\ldots,N.
$$
\end{frame}
% ----------------------------------------------------------------------
\begin{frame}{Stationary Distribution May Not Be Unique}
Consider a Markov chain with transition matrix $\P$ of the form
$$
\P=
\bordermatrix{%
  &  0  &  1  &  2  &  3  &  4 \cr
0 &  *  &  *  &  0  &  0  &  0 \cr
1 &  *  &  *  &  0  &  0  &  0 \cr
2 &  0  &  0  &  *  &  *  &  * \cr
3 &  0  &  0  &  *  &  *  &  * \cr
4 &  0  &  0  &  *  &  *  &  *
}
=
\begin{pmatrix}
\P_x & 0\\
0 & \P_y
\end{pmatrix}
$$
This Markov chain has 2 classes \{0,1\} and \{2, 3, 4\}; both are recurrent.
Note that this Markov chain can be reduced to two sub-Markov chains, one with state space \{0,1\} and the other \{2, 3, 4\}.
Their transition matrices are respectively $\P_x$ and $\P_y$.\par\medskip

Say $\pi_x=(\pi_0,\pi_1)$ and $\pi_y=(\pi_2,\pi_3,\pi_4)$ be respectively the stationary distributions of the two sub-Markov chains, i.e.,
$$
\pi_x\P_x=\pi_x,\quad \pi_y\P_y=\pi_y
$$
Verify that $\pi = (c\pi_0,c\pi_1,(1-c)\pi_2,(1-c)\pi_3,(1-c)\pi_4)$ is a stationary distribution of $\{X_n\}$ for any $c$ between 0 and 1.
\end{frame}
% ----------------------------------------------------------------------
\begin{frame}{Not All Markov Chains Have a Stationary Distribution}
For one-dimensional symmetric random walk, the transition probabilities are
$$P_{i,i+1}=P_{i,i-1}=1/2$$
The stationary distribution $\{\pi_j\}$ would satisfy the equation:
$$\pi_j=\Sum_{i\in\X}\pi_iP_{ij}=\frac{1}{2}\pi_{j-1}+\frac{1}{2}\pi_{j+1}.$$
Once $\pi_0$ and $\pi_1$ are determined, all $\pi_j$'s can be determined from the equations as
$$\pi_j=\pi_0+(\pi_1-\pi_0)j,\quad\text{ for all integer }j.$$
As $\pi_j\ge 0$ for all integer $j$, $\Rightarrow \pi_1=\pi_0$. Thus
$$\pi_j=\pi_0 \quad\text{ for all integer }j$$
Impossible to make $\Sum_{j=-\infty}^{\infty}\pi_j=1.$\bigskip

Conclusion: 1-dim symmetric random walk does not have a stationary distribution.
\end{frame}
% ----------------------------------------------------------------------
\begin{frame}{Limiting Distribution}
A probability distribution $\pi = [\pi_0, \pi_1, \pi_2, \ldots]$ is called the limiting distribution of a Markov chain $X_n$ if for all $i,j\in\X$,
\begin{align*}
\pi_j = \lim_{n\to\infty}P^{(n)}_{ij}
= \lim_{n \to \infty} \p (X_n = j \mid X_0 = i)
\end{align*}

Matrix version
$$\text{i.e.,}\quad\lim_{n\to\infty}\P^{(n)}=
\begin{pmatrix}
\pi_0 & \pi_1 & \pi_2 & \pi_3 &\cdots \cr
\pi_0 & \pi_1 & \pi_2 & \pi_3 &\cdots \cr
\pi_0 & \pi_1 & \pi_2 & \pi_3 &\cdots \cr
\vdots&\vdots&\vdots&\vdots&\ddots
\end{pmatrix}$$
\end{frame}


\begin{frame}{Example: Two-State Markov Chain}
$$
\X = \{0, 1\},\qquad
\P=\bordermatrix{%
 & 0  & 1\cr
0& 1-\alpha  & \alpha\cr
1& \beta & 1-\beta\cr
}$$
By induction, one can show that
\begin{align*}
\P^{(n)}&=
\begin{pmatrix}
\dfrac{\beta }{\alpha+\beta}+\dfrac{\alpha}{\alpha+\beta}(1\!-\!\alpha\!-\!\beta)^n &
\dfrac{\alpha}{\alpha+\beta}-\dfrac{\alpha}{\alpha+\beta}(1\!-\!\alpha\!-\!\beta)^n\\[8pt]
\dfrac{\beta }{\alpha+\beta}+\dfrac{\beta }{\alpha+\beta}(1\!-\!\alpha\!-\!\beta)^n &
\dfrac{\alpha}{\alpha+\beta}-\dfrac{\beta }{\alpha+\beta}(1\!-\!\alpha\!-\!\beta)^n
\end{pmatrix}\\
&\to
\begin{pmatrix}
\dfrac{\beta }{\alpha+\beta} &
\dfrac{\alpha}{\alpha+\beta}\\[8pt]
\dfrac{\beta }{\alpha+\beta} &
\dfrac{\alpha}{\alpha+\beta}
\end{pmatrix}\quad\text{as }n\to\infty
\end{align*}
The limiting distribution $\pi$ is $(\frac{\beta }{\alpha+\beta},\frac{\alpha}{\alpha+\beta}).$
\end{frame}

%
%\begin{frame}{Example 1: 2-state Markov Chain}
%$$
%\X = \{0, 1\},\qquad
%\P=\bordermatrix{%
% & 0  & 1\cr
%0& 1-\alpha  & \alpha\cr
%1& \beta & 1-\beta\cr
%}$$
%
%It can be shown that 
%$$\lim_{n\to\infty}\P^{(n)}=
%\begin{pmatrix}
%\frac{\beta}{\alpha +\beta} & \frac{\alpha}{\alpha +\beta}  \cr
%\frac{\beta}{\alpha +\beta} & \frac{\alpha}{\alpha +\beta}  \cr
%\end{pmatrix}$$
%
%The limiting distribution coincides with the stationary distribution
%\end{frame}

% ----------------------------------------------------------------------
\begin{frame}{Limiting Distribution is a Stationary Distribution}
The limiting distribution of a Markov chain is a stationary distribution of the Markov chain.\medskip

{\em \textbf{Proof} (not rigorous).} By Chapman Kolmogorov Equation,
$$P^{(n+1)}_{ij}=\Sum_{k\in\X}P^{(n)}_{ik}P_{kj}$$
Letting $n\to\infty$, we get
\begin{align*}
\pi_j=\lim_{n\to\infty}P^{(n+1)}_{ij}&=\lim_{n\to\infty}\Sum_{k\in\X}P^{(n)}_{ik}P_{kj}\\
&=^*\Sum_{k\in\X}\lim_{n\to\infty}P^{(n)}_{ik}P_{kj} \quad\text{(needs justification)}\\
&=\Sum_{k\in\X}\pi_kP_{kj}
\end{align*}
Thus the limiting distribution $\pi_j$'s satisfies the equations $\pi_j=\Sum_{k\in\X}\pi_kP_{kj}$ for all $j\in\X$
and is a stationary distribution.\smallskip
\hrule\smallskip

{\footnotesize See Karlin \& Taylor (1975), Theorem 1.3 on p.85-86 for a rigorous proof.}
\end{frame}
% ----------------------------------------------------------------------
%\begin{frame}{Limiting Distribution is Unique}
%If a Markov chain has a limiting distribution $\pi$, then $$\lim_{n\to\infty}\pi^{(n)}_j=\pi_j\text{ for all }j\in\X, \text{ whatever $\pi^{(0)}$ is}$$
%
%{\em \textbf{Proof} (not rigorous).} Since
%$$\pi^{(n)}_j=\Sum_{k\in\X}\pi^{(0)}_kP^{(n)}_{kj}$$
%Letting $n\to\infty$, we get
%\begin{align*}
%\lim_{n\to\infty}\pi^{(n)}_j&=\lim_{n\to\infty}\Sum_{k\in\X}\pi^{(0)}_kP^{(n)}_{kj}\\
%&=^*\Sum_{k\in\X}\pi^{(0)}_k\lim_{n\to\infty}P^{(n)}_{kj} \quad\text{(needs justification)}\\
%&=\underbrace{\Sum_{k\in\X}\pi^{(0)}_k}_{=1}\pi_j=\pi_j
%\end{align*}
%i.e.,
%if a limiting distribution exists, it is the unique stationary distribution.
%\end{frame}
% ----------------------------------------------------------------------

% ----------------------------------------------------------------------
\begin{frame}{Not All Markov Chains Have Limiting Distributions}

Consider the simple random walk $X_n$ on $\{0,1,2,3,4\}$ with absorbing boundary at 0 and 4. That is,
\[
X_{n+1}
=\begin{cases}
X_n+1 &\text{with probability 0.5 if } 0<X_n <4\\
X_n-1 &\text{with probability 0.5 if } 0<X_n <4\\
X_n & \text{if $X_n=$ 0 or 4}
\end{cases}
\]
The transition matrix is hence
{\small
\[
\P=
\bordermatrix{%
  &  0  &  1  &  2  &  3  &  4 \cr
0 &  1  &  0  &  0  &  0  &  0 \cr
1 & 0.5 &  0  & 0.5 &  0  &  0 \cr
2 &  0  & 0.5 &  0  & 0.5 &  0 \cr
3 &  0  &  0  & 0.5 &  0  & 0.5\cr
4 &  0  &  0  &  0  &  0  &  1
}
\]}
\end{frame}
% ----------------------------------------------------------------------
\begin{frame}{Not All Markov Chains Have Limiting Distributions}\linespread{1.1}
The $n$-step transition matrix of the simple random walk $X_n$ on $\{0,1,2,3,4\}$ with absorbing boundary at 0 and 4
can by shown by induction using the Chapman-Kolmogorov Equation to be
{\small
\begin{align*}
\P^{(2n-1)}&=
\bordermatrix{%
  &  0  &  1  &  2  &  3  &  4 \cr
0 &  1  &  0  &  0  &  0  &  0 \cr
1 & 0.75-0.5^{n+1} &  0  & 0.5^n &  0  & 0.25-0.5^{n+1}\cr
2 & 0.5-0.5^n & 0.5^n &  0  & 0.5^n & 0.5-0.5^n\cr
3 & 0.25-0.5^{n+1}  &  0  & 0.5^n &  0  & 0.75-0.5^{n+1}\cr
4 &  0  &  0  &  0  &  0  &  1
}\\
\P^{(2n)}&=
\bordermatrix{%
  &  0  &  1  &  2  &  3  &  4 \cr
0 &  1  &  0  &  0  &  0  &  0 \cr
1 & 0.75-0.5^{n+1} & 0.5^{n+1} &  0 & 0.5^{n+1} & 0.25-0.5^{n+1}\cr
2 & 0.5 -0.5^{n+1} &  0  & 0.5^n & 0 & 0.5-0.5^{n+1}\cr
3 & 0.25-0.5^{n+1} & 0.5^{n+1} &  0 & 0.5^{n+1} & 0.75-0.5^{n+1}\cr
4 &  0  &  0  &  0  &  0  &  1
}
\end{align*}}
\end{frame}
% ----------------------------------------------------------------------
\begin{frame}{Not All Markov Chains Have Limiting Distributions}
The limit of the $n$-step transition matrix as $n\to\infty$ is
$$
\P^{(n)}\to
\bordermatrix{%
  &  0   &  1  &  2  &  3  &  4 \cr
0 &  1   &  0  &  0  &  0  &  0 \cr
1 & 0.75 &  0  &  0  &  0  & 0.25\cr
2 & 0.5  &  0  &  0  &  0  & 0.5\cr
3 & 0.25 &  0  &  0  &  0  & 0.75\cr
4 &  0   &  0  &  0  &  0  &  1
}.
$$
Though $\lim_{n\to\infty}P^{(n)}_{ij}$ exists but the limit depends on the initial state $i$,
this Markov chain has no limiting distribution.\medskip

This Markov chain has two distinct absorbing states 0 and 4.
Other transient states may be absorbed to either 0 or 4 with
different probabilities depending how close those states are to 0 or 4.
\end{frame}

\begin{frame}
	\vfill
	When does a Markov chain have limiting distribution?
	\vfill
\end{frame}
% ----------------------------------------------------------------------
\begin{frame}{Periodicity}
A state of a Markov chain is said to have \structure{\bf period $d$} if
$$P^{(n)}_{ii}=0,\quad\text{whenever $n$ is not a multiple of $d$}$$
In other words, $d$ is the \structure{\em greatest common divisor} of all the $n$'s such that
$$
P^{(n)}_{ii}>0
$$
We say a state is \structure{\bf aperiodic} if $d=1$, and \structure{\bf periodic} if $d>1$.\bigskip


{\bf Fact}: Periodicity is a class property.\par That is, all states in the same class have the same period.\medskip

For a proof, see Problem 2\&3 on p.77 of Karlin \& Taylor (1975).
\end{frame}
% ----------------------------------------------------------------------
\begin{frame}{Examples (Periodicity)}
\begin{itemize}
\item All states in the Ehrenfest diffusion model are of period $d=2$ since
it's impossible to move back to the initial state in odd number of steps.
\item 1-D (2-D) Simple random walk on all integers (grids on a 2-d plane) are of period $d=2$
%\item Suppose a 2-D random walk can move to the nearest grid point in any direction, horizontally,
%vertically or diagonally, each with probability 1/8.
%\begin{center}\tabcolsep=2pt
%\begin{tabular}{ccccc}
%$\bullet$ &            & $\bullet$  &            & $\bullet$\\[-1pt]
%          & $\nwarrow$ & $\uparrow$ & $\nearrow$ &          \\[-1pt]
%$\bullet$ &$\leftarrow$& $\bullet$  & $\to$      & $\bullet$\\[-1pt]
%          & $\swarrow$ &$\downarrow$& $\searrow$ &          \\[-1pt]
%$\bullet$ &            & $\bullet$  &            &$\bullet$
%\end{tabular}
%\end{center}
%
%What is the period of this Markov chain?
\end{itemize}
\end{frame}
% ----------------------------------------------------------------------
\begin{frame}{Example (Periodicity)}
Specify the classes of a Markov chain
with the following transition matrix, and find the periodicity for each state. \[
\bordermatrix{%
  &  1  &  2  &  3  &  4  &  5  &  6  &  7 \cr
1 &  0  & 0.5 &  0  & 0.5 &  0  &  0  &  0 \cr
2 &  0  &  0  &  1  &  0  &  0  &  0  &  0 \cr
3 & 0.5 &  0  &  0  &  0  &  0  & 0.5 &  0 \cr
4 &  0  &  0  & 0.5 &  0  & 0.5 &  0  &  0 \cr
5 &  1  &  0  &  0  &  0  &  0  &  0  &  0 \cr
6 &  0  &  0  &  0  &  0  &  0  & 0.1 & 0.9\cr
7 &  0  &  0  &  0  &  0  &  0  & 0.7 & 0.3
}
\qquad
\onslide<2->{{\color{red}
\begin{array}{ccccc}
5        & \to      & 1        & \to      & 2\\
\uparrow & \swarrow & \uparrow & \swarrow &  \\
4        & \to      & 3        & &\\
         &          & \downarrow &&\\
7 &\leftrightarrow & 6 &&\\
\end{array}}
}
\]
\color{red}{Classes: \{1,2,3,4,5\}, \{6,7\}.\\\pause
Period is $d=1$ for state 6 and 7.\\\pause
Period is $d=3$ for state 1,2,3,4,5 since $\{1\}\to\{2,4\}\to\{3,5\}\to \{1\}.$
}
\end{frame}
% ----------------------------------------------------------------------
\begin{frame}{Periodic Markov Chains Have No Limiting Distributions}
For example, in the Ehrenfest diffusion model with 4 balls,
it can be shown by induction that the $(2n-1)$-step transition matrix is
{\footnotesize
\begin{align*}
\P^{(2n-1)}&=
\bordermatrix{%
  &  0  &  1  &  2  &  3  &  4 \cr
0 &  0  & 1/2\!+\!1/2^{2n-1} &  0  & 1/2\!-\!1/2^{2n-1} &  0 \cr
1 & 1/8\!+\!1/2^{2n+1} &  0  & 3/4 &  0  & 1/8\!-\!1/2^{2n+1}\cr
2 &  0  & 1/2 &  0  & 1/2 & 0 \cr
3 & 1/8\!-\!1/2^{2n+1} &  0  & 3/4 &  0  & 1/8\!+\!1/2^{2n+1}\cr
4 &  0  & 1/2\!-\!1/2^{2n-1} &  0  & 1/2\!+\!1/2^{2n-1} &  0
}\\
&\to
\bordermatrix{%
  &  0  &  1  &  2  &  3  &  4 \cr
0 &  0  & 1/2 &  0  & 1/2 &  0 \cr
1 & 1/8 &  0  & 3/4 &  0  & 1/8\cr
2 &  0  & 1/2 &  0  & 1/2 & 0 \cr
3 & 1/8 &  0  & 3/4 &  0  & 1/8\cr
4 &  0  & 1/2 &  0  & 1/2 &  0
}\quad\text{as }n\to\infty.
\end{align*}

}
\end{frame}
% ----------------------------------------------------------------------
\begin{frame}{Periodic Markov Chains Have No Limiting Distributions}
and the $2n$-step transition matrix is
\footnotesize
\begin{align*}
\P^{(2n)}&=
\bordermatrix{%
  &  0  &  1  &  2  &  3  &  4 \cr
0 & 1/8\!+\!1/2^{2n+1} &  0  & 3/4 &  0  & 1/8\!-\!1/2^{2n+1}\cr
1 &  0  & 1/2\!+\!1/2^{2n+1} &  0  & 1/2\!-\!1/2^{2n+1} &  0 \cr
2 & 1/8             &  0  & 3/4 &  0  & 1/8\cr
3 &  0  & 1/2\!-\!1/2^{2n+1} &  0  & 1/2\!+\!1/2^{2n+1} &  0 \cr
4 & 1/8\!-\!1/2^{2n+1} &  0  & 3/4 &  0  & 1/8\!+\!1/2^{2n+1}
}\\
&\to
\bordermatrix{%
  &  0  &  1  &  2  &  3  &  4 \cr
0 & 1/8 &  0  & 3/4 &  0  & 1/8\cr
1 &  0  & 1/2 &  0  & 1/2 &  0 \cr
2 & 1/8 &  0  & 3/4 &  0  & 1/8\cr
3 &  0  & 1/2 &  0  & 1/2 &  0 \cr
4 & 1/8 &  0  & 3/4 &  0  & 1/8
}\quad\text{as }n\to\infty.
\end{align*}
\end{frame}
% ----------------------------------------------------------------------
\begin{frame}{Periodic Markov Chains Have No Limiting Distributions}

In general for Ehrenfest diffusion model with $N$ balls, as $n\to\infty,$
\begin{align*}
P^{(2n)}_{ij}
&\to\begin{cases}
2{N\choose j}(\frac{1}{2})^{N}&\text{if $i+j$ is even}\\
 0            &\text{if $i+j$ is odd}\\
 \end{cases}\\
P^{(2n+1)}_{ij}
&\to\begin{cases}
 0            &\text{if $i+j$ is even}\\
2{N\choose j}(\frac{1}{2})^{N}&\text{if $i+j$ is odd}\\
 \end{cases}
\end{align*}
\bigskip

$\lim_{n\to\infty}P^{(n)}_{ij}$ doesn't exist for all $i,j\in\X$
\end{frame}
% ----------------------------------------------------------------------
\begin{frame}{Summary}
\begin{itemize}
\item Stationary distribution may not be unique if the Markov chain is not irreducible
\item Stationary distribution may not exist
\item A limiting distribution is always a stationary distribution
\item If it exists, limiting distribution is unique
\item Limiting distribution do not exist if the Markov chain is periodic
\end{itemize}
\end{frame}
% ----------------------------------------------------------------------
\end{document} 