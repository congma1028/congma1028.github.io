%\documentclass[letterpaper,draft]{beamer}
\documentclass[letterpaper,handout]{beamer}
%\documentclass[letterpaper]{beamer}

%---multiple pages on one sheet, ADD for handout--
%\usepackage{pgfpages}
%\pgfpagesuselayout{4 on 1}[letterpaper, landscape, border shrink=1mm]
%-------------------------------------------------
\usepackage{amsmath,amsfonts}
%\usepackage{booktabs}
%\usepackage{mdwlist}
\usepackage{amsfonts}
\usepackage{cancel}
%\usetheme{Copenhagen}
%\usetheme{warsaw}
\setbeamertemplate{navigation symbols}{}
\usepackage[english]{babel}
\def\ul{\underline}
% or whatever

\usepackage[latin1]{inputenc}
\subject{Talks}

\def\Sum{\sum\nolimits}
\def\Prod{\prod\nolimits}
\def\P{\mathbb{P}}
\def\p{\mathrm P}
\def\E{\mathbb E}
\def\V{\mathrm{Var}}
\def\CV{\mathrm{Cov}}
\def\X{\mathcal{X}}
\def\dt{\Delta}
\def\typo#1{\alert{#1}}
%-------------Answers------------
\def\Hide#1#2{\ul{~~~\onslide<#1>{\alert{#2}}~~~}}
\def\hide#1#2{\ul{~~\onslide<#1>{\alert{#2}}~~}}
%------Centered Page Number------
\defbeamertemplate{footline}{centered page number}
{%
  \hspace*{\fill}%
  %\usebeamercolor[fg]{page number in head/foot}%
  %\usebeamerfont{page number in head/foot}%
  \small Lecture \chapnum\ - \insertframenumber%
  \hspace*{\fill}\vskip2pt%
}

%\usepackage{tikz}
%\usebackgroundtemplate{%
%\tikz\node[opacity=0.3] {\includegraphics[height=\paperheight,widht=\paperwidth]{ctanlion}};}

%\usebackgroundtemplate{%
%  %\rule{0pt}{\paperheight}%
%  \parbox[c][\paperheight][c]{\paperwidth}{\centering\includegraphics[width=.65\paperwidth]{UClogo.pdf}}
%  %\hspace*{\paperwidth}
%}

\def\chapnum{25}
%--------------------------------
\setbeamertemplate{footline}[centered page number]

\title{STAT253/317 Lecture \chapnum} \date{} \author{Yibi Huang}
\begin{document}
% ----------------------------------------------------------------------
\begin{frame}\maketitle
\bigskip
\begin{center}\large
\begin{tabular}{ll}
10.5 & The Maximum of Brownian Motion with Drift\\
     & (11th edition only, not in 10th edition)
\end{tabular}
\end{center}
\end{frame}
% ----------------------------------------------------------------------
\begin{frame}{Maximum of a Brownian Motion with drift}
Let $\{X(t),t\ge 0\}$ be a Brownian Motion with drift coefficient $\mu$ and variance parameter $\sigma^2$.
Consider the maximum of the process up to time $t$
\[M(t)=\max_{0\le s\le t} X(s)\]
Also consider the hitting time to the value $a>0$
\[
T_a=\min\{t: X(t)=a\}.
\]

\vspace{-6pt}\begin{itemize}
\item It remains true that $P(T_a<t)=P(M(t)\ge a).$
\item Recall for Brownian motion without drift, we use the Reflection principle to find $P(T_a<t)$
\item Reflection principle doesn't apply to Brownian motion with drift.
We need other tools.
\end{itemize}
\end{frame}
% ----------------------------------------------------------------------
\begin{frame}{Theorem 10.2}
Let $X(t)$ be the Brownian motion process $\{B(t), t \ge 0\}$ with drift coefficient $\mu$ and variance
parameter $\sigma^2$.
Given that $X(t) = x$, the conditional distribution of
$\{X(s): 0 \le s \le t\}$ does not depend on the value of $\mu.$\medskip

{\it Proof.}
Given $X(t) = x$, $\{X(s): 0 \le s \le t\}$ remains a Gaussian process.
As a Gaussian process is uniquely determined by its mean function and the covariance function,
it suffices to show that the mean function
\[
m(s)=\E[X(s)|X(t)=x], \quad 0 \le s \le t
\]
and covariance
\[
C(s,u)=\CV(X(s),X(u)|X(t)=x), \quad 0 \le s,u \le t
\]
do not depend on the value of $\mu$.
\end{frame}
% ----------------------------------------------------------------------
\begin{frame}%{Theorem 10.2 (Proof Continued)}
\small
For jointly normal random variables, zero covariance implies independence.
If we can find a scalar $\alert{c}$ such that \[\CV(X(s)-\alert{c}X(t),X(t))=0,\]
then
$X(s)-\alert{c}X(t)$ and $X(t)$ would be indep..
The conditional distribution of of $X(s)-\alert{c} X(t)$ given $X(t)=x$ would simply be its unconditional distribution
\begin{align*}
X(s)&=c\underbrace{X(t)}_{x}+\underbrace{X(s)-cX(t)}_{\sim N(\mu s-c\mu t,\,\sigma^2(s-2cs+c^2t))}\\
&\sim N\left(cx+\mu s-c\mu t,\sigma^2(s-2cs+c^2t)\right).
\end{align*}

\vspace{-8pt}To make
\begin{align*}
\CV(X(s)-cX(t),X(t))&=\CV(X(s),X(t))-\CV(cX(t),X(t))\\
&=\sigma^2s-c\sigma^2t=\sigma^2(s-ct)=0,
\end{align*}
we must let $\alert{c=s/t}$. Thus given $X(t)=x$ for $s<t$,
\[ X(s)\sim N\bigg(\frac{sx}{t}+\underbrace{\mu s-(s/t)\mu t}_{=\mu s -\mu s = 0},\sigma^2\frac{s(t-s)}{t}\bigg)
=N\left(\frac{sx}{t},\;\sigma^2\frac{s(t-s)}{t}\right).\]
So the mean function $m(s)=\E[X(s)|X(t)=x]=\frac{sx}{t}$ and the covariance function
$C(s,u)=\CV(X(s),X(u)|X(t)=x)=\sigma^2\frac{s(t-s)}{t}$
don't depend on the drift coefficient $\mu.$
\end{frame}
% ----------------------------------------------------------------------
\begin{frame}{Theorem 10.3 on p.626-627}

\vspace{-20pt}\[
P(M(t) \ge y|X(t) = x) =
\begin{cases}
1 & \text{if }x\ge y\ge 0\\
e^{-2y(y-x)/t\sigma^2}&\text{if }x<y, y\ge 0
\end{cases}
\]
{\it Proof.}
\begin{itemize}
\item The equality is trivial when $x\ge y$ as $M(t)\ge X(t)= x\ge y$.
\item When $x<y$, as Theorem 10.2
implies the conditional distribution of
$M(t)= \max_{0 \le s \le t}X(s)$ given $X(t)=x$ is identical for all values of $\mu$,
we just need to show the identity for the case with drift $\mu=0$,
to which the \structure{Reflection Principle} is applicable.
\item For $h>0$ small enough that $y-x-h>0$, by the Reflection Principle,
\begin{align*}
&P(M(t) \ge y, x\le X(t) \le x+h)\\
=\,&P(M(t) \ge y, 2y-x-h\le X(t) \le 2y-x)\\
=\,&P(2y-x-h\le X(t) \le 2y-x)
\end{align*}
where the last equality is valid since
$M(t)\ge X(t)\ge 2y\!-\!x\!-\!h>y$ as $y-x-h>0$.
\end{itemize}
\end{frame}
% ----------------------------------------------------------------------
\begin{frame}

\begin{align*}
P(M(t) \ge y|X(t) = x)
&=\lim_{h\to 0}\frac{P(M(t) \ge y, x\le X(t) \le x+h)}{P(x\le X(t) \le x+h)}\\
&=\lim_{h\to 0}\frac{P(2y-x-h\le X(t) \le 2y-x)}{P(x\le X(t) \le x+h)}\\
&=\frac{f(2y-x)}{f(x)}
\end{align*}
where
\[
f(x)=\frac{1}{\sqrt{2\pi\sigma^2t}}\exp\left(-\frac{x^2}{2\sigma^2t}\right)
\]
is the density function of $X(t)\sim N(0,\sigma^2t)$ with drift $\mu=0$.
So
\begin{align*}
P(M(t) \ge y|X(t) = x)
&=\frac{f(2y-x)}{f(x)}
=\frac{\exp(-(2y-x)^2/(2\sigma^2t))}{\exp(-x^2/(2\sigma^2t))}\\
&=\exp\left(-\frac{(2y-x)^2-x^2}{2\sigma^2t}\right)
=e^{-\frac{2y(y-x)}{t\sigma^2}}
\end{align*}
\end{frame}
% ----------------------------------------------------------------------
\begin{frame}{Corollary 10.1 on p.627-628}\small
Conditioning on $X(t)$ and using Theorem 10.3 yields
\begin{align*}
P(M(t) \ge y)
&=\int_{-\infty}^{\infty}P(M(t) \ge y|X(t) = x)f_{X(t)}(x)dx\\
&=\int_{-\infty}^y\underbrace{e^{-\frac{2y(y-x)}{t\sigma^2}}f_{X(t)}(x)}_{\text{see below}}dx+
\underbrace{\int_y^{\infty}1\cdot f_{X(t)}(x)dx}_{=P(X(t)>y)}
\end{align*}
\[
e^{-\frac{2y(y-x)}{t\sigma^2}}f_{X(t)}(x)
=e^{-\frac{2y(y-x)}{t\sigma^2}}\frac{1}{\sqrt{2\pi\sigma^2t}}e^{-\frac{(x-\mu t)^2}{2\sigma^2t}}
=\frac{1}{\sqrt{2\pi\sigma^2t}}e^{-\frac{\alert{(x-\mu t)^2+4y(y-x)}}{2\sigma^2t}}
\]
in which
\begin{align*}
\alert{(x\!-\!\mu t)^2+4y(y\!-\!x)}&=x^2-2\mu tx+\mu^2t^2+4y^2-4xy\\
&=x^2\!-\!2(\mu t+y)x
+\underbrace{\mu^2t^2\overbrace{+4\mu t y}^{\text{add a term}}+4y^2}_{=(\mu t+2y)^2}\overbrace{-4\mu t y}^{\text{subtract a term}}\\
&=\underbrace{x^2\!-\!2(\mu t+y)x+(\mu t+2y)^2}_{=(x-(\mu t+2y))^2}-4\mu t y
\end{align*}
\end{frame}
% ----------------------------------------------------------------------
\begin{frame}{Corollary 10.1 on p.627-628 (Cont'd)}
Putting everything together, we get
\[
P(M(t) \ge y)
=e^{\frac{2\mu ty}{\sigma^2t}}\int_{-\infty}^y\frac{1}{\sqrt{2\pi\sigma^2t}}e^{-\frac{(x-(\mu t+2y))^2}{2\sigma^2t}}dx
+P(X(t)>y)
\]
Making the change of variable $u = x-2y$ gives
\begin{align*}
P(M(t) \ge y)
&=e^{\frac{2\mu yt}{\sigma^2t}}\int_{-\infty}^{-y}
\underbrace{\frac{1}{\sqrt{2\pi\sigma^2t}}e^{-\frac{(u-\mu t)^2}{2\sigma^2t}}}_{\text{density of }X(t)}du
+P(X(t)>y)\\
&=e^{\frac{2\mu y}{\sigma^2}}P(X(t)<-y)+P(X(t)>y)\\
&=e^{\frac{2\mu y}{\sigma^2}}\Phi\left(\frac{-y-\mu t}{\sigma\sqrt{t}}\right)+1-\Phi\left(\frac{y-\mu t}{\sigma\sqrt{t}}\right)
\end{align*}
since $X(t)\sim N(\mu t, \sigma^2 t).$\medskip

Note that for $\mu=0$, we get $P(M(t) \ge y)=P(X(t)<-y)+P(X(t)>y)=P(|X(t)|>y)$, which agrees with our calculation before.
\end{frame}
% ----------------------------------------------------------------------
\begin{frame}{Hitting Time for Brownian Motion with drift}
Also consider the hitting time to the value $y>0$
\[
T_y=\min\{t: X(t)=y\}.
\]
It remains true that $T_y<t$ if and only if $M(t)\ge y.$ So
\[
P(T_y<t)=
e^{\frac{2\mu y}{\sigma^2}}\Phi\left(\frac{-y-\mu t}{\sigma\sqrt{t}}\right)+1-\Phi\left(\frac{y-\mu t}{\sigma\sqrt{t}}\right)
\]
\end{frame}
% ----------------------------------------------------------------------
%\begin{frame}{Hitting Time for Brownian Motion with drift}
%Using the first Wald's identity $\E[B(T)]=\mu\E[T]$,
%\end{frame}
\end{document} 