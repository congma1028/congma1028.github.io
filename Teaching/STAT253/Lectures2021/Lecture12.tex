%\documentclass[letterpaper,draft]{beamer}
\documentclass[letterpaper,handout, mathserif]{beamer}
%\documentclass[letterpaper]{beamer}

%---multiple pages on one sheet, ADD for handout--
%\usepackage{pgfpages}
%\pgfpagesuselayout{4 on 1}[letterpaper, landscape, border shrink=1mm]
%-------------------------------------------------
\usepackage{amsmath,amsfonts}
%\usepackage{booktabs}
%\usepackage{mdwlist}
\usepackage{amsfonts}
%\usetheme{Copenhagen}
%\usetheme{warsaw}
\setbeamertemplate{navigation symbols}{}
\usepackage[english]{babel}
\def\ul{\underline}
% or whatever

\usepackage[latin1]{inputenc}
\subject{Talks}

\def\Sum{\sum\nolimits}
\def\p{\mathrm P}
\def\E{\mathbb E}
\def\V{\mathrm Var}
\def\X{\mathcal{X}}
\def\typo#1{\alert{#1}}
%-------------Answers------------
\def\Hide#1#2{\ul{~~~\onslide<#1>{\alert{#2}}~~~}}
\def\hide#1#2{\ul{~~\onslide<#1>{\alert{#2}}~~}}
%------Centered Page Number------
\defbeamertemplate{footline}{centered page number}
{%
  \hspace*{\fill}%
  %\usebeamercolor[fg]{page number in head/foot}%
  %\usebeamerfont{page number in head/foot}%
  \small Lecture \chapnum\ - \insertframenumber%
  \hspace*{\fill}\vskip2pt%
}

%\usepackage{tikz}
%\usebackgroundtemplate{%
%\tikz\node[opacity=0.3] {\includegraphics[height=\paperheight,widht=\paperwidth]{ctanlion}};}

%\usebackgroundtemplate{%
%  %\rule{0pt}{\paperheight}%
%  \parbox[c][\paperheight][c]{\paperwidth}{\centering\includegraphics[width=.65\paperwidth]{UClogo.pdf}}
%  %\hspace*{\paperwidth}
%}

\def\chapnum{12}
%--------------------------------
\setbeamertemplate{footline}[centered page number]

\title{STAT253/317 Lecture \chapnum} \date{} \author{Cong Ma}
\begin{document}
% ----------------------------------------------------------------------
\begin{frame}\maketitle\begin{center}Chapter 6 \ Continuous-Time Markov Chains\end{center}\end{frame}
% ----------------------------------------------------------------------
\begin{frame}{6.2 Continuous-Time Markov Chains (CTMC)}
\textbf{Definitions.} A stochastic process $\{X(t),t\ge 0\}$ with state space $\X$ is called a \structure{\em continuous-time Markov chain} if for any two states $i$, $j\in\X$,
\begin{align*}
&\p(\underbrace{X(t+s)=j}_{\mbox{future}}|\underbrace{X(s)=i}_{\mbox{present}},\underbrace{X(u)=x(u),\mbox{for }0\le u<s}_{\mbox{past}})\\
&=\p(\underbrace{X(t+s)=j}_{\mbox{future}}|\underbrace{X(s)=i}_{\mbox{present}})
\end{align*}
If $\p(X(t+s)=j|X(s)=i)$ does not depend on $s$ for all $i,j\in\X$, then it is denoted as
$$P_{ij}(t)=\p(X(t+s)=j|X(s)=i),$$
and we say the CTMC is \structure{\em homogeneous} in time.\medskip

In STAT253/317, we focus on homogeneous CTMC only.
\end{frame}
% ----------------------------------------------------------------------
\begin{frame}{Exponential Waiting Time}
Let $\{X(t),t\ge 0\}$ be a homogeneous continuous-time Markov chain.
For $i\in\X$, let $T_i$ denote the amount of time that $X(t)$ stays in state $i$ before making a transition into
a different state. \medskip

{\bf Claim}: $T_i$ has the \structure{\em memoryless property}.
\begin{align*}
&\p(T_i\ge t+s|T_i\ge s)\\
&=\p(X(u)=i,\mbox{ for } s\le u\le s+t|X(u)=i,\mbox{ for } 0\le u\le s)\\
&=\p(X(u)=i,\mbox{ for } s\le u\le s+t|X(s)=i)\quad(\mbox{Markov property})\\
&=\p(X(u)=i,\mbox{ for } 0\le u\le t|X(0)=i)\qquad(\mbox{time homogeneity})\\
&=\p(T_i\ge t) \quad\Rightarrow\quad  \mbox{So $T_i$ is memoryless.}
\end{align*}
Recall that the exponential distribution is the only continuous distribution having the memoryless property.

Thus \structure{$T_i\sim Exp(\nu_i)$} for some rate $\nu_i$.
\end{frame}
% ----------------------------------------------------------------------
\begin{frame}{An Alternative Definition of CTMC}
A stochastic process $\{X(t),t\ge 0\}$ with state space $\X$ is a \structure{\em continuous-time Markov chain} if
\begin{itemize}
\item (exponential waiting time)
 when the chain reaches a state $i$, the time it stays at state $i\sim Exp(\nu_i)$, where $\nu_i$ is the transition rate at state $i$
\item (embedded with a discrete time Markov chain)
 when the process leaves state $i$, it enters anther state $j$ with probability $P_{ij}$, such that
    $$P_{ii}=0, \quad\Sum_{j\in\X} P_{ij}=1\quad\mbox{for all }i,j \in\X.$$
\end{itemize}
\textbf{Remark}: The amount of time $T_i$ the process spends in
state $i$, and the next state visited, must be independent.
For if the next state visited were dependent on $T_i$, then information as to how long the
process has already been in state $i$ would be relevant to the prediction of the next
state---and this contradicts the Markovian assumption.
\end{frame}
% ----------------------------------------------------------------------
\begin{frame}{6.3 Birth and Death Processes}
Let $X(t)=$ the number of people in the system at time $t$.\\
Suppose that whenever there are $n$ people in the system, then
\begin{itemize}
\item [(i)] new arrivals enter the system at an exponential rate $\lambda_n$, and
\item [(ii)] people leave the system at an exponential rate $\mu_n.$
\end{itemize}
Such an $\{X(t),t\ge 0\}$ is called a \structure{\em birth and death process}.
$$
\arraycolsep=4pt
\begin{array}{ccccccccccccccc}
  & \lambda_0  && \lambda_1 && \lambda_2 && \cdots&  & \lambda_{n-1} && \lambda_n && \cdots\\[-3pt]
0 & \rightleftarrows & 1 & \rightleftarrows & 2 &\rightleftarrows& 3 & \cdots & n\!-\!1&\rightleftarrows & n &\rightleftarrows& n\!+\!1 & \cdots\\[-5pt]
  & \mu_1      && \mu_2     &&  \mu_3   && \cdots& & \mu_n && \mu_{n+1} && \cdots
\end{array}
$$
Suppose the process is at state $i>0$ at time $t$. Then
\begin{align*}
B_i &=\text{waiting time until the next birth} \sim \text{Exp}(\lambda_i)\\
D_i &=\text{waiting time until the next death} \sim \text{Exp}(\mu_i)
\end{align*}
Hence, the waiting time until the next transition out of state $i$ is
$\min(B_i, D_i)\sim \text{Exp}(\lambda_i+\mu_i)$, from which we can get
$$
\nu_i = \lambda_i + \mu_i,\;\text{for } i>0
$$
\end{frame}
% ----------------------------------------------------------------------
\begin{frame}{6.3 Birth and Death Processes (Cont'd)}

%\vspace{-12pt}$$
%\arraycolsep=4pt
%\begin{array}{ccccccccccccccc}
%  & \lambda_0  && \lambda_1 && \lambda_2 && \cdots&  & \lambda_{n-1} && \lambda_n && \cdots\\[-3pt]
%0 & \rightleftarrows & 1 & \rightleftarrows & 2 &\rightleftarrows& 3 & \cdots & n\!-\!1&\rightleftarrows & n &\rightleftarrows& n\!+\!1 & \cdots\\[-5pt]
%  & \mu_1      && \mu_2     &&  \mu_3   && \cdots& & \mu_n && \mu_{n+1} && \cdots
%\end{array}
%$$
Moreover, given the process is at state $i>0$ at time $t$,
the probability that the next transition is a birth rather than a death is
$$
P_{i,i+1}=P(B_i<D_i)= \frac{\lambda_i}{\lambda_i +\mu_i},
$$ which implies $P_{i,i-1}=P(D_i<B_i) = \frac{\mu_i}{\lambda_i +\mu_i}$, for $i>0.$\smallskip\pause

As only birth is possible at state $0$, we know $\nu_0 = \lambda_0$ and $P_{01}=1$.\smallskip\pause

To sum up, a birth and death process is a CTMC with state space $\X=\{0,1,2,\ldots\}$ such that
\begin{align*}
\nu_i &= \lambda_i + \mu_i, i>0,\quad \nu_0 = \lambda_0,\\
P_{i,i+1} &= \frac{\lambda_i}{\lambda_i +\mu_i},\; P_{i,i-1} = \frac{\mu_i}{\lambda_i +\mu_i}, i>0\\
P_{01} &= 1,\; P_{i,j}=0 \quad\mbox{if }|i-j|>1
\end{align*}
The parameters $\{\lambda_n\}^{\infty}_{n=0}$ and $\{\mu_n\}^{\infty}_{n=1}$ are called, respectively, the arrival (or birth) and departure (or death) rates.
\end{frame}
% ----------------------------------------------------------------------
\begin{frame}{Examples of Birth and Death Processes}
\begin{itemize}
\item Poisson Processes: $\mu_n=0$, $\lambda_n=\lambda$ for all $n\ge 0$
\item Pure Birth Process: $$\mu_n=0\quad \Rightarrow\quad \nu_i=\lambda_i,\; P_{i,i+1}=1,\; P_{i,i-1}=0$$
\item Yule Processes (Pure Birth Process with Linear Growth rate):
If there are $n$ people and each independently gives birth at at an exponential rate $\lambda$, then the total rate at which births occur is $n\lambda.$ $$\mu_n=0,\quad \lambda_n=n\lambda$$\pause

\vspace{-6pt}\textit{Reason}: Let
$$B_i = \text{time until the ith individual give birth}\sim Exp(\lambda),\; i=1,\ldots,n$$
So the time until the next (first) birth when there are $n$ individuals in the population is
$$
\min(B_1,B_2,\ldots,B_n) \sim Exp(\lambda+\lambda+\cdots+\lambda)=Exp(n\lambda)
$$
So the rate until the next birth is $\lambda_n=n\lambda$.
\end{itemize}
\end{frame}
% ----------------------------------------------------------------------
\begin{frame}{Example: Linear Growth Model with Immigration}
\begin{itemize}
\item each individual independently gives birth at an exponential rate $\lambda$
\item each individual independently die at at an exponential rate $\mu$
\item new immigrants come in at an exponential rate $\theta$
\end{itemize}
Such a process is a birth-death process with birth and death rates
$$\mu_n=n\mu,\quad \lambda_n=n\lambda+\theta$$

\vspace{-6pt}\textit{Reason}: Let
\begin{align*}
B_i &= \text{time until the ith individual give birth}\sim Exp(\lambda),\; i=1,\ldots,n\\
T &=\text{time until the next new immigrant comes in}\sim Exp(\theta)
\end{align*}
So the time until the population size increase from $n$ to $n+1$ is
$$
\min(B_1,\ldots,B_n, T) \sim Exp(\lambda+\cdots+\lambda+\theta)=Exp(n\lambda+\theta)
$$
So the rate until the next birth is $\lambda_n=n\lambda+\theta$.\par\smallskip

Similarly, one can show that the death rate is $\mu_n=n\mu$.

\end{frame}
% ----------------------------------------------------------------------
\begin{frame}{Example: $M/M/s$ Queueing Model}
\begin{itemize}
\item $s$ servers
\item Poisson arrival of customers, rate $=\lambda$
\item Exponential service time,  rate $=\mu$
\end{itemize}
$\Rightarrow$ a birth and death process with constant birth rate $\lambda_n=\lambda$, and death (departure)rate $\mu_n = \min(n,s)\mu.$\medskip\pause

\textit{Reason}:
Suppose, there are $n$ customer in the system at time $t$.
At most $\min(n,s)$ of them are being served. Let 
$S_i$ be remaining service time of the $i$th server $\sim$ Exp$(\mu).$
Then, the waiting time until the next departure is 
$$
\min(S_1, \ldots, S_{\min(s,n)})\sim \text{Exp}(\min(s,n)\mu).
$$
\end{frame}
% ----------------------------------------------------------------------
\begin{frame}{6.4 The Transition Probability Function $P_{ij}(t)$}
Recall the transition probability function $P_{ij}(t)$ of a CTMC $\{X(t),t\ge 0\}$ is
$$P_{ij}(t)=\p(X(t+s)=j|X(s)=i)$$

\textbf{Example.} (Poisson Processes with rate $\lambda$)
\begin{align*}
P_{ij}(t)&=\p(N(t+s)=j|N(s)=i)\\
&=\p(N(t+s)-N(s)=j-i)
=\begin{cases}
e^{-\lambda{t}}\frac{(\lambda t)^{j-i}}{(j-i)!} & \mbox{if }j\ge i\\
0&\mbox{if }j<i
\end{cases}
\end{align*}

\textbf{Properties of Transition Probability Functions}
\begin{itemize}
\item $P_{ij}(t)\ge 0$ for all $i,j\in\X$ and $t\ge 0$
\item (Row sums are 1) $\sum_j P_{ij}(t)=1$ for all $i\in\X$ and $t\ge 0$
\end{itemize}
\end{frame}

% ----------------------------------------------------------------------
\begin{frame}{Lemma 6.3 Chapman-Kolmogorov Equation}
For all $i,j\in\X$ and $t\ge 0$,
$$P_{ij}(t+s)=\sum_{k\in\X} P_{ik}(t)P_{kj}(s)$$
{\em Proof.}\small
\begin{align*}
&P_{ij} (t +s)\\
&= \p(X(t + s) = j |X(0) = i)\\
&=\sum_{k\in\X}\p(X(t + s) = j,X(t) = k|X(0) = i)\\
&=\sum_{k\in\X}\p(X(t + s) = j |X(t) = k,X(0) = i)\p(X(t) = k|X(0) = i)\\
&=\sum_{k\in\X}\p(X(t + s) = j |X(t) = k)\p(X(t) = k|X(0) = i)\;\mbox{(Markov Property)}\\
&=\sum_{k\in\X}P_{kj} (s)P_{ik}(t)
\end{align*}
\end{frame}


\begin{frame}
	\frametitle{The matrix notation}
	
	Let $\mathbf{P}(t) = [P_{ij}(t)]$ be the transition matrix at time $t$. 
	
	We have $\mathbf{P}(0) = \mathbf{I}$.
	And C-K equations read
	\begin{align*}
		\mathbf{P} (t+s) = \mathbf{P} (t) \mathbf{P}(s)
	\end{align*}
	
	\vfill
	One way to specify a CTMC is through $\{ \mathbf{P}(t) \}_{ t \geq 0}$. But 
	this requires an infinite number of matrices. Can we simplify it?
	
	\vfill
	Key: use derivatives $\mathbf{P}'(t)$
\end{frame}

\begin{frame}
	\frametitle{Transition rate matrix / infinitesimal generator $\mathbf{Q}$}
	Assume that 
	\begin{align*}
		\mathbf{P}'(0) = \lim_{h \to 0} \frac{\mathbf{P}(h) - \mathbf{P}(0)}{h} \quad \text{ exists}.
	\end{align*}
	
	In other words, for each $i,j$,
	\begin{align*}
		P_{ij}'(0) = \lim_{h \to 0} \frac{{P}_{ij}(h) - {P}_{ij}(0)}{h} \quad \text{ exists}.
	\end{align*}
	\vfill
	We will denote such limit as $\mathbf{Q} = [q_{ij}]$, the transition rate matrix.
	
	How about $\mathsf{P}'(t)$ for $t > 0$?
\end{frame}

\begin{frame}
	\frametitle{Kolmogorov's equations}
	
	
	By definition, one has
	\begin{align*}
		\mathsf{P}'(t) &= \lim_{h \to 0} \frac{\mathbf{P}(t+h) - \mathbf{P}(t) }{h} 
		= \lim_{h \to 0} \frac{\mathbf{P}(t)\mathbf{P}(h) - \mathbf{P}(t)}{h} \\
		       &=  \lim_{h \to 0} \frac{\mathbf{P}(t) ( \mathbf{P}(h) - \mathbf{I}) }{h} 
		       = \mathbf{P}(t) \mathbf{Q}.
	\end{align*}
	This is the so-called Kolmogorov's forward equations.
	
	\vfill
	Similarly you can prove backward equations
	\begin{align*}
		\mathbf{P}'(t) = \mathbf{Q} \mathbf{P}(t).
	\end{align*}
	
	\vfill These imply $\mathbf{P}(t) = \exp(t\mathbf{Q})$.
\end{frame}


\begin{frame}{Transition rate matrix}
\vfill
\large How to compute $\mathbf{Q}$?
\vfill
\end{frame}
% ----------------------------------------------------------------------
\begin{frame}{Lemma 6.2a}
For any $i,j\in\X$, we have
$$q_{ii} = \lim_{h\to 0}\frac{P_{ii}(h) - 1}{h} = -\nu_i$$

{\em Proof.}
Let $T_i$ be the amount of time the process stays in state $i$ before moving to other states.
\begin{align*}
P_{ii}(h)&=\p(X(h)=i|X(0) = i)\\
&=\p(X(h)=i,\mbox{no transition in (0,h]}|X(0) = i)\\
&\quad+\p(X(h)=i,\mbox{2 or more transition in (0,h]}|X(0) = i)\\
&=\p(T_i>h)+o(h)\\
&=e^{-\nu_ih}+o(h)\\
&=1-\nu_i h+o(h)
\end{align*}\vspace{-30pt}

\end{frame}
% ----------------------------------------------------------------------
\begin{frame}{Lemma 6.2b}
For any $i \neq j\in\X$, we have
$$q_{ij} = \lim_{h\to 0}\frac{P_{ij}(h)}{h}=\nu_iP_{ij}$$
\par\medskip

{\em Proof.}
\begin{align*}
P_{ij}(h)&=\p(X(h)=j|X(0) = i)\\
&=\p(X(h)=j,\mbox{1 transition in (0,h]}|X(0) = i)\\
&\quad+\p(X(h)=j,\mbox{2 or more transition in (0,h]}|X(0) = i)\\
&=\p(T_i<h)P_{ij}+o(h)\\
&=(1\!-\!e^{-\nu_ih})P_{ij}+o(h)\\
&=\nu_iP_{ij}h+o(h)
\end{align*}
\end{frame}
% ----------------------------------------------------------------------
%\begin{frame}{Theorem 6.1 Kolmogorov's Backward Equations}
%%For all states $i, j\in\X$ , and times $t\ge 0$,
%%$$P'_{ij} (t) =\Sum_{k\neq j}q_{ik}P_{kj}(t)-\nu_iP_{ij} (t)$$
%% {\em Proof.}
%From Lemma 6.3 (Chapman-Kolmogorov equations), we obtain
%\begin{align*}
%P_{ij}(h+t)-P_{ij}(t)&=\sum_{k\in\X} P_{ik}(h)P_{kj}(t)-P_{ij}(t)\\
%&=\sum_{k\in\X,k\neq i} P_{ik}(h)P_{kj}(t)-(1-P_{ii}(h))P_{ij}(t)
%\end{align*}
%and thus
%$$\lim_{h\to0}\frac{P_{ij} (t +h)-Pij (t)}{h}
%= \lim_{h\to0}\left\{\sum_{k\neq i} \frac{P_{ik}(h)}{h}P_{kj}(t)-\frac{1-P_{ii}(h)}{h}P_{ij}(t)\right\}$$
%Now assuming that we can interchange the limit and the summation in the preceding
%and applying Lemma 6.2, we obtain
%$$P'_{ij} (t) =\sum_{k\in\X,k\neq i}q_{ik}P_{kj}(t)-\nu_iP_{ij} (t)$$
%It turns out that this interchange can indeed be justified.
%\end{frame}
%% ----------------------------------------------------------------------
%\begin{frame}{Theorem 6.2 Kolmogorov's Forward Equations}
%From Lemma 6.3 (Chapman-Kolmogorov equations), we obtain
%\begin{align*}
%P_{ij}(t+h)-P_{ij}(t)&=\sum_{k\in\X} P_{ik}(t)P_{kj}(h)-P_{ij}(t)\\
%&=\sum_{k\in\X,k\neq j} P_{ik}(t)P_{kj}(h)-(1-P_{jj}(h))P_{ij}(t)
%\end{align*}
%and thus
%$$\lim_{h\to0}\frac{P_{ij} (h+t)-Pij (t)}{h}
%= \lim_{h\to0}\left\{\sum_{k\neq j} P_{ik}(t)\frac{P_{kj}(h)}{h}-\frac{1-P_{jj}(h)}{h}P_{ij}(t)\right\}$$
%Now assuming that we can interchange the limit and the summation in the preceding
%and applying Lemma 6.2, we obtain
%$$P'_{ij} (t) =\Sum_{k\neq j}P_{ik}(t)q_{kj}-\nu_jP_{ij} (t)$$
%Unfortunately, this interchange is not always justifiable. However, the forward equations do hold in most models,
%including all birth and death processes and all finite state models.
%\end{frame}
% ----------------------------------------------------------------------
\begin{frame}
%Recall that we define the instantaneous transition rates
%$$q_{ij}=\nu_iP_{ij},\quad\mbox{for} i,j\in\X, i\neq j$$
%If we define $q_{ii}$ as $-\nu_i$.
For finite state space case $\X=\{1,2,\ldots,m\}$, define the matrices
\begin{align*}
\mathbf{P}(t)&=
\begin{bmatrix}
P_{11}(t)&\cdots&P_{1m}(t)\\
\vdots   &      &\vdots\\
P_{m1}(t)&\cdots&P_{mm}(t)
\end{bmatrix},\quad
\mathbf{P}'(t)=
\begin{bmatrix}
P'_{11}(t)&\cdots&P'_{1m}(t)\\
\vdots   &      &\vdots\\
P'_{m1}(t)&\cdots&P'_{mm}(t)
\end{bmatrix},\\
\mathbf{Q}&=
\begin{bmatrix}
q_{11}&\cdots&q_{1m}\\
\vdots&      &\vdots\\
q_{m1}&\cdots&q_{mm}
\end{bmatrix}
=
\begin{bmatrix}
-\nu_1& \nu_1 P_{12} &\cdots&\nu_1 P_{1m}\\
\nu_2 P_{21}& -\nu_2 &\cdots&\nu_2 P_{2m}\\
\vdots&\vdots&      &\vdots\\
\nu_mP_{m1}&\nu_mP_{m2}&\cdots&-\nu_m
\end{bmatrix}
\end{align*}
In matrix notation,

Forward equation: $\mathbf{P}'(t)=\mathbf{P}(t)\mathbf{Q}$

Backward equation: $\mathbf{P}'(t)=\mathbf{Q}\mathbf{P}(t)$
\end{frame}
% ----------------------------------------------------------------------
\end{document}
\begin{frame}{}
\end{frame}
% ----------------------------------------------------------------------
