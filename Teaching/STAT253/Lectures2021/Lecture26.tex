%\documentclass[letterpaper,draft]{beamer}
\documentclass[letterpaper,handout]{beamer}
%\documentclass[letterpaper]{beamer}

%---multiple pages on one sheet, ADD for handout--
%\usepackage{pgfpages}
%\pgfpagesuselayout{4 on 1}[letterpaper, landscape, border shrink=1mm]
%-------------------------------------------------
\usepackage{amsmath,amsfonts}
%\usepackage[misc]{ifsym} % for the dice symbol \Cube{}
%\usepackage{booktabs}
%\usepackage{mdwlist}
%\usepackage{amsfonts}
%\usetheme{Copenhagen}
%\usetheme{warsaw}
\setbeamertemplate{navigation symbols}{}
\usepackage[english]{babel}
\def\ul{\underline}
% or whatever

\usepackage[latin1]{inputenc}
\subject{Talks}

\def\Sum{\sum\nolimits}
\def\Prod{\prod\nolimits}
\def\P{\mathbb{P}}
\def\p{\mathrm P}
\def\E{\mathbb E}
\def\V{\mathrm{Var}}
\def\CV{\mathrm{Cov}}
\def\X{\mathcal{X}}
\def\dt{\Delta}
\def\typo#1{\alert{#1}}
%-------------Answers------------
\def\Hide#1#2{\ul{~~~\onslide<#1>{\alert{#2}}~~~}}
\def\hide#1#2{\ul{~~\onslide<#1>{\alert{#2}}~~}}
%------Centered Page Number------
\defbeamertemplate{footline}{centered page number}
{%
  \hspace*{\fill}%
  %\usebeamercolor[fg]{page number in head/foot}%
  %\usebeamerfont{page number in head/foot}%
  \small Lecture \chapnum\ - \insertframenumber%
  \hspace*{\fill}\vskip2pt%
}

%\usepackage{tikz}
%\usebackgroundtemplate{%
%\tikz\node[opacity=0.3] {\includegraphics[height=\paperheight,widht=\paperwidth]{ctanlion}};}

%\usebackgroundtemplate{%
%  %\rule{0pt}{\paperheight}%
%  \parbox[c][\paperheight][c]{\paperwidth}{\centering\includegraphics[width=.65\paperwidth]{UClogo.pdf}}
%  %\hspace*{\paperwidth}
%}

\def\chapnum{26}
%--------------------------------
\setbeamertemplate{footline}[centered page number]

\title{STAT253/317 Winter 2014 Lecture \chapnum} \date{Mar 12, 2014} \author{Yibi Huang}
\begin{document}
% ----------------------------------------------------------------------
\begin{frame}\maketitle
\bigskip
\begin{center}\large
\begin{tabular}{ll}
$\bullet$ & Quadratic Variation\\
$\bullet$ &
\end{tabular}
\end{center}
\end{frame}
% ----------------------------------------------------------------------
\begin{frame}{Total Variation (First-Order Variation)}
For a function $f(t)$, we wish to compute the amount of up and down oscillation undergone by this function between 0 and $T$.

Let $\Pi =\{t_0,t_1,\ldots,t_n\}$ be a {\em partition} of $[0,T]$, which is a set of times
$$0= t_0 <t_1< t_2<\ldots t_n =T.$$
The mesh size of the partition is defined as $$\|\Pi\|=\max_{0\le j\le n-1}|t_{j+1}-t_j|.$$

The \structure{\bf total variation} of a function $f(t)$ on the interval $[0,T]$ is defined as
$$TV_T(f)=\lim_{\|\Pi\|\to 0}\Sum_{j=0}^{n-1}|f(t_{j+1})-f(t_j)|.$$
\end{frame}
% ----------------------------------------------------------------------
\begin{frame}{Total Variation (First-Order Variation) (Cont'd)}\,\par

{\bf Remark 1}. If the function $f(x)$ is monotone on $[0,T]$, then the total of $f$ on the on the interval $[0,T]$
is simply $|f(0)-f(T)|.$\vspace{0.75in}



{\bf Remark 2}. If the function $f(x)$ is monotone increasing on $[0,c]$ and monotone decreasing on $[c,T]$, then the total of $f$ on the on the interval $[0,T]$ is $|f(0)-f(c)|+|f(c)-f(T)|.$\vspace{.75in}


\medskip\hrule\medskip
\fbox{The total variation of Brownian motion in $[0,T]$ is $\infty$ for all $T>0$.}\smallskip

The proof will be given later
\end{frame}
% ----------------------------------------------------------------------
%\begin{frame}{Total Variation of Brownian Motion}
%For a differentiable function $f(t)$ with derivative $f'(t)$, by mean-value theorem, there exists
%some $t_j^*$ between $t_j$ and $t_{j+1}$ such that
%$$|f(t_{j+1})-f(t_j)|= |f'(t_j^*)|(t_{j+1}-t_j).$$
%The total variation is then
%$$TV_T(f)=\lim_{\|\Pi\|\to 0}\Sum_{j=0}^{n-1}|f'(t_j)^*|(t_{j+1}-t_j)=\int_0^T |f'(t)|dt.$$

%{\bf Proposition.}\par


%{\em Proof.}
% For any partition $\Pi =\{t_0,t_1,\ldots,t_n\}$, since $B(t_{j+1})-B(t_j)\sim N(0,\sigma^2(t_{j+1}-t_j))$ for %$j=0,1\ldots,n-1$, and are independent of each other,
%$\E|B(t_{j+1})-B(t_j)|=\sqrt{\frac{2}{\pi}(t_{j+1}-t_j)}\sigma$, we have
%\begin{align*}
%&\E\Big[\sum_{j=0}^{n-1}|B(t_{j+1})-B(t_j)|\Big]=\sqrt{\frac{2}{\pi}}\sigma \sum_{j=0}^{n-1}\sqrt{t_{j+1}-t_j}
%\to\infty \quad\mbox{as }\|\Pi\|\to 0\\
%&\V\bigg(\sum_{j=0}^{n-1}|B(t_{j+1})-B(t_j)|\bigg)=\sum_{j=0}^{n-1}\V(|B(t_{j+1})-B(t_j)|)\\
%=\,&\sum_{j=0}^{n-1}\E[B(t_{j+1})-B(t_j)]^2-(\E|B(t_{j+1})-B(t_j)|)^2\\
%=\,&\sum_{j=0}^{n-1}\sigma^2(t_{j+1}-t_j)-\frac{2}{\pi}\sigma^2(t_{j+1}-t_j)=(1-\frac{2}{\pi})\sigma^2T
%\end{align*}
%\end{frame}
% ----------------------------------------------------------------------
\begin{frame}{Quadratic Variation (Second-Order Variation)}
For a function $f(t)$ defined on the interval $[0,T]$, the \structure{\bf quadratic variation} of $f(t)$ in $[0,T]$ is defined as
$$[f,f](T)=\lim_{\|\Pi\|\to 0}\Sum_{j=0}^{n-1}[f(t_{j+1})-f(t_j)]^2.$$
For a smooth function $f$ on $[0, T]$ with continuous derivative $f'$,
by mean-value theorem, there exists
some $t_j^*$ between $t_j$ and $t_{j+1}$ such that
$$|f(t_{j+1})-f(t_j)|= |f'(t_j^*)|(t_{j+1}-t_j).$$
The quadratic variation is then
\begin{align*}
&\Sum_{j=0}^{n-1}[f'(t_j)^*]^2(t_{j+1}-t_j)^2\\
\le{}& \|\Pi\|\Sum_{j=0}^{n-1}[f'(t_j)^*]^2(t_{j+1}-t_j)\to\|\Pi\|\int_0^T |f'(t)|^2dt.
\end{align*}
If $f'(t)$ is continuous, then $\int_0^T |f'(t)|^2dt<\infty$.
As the mesh size $\|\Pi\|\to 0$, the quadratic variation of $f$ must be 0.
\end{frame}
% ----------------------------------------------------------------------
\begin{frame}{A Useful Result}
Several proofs in this lecture use the following results.\par\medskip

{\bf Proposition}. If $X_1, X_2,\ldots,X_n\ldots$ is a sequence of random variable with
$$
\lim_{n\to\infty}\E[X_n]=c\quad \text{and}\quad \lim_{n\to\infty}\V(X_n)=0
$$
then $X_n\to c$ in probability.
\vspace{1.5in}

\end{frame}
% ----------------------------------------------------------------------
\begin{frame}{Quadratic Variation of Standard Brownian Motion}
The quadratic variation of standard Brownian motion on the interval $[0,T]$ is $T$.
Here $T$ is a fixed constant.\par\medskip

{\em Proof.}
For any partition $\Pi =\{t_0,t_1,\ldots,t_n\}$, since $B(t_{j+1})-B(t_j)\sim N(0,t_{j+1}-t_j)$ for $j=0,1\ldots,n-1$, and is independent of each other, we have
\begin{align*}
\E\Big[\Sum_{j=0}^{n-1}[B(t_{j+1})-B(t_j)]^2\Big]&=\Sum_{j=0}^{n-1}(t_{j+1}-t_j)=T\\
\V\Big(\Sum_{j=0}^{n-1}[B(t_{j+1})-B(t_j)]^2\Big)&=\Sum_{j=0}^{n-1}3(t_{j+1}-t_j)^2\\
&\le 3 T \|\Pi\|\longrightarrow 0\quad\mbox{as }\|\Pi\|\to 0
\end{align*}
Thus $$\lim_{\|\Pi\|\to 0}\Sum_{j=0}^{n-1}[B(t_{j+1})-B(t_j)]^2=T.$$
\end{frame}
% ----------------------------------------------------------------------
\begin{frame}{Proof of that Brownian Motion Has Infinite Total Variation}
Suppose to the contrary that Brownian motion has finite total variation,
%and let V1(B; a, b) denote the total variation of B on the interval [a, b]. It then follows that
\begin{align*}
&\Sum_{j=0}^{n-1}[B(t_{j+1})-B(t_j)]^2\\
\le\,& \max_{0\le j\le n-1}|B(t_{j+1})-B(t_j)|\underbrace{\Sum_{j=0}^{n-1}|B(t_{j+1})-B(t_j)|}_{\to\; \text{total variation}}
\end{align*}
Since the Brownian motion path is continuous with probability 1 on $[0, T]$,
it is necessarily uniformly continuous on $[0, T]$. Therefore as the mesh size $\|\Pi\|\to 0$,
$$
\max_{0\le j\le n-1}|B(t_{j+1})-B(t_j)|\to 0 \text{ with prob. 1.}
$$
from which we conclude that $\sum_{j=0}^{n-1}[B(t_{j+1})-B(t_j)]^2\to 0$ with probability 1.
This is a contradiction to the result on the previous slide.
%\smallskip\hrule\smallskip

%It can be shown that all other variations of order $>2$ of the standard Brownian motion are 0
%$$\lim_{\|\Pi\|\to 0}\Sum_{j=0}^{n-1}[B(t_{j+1})-B(t_j)]^p=0\quad\mbox{if }p>2.$$
\end{frame}
% ----------------------------------------------------------------------
\begin{frame}{Review of Riemann--Stieltjes Integral}
The Riemann--Stieltjes integral of a real-valued function $f$ of a real variable with respect to a real function $g$ is defined as the limit of the approximating sum
$$\int_a^b f(t)dg(t)=\lim_{\|\Pi\|\to 0}\sum_{j=0}^{n-1}f(t_j^*)[g(t_{j+1})-g(t_j)]$$
where $t_j^*$ is in the $j$th subinterval $[t_{j+1}, t_{j}]$
and $\|\Pi\|$ is the mesh size $\displaystyle\max_{0\le j\le n-1}|t_{j+1}-t_j|$ of the partition $$\Pi =\{a=t_0<t_1<\ldots<t_n=b\}.$$
\end{frame}
% ----------------------------------------------------------------------
\begin{frame}{$\int_0^T B(t)dB(t)=$?}
In Riemann--Stieltjes integral, the limit does not depend on the selection of $t_j^*$ in the subinterval $[t_{j+1}, t_{j}]$.\bigskip

However, if $g(t)$ is not sufficiently smooth, the limit may depend on the  selection of $t_j^*$.
For example, if $f(t)=g(t)=$ the standard Brownian Motion $B(t)$, we will show that
\begin{align}
&\lim_{\|\Pi\|\to 0}\sum_{j=0}^{n-1}B(t_j^*)[B(t_{j+1})-B(t_j)]\cr
={}&
\begin{cases}
\frac{1}{2}B(T)^2-\frac{1}{2}T&\mbox{if }t_j^*=t_j\qquad\quad(\mbox{Ito integral})\\
\frac{1}{2}B(T)^2&\mbox{if }t_j^*=\frac{t_{j+1}+t_j}{2}\quad(\mbox{Stratonovich integral})\\
\frac{1}{2}B(T)^2+\frac{1}{2}T&\mbox{if }t_j^*=t_{j+1},
\end{cases}\label{eq:BdB}
\end{align}

Which definition should we choose?
\end{frame}
% ----------------------------------------------------------------------
\begin{frame}{Proof of Equation \eqref{eq:BdB}}
Observe that
$$\Sum_{j=0}^{n-1}B(t_j^*)[B(t_{j+1})-B(t_j)]=I+II$$
where
\begin{align*}
I &=\Sum_{j=0}^{n-1}\frac{1}{2}[B(t_{j+1})+B(t_j)][B(t_{j+1})-B(t_j)]\\
  &=\Sum_{j=0}^{n-1}\frac{1}{2}[B(t_{j+1})^2-B(t_j)^2]\\
  &=\frac{1}{2}[B(t_{n})^2-B(t_0)^2]=\frac{1}{2}B(T)^2\\
II&=\Sum_{j=0}^{n-1}\{B(t_j^*)-\frac{1}{2}[B(t_{j+1})+B(t_j)]\}[B(t_{j+1})-B(t_j)]\\
&=\frac{1}{2}\Sum_{j=0}^{n-1}\{B(t_j^*)-B(t_j)-[B(t_{j+1})-B(t_j^*)]\}\\
&\qquad\qquad\qquad\times[B(t_{j+1})-B(t_j^*)+B(t_j^*)-B(t_j)]\\
&=\frac{1}{2}\Sum_{j=0}^{n-1}\{[B(t_j^*)-B(t_j)]^2-[B(t_{j+1})-B(t_j^*)]^2\}
\end{align*}
\end{frame}
% ----------------------------------------------------------------------
\begin{frame}{Proof of Equation \eqref{eq:BdB} (Cont'd)}
For $t_j^*=t_j$, observe that
\begin{align*}
II&= \frac{1}{2}\Big\{
\Sum_{j=0}^{n-1}[B(t_j^*)-B(t_j)]^2-
\Sum_{j=0}^{n-1}[B(t_{j+1})-B(t_j^*)]^2\Big\}\\
&=\frac{1}{2}\Big\{
\Sum_{j=0}^{n-1}\underbrace{[B(t_j)-B(t_j)]^2}_{=0}-
\underbrace{\Sum_{j=0}^{n-1}[B(t_{j+1})-B(t_j)]^2}_{\text{quadratic variation}}\Big\}\\
&\to \frac{1}{2}(0-T)=-\frac{T}{2}\quad\text{in probability as }\max_{0\le j\le n-1}|t_{j+1}-t_j|\to 0
\end{align*}

Similarly, for $t_j^*=t_{j+1}$, observe that
\begin{align*}
II&= -\frac{1}{2}\Big\{
\underbrace{\Sum_{j=0}^{n-1}[B(t_{j+1})-B(t_j)]^2}_{\text{quadratic variation}}+
\Sum_{j=0}^{n-1}\underbrace{[B(t_{j+1})-B(t_{j+1})]^2}_{=0}\Big\}\\
&\to \frac{1}{2}(T-0)=\frac{T}{2}\quad\text{in probability as }\max_{0\le j\le n-1}|t_{j+1}-t_j|\to 0
\end{align*}
\end{frame}
% ----------------------------------------------------------------------
\begin{frame}{Proof of Equation \eqref{eq:BdB} (Cont'd)}
For $t_j^*=(t_{j+1}+t_j)/2$,
\begin{align*}
II&= \frac{1}{2}\Big\{
\sum_{j=0}^{n-1}[B(\frac{t_{j+1}+t_j}{2})-B(t_j)]^2-
\sum_{j=0}^{n-1}[B(t_{j+1})-B(\frac{t_{j+1}+t_j}{2})]^2\Big\}
\end{align*}
So
\begin{align*}
\E[II]&=\frac{1}{2}\Sum_{j=0}^{n-1}\frac{(t_{j+1}-t_j)}{2}-\frac{(t_{j+1}-t_j)}{2}=0\\
\V(II)&=\frac{1}{4}\Sum_{j=0}^{n-1}3(\frac{t_{j+1}-t_j}{2})^2+3(\frac{t_{j+1}-t_j}{2})^2\\
&=\frac{3}{8}\Sum_{j=0}^{n-1}(t_{j+1}-t_j)^2\le \frac{3}{8}\|\Pi\|T\to 0\mbox{ as }\|\Pi\|\to 0
\end{align*}
The above shows $II\longrightarrow 0$ in probability as $\|\Pi\|\to 0$.
\end{frame}
% ----------------------------------------------------------------------
\end{document}
% ----------------------------------------------------------------------
\begin{frame}
\end{frame}
% ----------------------------------------------------------------------
\begin{frame}
For $t_j^*=t_j$, observe that
\begin{align*}
&\frac{1}{2}\Sum_{j=0}^{n-1}[B(t_{j+1})-B(t_j)]^2\\
={}&\frac{1}{2}\Sum_{j=0}^{n-1}B(t_{j+1})^2-\Sum_{j=0}^{n-1}B(t_j)B(t_{j+1})+\frac{1}{2}\Sum_{j=0}^{n-1}B(t_{j})^2\\
={}&\frac{1}{2}B(t_n)^2+\frac{1}{2}\Sum_{j=0}^{n-1}B(t_{j})^2-\Sum_{j=0}^{n-1}B(t_j)B(t_{j+1})+\frac{1}{2}\Sum_{j=0}^{n-1}B(t_{j})^2\\
={}&\frac{1}{2}B(t_n)^2+\Sum_{j=0}^{n-1}B(t_{j})^2-\Sum_{j=0}^{n-1}B(t_j)B(t_{j+1})\\
={}&\frac{1}{2}B(T)^2+\Sum_{j=0}^{n-1}B(t_{j})[B(t_j)-B(t_{j+1})]
\end{align*}
Thus we have
$$\sum_{j=0}^{n-1}B(t_{j})[B(t_{j+1})-B(t_j)]
=\frac{1}{2}B(T)^2-\frac{1}{2}\underbrace{\sum_{j=0}^{n-1}[B(t_{j+1})-B(t_j)]^2}_{\to\; \mbox{quadratic variation}}$$
As $\|\Pi\|\to 0$, the last term approaches half of the quadratic variation $[B,B](T)$, which is $T$.
\end{frame}
% ----------------------------------------------------------------------
