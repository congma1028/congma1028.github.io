%\documentclass[letterpaper,draft]{beamer}
\documentclass[letterpaper,handout]{beamer}
%\documentclass[letterpaper]{beamer}

%---multiple pages on one sheet, ADD for handout--
%\usepackage{pgfpages}
%\pgfpagesuselayout{4 on 1}[letterpaper, landscape, border shrink=1mm]
%-------------------------------------------------
\usepackage{amsmath,amsfonts}
%\usepackage{booktabs}
%\usepackage{mdwlist}
\usepackage{amsfonts}
%\usetheme{Copenhagen}
%\usetheme{warsaw}
\setbeamertemplate{navigation symbols}{}
\usepackage[english]{babel}
\def\ul{\underline}
% or whatever

\usepackage[latin1]{inputenc}
\subject{Talks}

\def\Sum{\sum\nolimits}
\def\Prod{\prod\nolimits}
\def\p{\mathrm P}
\def\E{\mathbb E}
\def\V{\mathrm{Var}}
\def\X{\mathcal{X}}
\def\typo#1{\alert{#1}}
%-------------Answers------------
\def\Hide#1#2{\ul{~~~\onslide<#1>{\alert{#2}}~~~}}
\def\hide#1#2{\ul{~~\onslide<#1>{\alert{#2}}~~}}
%------Centered Page Number------
\defbeamertemplate{footline}{centered page number}
{%
  \hspace*{\fill}%
  %\usebeamercolor[fg]{page number in head/foot}%
  %\usebeamerfont{page number in head/foot}%
  \small Lecture \chapnum\ - \insertframenumber%
  \hspace*{\fill}\vskip2pt%
}

%\usepackage{tikz}
%\usebackgroundtemplate{%
%\tikz\node[opacity=0.3] {\includegraphics[height=\paperheight,widht=\paperwidth]{ctanlion}};}

%\usebackgroundtemplate{%
%  %\rule{0pt}{\paperheight}%
%  \parbox[c][\paperheight][c]{\paperwidth}{\centering\includegraphics[width=.65\paperwidth]{UClogo.pdf}}
%  %\hspace*{\paperwidth}
%}

\def\chapnum{16}
%--------------------------------
\setbeamertemplate{footline}[centered page number]

\title{STAT253/317 Lecture \chapnum} \date{} \author{Yibi Huang}
\begin{document}
% ----------------------------------------------------------------------
\begin{frame}\maketitle\begin{center}7.3\; Limit Theorems\end{center}\end{frame}
% ----------------------------------------------------------------------
\begin{frame}{Revision of Example 7.7}
A coin with probability $p$ to land heads (and $q=1-p$ to land tails) is tossed continually.
\begin{itemize}
\item What is the probability to get {\bf $k$ heads in a row} before getting {\bf $k$ tails in a row}?
\item How many tosses is expected to get $k$ heads in a row or $k$ tails in a row?
\end{itemize}\medskip

\textbf{Solution.}
\begin{itemize}
\item Suppose the coin is tossed at every integer time points $t=1,2,3\ldots.$
\item An {\em event} occurs whenever getting $k$ heads in a row.\\
Here we required the rows of heads for different events must be \underline{non-overlapping}.
\item Define
$N_H(t)=\mbox{\# of events occurred at or before time }t.$\\
$\{N_H(t),\; t\ge 0\}$ is a renewal processes (why?).
\item What is the mean length of the interarrival times for $\{N_H(t)\; t\ge 0\}$?
\end{itemize}
\end{frame}
% ----------------------------------------------------------------------
\begin{frame}{Revision of Example 7.7 (Cont'd)}
Let $T_k=$  the \# of tosses required to get $k$ heads in a row.\\
To get $k$ consecutive heads, one must first get $k-1$ consecutive heads, which takes $T_{k-1}$ steps.
So
$$
T_k=
\begin{cases}
T_{k-1}+1 &\mbox{w/ prob. } p, \quad\text{ (if heads in the next toss)}\\
T_{k-1}+1+T'_k   &\mbox{w/ prob. } 1-p \text{ (if tails in the next toss)}.
\end{cases}
$$
Here $T_k'\sim T_k$ and $T_k'$ is independent of the past, and hence is independent of $T_{k-1}.$
So
\begin{align*}
\E[T_k|T_{k-1}]
&=T_{k-1}+1+(1-p)\E[T'_k|T_{k-1}]\\
&=T_{k-1}+1+(1-p)\E[T'_k]\quad(\text{since $T'_k, T_{k-1}$ are indep.})\\
&=T_{k-1}+1+(1-p)\E[T_k]\quad(\text{since $T'_k\sim T_k$})
\end{align*}
and hence
\begin{align*}
\E[T_k]=\E[\E[T_k|T_{k-1}]]&=\E[T_{k-1}]+1+(1-p)\E[T_k]
\end{align*}
Subtracting $(1-p)\E[T_k]$ from both sides we get
$$p\E[T_k] = \E[T_{k-1}] + 1,\quad k=2, 3, 4, \ldots.$$
\end{frame}
% ----------------------------------------------------------------------
\begin{frame}{Revision of Example 7.7 (Cont'd)}
Observe that $T_1$ has a geometric distribution and hence$\E[T_1]=1/p.$
Using the iterative relation, we get that
\begin{align*}
\E[T_2]&=(\E[T_1] + 1)/p=1/p^2+1/p\\
\E[T_3]&=(\E[T_2] + 1)/p=1/p^3+1/p^2+1/p\\[-3pt]
\vdots\\[-8pt]
\E[T_k]&=(\E[T_{k-1}] + 1)/p=1/p^k+\ldots+1/p^2+1/p=\frac{1-p^k}{p^k(1-p)}
\end{align*}
By Proposition 7.1, we know
$$\lim_{t\to\infty}\frac{N_H(t)}{t}=\frac{1}{\E[T_k]}=\frac{p^k(1-p)}{1-p^k}=\frac{p^kq}{1-p^k}.$$

Similarly, considering the renewal process
$$N_T(t)=\mbox{\# of times to see $k$ tails in a row by time }t.$$
By Proposition 7.1, we also have
$\displaystyle\lim_{t\to\infty}\frac{N_T(t)}{t}=\frac{q^kp}{1-q^k}.$
\end{frame}
% ----------------------------------------------------------------------
\begin{frame}{Example 7.7  Solution (Cont'd)}
Consider the counting process $N(t)=N_H(t)+N_T(t).$
\begin{itemize}
\item $\{N(t): t\ge 0\}$ is also a renewal process (Why?)\\
Here an event occurs whenever one gets $k$ heads in a row  or $k$ tails in a row.
\item As $t\to\infty,$
$$
\frac{N(t)}{t}=\frac{N_H(t)}{t}+\frac{N_T(t)}{t}\to\frac{p^kq}{1-p^k}+\frac{q^kp}{1-q^k}.
$$
which is the reciprocal of mean length $\E[T]$ of interarrival times for $\{N(t): t\ge 0\}$.
\item From the above we can answer the 2nd question:
the expected number of tosses required to get $k$ heads in a row or $k$ tails in a row is
$$\E[T]=\frac{1}{p^kq/(1-p^k)+q^kp/(1-q^k)}.$$
\end{itemize}
\end{frame}
% ----------------------------------------------------------------------
\begin{frame}{Example 7.7  Solution (Cont'd)}
With probability 1, the long run proportion of events that are $k$-heads is
\begin{align*}
\lim_{t\to\infty}\frac{N_H(t)}{N(t)}&=\lim_{t\to\infty}\frac{N_H(t)}{N_H(t)+N_T(t)}
=\lim_{t\to\infty}\frac{N_H(t)/t}{N_H(t)/t+N_T(t)/t}\\
&=\frac{p^kq/(1-p^k)}{p^kq/(1-p^k)+q^kp/(1-q^k)}
\end{align*}

The probability of getting $k$-heads before $k$-tails is,
by SLLN,
equal to the long-run proportion of events that are $k$-heads.
$$\p(\text{$k$-heads before $k$-tails})=\frac{p^kq/(1-p^k)}{p^kq/(1-p^k)+q^kp/(1-q^k)}$$
\end{frame}
% ----------------------------------------------------------------------
\begin{frame}{Theorem 7.2\; CLT for Renewal Processes}
Suppose that $\mu$ and $\sigma^2$ are, respectively, the mean and variance
of the interarrival times of a renewal process $\{N(t),t\ge 0\}$. Then
$$\lim_{t\to\infty}\frac{\V(N(t))}{t}=\frac{\sigma^2}{\mu^3},$$
and $N(t)$ is asymptotically $N(t/\mu, \sigma^2t/\mu^3)$, i.e.,
$$
\lim_{t\to\infty}\p\left(\frac{N(t)-t/\mu}{\sqrt{t\sigma^2/\mu^3}}<x\right)=\Phi(x)
$$
where $\Phi(x)=\int_{-\infty}^x\frac{1}{\sqrt{\pi}}e^{-z^2/2}dz$ is the CDF of the standard normal distribution.
\end{frame}
% ----------------------------------------------------------------------
\begin{frame}{Proof of Theorem 7.2}

\vspace{-10pt}
\begin{align*}
&\p\left(\frac{N(t)-t/\mu}{\sqrt{t\sigma^2/\mu^3}}<x\right)\\
=\,&\p\left(N(t)<\frac{t}{\mu}+\frac{\sigma}{\mu}\sqrt{\frac{t}{\mu}}x\right)\\
=\,&\p(N(t)\le n)&\Big(\mbox{Let }n=\Big\lfloor\frac{t}{\mu}+\frac{\sigma}{\mu}\sqrt{\frac{t}{\mu}}x\Big\rfloor\Big)\\
=\,&\p(S_n\ge t)&\mbox{(Recall } N(t)\le n \Leftrightarrow S_n\ge t)\\
=\,&\p\left(\frac{S_n-n\mu}{\sqrt{n}\sigma}\ge \frac{t-n\mu}{\sqrt{n}\sigma}\right)\\
\longrightarrow\,&1-\Phi\left(\frac{t-n\mu}{\sqrt{n}\sigma}\right)&\mbox{as }n\to\infty\mbox{ by the CLT for }S_n\\
=\,&\Phi\left(-\frac{t-n\mu}{\sqrt{n}\sigma}\right)&(\mbox{since }1-\Phi(z)=\Phi(-z))
\end{align*}
\hrule\medskip
\footnotesize{\quad Here $\lfloor y\rfloor$ means the greatest integer less or equal to $y$}
\end{frame}
% ----------------------------------------------------------------------
\begin{frame}{Proof of Theorem 7.2 (Cont'd)}
It remains to show that
$$-\frac{t-n\mu}{\sqrt{n}\sigma}\longrightarrow x\quad\mbox{as }t\to\infty.
$$
Since $n\le \frac{t}{\mu}+\frac{\sigma}{\mu}\sqrt{\frac{t}{\mu}}x$, as $t\to\infty$, we have
$$\frac{t-n\mu}{\sqrt{n}\,\sigma}\ge \frac{t-(\frac{t}{\mu}+\frac{\sigma}{\mu}\sqrt{\frac{t}{\mu}}x)\mu}{\sigma\sqrt{\frac{t}{\mu}+\frac{\sigma}{\mu}\sqrt{\frac{t}{\mu}}x}}
=\frac{-\sigma x\sqrt{t/\mu}}{\sigma\sqrt{\frac{t}{\mu}+\frac{\sigma}{\mu}\sqrt{\frac{t}{\mu}}x}}\longrightarrow -x.$$
Similarly because $n\ge \frac{t}{\mu}+\frac{\sigma}{\mu}\sqrt{\frac{t}{\mu}}-1$, we have
$$\frac{t-n\mu}{\sqrt{n}\,\sigma}\le \frac{t-(\frac{t}{\mu}+\frac{\sigma}{\mu}\sqrt{\frac{t}{\mu}}x-1)\mu}{\sigma\sqrt{\frac{t}{\mu}+\frac{\sigma}{\mu}\sqrt{\frac{t}{\mu}}x}}
=\frac{-\sigma x\sqrt{t/\mu}-\mu}{\sigma\sqrt{\frac{t}{\mu}+\frac{\sigma}{\mu}\sqrt{\frac{t}{\mu}}x-1}}\longrightarrow -x$$
as $t\to\infty$.
\end{frame}
% ----------------------------------------------------------------------
\end{document}
\begin{frame}{7.4 Renewal Reward Processes}
Let $\{N(t),t\ge 0\}$ be a renewal process with i.i.d. interarrival times $X_i$, $i=1,2,\ldots.$
Let $R_i$, $i=1,2,\ldots$ be i.i.d random variables.
$R_i$ may depend on the $i$th interarrival time $X_i$,
but $(X_i,R_i)$ are i.i.d. random variable pairs.
The compound process
$$R(t)=\Sum_{i=1}^{N(t)}R_i$$
is called a \structure{\em renewal reward process}.
$R_i$ may be considered as reward earned during the $i$th cycle,
and $R(t)$ represents the total reward earned up to time $t$.\bigskip

\textbf{Proposition 7.3} If $\E[R_1] <\infty$ and $\E[X_1] <\infty$, then

(a) with probability 1,
$\displaystyle{\lim_{t\to\infty}\frac{R(t)}{t}= \frac{\E[R_1]}{\E[X_1]}}$

(b) $\displaystyle{\lim_{t\to\infty}\frac{\E[R(t)]}{t}= \frac{\E[R_1]}{\E[X_1]}}$
\end{frame}
% ----------------------------------------------------------------------
\begin{frame}{Proof of Proposition 7.3(a)}
We give the proof for (a) only. To prove this, write
$$\frac{R(t)}{t}=\frac{\sum^{N(t)}_{n=1} R_n}{t}=\frac{\sum^{N(t)}_{n=1} R_n}{N(t)}\times\frac{N(t)}{t}$$
By the strong law of large numbers we obtain
$$\frac{\sum^{N(t)}_{n=1} R_n}{N(t)}\to\E[R]\quad\mbox{as }t\to\infty$$
and by Proposition 7.1
$$\frac{N(t)}{t}\to\frac{1}{\E[X_1]}\quad\mbox{as }t\to\infty$$
The result thus follows.
\end{frame}
% ----------------------------------------------------------------------
\begin{frame}{Example 7.12 (A Car Buying Model)}
\begin{itemize}
\item Mr. Brown buy a new car whenever his old one breaks down or reaches the age of $T$ years
\item Let $Y_i$ be the lifetime of his $i$th car. Suppose $Y_i$'s are i.i.d with
$$\mbox{CDF }\;H(y)=\p(Y\le y),\quad \mbox{and density } h(y)=H'(y).$$
\item Cost to by a new car $=C_1$;
\item If the car breaks down, an additional cost of $C_2$ is incurred.
\item What's Mr. Brown's long run average cost?
\end{itemize}
\end{frame}
% ----------------------------------------------------------------------
\begin{frame}{Example 7.12 (A Car Buying Model) Solution}
\begin{itemize}
\item An event occurs whenever Mr. Brown buys a new car
\item Interarrival times: $X_i=\min(Y_i,T)$
\item Cost incurred in the $i$th cycle: $R_i=C_1+C_2\mathbf{1}_{\{Y_i\le T\}}$
\item Are $(X_i, R_i)$, $i=1,2,\ldots$ i.i.d?
\item $$\E[X_i]=\int_0^{\infty} \min(y,T)h(y)dy=\int_0^{T} yh(y)dy+T(1-H(T))$$
\item $$\E[R_i]=C_1+C_2\p(Y_i\le T)=C_1+C_2H(T)$$
\item long-run average cost
$$=\frac{C_1+C_2H(T)}{\int_0^{T} yh(y)dy+T(1-H(T))}$$
\end{itemize}
\end{frame}
% ----------------------------------------------------------------------
\begin{frame}{Example 7.16 \& 7.17}
Let $\{N(t),t\ge 0\}$ be a renewal process with i.i.d. interarrival times $X_i$, $i=1,2,\ldots.$
Consider the current age of the item in use at time $t$
$$A(t)=t-S_{N(t)}.$$
What is the long-run average of age $$\lim_{t\to\infty}\frac{\int_0^t A(s)ds}{t}?$$
Also consider the residual life of the item in use at time $t$
$$Y(t)=S_{N(t)+1}-t.$$
What is the long-run average of residual life $$\lim_{t\to\infty}\frac{\int_0^t Y(s)ds}{t}?$$
\end{frame}
% ----------------------------------------------------------------------
\begin{frame}{Solution to Example 7.16}
Let's try to turn $\int_0^t A(s)ds$ into a renewal reward process:

Note $\int_0^{S_{N(t)}} A(s)ds\le\int_0^t A(s)ds<\int_0^{S_{N(t)+1}} A(s)ds,$
and
\begin{align*}
\int_0^{S_{N(t)}} A(s)ds&=\sum_{i=1}^{N(t)}\int_{S_{i-1}}^{S_i} A(s)ds=\sum_{i=1}^{N(t)}\int_{S_{i-1}}^{S_i} s-S_{i-1}ds\\
&=\sum_{i=1}^{N(t)}\int_{0}^{X_i} udu\quad(\mbox{let }u=s-S_{i-1})\\
&=\sum_{i=1}^{N(t)}\frac{X_i^2}{2}=R(t),
\end{align*}
where $R(t)=\Sum_{i=1}^{N(t)}R_i$ is a renewal reward process with $R_i=X_i^2/2$.
Similarly, one can show that
$$\int_0^{S_{N(t)+1}} A(s)ds=\Sum_{i=1}^{N(t)+1}R_i=R(t)+R_{N(t)+1}.$$
\end{frame}
% ----------------------------------------------------------------------
\begin{frame}{Solution to Example 7.16 (Cont'd)}
Since $$R(t)\le\int_0^t A(s)ds<R(t)+R_{N(t)+1},$$
and $$\frac{R_{N(t)+1}}{t}=\frac{X_{N(t)+1}^2}{2t}\to 0\quad\mbox{ as }t\to\infty,$$
by Proposition 7.3, the long-run average age of the item in use is
$$
\lim_{t\to\infty}\frac{\int_0^t A(s)ds}{t}
=\lim_{t\to\infty}\frac{R(t)}{t}=\frac{\E[R_1]}{\E[X_1]}=\frac{\E[X_1^2]}{2\E[X_1]}
$$
\end{frame}
% ----------------------------------------------------------------------
\begin{frame}{Solution to Example 7.17}
Similarly, from that
$$\int_0^{S_{N(t)}} Y(s)ds\le\int_0^t Y(s)ds<\int_0^{S_{N(t)+1}} Y(s)ds,$$
one can show that
\begin{align*}
\int_0^{S_{N(t)}} Y(s)ds &=\sum_{i=1}^{N(t)}\int_{S_{i-1}}^{S_i}(S_{i}-s)ds=\sum_{i=1}^{N(t)}\int_{0}^{X_i} udu\quad(\mbox{let }u=S_{i}-s)\\
&=\sum_{i=1}^{N(t)}\frac{X_i^2}{2}=R(t)
\end{align*}
and that $\int_0^{S_{N(t)+1}} Y(s)ds=\sum_{i=1}^{N(t)+1}\frac{X_i^2}{2}=R(t)+R_{N(t)+1}.$
By the same argument, the long-run average of residual life of the item in use is
$$
\lim_{t\to\infty}\frac{\int_0^t Y(s)ds}{t}
=\lim_{t\to\infty}\frac{R(t)}{t}=\frac{\E[R_1]}{\E[X_1]}=\frac{\E[X_1^2]}{2\E[X_1]}
$$
\end{frame}
% ----------------------------------------------------------------------
\end{document}
\begin{frame}{}
\end{frame}
% ----------------------------------------------------------------------
\begin{frame}
\end{frame}
% ----------------------------------------------------------------------
